\htmlhr
\section{Tainting checker\label{tainting-checker}}

The tainting checker prevents certain kinds of trust errors.  If the
checker issues no warning for a given program, then the program only
uses trusted values and sanitizes all tainted references before use.

Using tainted references before sanitizing them expose the application
to security vulnerabilities.  For example, web applications that construct
database queries with user input are venerable to
\ahref{http://en.wikipedia.org/wiki/Sql_injection}{SQL injection}.

Note: This does \emph{not} mean that your program is free from trust errors.
(You might have forgotten to annotation a particular method requires trusted
references only.)

\subsection{Tainting annotations\label{tainting-annotations}}

The Tainting type system uses one annotation:
\code{@\refclass{tainting/quals}{Untainted}}.  The annotation indicates
a type that includes only untainted trusted values.

Any type not marked as \code{Untainted} is treated as tainted untrusted.

\subsection{Writing \code{@Untainted} annotations\label{writing-untainted}}

In order to use the tainting checker, you must annotate your code with
\code{@\refclass{tainting/quals}{Untainted}} type annotation, to mark
operations that require trusted operations.

It is helpful to start annotating the secure kernel boundary entry
points.  To secure against SQL injection attacks, it is useful to
start annotating the \sunjavadoc{java/sql/Statement.html}{Statement}
class; the execute operations may only operate on untainted queries
(Section~\ref{annotating-libraries} describes how you can annotate
external libraries)

\begin{Verbatim}
  public boolean execute(@Untainted String sql) throws SQLException;
  public boolean executeUpdate(@Untainted String sql) throws SQLException; 
\end{Verbatim}

The Tainting checker, in turn, will verify that no untainted value may flow into
any of these methods.

% LocalWords:  quals
