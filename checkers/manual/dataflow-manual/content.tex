
\section{Introduction}

This document describes a \emph{dataflow analysis framework} for the Java Programming Language.
In particular, the framework is designed to be used in the Checker Framework, a framework
that adds pluggable types to the Java type system.

The primary purpose of the dataflow framework is to enable flow-sensitive type checking in the Checker Framework, which serves to reduce the burden of annotating a program.  The dataflow framework was designed with several goals in mind.  First, to encourage other uses of the framework, it is written as a separate package that can be built and used with no dependence on the Checker Framework.  Second, the framework is currently intended to support analysis but not transformation, so it provides information that can be used by a type checker or an IDE, but it does not support optimization.  Third, the framework aims to minimize the burden on developers who build on top of it.  In particular, the hierarchy of analysis classes is designed to reduce the effort required to implement a new flow-sensitive type checker in the Checker Framework.

The Dataflow Framework's result (\autoref{sec:analysis_result_class}) is an
abstract value for each expression
(an estimate of the expression's run-time value) and a store at each
program point.  A store maps variables and other distinguished expressions
to abstract values.  As a pre-pass, the Dataflow Framework transforming an
input AST into a control flow graph (\autoref{sec:cfg}) consisting of basic
blocks made up of
nodes representing single operations.  To produce its output, the Checker
Framework performs iterative data flow analysis over the control flow
graph.  The effect of a single node on the dataflow store is represented by
a transfer function, which takes an input store and a node and produces an
output store.  Once the analysis reaches a fix point, the result can be
accessed by client code.

In the Checker Framework, the abstract values to be computed are annotated
types.  An individual checkers can customize its analysis by extending the
abstract value class and by overriding the behavior of the transfer
function for particular node types.

\begin{workinprogress}    
    Paragraphs colored in gray with a gray bar on the left side (just like this one)
    contain questions, additional comments or indicate missing parts. Eventually, these
    paragraphs will be removed.
\end{workinprogress}





\section{Organization}

\subsection{Projects}

The source code of the combined Checker Framework and Dataflow Framework is divided into three projects: \code{javacutils}, \code{dataflow}, and \code{checkers}, which can be built into distinct jar files. 

\code{javacutils} provides convenient interfaces to routines in Oracle's javac library.  There are utility classes for interacting with annotations, elements, trees and types, as well as \code{InternalUtils}, which gives direct access to internal features of javac that are not part of a supported interface.  There are interfaces or abstract classes for reporting errors, for processing types in an AST, and for providing the annotations present on an Element.  The \code{javacutils.trees} package provides a class to parse expressions into javac Trees, a class to build new Trees from scratch, and a class to represent newly introduced variables that are not part of an input program.
 
\code{dataflow} contains the classes to represent and construct control flow graphs and the base classes required for flow analysis.   These classes are described in detail in \autoref{sec:node_classes}.
  
\code{checkers} contains the Checker Framework and the type system-specific checkers, including the derived classes for flow analysis of annotated types which are described later in this document.

The \code{dataflow} project depends on \code{javacutils}, while the \code{checkers} project depends on both \code{dataflow} and \code{javacutils}.

\subsection{Classes}

This section gives an overview of the major Java classes and interfaces in the implementation of the Dataflow Framework and the flow-sensitive type checking feature of the Checker Framework.  It includes both the base classes in the \code{dataflow} project and the derived classes in the \code{checkers} project.  The class and interface declarations are given with full package names to indicate which project they belong to.

\subsubsection{Nodes}
\label{sec:node_classes}

    A Node class represents an individual operation of a program, including arithmetic operations, logical operations, method calls, variable references, array accesses, etc.  \autoref{tab:nodes} lists the Node types.

\begin{workinprogress}
Need to double-check whether the code adds exceptions properly.  Also, does
division possibly throw a DivideByZero exception?
\end{workinprogress}


    \code{class dataflow.cfg.node.*Node}

\subsubsection{Blocks}
\label{sec:block_classes}

    Nodes are grouped into basic blocks using a hierarchy of Block classes.  The hierarchy is composed of five interfaces, two abstract classes, and four concrete classes.  

             \code{interface dataflow.cfg.block.Block}
    
              \code{abstract class dataflow.cfg.block.BlockImpl implements Block}

              \code{interface dataflow.cfg.block.SingleSuccessorBlock extends Block}

              \code{abstract class dataflow.cfg.block.SingleSuccessorBlockImpl extends BlockImpl implements
        SingleSuccessorBlock}

    A RegularBlock contains no exception-raising operations and has a single control-flow successor.

    \code{interface dataflow.cfg.block.RegularBlock extends SingleSuccessorBlock}
        
    \code{class dataflow.cfg.block.RegularBlockImpl extends SingleSuccessorBlockImpl implements
        RegularBlock}
        
    An ExceptionBlock contains a single operation that may raise an exception, with one or more exceptional successors and a single normal control-flow successor.
            
    \code{interface dataflow.cfg.block.ExceptionBlock extends SingleSuccessorBlock}

    \code{class dataflow.cfg.block.ExceptionBlockImpl extends SingleSuccessorBlockImpl implements
        ExceptionBlock}
        
    A SpecialBlock represents method entry or exit, including exceptional
    exit which is represented separately from normal exit.

    \code{interface dataflow.cfg.block.SpecialBlock extends SingleSuccessorBlock}

    \code{class dataflow.cfg.block.SpecialBlockImpl extends SingleSuccessorBlockImpl implements
        SpecialBlock}

    A ConditionalBlock contains no operations at all.  It represents a control-flow split to either a `then' or an `else' successor based on the immediately preceding boolean-valued Node.

    \code{interface dataflow.cfg.block.ConditionalBlock extends Block}
    
    \code{class dataflow.cfg.block.ConditionalBlockImpl extends BlockImpl implements ConditionalBlock}
    


\subsubsection{ControlFlowGraph}
\label{sec:control_flow_graph_class}

     A ControlFlowGraph represents the body of a method or an initializer expression as a graph of Blocks with distinguished entry, exit, and exceptional exit SpecialBlocks.  ControlFlowGraphs are produced by the CFGBuilder classes and are treated as immutable once they are built.

        \code{class dataflow.cfg.ControlFlowGraph}

\subsubsubsection{CFGBuilder}
\label{sec:cfg_builder_classes}
    
    The CFGBuilder classes visit an AST and produce a corresponding ControlFlowGraph as described in \autoref{sec:ast_to_cfg_translation}.
    
        \code{class dataflow.cfg.CFGBuilder}
        
    The Checker Framework derives from CFGBuilder in order to desugar enhanced for loops that make explicit use of type annotations provided by the checker in use.
        
        \code{class checkers.flow.CFCFGBuilder extends CFGBuilder}


\subsubsection{FlowExpressions}
\label{sec:flow_expressions_class}

    The Dataflow Framework records the abstract values of certain
    expressions, called FlowExpressions:  local variables, field accesses,
    array accesses, references to \code{this}, and pure method calls.
    FlowExpressions are keys in the store of abstract values.

    \code{class dataflow.analysis.FlowExpressions}
    
    Java expressions that appear in method pre- and postconditions are parsed into FlowExpressions using helper routines in \code{checkers.util.FlowExpressionParseUtil}.

    
\subsubsection{AbstractValue}
\label{sec:abstract_value_classes}

    AbstractValue is the internal representation of dataflow information
    produced by an analysis.  An AbstractValue is an estimate about the
    run-time values that an expression may evaluate to.  The client of the
    Dataflow Framework defines the abstract value, so the information may
    vary widely among different users of the Dataflow Framework, but they
    share a common feature that one can compute the least upper bound of
    two AbstractValues.

        \code{interface dataflow.analysis.AbstractValue<V extends AbstractValue<V>>}
        
    For the Checker Framework, abstract values are essentially AnnotatedTypeMirrors.
        
        \code{abstract class checkers.flow.CFAbstractValue<V extends CFAbstractValue<V>> implements AbstractValue<V>}
        
        \code{class checkers.flow.CFValue extends CFAbstractValue<CFValue>}
        
    For the NullnessCheckers, abstract values additionally track the meaning of PolyNull, which may be either Nullable or NonNull.
        
        \code{class checkers.nullness.NullnessValue extends CFAbstractValue<NullnessValue>}
        
\begin{workinprogress}    
Don't (nearly) all type systems support \code{Poly...}, so wouldn't that
statement be true of most type checkers, not just the NullnessCheckers?
\end{workinprogress}



\subsubsection{Store}
\label{sec:store_classes}

    A Store is a set of dataflow facts computed by an analysis, so it is a mapping from FlowExpressions to AbstractValues.  As with AbstractValues, one can take the least upper bound of two Stores.

    \code{interface dataflow.analysis.Store<S extends Store<S>>}

    The Checker Framework store restricts the type of abstract values it may contain.

    \code{abstract class checkers.flow.CFAbstractStore<V extends CFAbstractValue<V>,
        S extends CFAbstractStore<V, S>>
    implements Store<S>}
    
    \code{class checkers.flow.CFStore extends CFAbstractStore<CFValue, CFStore>}
    
    An InitializationStore tracks which fields of the `self' reference have been initialized.
    
    \code{class checkers.initialization.InitializationStore<
        V extends CFAbstractValue<V>, S extends InitializationStore<V, S>>
    extends CFAbstractStore<V, S>}
    
    A NullnessStore additionally tracks the meaning of PolyNull.
    
    \code{class checkers.nullness.NullnessStore extends
        InitializationStore<NullnessValue, NullnessStore>}


\subsubsection{Transfer functions}
\label{sec:transfer_functions}

    A transfer function (\autoref{sec:transfer_function_classes}) is
    explicitly represented as a node visitor that takes a TransferInput
    (\autoref{sec:transfer_input_classes}) and produces a TransferResult
    (\autoref{sec:transfer_result_classes}).

\subsubsubsection{TransferInput}
\label{sec:transfer_input_classes}

The TransferInput represents the set of dataflow facts known to be true immediately before the node to be analyzed.  TransferInputs may contain a single store, or a pair of `then' and `else' stores when following a boolean-valued expression.

    \code{class dataflow.analysis.TransferInput<
        A extends AbstractValue<A>,
        S extends Store<S>>}

\subsubsubsection{TransferResult}
\label{sec:transfer_result_classes}

    A TransferResult is the output of a transfer function.  In other words, it is the set of dataflow facts known to be true immediately after a node.  A Boolean-valued expression produces a ConditionalTransferResult that contains both a `then' and an `else' store, while most other Nodes produce a RegularTransferResult with a single store.

    \code{abstract class dataflow.analysis.TransferResult<
        A extends AbstractValue<A>,
        S extends Store<S>>}
        
    \code{class dataflow.analysis.ConditionalTransferResult<
        A extends AbstractValue<A>,
        S extends Store<S>>
    extends TransferResult<A, S>}

    \code{class dataflow.analysis.RegularTransferResult<
        A extends AbstractValue<A>,
        S extends Store<S>>
    extends TransferResult<A, S>}

\subsubsubsection{TransferFunction}
\label{sec:transfer_function_classes}

    A TransferFunction is a NodeVisitor that takes an input and produces an output.

    \code{interface dataflow.analysis.TransferFunction<
        A extends AbstractValue<A>,
        S extends Store<S>>
    extends NodeVisitor<TransferResult<A, S>, TransferInput<A, S>>}

    The Checker Framework defines a derived class of TransferFunction to serve as the default for most checkers.  The class constrains the type of abstract values and it overrides many node visitor methods to refine the abstract values in their TransferResults.

    \code{abstract class checkers.flow.CFAbstractTransfer<
        V extends CFAbstractValue<V>,
        S extends CFAbstractStore<V, S>,
        T extends CFAbstractTransfer<V, S, T>>
    extends AbstractNodeVisitor<TransferResult<V, S>, TransferInput<V, S>>
    implements TransferFunction<V, S>}

    \code{class checkers.flow.CFTransfer
    extends CFAbstractTransfer<CFValue, CFStore, CFTransfer>}
    
    The Initialization Checker's transfer function tracks which fields of the `self' reference have been initialized.
    
    \code{class checkers.initialization.InitializationTransfer<
        V extends CFAbstractValue<V>,
        T extends InitializationTransfer<V, T, S>,
        S extends InitializationStore<V, S>>
    extends CFAbstractTransfer<V, S, T>}

    The Regex checker's transfer function overrides visitMethodInvocation to special case isRegex and asRegex methods.

    \code{class checkers.regex.RegexTransfer
    extends CFAbstractTransfer<CFValue, CFStore, RegexTransfer>}


\subsubsection{Analysis}
\label{sec:analysis_classes}

    An Analysis performs iterative dataflow analysis over a control flow graph using a given transfer function.  Currently only forward analyses are supported.

    \code{class dataflow.analysis.Analysis<
        A extends AbstractValue<A>,
        S extends Store<S>,
        T extends TransferFunction<A, S>>}

    The Checker Framework defines a derived class of Analysis for use as the default analysis of most checkers.  This class adds information about the type hierarchy being analyzed and acts as a factory for abstract values, stores, and the transfer function.

    \code{abstract class dataflow.analysis.CFAbstractAnalysis<
        V extends CFAbstractValue<V>,
        S extends CFAbstractStore<V, S>,
        T extends CFAbstractTransfer<V, S, T>>
    extends Analysis<V, S, T>}
    
    \code{class checkers.flow.CFAnalysis
    extends CFAbstractAnalysis<CFValue, CFStore, CFTransfer>}

    The Nullness Checkers' analysis overrides the factory methods for abstract values, stores, and the transfer function.

    \code{class checkers.nullness.NullnessAnalysis
    extends CFAbstractAnalysis<NullnessValue, NullnessStore, NullnessTransfer>}
    
    The RegexChecker's analysis overrides the factory methods for abstract values, stores, and the transfer function.

    \code{class checkers.regex.RegexAnalysis
    extends CFAbstractAnalysis<CFValue, CFStore, RegexTransfer>}


\subsubsection{AnalysisResult}
\label{sec:analysis_result_class}

    An AnalysisResult preserves the dataflow information computed by an Analysis for later use by clients.  The information consists of an AbstractValue for each node in the CFG and a Store that is valid at the start of each Block.  The AnalysisResult class can return AbstractValues for either Nodes or Trees and it can re-run the transfer function to compute Stores that are valid immediately before or after any Tree.

    \code{class dataflow.analysis.AnalysisResult<A extends AbstractValue<A>,
                                       S extends Store<S>>}


\subsubsection{AnnotatedTypeFactory}
\label{sec:annotated_type_factory_classes}

    AnnotatedTypeFactorys are not part of the Dataflow Framework, per se, but they are parameterized by the Dataflow Framework classes that they use.

    \code{class checkers.types.AnnotatedTypeFactory
    implements AnnotationProvider}
    
    In the Checker Framework, dataflow analysis is performed on demand, one class at a time, the first time that a ClassTree is passed to getAnnotatedType.   This is implemented in the abstract class AbstractBasicAnnotatedTypeFactory with concrete implementation in BasicAnnotatedTypeFactory.
    
    \code{abstract class checkers.types.AbstractBasicAnnotatedTypeFactory<
        Checker extends BaseTypeChecker<?>,
        Value extends CFAbstractValue<Value>,
        Store extends CFAbstractStore<Value, Store>,
        TransferFunction extends CFAbstractTransfer<Value, Store, TransferFunction>,
        FlowAnalysis extends CFAbstractAnalysis<Value, Store, TransferFunction>>
    extends AnnotatedTypeFactory}
    
    \code{class checkers.types.BasicAnnotatedTypeFactory<
        Checker extends BaseTypeChecker<?>>
    extends AbstractBasicAnnotatedTypeFactory<Checker, CFValue, CFStore, CFTransfer, CFAnalysis>}






\section{The Control-Flow Graph}
\label{sec:cfg}

%In this section, we describe the control-flow graph (CFG), and the translation from the
%Java abstract syntax tree (AST) to the CFG.

This section
describes the control-flow graph (CFG), which is used to represent a single method or field initialization, and the
translation from the abstract syntax tree (AST) to the CFG\@.
(The dataflow framework described here is designed to perform an intra-procedural analysis.
This analysis is modular and every method is considered in isolation.)
 We start with a simple
example, then give a more formal definition of the CFG and its properties, and finally
describe the translation from the AST to the CFG.

As is standard, a control-flow graph in our framework is a set of basic blocks
that are linked by control-flow edges. Possibly less standard,
every basic block consists of a sequence
of so-called nodes, which correspond to a minimal Java operations or expression.


\flow{Simple}{A simple Java code snippet to introduce the CFG.}

Consider the method \code{test} of \autoref{lst:CFGSimple} whose control-flow
graph is shown in \autoref{fig:CFGSimple}. The if conditional got translated to
a \emph{conditional basic block} (octagon) with two successors. There are also two
special basic blocks (ovals) to denote the entry and exit point of the method.


\subsection{Formal Definition of the Control-Flow Graph}
\label{sec:cfg-formal}

The control-flow graph models all paths that can possibly be taken by an
execution of the method.

\begin{definition}[Control-Flow Graph]
    A \emph{control-flow graph} consists of a set of \emph{basic blocks} and
    a set of directed edges between these basic blocks, some of which are labeled.
\end{definition}

\begin{definition}[Basic Block]
    A \emph{basic block} is a sequence of \emph{nodes}, where the only
    control flow between the nodes inside
    the basic block is sequential.  Furthermore, there is no
    control flow occurring between those nodes and nodes of other basic
    blocks, except between the last node of one block $b_1$ and the first node
    of another block $b_2$, if $b_2$ is a successor of $b_1$.  A basic
    block may have multiple successors.
\end{definition}



\begin{definition}[Types of Basic Blocks]
    There are four \emph{types} of basic blocks in a control-flow graph:
    \begin{enumerate}
        \item \textbf{Special basic blocks.} A \emph{special basic block} contains
        the empty sequence of nodes (i.e., is empty) and denotes either the
        entry or one of the exit blocks of a method. There are three types
        of special basic blocks:
        \begin{itemize}
            \item Entry block. This basic block is the (only) entry point of the
            method and thus is the only basic block without predecessors.
            \item Exit block. This basic block denotes the (normal) exit of
            a method, and it does not have successors.
            \item Exceptional exit block, which indicates exceptional termination
            of the method. As an exit block, this block does not have successors.
        \end{itemize}
        Every method has exactly one entry, zero or one exit blocks, and zero or one
        exceptional exit blocks. However, there is either an exit block or an exceptional
        exit block.
        \item \textbf{Regular basic block.} A \emph{regular basic block} contains
        any non-empty sequence of nodes and has exactly one successor.
        None of the nodes in the block can throw an exception at runtime.
        \item \textbf{Exception basic block.} An \emph{exception basic block} contains exactly
        one node that \emph{might} throw an exception at runtime (e.g., a method call).
        There are zero or one non-exceptional successors, and one
        or more exceptional successors (see \autoref{def:edges}). But in all cases there
        is at least one successor (regular or exceptional), and only a basic block
        containing a \code{throw} statement does not have a non-exceptional successor.
        \item \textbf{Conditional basic block.} A \emph{conditional basic block}
        does not contain any nodes and is used as a \emph{split point} after the execution
        of a node of boolean type. It has exactly two successors (both non-exceptional): one
        if the previous node evaluates to true, and another
        if the node evaluates to false. Both successors are labeled accordingly.
\begin{workinprogress}
Successors are basic blocks, but edges are labeled.
\end{workinprogress}


        There is always exactly a single predecessor block for every conditional basic block, which
        is either a regular basic block or an exception basic block. In both cases, the last node
        in the predecessor will be of boolean type
        and the boolean value controls which successor of the conditional block is executed.
    \end{enumerate}
\end{definition}

The Java implementation of the four block types above is described in \autoref{sec:block_classes}.

In the visualizations used in this document (e.g., in \autoref{fig:CFGSimple}), special basic blocks are
shown as ovals, conditional basic blocks are polygons with eight sides, and any
other basic block appears as a rectangle.

\begin{definition}[Control-Flow Graph Edges]
\label{def:edges}
The basic blocks of a control-flow graph are connected by directed \emph{edges}.
If $b_1$ and $b_2$ are connected by a directed edge $(b_1,b_2)$, we call
$b_1$ the predecessor of $b_2$, and we call $b_2$ the successor of $b_1$.
In a control-flow graph, there are three types of edges:
\begin{enumerate}
    \item \textbf{Exceptional edges}. An \emph{exceptional edge} connects an exception basic block
    with its exceptional successors, and it is labeled by the most
    general exception that might cause execution to take this edge during runtime.
    Note that the outgoing exceptional edges of a basic block do not need to have mutually
    exclusive labels; the semantics is that the control flow follows the most specific
    edge. For instance, if one edge is labeled with type \code{A} and another is labeled
    with type \code{B} where \code{B} is a subtype of \code{A}, then the execution only takes
    the first edge if the exception is of a subtype of \code{A}, but not a subtype of \code{B}.
    
\begin{workinprogress}
There is not necessarily a most specific exception type in the program
text; in that case, does the translation add a most specific case that will
never be executed at run time?
\end{workinprogress}

    There is at most one successor for every exception type.
    \item \textbf{Conditional edges}. A \emph{conditional edge} is a non-exceptional
    edge that connects a conditional
    basic block with one of its successors, and is labeled with either ``true'' or
    ``false''.
    \item \textbf{Regular, non-conditional edge.} Any other edge is a \emph{regular edge}, and does not carry
    a label. Only regular basic blocks, the entry basic block, and exception basic blocks have outgoing
    regular edges.
\end{enumerate}
\end{definition}


\begin{definition}[Nodes]
    \label{def:node}
    A \emph{node} is a minimal Java operation or expression.
    It is minimal in the sense that it cannot be decomposed further into subparts
    between which control flow occurs. Examples for such nodes include integer literals,
    an addition node (that performs the mathematical addition of two nodes) or a method call.
    Control flow such as \code{if} and \code{goto} are not represented as nodes.
    The full list of nodes is given in \autoref{tab:nodes} and several of them
    are described in more detail in \autoref{sec:noteworthy-translations}.
    
    It is important to note that, even though nodes can contain references to other nodes,
    it is only the ``top-level'' node which is considered at that point in the basic
    block. In our example of the addition node, this means that only the addition operation
    is to be executed, and its operands would occur earlier in the control-flow graph
    (as they are evaluated first, before performing the addition).
\end{definition}

In the visualization, a string representation of the node is used, followed by
the node type in square brackets. Note that the string representation often also includes
more than just the ``top-level'' node. For instance, an addition node also displays its
left and right operator. This is done for clarity, so that it is easy to see what
expressions are summed up.

\begin{workinprogress}
  An example would be helpful to make the ideas concrete.  Should you
  forward-reference to \code{||} in one of the examples?  (I think that's
  OK because short-circuited operators are treated the same for the
  purposes of this discussion.)
\end{workinprogress}

\autoref{tab:nodes} lists all node types in our framework. We use the
Java class name of the implementation, but leave out the suffix \code{Node}, which is present
for every type.

    \begin{longtable}{lp{0.5\linewidth}l}
        \midrule
        \multicolumn{3}{c}{\autoref{tab:nodes}: All node types in our dataflow framework.} \\ \\
        \textbf{Node type} & \textbf{Notes} & \textbf{Example} \\ \midrule \endfirsthead

        \textbf{Node type} & \textbf{Notes} & \textbf{Example} \\ \midrule \endhead
        \hline \multicolumn{3}{|c|}{{Continued on next page}} \\ \hline \endfoot
        \endlastfoot

        \code{Node} & The base class of all nodes. & \\
        \midrule

        \code{ValueLiteral} & The base class of literal value nodes. & \\
        \code{BooleanLiteral} & & \code{true} \\
        \code{CharacterLiteral} & & \code{'c'} \\
        \code{DoubleLiteral} & & \code{3.14159} \\
        \code{FloatLiteral} & & \code{1.414f} \\
        \code{IntegerLiteral} & & \code{42} \\
        \code{LongLiteral} & & \code{1024L} \\
        \code{NullLiteral} & & \code{null} \\
        \code{ShortLiteral} & & \code{512} \\
        \code{StringLiteral} & & \code{"memo"} \\
        \midrule
        
        & Accessor expressions & \\
        \code{ArrayAccess} & & \code{args[i]} \\
        \code{FieldAccess} & & \code{f}, \code{obj.f} \\
        \code{MethodAccess} & & \code{obj.hashCode} \\
        \code{ThisLiteral} & Base class of references to \code{this} & \\
        \code{ExplicitThisLiteral} & Explicit use of \code{this} in an expression & \\
        \code{ImplicitThisLiteral} & Implicit use of \code{this} in an expression & \\
        \code{Super} & Explicit use of \code{super} in expression. & \code{super(x, y)} \\
        \code{LocalVariable} & Use of a local variable, either as l-value or r-value & \\
        \midrule
        
        \code{MethodInvocation} & Note that access and invocation are distinct. & \code{hashCode()} \\
        \midrule

        & Arithmetic and logical operations. & \\
        \code{BitwiseAnd} & & a \& \code{b} \\
        \code{BitwiseComplement} & & \verb|~b| \\
        \code{BitwiseOr} & & \code{a | b} \\
        \code{BitwiseXor} & & \code{a ^ b} \\
        \code{ConditionalAnd} & Short-circuiting. & a \&\& \code{b} \\
        \code{ConditionalNot} & & \code{!a} \\
        \code{ConditionalOr} & Short-circuiting. & \code{a || b} \\
        \code{FloatingDivision} & & \code{1.0 / 2.0} \\
        \code{FloatingRemainder} & & \code{13.0 \% 4.0} \\
        \code{LeftShift} & & \code{x << 3} \\
        \code{IntegerDivision} & & \code{3 / 2} \\
        \code{IntegerRemainder} & & \code{13 \% 4} \\
        \code{NumericalAddition} & & \code{x + y} \\
        \code{NumericalMinus} & & \code{-x} \\
        \code{NumericalMultiplication} & & \code{x * y} \\
        \code{NumericalPlus} & & \code{+x} \\
        \code{NumericalSubtraction} & & \code{x - y} \\
        \code{SignedRightShift} & & \code{x >> 3} \\
        \code{StringConcatenate} & & \code{s + ".txt"} \\
        \code{TernaryExpression} & & \code{c ? t : f} \\
        \code{UnsignedRightShift} & & \code{x >>> 5} \\
        \midrule

        & Relational operations & \\
        \code{EqualTo} & & \code{x == y} \\
        \code{NotEqual} & & \code{x != y} \\
        \code{GreaterThan} & & \code{x > y} \\
        \code{GreaterThanOrEqual} & & \code{x >= y} \\
        \code{LessThan} & & \code{x < y} \\
        \code{LessThanOrEqual} & & \code{x <= y} \\

        \code{Case} & Case of a switch.  Acts as an equality test. & \\
        \midrule

        \code{Assignment} & & \code{x = 1} \\
%        \midrule
	
        \code{StringConcatenateAssignment} & A compound assignment. & \code{s += ".txt"} \\
        \midrule

        \code{ArrayCreation} & & \code{new double[]} \\
        \code{ObjectCreation} & & \code{new Object()} \\
        \midrule

        \code{TypeCast} & & \code{(float) 42} \\
        \code{InstanceOf} & & \code{x instanceof Float} \\
        \midrule
        
        & Conversion nodes. & \\
        \code{NarrowingConversion} & Implicit conversion. & \\
        \code{StringConversion} & Might be implicit. & \code{obj.toString()} \\
        \code{WideningConversion} & Implicit conversion. & \\
        \midrule

        \midrule
        & \textbf{Non-value nodes} & \\

        & Types appearing in expressions, such as \code{MyType.class} & \\
        \code{ArrayType} & & \\
        \code{ParameterizedType} & & \\
        \code{PrimitiveType} & & \\
        \midrule
        
        \code{ClassName} & Identifier referring to Java class or interface. & \code{java.util.HashMap} \\
        \code{PackageName} & Identifier referring to Java package. & \code{java.util} \\
        \midrule

        \code{Throw} & Throw an exception. & \\
        \code{Return} & Return from a method. & \\
        \midrule

        \code{AssertionError} & & \code{assert x != null : "Hey"} \\
        \midrule

        \code{Marker} & No-op nodes used to annotate a CFG & \\
        \midrule
        
        \code{NullChk} & Null checks inserted by javac & \\
        \midrule
        
        \code{VariableDeclaration} & Declaration of a local variable & \\
        \midrule
	

        \caption{All node types in our dataflow framework.}
        \label{tab:nodes}
    \end{longtable}
    
\begin{workinprogress}
ThisLiteral shouldn't be considered a literal value node, because a use of
\code{this} is not a literal expression.  I would change the name and group
it with accessor expressions.
\end{workinprogress}

\begin{workinprogress}
Can \code{StringConversion} be implicit?  I think so, but in any event discuss.
\end{workinprogress}

\begin{workinprogress}
What is \code{Marker} used for?
It's for drawing diagrams, to show where try-catch and switch nodes were in
the original program.
\end{workinprogress}

\begin{workinprogress}
I think it might be helpful to indicate which nodes can possibly throw an
exception.  This could be done by segretating them to a different part of
the table, or by having another column or an asterisk.
\end{workinprogress}

\begin{workinprogress}
For each non-expression, explain its purpose, just like the explanation for
Marker that still needs to be fleshed out.
\end{workinprogress}

\begin{workinprogress}
``Could be desugared'' on ``StringConcatenateAssignment'' and ``Conversion
nodes'' was confusing.  What is the design rationale for
desugaring in the Dataflow Framework?   Discuss that.  Here, at least
forward-reference to \autoref{sec:desugaring}, if that's relevant. \\
More generally, for any cases that will be discussed in the text, add a
forward reference to the section with the discussion.
\end{workinprogress}


\subsection{Noteworthy Translations and Node Types}
\label{sec:noteworthy-translations}

In this section we mention any non-straightforward translations from the AST to
the CFG, or special properties about individual nodes.


\subsubsection{Program Structure}
\label{sec:prog-structure}

Java programs are structured using high-level programming constructs such as
different variants of loops, if-then-else constructs, try-catch-finally blocks
or switch statements.  During the translation from the AST to the CFG some
of this program structure is lost and all non-sequential control flow is
represented by two low-level constructs: conditional basic blocks and
control-flow edges between basic blocks. For instance, a \code{while} loop is translated
into its condition followed by a conditional basic block that models the two
possible outcomes of the condition: either, the control flow follows the
`true' branch and continues with the loops body, or goes to the `false'
successor and executes the first statement after the loop.


\subsubsection{Assignment}

As described in \jlsref{15.26.1}, the execution of an assignment is in general
not strictly left-to-right. Rather, the right-hand side might be evaluated even
if the left-hand side of the assignment causes an exception. This semantics is faithfully
represented in the CFG produced by our translation.
An example of a field assignment exhibiting this behavior is shown in \autoref{lst:CFGFieldAssignment}.

\flow{FieldAssignment}{Control flow for a field assignment is not strictly
left-to-right (cf.\ \jlsref{15.26.1}),
which is properly handled by our translation.}


\subsubsection{Conditional stores}

The dataflow framework extracts information from control-flow splits that occur in if, for, while, and switch statements.  In order to have the information available at the split, we eagerly produce two stores contained in a \code{ConditionalTransferResult} after certain boolean-valued expressions.  The stores are called the \emph{then} and \emph{else} stores.  So, for example, after the expression \code{x == null}, two different stores will be created.  The Nullness Checkers would produce a then store that maps \code{x} to @Nullable and an else store that maps \code{x} to @NonNull.

The dataflow framework allows a maximum of two stores and when there are two distinct stores, they always refer to the most recent boolean-valued expression.  Stores are propagated through most nodes and they are reversed for conditional not expressions.  The transfer functions for many nodes merge conditional stores back together because they cannot maintain the distinction between them.  Merging just means taking the least upper bound of the two stores and it happens automatically by
calling \code{TransferInput.getRegularStore}.


\subsubsection{Branches}

The control flow graph represents all non-exceptional control-flow splits, or branches, as \code{ConditionalBlocks} that contain no nodes.  If there is one store flowing into a conditional block, then it is duplicated to both successors.  If there are two stores flowing into a conditional block, the then store is propagated to the block's true successor and the else store is propagated to the block's false successor.

A key point is that while we \emph{produce} multiple stores after several kinds of boolean-valued expressions, we can only \emph{use} them after a conditional block has split them apart, making them the unique store at some program point.  This constraint dictates how we build the CFG for conditional and and or nodes, as described in \autoref{sec:cond-exp}.  We split the control-flow immediately after computing the value of the left operand and we do not merge it until after the operations of the then or else successor.

\begin{workinprogress}
  Consider restructuring to make the relationship to \autoref{sec:cond-exp}
  clearer.
\end{workinprogress}

\subsubsection{Conditional Expressions}
\label{sec:cond-exp}

The conditional and (\code{&&}, cf. \jlsref{15.23}) and the conditional
or (\code{||}, cf. \jlsref{15.24}) expressions are subject to short-circuiting:
if evaluating the left-hand side already determines the result, then the right-hand
side is not evaluated. This semantics is represented in the CFG, and an example
is given in \autoref{lst:CFGConditionalOr}.

To differentiate the paths taken by the execution, given how the conditional or evaluates,
there are two different nodes for the conditional or expression in the AST
(as shown in \autoref{fig:CFGConditionalOr}). However, this is only the case when
two nodes are needed to distinguish execution paths. If the outgoing edges
of the two nodes for the same expression are merged right away, then there is
no need to have a conditional basic block. In such a case, only one
node is introduced. In the visualization, the truth value is indicated in
parentheses after the node type. An example of only a single node is given in
\autoref{lst:CFGConditionalOr2}.

\begin{workinprogress}
A more directed (that is, simple) example and expanded explanation would be helpful here.
\end{workinprogress}

\flow{ConditionalOr}{Example of a conditional or expression (\code{||}) with short-circuiting
and usage of two \code{ConditionalOrNode}s.}

\begin{workinprogress}
  ``two \code{ConditionalOrNode}s:  I see 4.  ``a single
  \code{ConditionalOrNode}'': I see 3.  It would be helpful to indicate
  which ones you are referring to.  Are you only talking about the
  top-level \code{||}?  Showing an example with just one \code{||} might
  make the example clearer.
\end{workinprogress}

\flow{ConditionalOr2}{Example of a conditional or expression (\code{||}) with short-circuiting
and usage of only a single \code{ConditionalOrNode}.}


\begin{workinprogress}
Here is Mike's take on how the translation happens.  Someone should
double-check it and improve the presentation.
\begin{itemize}
\item
Create a boolean expression.
\item
Create its labels (but they aren't attached anywhere yet).
\item
Later, attach each label to some code.
Maybe both labels get attached to the same code, in which case the last
pass will simplify the CFG.
\end{itemize}

A ``conditional node'' is used when generating the CFG for the predicate of
a conditional (\code{if}, etc.), but not when generating the CFG for an
arbitrary boolean expression.  When generating this node, the Checker
Framework knows its then and else successors because the conditional
statement (the \code{if}, etc.) passed that information in.

\end{workinprogress}


\subsubsection{Implicit \code{this} access}

The Java compiler AST uses the same type (\code{IdentifierTree}) for local variables
and implicit field accesses (where \code{this.} is left out).
To relieve the user of our dataflow framework from manually determining
the two cases, we consistently use \code{FieldAccessNode} for field accesses,
where the receiver might be an \code{ImplicitThisNode}.
For instance, this is shown in the earlier example \autoref{lst:CFGFieldAssignment}.



\subsection{AST to CFG Translation}
\label{sec:ast_to_cfg_translation}

This section gives a high-level overview of the translation process from the
abstract syntax tree to the control-flow graph as described in \autoref{sec:cfg-formal}.

First, we define several entities, which will be used in the translation.

\begin{workinprogress}
What is the intuition behind and extended node?  What does in mean?  How is
it different than a regular node?  How is it used?  This definition is
precise, but it doesn't help the reader to place the concept in context and
therefore it is confusing.
\end{workinprogress}

\begin{workinprogress}
Nodes themselves never contain control flow information; they only
represent computation.

An extended node is a wrapper around a node that represents control flow
information.  It contains:  a node, a label, a predecessor, and a
successor.

An extended node is temporarily used to keep track of some control flow
information.  Later, the basic block data structures are created, and they
represent the control flow.  (And at that point the extended nodes can be
discarded.)
\end{workinprogress}

\begin{definition}[Extended Node]
    In the translation process the data type \emph{extended node} is used.
    An extended node can be one of four possibilities:
    \begin{itemize}
        \item \textbf{Simple extended node.} An extended node can just be a
          wrapper for a node that does not throw an exception,
        as defined in Definition~\ref{def:node}.
        \begin{workinprogress}
          Say how non-exception-throwing nodes are distinguished in
          Table~\autoref{tab:nodes}.  Exception-throwing ones need to
          include throw, call, field access, typecast, division, ...
        \end{workinprogress}
        \item \textbf{Exception extended node.} Similar to a simple node, an exception extended
        node contains a node, but this node might throw an exception at runtime.
        \item \textbf{Unconditional jump.} An unconditional jump indicates that control
        flow proceeds non-sequentially to a location indicated by a target label.
        \item \textbf{Conditional jump.} A conditional jump can follow an extended node
        that contains a node of boolean type. It contains two target labels, one if the
        node evaluates to true and one for false.
    \end{itemize}
\end{definition}
\begin{definition}[Label]
    A \emph{label} is a marker that is used to refer to extended nodes.
\end{definition}

\begin{workinprogress}
Is a label used temporarily like an extended node, long-term like a basic
block, or both?
\end{workinprogress}

\begin{workinprogress}
I find it useful to never name phases ``one'', ``two'', ``three'', but
always to give them Inglish names.  This makes the meaning clearer to
readers --- often because it forces the writer to think harder about the
meaning of each.  It can still be useful to number the phases even after
they have more mnemonic names.
\end{workinprogress}


The process of translating an AST to a CFG proceeds in three distinct phases.
\begin{enumerate}
    \item \textbf{Phase one.} In the first phase, a single linear sequence of extended nodes is created. The control flow is implicitly
    assumed to be sequential through the sequence of extended nodes, until a (conditional
    or unconditional) jump is encountered in an if, for, while or switch statement, in which
    case the jump target decides where execution proceeds.
    
    The labels used as targets of jumps are associated with positions in this sequence and are managed
    by maintaining a binding function from labels to sequence positions. The advantage of
    having this indirection is that one can create a label and associate with the next free
    position in the sequence, without knowing which exact extended node will be placed there.
    Furthermore, labels can be created and used before they are actually bound to their correct
    position in the sequence (e.g., when that position is not yet known).
    At the end, the binding function can be used to resolve labels to
    extended nodes.
    
    Furthermore, phase one also computes a mapping from AST tree elements to nodes, as well
    as a set of leaders. A \emph{leader} is an extended node for which one of the following
    conditions applies:
    \begin{itemize}
    \item It is the first extended node in the sequence.
    \item It is the target of a jump (i.e. there is a label bound to the location of the node
    in the sequence).
    \item It is the first node following a jump.
    \end{itemize}
    
    \item \textbf{Phase two.} Phase two translates the linear representation to a control-flow graph
    by performing the following transformations:
    \begin{itemize}
        \item Simple extended nodes are translated to regular basic blocks, where multiple nodes
        can be grouped in one regular basic block.
        \item Exception extended nodes are translated to exception basic blocks with the correct edges.
        \item Unconditional jumps are replaced with edges between the correct basic blocks.
        \item Conditional jumps are replaced by a conditional basic block.
    \end{itemize}
    To greatly simplify the implementation, phase two is allowed to produce a degenerated control-flow
    graph. In particular, the following deficiencies are possible:
    \begin{itemize}
    \item Regular basic blocks might be empty.
    \item Some conditional basic blocks might be unnecessary, in that they have the same target for both
    the `then' as well as the `else' branch.
    \item Two consecutive, non-empty, regular basic blocks can exist, even if the second block has only
    exactly one predecessor and the two blocks could thus be merged.
    \end{itemize}
    \item \textbf{Phase three.} In the third and last phase, the control-flow graph is transformed such that
    the deficiencies remaining from phase two are removed. It is ensured that removing one kind of deficiency
    does not create another degenerate case.
\end{enumerate}



\subsubsection{Desugaring}
\label{sec:desugaring}

Desugaring means replacing complicated source language constructs by simpler ones, or removing syntactic sugar from an input program.  Originally, we intended for the control flow graph representation to be as close as possible to the Java abstract syntax tree to simplify the mapping from tree to CFG node and back and to reuse existing checker code written in terms of trees.  However, we ran into several cases that were better handled by desugaring.

    \begin{itemize}
    \item We decided to represent implicit conversion operations like boxing, unboxing, widening,
    and narrowing as their own CFG nodes (not as method calls) because they change the type of a value.  The pre-conversion
    type can be associated with the original node and the post-conversion type can be associated
    with the explicit conversion node.  It also makes it possible for the transfer function to operate
    on the conversion nodes.
    
\begin{workinprogress}
Is the ``not as method calls'' bit above still correct?
\end{workinprogress}


    \item Enhanced for loops are defined in terms of a complicated translation into simpler
    operations, including field accesses, branches, and method calls that could affect dataflow
    information.  It would be prohibitively difficult for a checker writer to write a transfer function that
    correctly accounted for all of those operations, so we desugar enhanced for loops.
    
    \item Once we decided to make conversion nodes explicit it made sense to desugar
    compound assignments.  A compound assignment like \code{Integer i; i += 3;} performs
    both an unboxing and a boxing operation on \code{i}.  Desugaring all compound assignments
    greatly reduced the total number of node classes.

    \end{itemize}

In order to desugar code and still maintain the invariant that every CFG node maps to a tree, 
we needed to create new AST tree nodes that were not present in the input program.  Javac
allows us to do this through non-supported APIs and we wrote some utility classes in
\code{javacutils} to make the process easier.  The new trees are created during CFG building
and they persist as long as some CFG node refers to them.  However, the trees are not
inserted into the AST, so they are not type-checked or seen by other tree visitors.  Their
main purpose is to carry Java types and to satisfy AnnotatedTypeFactory methods.

A further complication is that these newly-introduced AST trees are not part of the TreePath
when visiting the AST.  We work around this problem by giving the AnnotatedTypeFactory a
mapping, called the \code{pathHack}, from newly-introduced trees to their containing
MethodTree and ClassTree.

Possibly even worse, we needed to create fake symbols for variables created when
desugaring enhanced for loops.  Javac does not expose the ability to create a symbol,
so we created a new subclass of Symbol.VarSymbol called \code{javacutils.tree.DetachedVarSymbol} for this purpose.  AnnotatedTypeFactory
explicitly checks for DetachedVarSymbols in its DeclarationFromElement method.



\subsubsection{Conversions and node-tree mapping}
\label{sec:conversions}

As mentioned in \autoref{sec:desugaring}, we represent implicit Java type conversions such as boxing, unboxing, widening, and narrowing by explicit CFG nodes.  This means that some 
AST tree nodes correspond to multiple CFG nodes: a pre-conversion node and a post-conversion
node.  We will describe how the conversions work and how the node-tree mappings are
implemented.

Boxing and unboxing are represented in terms of calls to Java standard library methods.  Boxing
corresponds to a call to \code{BoxedClass.valueOf} while unboxing corresponds to a call to
\code{BoxedClass.\*Value}.  This allows annotations on the library methods, as well as transfer functions for method invocations, to apply to the conversions with no special work on the part of
a checker developer.

Widening and narrowing conversions are still represented as special node types, although
it would be more consistent to change them into type casts.

We maintain the invariant that a CFG node maps to zero or one AST tree and almost all of them map to a single tree.  But we can't maintain a unique inverse mapping because some trees have both pre- and post-conversion nodes.  Instead, we remember two mappings, one from tree to pre-conversion node and, for those trees that were converted, one from tree to post-conversion node.  Both the CFGBuilder and the ControlFlowGraph store two separate mappings.  The Analysis class explicitly stores the tree to pre-conversion node mapping as \code{treeLookup} and it indirectly uses the tree to post-conversion mapping in \code{Analysis.getValue(Tree)}.  This has effectively hidden the distinction between pre and post-conversion nodes from the Checker Framework, but in the long run
it may be necessary to expose it.



\section{Dataflow Analysis}

This section describes how the dataflow analysis over the control-flow
graph is performed and what the user of our framework has to implement to
define a particular analysis.


\subsection{Overview}

Roughly, a dataflow analysis in our framework works as follows. Given the abstract syntax
tree of a method, the framework computes the corresponding control-flow graph as
described in \autoref{sec:cfg}. Then, a simple forward iterative algorithm is used to compute
a fix-point, by iteratively applying a set of transfer functions to the nodes in the
CFG\@.
(For our initial application, type-checking, we do not need to support
backwards analyses; in the future, we may wish to do so.)
 These transfer functions are specific to the particular analysis and are used
to approximate the runtime-behavior of different statements and expressions.

An analysis maintains dataflow information in two forms.  First, for every node that can take on an abstract value, there is a mapping (\code{Analysis.nodeValues}) from node to abstract value.  For example, in the Checker Framework, the mapping is from expression nodes to annotated types.  Second, for every flow expression, there is a data structure called a \emph{store} that tracks the abstract value of the flow expression at specified program points.  An analysis keeps explicit stores for the start of each basic block (\code{Analysis.stores}) and computes the store for other program points on the fly.

After an analysis has iterated to a fix-point, the computed dataflow information is maintained in an AnalysisResult, which can map either nodes or trees to abstract values.


\subsection{Concurrency}

The dataflow framework can operate under the assumption that the code is only
executed sequentially, or work in the general setting where concurrency is allowed.
The difference between these two modes is the treatment of fields:  In the
presence of concurrency, the dataflow analysis can in general not infer any type information
of field based on local information.  Another thread might have already changed the value
of that field before the next use, potentially changing its type.

The command line option to enable concurrent semantics is -AconcurrentSemantics.  The default is to use sequential semantics and the option takes effect at the point that an AbstractBasicAnnotatedTypeFactory creates a new empty store.

An exception to this are monotonic type properties, such as the \code{@MonotonicNonNull} annotation
of the non-null type system.  There is now a meta-annotation \code{@MonotonicQualifier} that
allows one to declare that a qualifier behaves monotonically, however it is not yet used to
preserve dataflow information about fields under concurrent semantics.


\subsection{Managing Intermediate Results of the Analysis}
\label{sec:node-mapping}
\label{sec:store-management}

Conceptually, the dataflow analysis computes an abstract value for every node and flow expression\footnote{Certain
dataflow analysis might choose not to produce an abstract value for every node.  For instance,
a constant propagation analysis would only be concerned with nodes of a numerical type, and
ignore other nodes.}.
The transfer function (described in more detail in \autoref{sec:transfer-fnc}) produces these abstract
values based on the abstract values computed earlier for sub-nodes.
For instance, in a constant propagation analysis, the transfer function of addition would look at
the abstract values for the left and right operand, and determine that the \code{AdditionNode}
is a constant if and only if both operands are constant.

The dataflow framework manages these abstract values that have been computed for nodes and
makes them available to the transfer function.  To this end, a mapping called the
\emph{node-value mapping} is used that maps \code{Node}s to their abstract values.
The design of the framework consciously avoids storing the abstract value directly in
nodes to remove any coupling between the control-flow graph and a particular analysis.
This allows the control-flow graph to be constructed only once, and then reused for different
dataflow analyses.

The stores tracked by an analysis implement the \code{Store} interface, which defines the
following operations:
\begin{itemize}
\item Least upper bound: Compute the least upper bound of two stores (e.g., at a merge-point
in the control-flow graph).
\item Equivalence: Compare two stores if they are (semantically) different, which is used
to determine if a fix-point is reached in the dataflow analysis. Note that reference-equality
is most likely not sufficient.
\item Copy mechanism: Clone stores to get an exact copy.
\end{itemize}
The store is analysis-dependent, but the framework provides a default store implementation
which can be reused.

Every store is associated with a particular point in the control-flow graph, and
all stores are managed by the framework. It maintains
a single store for every basic block that represents the information available at the
beginning of that block.  When dataflow information is requested for a later point in a block, the analysis applies the transfer function to compute it from the initial store.



\subsection{Answering Questions}
After the flow analysis for a particular method has been computed, there are two kinds of
information that have been computed.  Firstly, the node-value mapping stores an abstract
value for every node, and secondly, the information maintained in various stores is
available.

Two kinds of queries are possible to the dataflow analysis after the analysis is complete:
\begin{enumerate}
    \item For a given AST tree node, what is its abstract value?  Both pre- and postconversion
    values can be retrieved.
    \item For a given AST tree node, it is possible to ask an analysis-specific question
    about the state right after this AST tree node. For instance, examples of questions
    include:
    \begin{itemize}
        \item Which locks are currently held?
        \item Are all fields of a given object initialized?
    \end{itemize}
\end{enumerate}

The implementor
of the dataflow analysis can implement arbitrary queries and how to answer them
given a store. Then, the framework will allow issuing those queries for
any given AST node by looking up (or computing) the correct store and applying the query
implementation.
The store may first need to be computed, as the framework does not store all intermediate
stores but rather only those for key positions as described in \autoref{sec:store-management}.

To support both kinds of queries, our framework builds a map from AST tree nodes (of type
\code{com.sun.source.tree.Tree}) to CFG nodes.  To answer questions of the first type it is
then possible to go from the AST tree node to the CFG node and look up its abstract value in
the node-value mapping.  By default, the abstract value returned for a tree by \code{Analysis.getValue(Tree)} includes any implicit conversions because it uses the mapping
from tree node to post-conversion CFG node.  To request the pre-conversion value, one currently uses the \code{ControlFlowGraph.treelookup} map directly.

To support questions of the second kind, every node has a reference
to the basic block it is part of. Thus, for a given AST tree node, the framework
can determine the CFG node and thereby the CFG basic block, and compute the necessary store
to answer the question.



\subsection{Transfer Function}
\label{sec:transfer-fnc}

A transfer function has to provide the following:
\begin{itemize}
\item A method that returns the initial store to be used by the dataflow analysis, given
the list of arguments (as \code{LocalVariableNode}'s) and the \code{MethodTree} (not expected to be used often).
\item A transfer method for every \code{Node} type that takes a store, the node and produces
an updated store. This is achieved by implementing the \code{NodeVisitor<S, S>} interface
for the store type \code{S}.

These transfer methods also get access to the abstract value of any sub-node of the node \code n
under consideration.  This is not limited to immediate children, but the abstract value for any node contained
in \code n can be queried.
\end{itemize}


\section{Example: Constant Propagation}

    As a proof-of-concept, I (Stefan) implemented a constant propagation analysis for local variables
    and integer values.  The main class is 
    \code{dataflow.cfg.playground.ConstantPropagationPlayground}. 
    I describe the most important aspects here.
    
    \textbf{Abstract values.} A class \code{Constant} is used as an abstract value, which can
    either be \emph{top} (more than one integer value seen), \emph{bottom} (no value seen yet),
    or \emph{constant} (exactly one value seen; in which case the value is also stored).
    
    \textbf{The store.} The store for the constant propagation analysis maps \code{Node}'s to \code{Constant},
    where only \code{LocalVariableNode}'s and \code{IntegerLiteralNode}'s are actually stored. Only those
    two nodes actually are of interest (there is no addition/multiplication/etc. yet, and other constructs like
    fields are not supported by the analysis).
    
    Two different instances of \code{LocalVariableNode} can refer to the same local variable, and thus
    the \code{equals} method has been implemented accordingly. Therefore, every local variable occurs
    at most once in the store, even if multiple \code{LocalVariableNode} for it exist.
    
    \textbf{The transfer function.} The transfer function is very simple. The initial store contains
    \emph{top} for all parameters, as any value could have been passed in. Then, for
    an integer literal, the store is extended to indicate what abstract value this literal
    stands for. Furthermore, for an assignment, I check if the left-hand side is
    a local variable, and if so update its abstract value in the
    store with the abstract value of the right-hand side (which can be looked up in the store).
    
    To illustrate how we can have different information in the then and else block of a conditional,
    I also implemented another transfer function that considers the \code{EqualToNode}, and if
    it is of the form \code{a == e} for a local variable \code{a} and constant \code{e}, passes
    the correct information to one of the branches. This is also shown in the example of \autoref{fig:ConstSimple}.
    
    \text{Example.} A small example is shown in \autoref{fig:ConstSimple}.

\constantpropagation{Simple}{Simple sequential program to illustrate constant propagation.}



\section{Default Analysis}

\subsection{Overview}

The default flow-sensitive analysis \code{checkers.flow.CFAnalysis} works for the qualifier hierarchy of any checker defined in 
the Checker Framework.  This generality is both a strength and a weakness because the 
default analysis can
always run but the facts it can deduce are limited.  The default analysis
will be extensible so checkers can add logic specific to their own qualifiers.

The default flow-sensitive analysis takes advantage of several forms of control-flow to improve the precision of type qualifiers.  It tracks assignments to flow expressions, propagating type qualifiers from the right-hand side of the assignment.  It considers equality and inequality tests to propagate the most precise qualifiers from the left or right-hand side to the true (resp. false) successor.  It also applies type qualifiers from method postconditions after calls.



\subsection{Interaction of the Checker Framework and the Dataflow Analysis}
\label{sec:flow-cf-interaction}

The \code{AnnotatedTypeFactory} of a checker already implements how different annotations
need to be combined to determine the type of a larger expression.  For instance,
the regex checker can determine that the string concatenation of two regular expression
string (\code{@Regex}) are again a regular expression.

The dataflow analysis will have to perform very similar operations for the transfer function
for intermediary nodes.  To reuse the code in the \code{AnnotatedTypeFactory}, we
use the following default implementation for all transfer methods, and let
the dataflow framework interact with the factory in the following way.
For a given node \code{n}, perform the following steps:
\begin{enumerate}
    \item Check whether the node has a corresponding AST tree node. If not, no information is
    available and ``top'' has to be used as abstract value.
    \item Otherwise, use this AST tree node and ask the \code{AnnotatedTypeFactory} about
    the type of the tree node.  The factory will then use its checker-dependent logic
    to compute this type.  Typical implementations will look at the type
    of sub-trees and compute the overall type based on the information about these sub-trees.
    
    Note that the factory recursively uses information provided by the flow analysis to determine
    the types of sub-trees.  There is a check in \code{Analysis.getValue} that the node whose type
    is being requested is a sub-node of the node to which the transfer function is currently being
    applied.  For other nodes, the analysis will return a null abstract value and the factory will
    return the flow-insensitive annotated type.
\end{enumerate}



\subsection{The Checker-Framework Store and Dealing with Aliasing}

The dataflow framework provides a default implementation of a store with the
class \code{CFAbstractStore}, which is used (as \code{CFStore}) as the
default store if a checker does not provide its own implementation.
This implementation of a store tracks the following information:
\begin{itemize}
	\item Abstract values of local variables.
	\item Abstract values of fields where the receiver is an access sequence
	compose of the following:
	\begin{itemize}
	\item Field access.
	\item Local variable.
	\item Self reference (i.e., \code{this}).
	\item Pure method call.
	\end{itemize}
\end{itemize}

The most challenging part is ensuring that the information about
field accesses is kept up to date in the face of incomplete aliasing
information.  In particular, at method calls and assignments care needs to be
taken about which information is still valid afterwards.

\subsubsection{Internal Representation}

To keep track of the abstract values of fields, we use our own representation
(that is different from the \code{Node} type introduced earlier).  This
data type is defined inductively as follows:

\begin{bnfgrammar}
    \production{\nonterminal{FieldAccess}}
        {\nonterminal{Receiver} \nonterminal{Field}}
    \production{\nonterminal{Receiver}}
        {\nonterminal{SelfReference}}
    \altline{\nonterminal{LocalVariable}}
    \altline{\nonterminal{FieldAccess}}
    \altline{\nonterminal{PureMethodCall}}
    \altline{\nonterminal{Unknown}}
    \production{\nonterminal{PureMethodCall}}
        {\nonterminal{Receiver} \literal{.} \nonterminal{Method}
            \literal{(} \nonterminal{Receiver}$^{,*}$ \literal{)}}
\end{bnfgrammar}

where \nonterminal{Field} is a Java field (identified by its \code{Element}),
\nonterminal{SelfReference} is the self reference in Java (i.e., \code{this})
and \nonterminal{LocalVariable} is used to refer to local variables (again
identified by their \code{Element}).
\nonterminal{PureMethodCall} is a Java method call of a \textit{pure} method
(as defined in the Checker Framework), with a receiver and zero or more
arguments.  Finally, \nonterminal{Unknown} is used to represent any other
Java expressions that might appear.  It is only used to determine which
information needs to be removed (e.g., after an assignment), but no field
access that contains \nonterminal{Unknown} is stored in the mapping to
abstract values.  For instance, \nonterminal{Unknown} could stand for
a non-pure method call, an array access or a ternary expression.

The store maintains a mapping from field accesses (as defined above) to
abstract values, and in the following we describe the operations that
keep this mapping up-to-date.

\subsubsection{Updating Information in the Store}

\newcommand{\alias}{\operatorname{might\_alias}}

In the following, let $o$ be any \nonterminal{Receiver},
$x$ a local variable, $f$ a
field and $m$ a pure method.  Furthermore, we assume to have access to
a predicate $\alias(o_1,o_2)$ that returns true if and only
if~$o_1$ might alias~$o_2$.

\subsubsubsection{At Field Assignments}

For a field update of the form $o_1.f_1 = e$ for some expression $e$, the
dataflow analysis first determines the abstract value $e_\text{val}$ for $e$.
Then, the following steps need to be taken in order to correctly update
the information in the store.
\begin{enumerate}
    \item For every field access $o_2.f_2$, remove its information if
    the two field $f_1$ and $f_2$ are the same, and either of the following
    conditions holds:
    \begin{itemize}
        \item $o_1$ and $o_2$ are (syntactically) the same.  In this case, the
        new abstract value will be added in the second step.
        \item $o_2$ contains a \nonterminal{Receiver} that \emph{might}
        alias $o_1.f$ as determined by the $\alias$
        predicate (this includes the case where $o_2$ and $o_1.f$ might be
        aliases).  Note that the ``contains'' notion for pure method calls
        includes both the receiver as well as the arguments.
    \end{itemize}
    \item Add the fact that $o_1.f$ now has the abstract value $e_\text{val}$
    to the mapping.
\end{enumerate}


\subsubsubsection{At Local Variable Assignments}

For a local variable assignment of the form $x = e$ for some expression $e$,
the dataflow analysis first determines the abstract value $e_\text{val}$ for
$e$. Then, the following steps need to be taken in order to correctly update
the information in the store.
\begin{enumerate}
    \item For every field access $o_2.f_2$, remove its information if
    $o_2$ contains a \nonterminal{Receiver} that \emph{might}
    alias $x$ as determined by the $\alias$
    predicate.
    \item Add the fact that $x$ now has the abstract value $e_\text{val}$
    to the mapping for local variables.
\end{enumerate}

\subsubsubsection{At Other Assignments}

For any other assignment of the for $z = e$ for some expression $e$ and some
assignment target $z$ (which is not a field or local variable),
the dataflow analysis first determines the abstract value $e_\text{val}$ for
$e$. Then, the following steps need to be taken in order to correctly update
the information in the store.
\begin{enumerate}
    \item For every field access $o_2.f_2$, remove its information if
    $o_2$ contains a \nonterminal{Receiver} that \emph{might}
    alias $z$ as determined by the $\alias$ predicate.
\end{enumerate}


\subsubsubsection{At Non-Pure Method Calls}

A non-pure method call might modify the value of any field arbitrarily.
Therefore, at a method call, any information about fields is lost.


\subsubsubsection{Alias Information}

The Checker Framework does not include an aliasing analysis, which could
provide precise aliasing information.  For this reason, we implement the
predicate $\alias$ as follows:
\[ \alias(o_1,o_2) :=
\left(\operatorname{type}(o_1) <: \operatorname{type}(o_2)
\;\;\text{or}\;\;
\operatorname{type}(o_2) <: \operatorname{type}(o_1) \right) \]
where $\operatorname{type}(o)$ determines the Java type of a reference $o$
and $<:$ denotes standard Java subtyping.


%%% Local Variables: 
%%% mode: latex
%%% TeX-master: "dataflow"
%%% TeX-command-default: "PDF"
%%% End: 
