\htmlhr
\chapter{Handling warnings and legacy code\label{warnings-and-legacy}}

Section~\ref{get-started-with-legacy-code} describes a methodology for
applying annotations to legacy code.  This chapter tells you what to do if,
for some reason, you cannot change your code in such a way as to eliminate
a checker warning.


\section{Checking partially-annotated programs:  handling unannotated code\label{unannotated-code}}

Sometimes, you wish to type-check only part of your program.  
You might focus on the most mission-critical or error-prone part of your
code.  When you start to use a checker, you may not wish to annotate
your entire program right away.
% Not having source code is *not* a reason.
You may not have 
enough knowledge to annotate poorly-documented libraries that your program uses.

If annotated code uses unannotated code, then the checker may issue
warnings.  For example, the Nullness checker (Chapter~\ref{nullness-checker}) will
warn whenever an unannotated method result is used in a non-null context:

\begin{Verbatim}
  @NonNull myvar = unannotated_method();   // WARNING: unannotated_method may return null
\end{Verbatim}

If the call \emph{can} return null, you should fix the bug in your program by
removing the \code{@\refclass{nullness/quals}{NonNull}} annotation in your own program.

If the library call \emph{never} returns null,
there are several ways to eliminate the compiler warnings.
\begin{enumerate}
\item Annotate \code{unannotated\_method} in full.  This approach provides
  the strongest guarantees, but may require you to annotate additional
  methods that \code{unannotated\_method} calls.  See
  Chapter~\ref{annotating-libraries} for a discussion of how to annotate
  libraries for which you have no source code.
\item Annotate only the signature of \code{unannotated\_method}, and
  suppress warnings in its body.  Two ways to suppress the warnings are via a
  \code{@SuppressWarnings} annotation or by not running the checker on that
  file (see Section~\ref{suppressing-warnings}).
\item Suppress all warnings related to uses of \code{unannotated\_method}
  via the \code{skipClasses} processor option
  (see Section~\ref{suppressing-warnings}).
  Since this can suppress more warnings than you may expect,
  it is usually better to annotate at least the method's signature.  If you
  choose the boundary between the annotated and unannotated code wisely,
  then you only have to annotate the signatures of a limited number of
  classes/methods
  (e.g., the public interface to a library or package).
  
\end{enumerate}

Chapter~\ref{annotating-libraries} discusses adding annotations to
signatures when you do not have source code available.
Section~\ref{suppressing-warnings} discusses suppressing warnings.


If you annotate a third-party library, please share it with us so that we
can distribute the annotations with the Checker Framework; see
Section~\ref{reporting-bugs}.


\section{Suppressing warnings\label{suppressing-warnings}}

You may wish to suppress checker warnings because of unannotated libraries
or un-annotated portions of your own code, because of application
invariants that are beyond the capabilities of the type system, because of
checker limitations, because you are interested in only some of the
guarantees provided by a checker, or for other reasons.  You can suppress
warnings via
\begin{itemize}
\item
  the \code{@SuppressWarnings} annotation,
\item
  the \code{-AskipClasses} command-line option,
\item
  the javac \code{-Alint} command-line option,
\item
  not using the \code{-processor} switch to \code{javac}, or
\item
  checker-specific mechanisms.
\end{itemize}

\noindent
We now explain these mechanisms in turn.

You can suppress specific errors and warnings by use of the
\code{@SuppressWarnings("\emph{checkername}")} annotation, for example
\code{@SuppressWarnings("interning")} or \code{@SuppressWarnings("nullness")}.
This may be placed on program elements such as a local
variable declaration, a method, or a class.  For instance, one common use is
to suppress warnings at a cast that you know is safe.  Here is an example
that uses the Tainting Checker (Section~\ref{tainting-checker}):

\begin{Verbatim}
  @SuppressWarnings("tainting")
  String myvar = (@Untainted String) expr;  // expr has type: @Tainted String
\end{Verbatim}

It is good practice to suppress warnings in the
smallest possible scope.  For example, if a particular expression causes a
false positive warning, you should extract that expression into a local variable
and place a \code{@SuppressWarnings} annotation on the variable
declaration.
As another example, if you have annotated the signatures but not the bodies
of the methods in a class or package, put a \code{@SuppressWarnings}
annotation on the class declaration or on the package's
\code{package-info.java} file.

You can suppress all errors and warnings at all uses of a given class.
Set the \code{-AskipClasses} command-line option to a
regular expression that matches classes for which warnings and errors
should be suppressed.  For example, if you use
``{\codesize\verb|-AskipClasses=^java\.|}'' on the command line
(with appropriate quoting) when invoking
\code{javac}, then the checkers will suppress all warnings within
classes whose fully-qualified name starts with \codesize\verb|java.|, all
warnings relating to invalid arguments, and all warnings relating to incorrect
use of the return value.


\label{lint-options}

The \code{-Alint} option enables or disables optional checks, analogously to 
javac's \code{-Xlint} option.
Each of the distributed checkers supports at least the following lint options:

\begin{itemize}

\item
  \code{cast:unsafe} (default: on) warn about unsafe casts that are not
  checked at run time, as in \code{((@NonNull String) myref)}.  Such casts
  are generally not necessary when flow-sensitive local type refinement is
  enabled.

\item
  \code{cast:redundant} (default: on) warn about redundant
  casts that are guaranteed to succeed at run time,
  as in \code{((@NonNull String) "m")}.  Such casts are not necessary,
  because the target expression of the cast already has the given type
  qualifier.

\item
  \code{cast} Enable or disable all cast-related warnings.

\item
  \code{all} Enable or disable all lint warnings, including
  checker-specific ones if any.  Examples include \code{nulltest} for the
  Nullness Checker (see Section~\ref{lint-nulltest}) and \<dotequals> for
  the Interning Checker (see Section~\ref{lint-dotequals}).  This option
  does not enable/disable the checker's standard checks, just its optional
  ones.

\item
  \code{none} The inverse of \<all>:  disable or enable all lint warnings,
  including checker-specifc ones if any.

\end{itemize}

% This syntax is different from -Xlint that uses a colon instead of an
% equals sign, because javac forces the use of the equals sign.

\noindent
To activate a lint option, write \code{-Alint=} followed by a
comma-delimited list of check names.  If the option is preceded by a
hyphen (\code{-}), the warning is disabled.  For example, to disable all
lint options except redundant casts, you can pass
\code{-Alint=-all,cast:redundant} on the command line.


You can also compile parts of your code without use of the
\code{-processor} switch to \code{javac}.  No checking is done during
such compilations.

Finally, some checkers have special rules.  For example, the Nullness
checker (Chapter~\ref{nullness-checker}) uses \code{assert} statements that contain
null checks, and the special \<castNonNull> method, to suppress warnings
(Section~\ref{suppressing-warnings-with-assertions}).



\section{Writing annotations in comments for backward compatibility\label{annotations-in-comments}}

Sometimes, your code needs to be compilable by people who are not
using a Java 7 compiler.


\subsection{Annotations in comments}

A Java 4 compiler does not permit use of
annotations, and a Java 5 compiler only permits annotations on
declarations (but not on generic arguments, casts, \<extends> clauses, method receiver, etc.).

So that your code can be compiled by any Java compiler (for any version of
the Java language), you may write any annotation inside a
\code{/*}\ldots\code{*/} Java comment, as in \code{List</*@NonNull*/ String>}.
The Type Annotations compiler treats the code exactly as if you had not written the
\code{/*} and \code{*/}.
In other words, the Type Annotations compiler will recognize the
annotation, but your code will still compile with any other Java compiler.

(\textbf{Note:} This is a feature of the Type Annotations compiler that is
distributed along with the Checker Framework.  It is \emph{not} supported by the
mainline OpenJDK compiler, which ignores annotations written in
comments.  This is the key difference between the Type Annotations
compiler and the OpenJDK compiler.)
%   For more details
% about the differences, see file \code{README-jsr308.html} in the Type
% Annotations distribution.

In a single program, you may write some annotations in comments, and others
without comments.

By default, the compiler ignores any comment that contains spaces at the
beginning or end, or between the \code{@} and the annotation name.  
In other words, it reads \code{/*@NonNull*/} as an annotation but ignores
\code{/* @NonNull*/} or \code{/*@ NonNull*/} or \code{/*@NonNull */}.
This
feature enables backward compatibility with code that contains comments
that start with \code{@} but are not annotations.  (The
ESC/Java~\cite{FlanaganLLNSS02}, JML~\cite{LeavensBR2006:JML}, and
Splint~\cite{Evans96} tools all use ``\code{/*@}'' or ``\code{/*~@}'' as a
comment marker.)
Compiler flag
\code{-XDTA:spacesincomments} causes the compiler to parse annotation comments
even when they contain spaces.  You may need to use
\code{-XDTA:spacesincomments} if you use Eclipse's ``Source $>$ Correct
Indentation'' command, since it inserts space in comments.  But the
annotation comments are less readable with spaces, so you may wish to disable
inserting spaces:  in the Formatter preferences, in the Comments tab,
unselect the ``enable block comment formatting'' checkbox.


\subsection{Implicit import statements\label{implicit-import-statements}}

When writing source code with annotations, it is more convenient to write a
short form such as \code{@NonNull} instead of
\code{@checkers.nullness.quals.NonNull}.

The traditional way to do this is to write an import statement like
``\code{import checkers.nullness.quals.*;}''.  This works, but everyone who
compiles the code (no matter what compiler they use, and even if the
annotations are in comments) must have the annotation definitions (e.g.,
the \code{checkers.jar} or \code{checkers-quals.jar} file) on their
classpath.  The reason is that a Java compiler issues an error if an
imported package is not on the classpath.  See Section~\ref{distributing}.

\label{jsr308_imports}

An alternative is to set the shell environment variable
\code{jsr308\_imports} when you compile the code.
The Type Annotations compiler treats this as if the given packages were
imported, but other compilers
ignore the
\code{jsr308\_imports} environment variable --- they do not need it, since
they do not support annotations in comments.  Thus, your code can compile
whether or not the Type Annotations compiler is being used.

You can specify multiple packages separated by the classpath separator
(same as the file path separator:  \<;> for Windows, and \<:> for Unix and
Mac.).  For example, to implicitly import the Nullness and Interning
qualifiers, set \code{jsr308\_imports} to
\code{checkers.nullness.quals.*:checkers.interning.quals.*}.

In bash, you could write \code{export
  jsr308\_imports='checkers.nullness.quals.*'}, or prefix the \code{javac}
command by \code{jsr308\_imports='checkers.nullness.quals.*'} .
Alternately, you can set the environment variable via the javac
command-line argument \code{-J-Djsr308\_imports="checkers.nullness.quals.*"}.


\subsection{Migrating away from annotations in comments}

Suppose that your codebase currently uses annotations in comments, but you
wish to remove the comment characters around your annotations, because in
the future you will use only Java 7 compilers.  This Unix command removes
the comment characters, for all Java files in the current
working directory or any subdirectory.

\begin{Verbatim}
   find . -type f -name '*.java' -print \
     | xargs grep -l -P '/\*\s*@([^ */]+)\s*\*/' \
     | xargs perl -pi.bak -e 's|/\*\s*@([^ */]+)\s*\*/|@\1|g'
\end{Verbatim}

You can customize this command:
\begin{itemize}
\item
To process comments with embedded spaces and asterisks, change 
two instances of ``\verb|[^ */]|'' to ``\verb|[^/]|''.
\item
To ignore comments with leading or trailing spaces, remove the four
instances of ``\verb|\s*|''.  
\item
  To not make backups, remove
``\verb|.bak|''.
\end{itemize}


If you are using implicit import statements
(Section~\ref{implicit-import-statements}), you may also need to introduce
explicit import statements into your code.



% LocalWords:  quals skipClasses un AskipClasses Alint annotationname javac's
% LocalWords:  Xlint dotequals castNonNull XDTA spacesincomments Formatter jsr
% LocalWords:  unselect checkbox classpath Djsr bak
