\htmlhr
\chapter{Internationalization checker\label{i18n-checker}}

The Internationalization Checker verifies that your code is properly
internationalized.  Internationalization is the process of adapting
software to different languages and locales.  Internationalization is
sometimes called localization (though the terms are not identical), and is
sometimes called i18n (because the word starts with ``i'', ends with ``n'',
and has 18 characters in between).

The checker focuses on one aspect of localization:  user-visible strings
should be presented in the user's own language, such as English, French, or
German.  This is achieved by looking up keys in a localization resource,
which maps keys to user-visible strings.  For instance, one version of a
resource might map \code{"CANCEL\_STRING"} to
\code{"Cancel"}, and another version of the same resource might map
\code{"CANCEL\_STRING"} to \code{"Abbrechen"}.

There are other aspects to localization, such as formatting of dates (3/5
vs.~5/3 for March 5), that the checker does not check.

The Internationalization Checker verifies these two properties:

\begin{enumerate}

\item
  Any user-visible text should be obtained from a localization resource.
  For example, \code{String} literals should not be output to the user.

\item
  When looking up keys in a localization resource, the key should exist in
  that resource.  This check catches incorrect or misspelled localization
  keys.

\end{enumerate}


\section{Internationalization annotations\label{i18n-annotations}}

The Internationalization Checker supports two annotations:

\begin{enumerate}
\item \code{@\refclass{i18n/quals}{Localized}}: indicates that the qualified
\code{String} is a message that has been localized and/or formatted with
respect to the used locale.

\item \code{@\refclass{i18n/quals}{LocalizableKey}}: indicates that the
qualified \code{String} or \code{Object} is a valid key found in the
localization resource.
\end{enumerate}

You may need to add the \code{@Localized} annotation to more methods in the
JDK or other libraries, or in your own code.


\section{Running the Internationalization Checker\label{i18n-running}}

The Internationalization Checker can be invoked by running the following
command:

\begin{Verbatim}
  javac -processor checkers.i18n.I18nChecker -Abundlename=MyResource MyFile.java ...
\end{Verbatim}

You must specify the localization resource, which maps keys to user-visible
strings.  The checker supports two types of localization resource:
ResourceBundle or property file.  You should specify just one of the
following two command-line options:

\begin{enumerate}

\item \code{-Abundlename=\emph{resource\_name}}

  \emph{resource\_name} is the name of the resource to be used with
  \sunjavadoc{java/util/ResourceBundle.html#getBundle(java.lang.String, java.util.Locale, java.lang.ClassLoader)}{ResourceBundle.getBundle()}.
  The checker uses the default \code{Locale} and \code{ClassLoader} in the
  compilation system.
  (For a tutorial about \code{ResourceBundle}s, see
  \myurl{http://java.sun.com/developer/technicalArticles/Intl/ResourceBundles/}.)

\item \code{-Apropfile=\emph{prop\_file}}

  \emph{prop\_file} is the name of a properties file that maps
  localization keys to localized message.  The file format is described in
  the Javadoc for 
  \sunjavadoc{java/util/Properties.html#load(java.io.Reader)}{Properties.load()}.

\end{enumerate}
