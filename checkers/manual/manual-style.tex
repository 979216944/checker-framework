\usepackage{pslatex}
\usepackage{fullpage}
\usepackage{graphicx}
\usepackage{hevea}
\usepackage{listings}
\usepackage{url}
\usepackage{alltt}
% Not supported by Hevea, so don't bother: \usepackage{minitoc}

\usepackage{relsize}
% \def\codesize{\smaller}
\def\codesize{\relax}           % for "pslatex"
%HEVEA \def\codesize{\relax}
\newcommand{\code}[1]{\ifmmode{\mbox{\codesize\ttfamily{#1}}}\else{\codesize\ttfamily #1}\fi}
% This can't handle a URL with an embedded "#" -- at least at UW CSE
\newcommand{\myurl}[1]{{\codesize\url{#1}}}
%HEVEA \def\myurl{\url}
\def\<#1>{\code{#1}}

\usepackage{fancyvrb}
%BEGIN LATEX
\RecustomVerbatimEnvironment{Verbatim}{Verbatim}{fontsize=\codesize}
%END LATEX

%HEVEA \footerfalse    % Disable hevea advertisement in footer

\newcommand{\htmlhr}{\relax}
%HEVEA \renewcommand{\htmlhr}{\@hr{}{}}

% Problem with using "\newcommand" or "\renewcommand": Hevea writes this
% into manual.image.tex, and invokes LaTeX on it.  Sometimes running "make"
% leads to an error as a result, sometimes not.  I don't know the pattern
% of the failures, though running "make clean" and then "make" seems to
% work.  So maybe there's a problem with an auxiliary file.  Solve it by
% always defining \discretionary, so that \renewcommand works.
%HEVEA \def\discretionary{\relax}\renewcommand{\discretionary}[3]{\relax}

%HEVEA \newstyle{.lstframe}{margin:auto;margin-bottom:2em}

% Make images larger and thus less pixelated.  This is not really a solution.
%HEVEA\@addimagenopt{-mag 2000}
% This avoids Hevea's built in conversion.
\newcommand{\includeimage}[2]{
\begin{center}
\ifhevea\imgsrc{#1.png}\else
\resizebox{!}{#2}{\includegraphics{figures/#1}}
\vspace{-1.5\baselineskip}
\fi
\end{center}}


% At least 80% of every float page must be taken up by
% floats; there will be no page with more than 20% white space.
\def\topfraction{.8}
\def\dbltopfraction{\topfraction}
\def\floatpagefraction{\topfraction}     % default .5
\def\dblfloatpagefraction{\topfraction}  % default .5
\def\textfraction{.2}


% Left and right curly braces and backslash, in tt font
\newcommand{\ttlcb}{\texttt{\char "7B}}
\newcommand{\ttrcb}{\texttt{\char "7D}}
\newcommand{\ttbs}{\texttt{\char "5C}}


%BEGIN LATEX
  %% Bring items closer together in list environments
  % Prevent infinite loops
  \let\Itemize =\itemize
  \let\Enumerate =\enumerate
  \let\Description =\description
  % Zero the vertical spacing parameters
  \def\Nospacing{\itemsep=0pt\topsep=0pt\partopsep=0pt\parskip=0pt\parsep=0pt}
  % Redefine the environments in terms of the original values
  \renewenvironment{itemize}{\Itemize\Nospacing}{\endlist}
  \renewenvironment{enumerate}{\Enumerate\Nospacing}{\endlist}
  \renewenvironment{description}{\Description\Nospacing}{\endlist}

  % Add line between figure and text
  \makeatletter
  \def\topfigrule{\kern3\p@ \hrule \kern -3.4\p@} % the \hrule is .4pt high
  \def\botfigrule{\kern-3\p@ \hrule \kern 2.6\p@} % the \hrule is .4pt high
  \def\dblfigrule{\kern3\p@ \hrule \kern -3.4\p@} % the \hrule is .4pt high
  \makeatother
%END LATEX


% Reference to Checker Framework Javadoc for a class (not a method, etc.).
% Arg 1: directory under checkers/, including internal "/", but no leading
% or trailing "/".
% Arg 2: class name.
% In the printed version, only the base class name appears.
% In the HTML version, it's a link to the Javadoc.
\newcommand{\refclass}[2]{\ahref{doc/checkers/#1/#2.html}{\<#2>}}
% Reference to Checker Framework Javadoc for a method or field.).
% Arg 1: directory under checkers/, including internal "/", but no leading
% or trailing "/".
% Arg 2: class name.
% Arg 3: method name.
% Arg 4: fully-qualified arguments.  Example: "(T)"
% In the printed version, only "class.method" appears.
% In the HTML version, it's a link to the Javadoc.
\newcommand{\refmethod}[4]{\ahref{doc/checkers/#1/#2.html\##3#4}{\<#2.#3>}}
% Reference to Sun Javadoc.
% Arg 1: .html reference, without the .../api/ prefix
% Arg 2: What will appear in the formatted manual.
% Problem:  the "?is-external=true" must appear before any "#".  But why is
% it necessary at all?
% \newcommand{\sunjavadoc}[2]{\ahref{http://docs.oracle.com/javase/7/docs/api/#1?is-external=true}{\<#2>}}
\newcommand{\sunjavadoc}[2]{\ahref{http://docs.oracle.com/javase/7/docs/api/#1}{\<#2>}}


\newcommand{\refwithpage}[1]{\ref{#1}, page~\pageref{#1}}
%HEVEA \renewcommand{\refwithpage}[1]{\ref{#1}}
\newcommand{\refwithpageparen}[1]{\ref{#1} (page~\pageref{#1})}
%HEVEA \renewcommand{\refwithpageparen}[1]{\ref{#1}}
\newcommand{\chapterpageref}[1]{Chapter~\refwithpage{#1}}
