\htmlhr
\chapter{Frequently Asked Questions (FAQs)\label{faq}}

These are some common questions about the Checker Framework and about
pluggable type-checking in general.  Feel free to suggest improvements to
the answers, or other questions to include here.

There is a separate FAQ for the type annotations syntax
(\url{http://types.cs.washington.edu/jsr308/jsr308-faq.html}).

% Not supported by Hevea, so don't bother; instead do by hand:
% \minitoc

%BEGIN LATEX
~
%END LATEX

\noindent
\textbf{Contents:} \\
\ref{faq-ease-of-use}: Are type annotations easy to read and write? \\
\ref{faq-code-clutter}: Will my code become cluttered with type annotations? \\
\ref{faq-no-absolute-guarantee}: Can a pluggable type-checker give an absolute guarantee of correctness? \\
\ref{never-make-type-errors}: I don't make type errors, so pluggable type checking would not help me \\
\ref{faq-qualifiers-vs-subclasses}: When should I use type qualifiers, and when should I use subclasses? \\
\ref{faq-create-a-checker}: How do I create a new checker? \\
\ref{faq-declarative-syntax-for-type-rules}: Why is there no declarative syntax for writing type rules? \\
\ref{faq-type-checking-vs-bug-detectors}: Why not just use a bug detector (like FindBugs)? \\
\ref{faq-jml}: How does pluggable type-checking compare with JML? \\
\ref{faq-list-map-nonnull-typeargs}: Why are the type parameters to \<List> and \<Map> annotated as \<@NonNull>? \\
\ref{faq-run-time-checking}: How can I do run-time monitoring of properties that were not statically checked?



% This FAQ also appears in the JSR 308 FAQ.
\section{Are type annotations easy to read and write?\label{faq-ease-of-use}}

The paper
\ahref{http://www.cs.washington.edu/homes/mernst/pubs/pluggable-checkers-issta2008-abstract.html}{``Practical
  pluggable types for Java''}~\cite{PapiACPE2008} discusses case studies in
which programmers
found type annotations to be natural to read and write.  The code
continued to feel like Java, and the type-checking errors were easy to
comprehend and often led to real bugs.

You don't have to take our word for it, though.  You can try the
Checker Framework for yourself.

The difficulty of adding and verifying annotations depends on your program.
If your program is well-designed and -documented, then skimming the
existing documentation and writing type annotations is extremely easy.
Otherwise, you may find yourself spending a lot of time trying to
understand, reverse-engineer, or fix bugs in your program, and then just a
moment writing a type annotation that describes what you discovered.  This
process inevitably improves your code.  You must decide whether it is a
good use of your time.  For code that is not causing trouble now and is
unlikely to do so in the future (the code is bug-free, and you do not
anticipate changing it or using it in new contexts), then the
effort of writing type annotations for it may not be justified.


% This FAQ also appears in the JSR 308 FAQ.
\section{Will my code become cluttered with type annotations?\label{faq-code-clutter}}

As with any language feature, it is possible to write ugly code that
over-uses annotations.  However, in normal use, very few annotations need
to be written.  Figure 1 of the paper
\ahref{http://www.cs.washington.edu/homes/mernst/pubs/pluggable-checkers-issta2008-abstract.html}{Practical
  pluggable types for Java}~\cite{PapiACPE2008} reports data for over
350,000 lines of type-annotated code:

\begin{itemize}
\item
    1 annotation per 62 lines for nullness annotations (\<@NonNull>, \<@Nullable>, etc.)
    % (/ (+ 4640 3961 10798) (+ 107 35 167))
\item
    1 annotation per 1736 lines for interning annotations (\<@Interned>)
    % (/ 224048 129)
\item
    1 annotation per 27 lines for immutability annotations (IGJ type system)
    % (/ (+ 6236 18159 30507 8691 59221 26828) (+ 315 1125 1386 384 1815 450))
\end{itemize}

Furthermore, these numbers are for annotating existing code.  New code that
is written with the type annotation system in mind is cleaner and more
correct, so it requires even fewer annotations.

In other words, annotations do not clutter code, and they are used much
less frequently than generic types, which Java programmers find acceptable.


\section{Can a pluggable type-checker give an absolute guarantee of correctness?\label{faq-no-absolute-guarantee}}

Each checker looks for certain errors.  You can use multiple checkers, but
even then your program might still contain other kinds of errors.

If you run a pluggable checker on only part of the code of a program, then
you do not get a guarantee that all parts of the program satisfy the type
system (that is, are error-free).  An example is a framework that clients
are intended to extend.  In this case, you should recommend that clients
run the pluggable checker.  There is no way to force users to do so, so you
may want to retain dynamic checks or use other mechanisms to detect errors.

There are other circumstances in which a static type-checker may fail to
detect a possible type error.  In Java, these include arrays, casts, raw
types, reflection, separate compilation (bytecodes from unverified sources),
native code, etc.  (For details, see section~\ref{checker-guarantees}.)
Java uses dynamic checks for most of these, so that the
type error cannot cause a security vulnerability or a crash.  The pluggable
type-checkers inherit many (not all) of these weaknesses of Java
type-checking, but do not currently have built-in dynamic checkers.
Writing dynamic checkers would be an interesting and valuable project.

% This paragraph is weak.

Even if a tool such as a pluggable checker cannot give an ironclad
guarantee of correctness, it is still useful.  It can finding errors, 
excluding certain types of possible problems (e.g., restricting the
possible class of problems), and increasing confidence in a piece of
software.


\section{I don't make type errors, so pluggable type checking would not help me\label{never-make-type-errors}}

Occasionally, a developer says that he makes no errors that typechecking
could catch, or that any such errors are unimportant because they have low
impact and are easy to fix.  When I investigate the claim, I invariably
find that the developer is mistaken.

Very frequently, the developer has underestimated what typechecking can
discover.  Not every type error leads to an exception being thrown; and
even if an exception is thrown, it may not seem related to classical types.
Remember that a type system can discover
null pointer dereferences,
incorrect side effects, 
security errors such as information leakage or SQL injection,
partially-initialized data,
and many other errors.  Even where type-checking does not discover a
problem directly, it can indicate code with bad smells, thus revealing
problems, improving documentation, and making future maintenance easier.

There are other ways to discover errors, including extensive testing and
debugging.  But type-checking is a good complement to these.  It is more
effective for some problems, and less effective for other problems.  It can
reduce (but not eliminate) the time and effort that you spend on other
approaches.  There are many important errors that type checking and other
automated approaches cannot find; pluggable typechecking gives you more
time to focus on those.


\section{When should I use type qualifiers, and when should I use subclasses?\label{faq-qualifiers-vs-subclasses}}

In brief, use subtypes when you can, and use type qualifiers when you cannot
use subtypes.
For more details, see section~\ref{when-to-use-type-qualifiers}.


\section{How do I create a new checker?\label{faq-create-a-checker}}

In addition to using the checkers that are distributed with the Checker
Framework, you can write your own checker to check specific properties that
you care about.  Thus, you can find and prevent the bugs that are most
important to you.

Chapter~\ref{writing-a-checker} gives
complete details regarding how to write a checker.  It also suggests places
to look for more help, such as the \ahref{doc/}{Checker Framework
API documentation (Javadoc)} and the source code of the distributed
checkers.

To whet your interest and demonstrate how easy it is to get started, here
is an example of a complete, useful type checker.

\begin{Verbatim}
  @TypeQualifier
  @SubtypeOf(Unqualified.class)
  public @interface Encrypted { }
\end{Verbatim}

Section~\ref{basic-example} explains this checker and tells
you how to run it.


\section{Why is there no declarative syntax for writing type rules?\label{faq-declarative-syntax-for-type-rules}}

A type system implementer can declaratively specify the type qualifier
hierarchy (Section~\ref{declarative-hierarchy}) and the type introduction rules
(Section~\ref{declarative-type-introduction}).  However, the Checker
Framework does not offer a declarative syntax for specifying type-checking
rules (Section~\ref{extending-visitor}).
A declarative syntax might be more  concise, more readable, and more
verifiable than a procedural syntax.

Previous attempts to devise a declarative syntax 
for realistic type systems have failed; see a technical
paper~\cite{PapiACPE2008} for a discussion.  When an
adequate syntax exists, then the Checker Framework can be extended to
support it.


\section{Why not just use a bug detector (like FindBugs)?\label{faq-type-checking-vs-bug-detectors}}

Pluggable type-checking finds more bugs than a bug detector does, for any
given variety of bug.

A bug detector like \ahref{http://findbugs.sourceforge.net/}{FindBugs}~\cite{HovemeyerP2004,HovemeyerSP2006},
\ahref{http://artho.com/jlint/}{JLint}~\cite{Artho2001}, or
\ahref{http://pmd.sourceforge.net/}{PMD}~\cite{Copeland2005} aims to find \emph{some}
of the most obvious bugs in your program.  It uses a lightweight analysis,
then uses heuristics to discard some of its warnings.  Thus, even if the tool
prints no warnings, your code might still have errors --- maybe the
analysis was too weak to find them, or the tool's heuristics classified the
warnings as likely false positives and discarded them.

A type checker aims to find \emph{all} the bugs (of certain varieties).
It requires you to write type qualifiers in your program, or to use a tool
that infers types.  Thus, it requires more work from the programmer, and in
return it gives stronger guarantees.

Each tool is useful in different circumstances, depending on how important
your code is and your desired level of confidence in your code.  For more
details on the comparison, see section~\ref{other-tools}.  For a case study
that compared the nullness analysis of FindBugs, JLint, PMD, and the
Checker Framework, see section 6 of the paper
\ahref{http://www.cs.washington.edu/homes/mernst/pubs/pluggable-checkers-issta2008.pdf}{``Practical pluggable types for Java''}~\cite{PapiACPE2008}.


\section{How does pluggable type-checking compare with JML?\label{faq-jml}}

\ahref{http://www.cs.ucf.edu/~leavens/JML/}{JML}, the Java Modeling
Language~\cite{LeavensBR2006:JML}, is a language for writing formal
specifications.  JML aims to be more expressive than pluggable
type-checking.  JML is not as practical as pluggable type-checking.

A programmer can write a JML specification that
describes arbitrary facts about program behavior.  Then, the programmer can
use formal reasoning or a theorem-proving tool to verify that the code
meets the specification.  Run-time checking is also possible.
By contrast, pluggable type-checking can express a more limited set of
properties about your program.

The JML toolset is less mature.  For instance, if your code uses
generics or other features of Java 5, then you cannot use JML.  
However, JML has a run-time checker, which the Checker Framework currently
lacks.


\section{Why are the type parameters to \<List> and \<Map> annotated as \<@NonNull>?\label{faq-list-map-nonnull-typeargs}}

If you try to write \code{List<@Nullable Object>}, you will get a type
error, because \<List>'s type parameter is annotated as \<@NonNull>.
You will also get an error if you try to use a nullable
type parameter to certain other collections, including
\<AbstractCollection>, \<Collection>, \<List>, \<Map>, and \<Queue>.

% AbstractCollection has no documentation of its own regarding nullness,
% but it implements Collection.

% The JML specifications of the add() method says
%       @   signals (NullPointerException)
%       @             (* not allowed to add null *);
%       ...
%       @   signals (NullPointerException)
%       @             (* not allowed to add null *);
% In other words, the method might throw NullPointerException, but the JML
% spec does not say under what circumstances.


This is in conformance with the Javadoc specification of those classes.
The Javadoc for \<List> states:

\begin{quote}
  Some list implementations have restrictions on the elements that they may
  contain. For example, some implementations prohibit null elements,
\end{quote}

In particular, calling this method might result in a null pointer exception:

\begin{Verbatim}
  static void addNull(List l) {
    l.add(null);
  }
\end{Verbatim}

The Checker Framework is designed to warn you whenever your code might
throw a null pointer exception.  If you want to be safe, you will never put
\<null> in a \<List> of unknown provenance, because that \<List> might not
accept null.

By contrast, this code is OK because \<ArrayList> is documented to support
null elements:

\begin{Verbatim}
  static void addNull(ArrayList l) {
    l.add(null);
  }
\end{Verbatim}

A similar argument applies to \<Map>.
For example, \<ConcurrentHashMap> and \<Hashtable> implement \<Map> but do
not permit \<null> to be used as a key or value.

Suppose your program has a list variable, and you know that any list referenced
by that variable will
definitely support null.  Then, you can suppress the warning:

\begin{Verbatim}
  @SuppressWarnings("nullness:generic.argument")
  static void addNull(List l) {
    l.add(null);
  }
\end{Verbatim}

\noindent
You need to use \<@SuppressWarnings("nullness:generic.argument")>
whenever you use a collection that may contain \<null> elements in
contradiction to its documentation.  Fortunately, such uses are relatively
rare.

For more details, see Section~\ref{suppressing-warnings-nullness}.


\section{How can I do run-time monitoring of properties that were not statically checked?\label{faq-run-time-checking}}

Currently, the Checker Framework has no support for adding code to check,
at run time, code that was not checked (see
Chapter~\ref{warnings-and-legacy} for reasons that code might not be
checked).  An exception is the Nullness Checker, which has ways to
dynamically check nullness via assertions and casts (the
\refmethod{nullness}{NullnessUtils}{castNonNull}{(T)} method); see
Section~\ref{suppressing-warnings-with-assertions}.

More general support would be an interesting and valuable project.  If you
are able to add run-time verification functionality, we would gladly
welcome it as a contribution to the Checker Framework.


% LocalWords:  IGJ toolset AbstractCollection ConcurrentHashMap NullnessUtils
% LocalWords:  castNonNull
