\htmlhr
\chapter{Linear checker for preventing aliasing\label{linear-checker}}

The Linear Checker implements type-checking for a linear type system.  A
linear type system prevents aliasing:  there is only one (usable) reference
to a given object at any time.  Once a reference appears on the right-hand
side of an assignment, it may not be used any more.  The same rule applies
for pseudo-assignments such as procedure argument-passing (including as the
receiver) or return.

One way of thinking about this is that a reference can only be used once,
after which it is ``used up''.  This property is checked statically at
compile time.  The single-use property only applies to use in an
assignment, which makes a new reference to the object; ordinary field
dereferencing does not use up a reference.

By forbidding aliasing, a linear type system can prevent problems such as
unexpected modification (by an alias), or ineffectual modification (after a
reference has already been passed to, and used by, other code).


% TODO: need better motivation.


To run the Linear Checker, supply the \code{-processor
  checkers.Linear.LinearChecker} command-line option to javac.


Figure~\ref{fig:linear-example} gives an example of the Linear Checker's rules.

\begin{figure}
%BEGIN LATEX
\begin{smaller}
%END LATEX
\begin{Verbatim}
class Pair {
  Object a;
  Object b;
  public String toString() {
    return "<" + String.valueOf(a) + "," + String.valueOf(b) + ">";
  }
}

void print(@Linear Object arg) {
  System.out.println(arg);
}

@Linear Pair printAndReturn(@Linear Pair arg) {
  System.out.println(arg.a);
  System.out.println(arg.b);      // OK: field dereferencing does not use up the reference arg
  return arg;
}

@Linear Object m(Object o, @Linear Pair lp) {
  @Linear Object lo2 = o;         // ERROR: aliases may exist
  @Linear Pair lp3 = lp;          
  @Linear Pair lp4 = lp;          // ERROR: reference lp was already used
  lp3.a;                            
  lp3.b;                          // OK: field dereferencing does not use up the reference
  print(lp3);
  print(lp3);                     // ERROR: reference lp3 was already used
  lp3.a;                          // ERROR: reference lp3 was already used
  @Linear Pair lp4 = new Pair(...);
  lp4.toString();
  lp4.toString();                 // ERROR: reference lp4 was already used
  lp4 = new Pair();               // OK to reassign to a used-up reference
  // If you need a value back after passing it to a procedure, that
  // procedure must return it to you.
  lp4 = printAndReturn(lp4);
  if (...) {
    print(lp4);
  }
  if (...) {
    return lp4;                   // ERROR: reference lp4 may have been used
  } else {
    return new Object();
  }
}
\end{Verbatim}
%BEGIN LATEX
\end{smaller}
%END LATEX
\caption{Example of Linear Checker rules.}
\label{fig:linear-example}
\end{figure}


\section{Linear annotations\label{linear-annotations}}

The linear type system uses one user-visible annotation:
\code{@\refclass{linear/quals}{Linear}}.  The annotation indicates
a type for which each value may only have a single reference ---
equivalently, may only be used once on the right-hand side of an
assignment.

The full qualifier hierarchy for the linear type system includes three
types:
\begin{itemize}
\item
\code{@UsedUp} is the type of references whose object has been assigned to
another reference.  The reference may not be used in any way, including
having its fields dereferenced, being tested for equality with \<==>, or
being assigned to another reference.  Users never need to write this
qualifier.
\item
\code{@Linear} is the type of references that have no aliases, and that may
be dereferenced at most once in the future.  The type of \<new \emph{T}()> is
\code{@Linear \emph{T}} (the analysis does not account for the slim
possibility that an alias to \<this> escapes the constructor).
\item
\code{@NonLinear} is the type of references that may be dereferenced, and
aliases made, as many times as desired.  This is the default, so users only
need to write \code{@NonLinear} if they change the default.
\end{itemize}

% TODO: Should draw a picture rather than leaving this as text.

\noindent
\code{@UsedUp} is a supertype of \code{@NonLinear}, which is a
supertype of \code{@Linear}.

This hierarchy makes an assignment like

\begin{Verbatim}
  @Linear Object l = new Object();
  @NonLinear Object nl = l;
  @NonLinear Object nl2 = nl;
\end{Verbatim}

\noindent
legal.  In other words, the fact that an object is referenced by a
\<@Linear> type means that there is only one usable reference to it \emph{now},
not that there will \emph{never} be multiple usable references to it.
(The latter guarantee would be possible to enforce, but it is not what the
Linear Checker does.)


\section{Limitations\label{linear-limitations}}

The \<@Linear> annotation is supported and checked only on method
parameters (including the receiver), return types, and local variables.
Supporting \<@Linear> on fields would require a sophisticated alias
analysis or type system, and is future work.
% One could imagine "this-linear" for handling fields.

No annotated libraries are provided for linear types.  Most libraries would
not be able to use linear types in their purest form.  For example, you
cannot put a linearly-typed object in a hashtable, because hashtable
insertion calls \<hashCode>; \<hashCode> uses up the reference and does not
return the object, even though it does not retain any pointers to the
object.  For similar reasons, a collection of linearly-typed objects could
not be sorted or searched.

Our lightweight implementation is intended for use in the parts of your
program where errors relating to aliasing and object reuse are most likely.
You can use manual reasoning (and possibly an unchecked cast or warning
suppression) when objects enter or exit those portions of your program, or
when that portion of your program uses an unannotated library.


% LocalWords:  quals UsedUp hashCode NonLinear
