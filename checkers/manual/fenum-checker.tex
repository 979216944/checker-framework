\htmlhr
\chapter{Fake Enum checker\label{fenum-checker}}

Java's 
\ahref{http://java.sun.com/docs/books/jls/third\_edition/html/classes.html\#8.9}{\code{enum}}
keyword lets you define an enumeration type: a finite set of distinct values
that is disjoint from other enumerations.
Using a true enumerated type has advantages over using a set of \code{int} or
\code{String} constants; this pattern (known as the fake enum pattern) relies on
a careful, manual encoding and provides no type safety.
However, sometimes you cannot use enums.
A public API that predates Java's enum keyword may use \code{int} constants;
it cannot be changed, because doing so would break existing clients.
For example, Java's JDK still uses \code{int} constants in the AWT and Swing
frameworks.
In environments with limited resources, for example in the Android mobile phone
platform,
\ahref{http://developer.android.com/guide/practices/design/performance.html\#avoid\_enums}{avoiding enums}
can improve performance.

In cases when code has to use the fake enum pattern, the fake enum (Fenum)
checker gives the same safety guarantees as a true enumeration type. 
The developer can introduce new types that are distinct from all values of the
base type. Fenums can be introduced for primitive types as well as for reference types.


\section{Fake enum annotations}

The checker supports two ways to introduce a new fenum:

\begin{enumerate}
\item Introduce your own specialized fenum annotation with code like this:

\begin{Verbatim}[commandchars=\\\[\]]
package \textit[myproject].quals;

import java.lang.annotation.*;
import checkers.quals.SubtypeOf;
import checkers.quals.TypeQualifier;

@Documented
@Retention(RetentionPolicy.RUNTIME)
@TypeQualifier
@SubtypeOf( { FenumTop.class } )
public @interface \textit[MyFenum] {}
\end{Verbatim}

\item Use the provided \code{@\refclass{fenum/quals}{Fenum}} annotation, that takes a
\code{String} argument to distinguish different fenums.
For example, \code{@Fenum("A")} and \code{@Fenum("B")} are two distinct fenums.
\end{enumerate}


The first approach allows you to define a short, meaningful name suitable for
your project, whereas the second approach allows quick prototyping.



\section{What the Fenum checker checks}

The Fenum checker ensures that all types with a particular fenum annotation or
\code{String} argument are different from unannotated types and types with a
different fenum annotation or \code{String} argument.

The checker additionally forbids method calls on fenum types and ensures that
only one fenum type is used in comparisons and arithmetic operations (if
applicable to the annotated type).



\section{Running the Fenum checker}

The Fenum checker can be invoked by running the following commands.

If you define your own annotation, provide the name of the annotation using the
\code{-Aquals} option:

\begin{Verbatim}[commandchars=\\\[\]]
  javac -processor checkers.fenum.FenumChecker
        \textit[-Aquals=myproject.quals.MyFenum] MyFile.java ...
\end{Verbatim}

If you use the \code{@\refclass{fenum/quals}{Fenum}} annotation with a
\code{String} argument, you do not need a checker option:

\begin{Verbatim}
  javac -processor checkers.fenum.FenumChecker MyFile.java ...
\end{Verbatim}



\section{Suppressing Warnings}

One example of when you need to suppress warnings is when you initialize the
fenum constants to literal values.
To remove this warning message, add the corresponding \code{@SuppressWarnings} to either
the field or class declaration, for example:

\begin{Verbatim}
@SuppressWarnings("fenum:assignment.type.incompatible")
class MyConsts {
  public static final @Fenum("A") int ACONST1 = 1;
  public static final @Fenum("A") int ACONST2 = 2;  
}
\end{Verbatim}



\section{Example}

The following example introduces two fenums in class TestStatic
and then performs a few typical operations.

\begin{Verbatim}
@SuppressWarnings("fenum:assignment.type.incompatible")
public class TestStatic {
  public static final @Fenum("A") int ACONST1 = 1;
  public static final @Fenum("A") int ACONST2 = 2;

  public static final @Fenum("B") int BCONST1 = 4;
  public static final @Fenum("B") int BCONST2 = 5;
}

class FenumUser {
  @Fenum("A") int state1 = TestStatic.ACONST1;
  @Fenum("B") int state2 = TestStatic.ACONST1;     // Incompatible fenums forbidden!

  void fenumArg(@Fenum("A") int p) {}
	
  void foo() {
    state1 = 4;                     // Direct use of value forbidden!
    state1 = TestStatic.BCONST1;    // Incompatible fenums forbidden!
    state1 = TestStatic.ACONST2;    // ok

    fenumArg(5);                    // Direct use of value forbidden!
    fenumArg(TestStatic.BCONST1);   // Incompatible fenums forbidden!
    fenumArg(TestStatic.ACONST1);   // ok
  }
 }
\end{Verbatim}

