\htmlhr
\chapter{How to create a new checker\label{writing-a-checker}}

\newcommand{\TreeAPIBase}{http://types.cs.washington.edu/checker-framework/api/javac/tree/com/sun/source}
\newcommand{\refTreeclass}[2]{\ahref{\TreeAPIBase{}/#1/#2.html?is-external=true}{\<#2>}}
\newcommand{\ModelAPIBase}{http://types.cs.washington.edu/checker-framework/api/javac/tree/javax/lang/model}
\newcommand{\refModelclass}[2]{\ahref{\ModelAPIBase{}/#1/#2.html?is-external=true}{\<#2>}}

This chapter describes how to create a checker
--- a type-checking compiler plugin that detects bugs or verifies their
absence.  After a programmer annotates a program,
the checker plugin verifies that the code is consistent
with the annotations.
If you only want to \emph{use} a checker, you do not need to read this
chapter.

Writing a simple checker is easy!  For example, here is a complete, useful
type checker:

\begin{Verbatim}
@TypeQualifier
@SubtypeOf(Unqualified.class)
@Target({ElementType.TYPE_USE, ElementType.TYPE_PARAMETER})
public @interface Encrypted {}
\end{Verbatim}

This checker is so short because it builds on the Basic Checker
(Chapter~\ref{basic-checker}).
See Section~\ref{basic-example} for more details about this particular checker.
When you wish to create a new checker, it is often easiest to begin by
building it declaratively on top of the Basic Checker, and then return to
this chapter when you need more expressiveness or power than the Basic
Checker affords.

You can also create your own checker by customizing a different existing
checker.  Specific checkers that are designed for extension (besides the Basic
Checker) include the Fake Enumeration Checker
(\chapterpageref{fenum-checker}), the Units Checker
(\chapterpageref{units-checker}), and the typestate checkers
(\chapterpageref{typestate-checker}).
Or, you can copy and then modify a different existing checker --- whether
one distributed with the Checker Framework or a third-party one.

You can place your checker's source files wherever you like.  When you
compile your checker, \<\$CHECKERS/binary/checkers.jar> and \<\$CHECKERS/binary/javac.jar>
should be on your classpath.  (If you wish to modify an existing checker in place,
or to place the source code for your new checker in your own private copy of the
Checker Framework source code, then you need to be able to re-compile the
Checker Framework, as described in Section~\ref{build-source}.)

The rest of this chapter contains many details for people who want to write more powerful
checkers.
You do not need all of the details, at least at first.
In addition to reading this chapter of the manual, you may find it helpful
to examine the implementations of the checkers that are distributed with
the Checker Framework.  You can even create your checker by modifying one
of those.
The Javadoc documentation of the framework and the checkers is in the
distribution and is also available online at
\myurl{http://types.cs.washington.edu/checker-framework/current/doc/}.

If you write a new checker and wish to advertise it to the world, let us
know so we can mention it in the Checker Framework Manual, link to
it from the webpages, or include it in the Checker Framework distribution.
For examples, see Chapters~\ref{typestate-checker}
and~\ref{third-party-checkers}.


\section{Relationship of the Checker Framework to other tools\label{tool-relationships}}

This table shows the relationship among various tools.
All of the tools use the Type Annotations (JSR 308) syntax.
You use the Checker Framework to build pluggable type systems, and the
Annotation File Utilities to manipulate \code{.java} and \code{.class} files.

\newlength{\bw}
\setlength{\bw}{.5in}

%% Strictly speaking, "Basic Checker" should sit on top of Checker
%% Framework and below all the specific checkers.  But omit it for simplicity.

% Unfortunately, Hevea inserts a horizontal line between every pair of rows
% regardless of whether there is a \hline or \cline.  So, make paragraphs.
\begin{center}
\begin{tabular}{|p{\bw}|p{\bw}|p{\bw}|p{\bw}|p{.4\bw}|p{\bw}|p{1.5\bw}|p{1\bw}|}
\cline{1-4} \cline{6-6}
\centering Basic \par Checker &
\centering Nullness \par Checker &
\centering Mutation \par Checker &
\centering Tainting \par Checker &
\centering \ldots &
\centering Your \par Checker &
\multicolumn{2}{c}{} 
\\ \hline
\multicolumn{6}{|p{6\bw}|}{\centering Base Checker \par (enforces subtyping rules)} &
\centering Type \par inference &
% Adding "\centering" here causes a LaTeX alignment error
Other \par tools
\\ \hline
\multicolumn{6}{|p{6\bw}|}{\centering Checker Framework \par (enables creation of pluggable type-checkers)} &
\multicolumn{2}{p{3\bw}|}{\centering \ahref{http://types.cs.washington.edu/annotation-file-utilities/}{Annotation File Utilities} \par (\code{.java} $\leftrightarrow$ \code{.class} files)} 
\\ \hline
\multicolumn{8}{|p{8.5\bw}|}{\centering
  \ahref{http://types.cs.washington.edu/jsr308/}{Type Annotations} syntax
  and classfile format (``JSR 308'') \par \centering (no built-in semantics)} \\ \hline
\end{tabular}
\end{center}


The Base Checker enforces the standard subtyping rules on extended types.
The Basic Checker is a simple use of the Base Checker that supports
providing type qualifiers on the command line.
You usually want to build your checker on the Base Checker.


\section{The parts of a checker\label{parts-of-a-checker}}

The Checker Framework provides abstract base classes (default
implementations), and a specific checker overrides as little or as much of
the default implementations as necessary.
%
Sections~\ref{writing-typequals}--\ref{writing-compiler-interface} describe
the components of a type system as written using the Checker Framework:

\begin{description}

\item{\ref{writing-typequals}}
  \textbf{Type qualifiers and hierarchy.}  You define the annotations for
  the type system and the subtyping relationships among qualified types
  (for instance, that \<@NonNull Object> is a subtype of \<@Nullable
  Object>).

\item{\ref{writing-type-introduction}}
  \textbf{Type introduction rules.}  For some types and
  expressions, a qualifier should be treated as implicitly present even if a
  programmer did not explicitly write it.  For example, in the Nullness
  type system every literal
  other than \<null> has a \<@\refclass{nullness/quals}{NonNull}> type;
  examples of literals include \<"some string"> and \<java.util.Date.class>.

\item{\ref{extending-visitor}}
  \textbf{Type rules.}  You specify the type system semantics (type
  rules), violation of which yields a type error.  There are two types of
  rules.
\begin{itemize}
\item
  Subtyping rules related to the type hierarchy, such as that every
  assignment and pseudo-assignment satisfies a subtyping relationship.
  Your checker automatically inherits these subtyping rules from the Base
  Checker (Chapter~\ref{basic-checker}).
\item
  Additional rules that are specific to your particular checker.  For
  example, in the Nullness type system, only references with a
  \<@\refclass{nullness/quals}{NonNull}> type may be dereferenced.  You
  write these additional rules yourself.
\end{itemize}

\item{\ref{writing-compiler-interface}}
  \textbf{Interface to the compiler.}  The compiler interface indicates
  which annotations are part of the type system, which command-line options
  and \<@SuppressWarnings> annotations the checker recognizes, etc.
\end{description}


\section{Annotations: Type qualifiers and hierarchy\label{writing-typequals}}

A type system designer specifies the qualifiers in the type system and
the type hierarchy that relates them.

%% True, but seems irrelevant here, so it detracts from the message.
% Each qualifier restricts the values that
% a type can represent.  For example \<@NonNull String> type can only
% represent non-null values, indicating that the variable may not hold
% \<null> values.

Type qualifiers are defined as Java annotations~\cite{JSR269}.  In Java, an
annotation is defined using the Java \code{@interface} keyword.
For example:

\begin{Verbatim}
  // Define an annotation for the @NonNull type qualifier.
  @TypeQualifier
  @Target({ElementType.TYPE_USE, ElementType.TYPE_PARAMETER})
  public @interface NonNull { }
\end{Verbatim}

\noindent
Write the \<@\refclass{quals}{TypeQualifier}> meta-annotation on the annotation definition
to indicate that the annotation represents a type qualifier
and should be processed by the checker.
Also write a \<@\sunjavadoc{java/lang/annotation/Target.html}{Target}>
meta-annotation to indicate where the annotation may be written.
(An annotation that is written on an annotation
definition, such as \<@\refclass{quals}{TypeQualifier}>, is called a \emph{meta-annotation}.)

% \noindent
% The \<@TypeQualifier> meta-annotation
% distinguishes it from an ordinary
% annotation that applies to a declaration (e.g., \<@Deprecated> or
% \<@Override>).
% The framework ignores any annotation whose
% declaration does not bear the \<@TypeQualifier> meta-annotation (with minor
% exceptions, such as \<@SuppressWarnings>).

The type hierarchy induced by the qualifiers can be defined either
declaratively via meta-annotations (Section~\ref{declarative-hierarchy}), or procedurally through
subclassing \refclass{types}{QualifierHierarchy} or
\refclass{types}{TypeHierarchy} (Section~\ref{procedural-hierarchy}).


\subsection{Declaratively defining the qualifier and type hierarchy\label{declarative-hierarchy}}

Declaratively, the type system designer uses two meta-annotations (written
on the declaration of qualifier annotations) to specify the qualifier
hierarchy.

\begin{itemize}

\item \code{@\refclass{quals}{SubtypeOf}} denotes that a qualifier is a subtype of
  another qualifier or qualifiers, specified as an array of class
  literals.  For example, for any type $T$,
  \code{@\refclass{nullness/quals}{NonNull}} $T$ is a subtype of \code{@\refclass{nullness/quals}{Nullable}} $T$:

  \begin{Verbatim}
    @TypeQualifier
    @Target({ElementType.TYPE_USE, ElementType.TYPE_PARAMETER})
    @SubtypeOf( { Nullable.class } )
    public @interface NonNull { }
  \end{Verbatim}

  % (The actual definition of \refclass{nullness/quals}{NonNull} is slightly more complex.)


  %% True, but a distraction.  Move to Javadoc?
  % (It would be more natural to use Java subtyping among the qualifier
  % annotations, but Java forbids annotations from subtyping one another.)
  %
  \<@\refclass{quals}{SubtypeOf}> accepts multiple annotation classes as an argument,
  permitting the type hierarchy to be an arbitrary DAG\@.  For example,
  in the IGJ type system (Section~\ref{igj-annotations}), \<@\refclass{igj/quals}{Mutable}>
  and \<@\refclass{igj/quals}{Immutable}> induce two mutually exclusive subtypes of the
  \<@\refclass{igj/quals}{ReadOnly}> qualifier.
%TODO: In the IGJ hierarchy I didn't find a use of multiple supertypes. Like
% this the previous paragraph is confusing, as it does not give a correct
% example.

% TODO: describe multiple type hierarchies
% TODO: describe multiple polymorphic qualifiers and PolyAll
% TODO: the code consistently uses "top" for type qualifiers and
%       "root" for ASTs, in particular for CompilationUnitTrees.

  All type qualifiers, except for polymorphic qualifiers (see below and
  also Section~\ref{qualifier-polymorphism}), need to be
  properly annotated with \refclass{quals}{SubtypeOf}.

  The top qualifier is annotated with
  \<@SubtypeOf( \{ \} )>.  The top qualifier is the qualifier that is
  a supertype of all other qualifiers.  For example, \<@\refclass{nullness/quals}{Nullable}>
  is the top qualifier of the Nullness type system, hence is defined as:

  \begin{Verbatim}
    @TypeQualifier
    @Target({ElementType.TYPE_USE, ElementType.TYPE_PARAMETER})
    @SubtypeOf( { } )
    public @interface Nullable { }
  \end{Verbatim}

  If the top qualifier of the hierarchy is the unqualified type, then its children
  will use \code{@SubtypeOf(Unqualified.class)}, but no \code{@SubtypeOf(
    \{ \} )} annotation on the top qualifier is necessary.  For an example, see the
  \<Encrypted> type system of Section~\ref{encrypted-example}.

\item \<@\refclass{quals}{PolymorphicQualifier}> denotes that a qualifier is a
  polymorphic qualifier.  For example:

  \begin{Verbatim}
    @TypeQualifier
    @Target({ElementType.TYPE_USE, ElementType.TYPE_PARAMETER})
    @PolymorphicQualifier
    public @interface PolyNull { }
  \end{Verbatim}

  For a description of polymorphic qualifiers, see
  Section~\ref{qualifier-polymorphism}.  A polymorphic qualifier needs
  no \<@\refclass{quals}{SubtypeOf}> meta-annotation and need not be
  mentioned in any other \<@\refclass{quals}{SubtypeOf}>
  meta-annotation.

\end{itemize}

\urldef{\isSubtypeURL}\url{doc/checkers/util/GraphQualifierHierarchy.html#isSubtype(java.util.Collection,%20java.util.Collection)}

The declarative and procedural mechanisms for specifying the hierarchy can
be used together.  In particular, when using the \<@\refclass{quals}{SubtypeOf}>
meta-annotation, further customizations may be
performed procedurally (Section~\ref{procedural-hierarchy})
by overriding the \ahref{\isSubtypeURL}{\code{isSubtype}} method in the checker class
(Section~\ref{writing-compiler-interface}).
However, the declarative mechanism is sufficient for most type systems.


\subsection{Procedurally defining the qualifier and type hierarchy\label{procedural-hierarchy}}

\urldef{\createQualifierHierarchyURL}\url{doc/checkers/basetype/BaseTypeChecker.html#createQualifierHierarchy()}
\urldef{\createTypeHierarchyURL}\url{doc/checkers/basetype/BaseTypeChecker.html#createTypeHierarchy()}

While the declarative syntax suffices for many cases, more complex
type hierarchies can be expressed by overriding, in \refclass{basetype}{BaseTypeChecker},
either \ahref{\createQualifierHierarchyURL}{\<createQualifierHierarchy>} or \ahref{\createTypeHierarchyURL}{\<createTypeHierarchy>} (typically
only one of these needs to be overridden).
For more details, see the Javadoc of those methods and of the classes
\refclass{types}{QualifierHierarchy} and \refclass{types}{TypeHierarchy}.

The \refclass{types}{QualifierHierarchy} class represents the qualifier hierarchy (not the
type hierarchy), e.g., \refclass{igj/quals}{Mutable}
is a subtype of \refclass{igj/quals}{ReadOnly}.  A type-system designer may subclass
\refclass{types}{QualifierHierarchy} to express customized qualifier
relationships (e.g., relationships based on annotation
arguments).

The \refclass{types}{TypeHierarchy} class represents the type hierarchy ---
that is, relationships between
annotated types, rather than merely type qualifiers, e.g., \<@Mutable
Date> is a subtype of \<@ReadOnly Date>.  The default \refclass{types}{TypeHierarchy} uses
\refclass{types}{QualifierHierarchy} to determine all subtyping relationships.
The default \refclass{types}{TypeHierarchy} handles
generic type arguments, array components, type variables, and
wildcards in a similar manner to the Java standard subtype
relationship but with taking qualifiers into consideration.  Some type
systems may need to override that behavior.  For instance, the Java
Language Specification specifies that two generic types are subtypes only
if their type arguments are identical:  for example,
\code{List<Date>} is not a subtype of \code{List<Object>}, or of any other
generic \code{List}.
(In the technical jargon, the generic arguments are ``invariant'' or ``novariant''.)
The Javari type system overrides this
behavior to allow some type arguments to change covariantly in a type-safe
manner (e.g.,
\code{List<@Mutable Date>} is a subtype of \code{List<@QReadOnly Date>}).


\subsection{Defining a default annotation\label{typesystem-defaults}}

% This paragraph is out of place.

\urldef{\addAbsoluteDefaultURL}\url{doc/checkers/util/QualifierDefaults.html#addAbsoluteDefault(javax.lang.model.element.AnnotationMirror,%20checkers.quals.DefaultLocation)}

A type system designer may set a default annotation.  A user may override
the default; see Section~\ref{defaults}.

The type system designer may specify a default annotation declaratively,
using the \code{@\refclass{quals}{DefaultQualifierInHierarchy}}
meta-annotation.
Note that the default will apply to any source code that the checker reads,
including stub libraries, but will not apply to compiled \code{.class}
files that the checker reads.

Alternately, the type system designer may specify a default procedurally,
by calling the
\ahref{\addAbsoluteDefaultURL}{\<QualifierDefaults.addAbsoluteDefault>}
method.  You may do this even if you have declaratively defined the
qualifier hierarchy; see the Nullness Checker's implementation for an
example.


Recall that defaults are distinct
from implicit annotations; see Sections~\ref{effective-qualifier}
and~\ref{writing-type-introduction}.


\subsection{Completeness of the type hierarchy\label{bottom-and-top-qualifier}}

When you define a type system, its type hierarchy must be a
complete lattice --- that is, there must be a top type that is a
supertype of all other types, and there must be a bottom type that is a
subtype of all other types.
Furthermore, it is best if the top type and bottom type are defined
explicitly for the type system, rather than (say) reusing a qualifier from the
Checker Framework such as \<@Unqualified>.

It is possible that a single type-checker checks multiple type hierarchies.
An example is the Nullness Checker that has separate type hierarchies for
nullness, initialization, and map keys.  In this case, each type hierarchy
would have its own top qualifier and its own bottom qualifier; they don't
all have to share a single top qualifier or a single bottom qualifier.


\label{bottom-qualifier}

\paragraph{Bottom qualifier}
Your type hierarchy must have a bottom qualifier
--- a qualifier that is a (direct or indirect) subtype of every other
qualifier.

Your type system must give \<null> the bottom type.  (The only exception
is if the type system has special treatment for \<null> values, as the
Nullness Checker does.)  This legal code
will not type-check unless \<null> has the bottom type:
\begin{Verbatim}
<T> T f() {
    return null;
\end{Verbatim}

You don't necessarily have to define a new bottom qualifier.  But, You can
use \<checkers.quals.Bottom> if your type system does not already have an
appropriate bottom qualifiere

If your type system has a special bottom type that is used \emph{only} for
the \code{null} value, then users should never write the bottom qualifier
explicitly.  To ensure this, write \<@Target(\{\})> on the definition of
the bottom qualifier.


The hierarchy shown in Figure~\ref{fig:igj-hierarchy} lacks
a bottom qualifier, because there is no qualifier that is a subtype of both
\<@Immutable> and \<@Mutable>.
The actual IGJ hierarchy does contain a (non-user-visible) bottom qualifier,
defined like this:

\begin{Verbatim}
  @TypeQualifier
  @SubtypeOf({Mutable.class, Immutable.class, I.class})
  @Target({}) // forbids a programmer from writing it in a program
  @ImplicitFor(trees = { Kind.NULL_LITERAL, Kind.CLASS, Kind.NEW_ARRAY },
               typeClasses = { AnnotatedPrimitiveType.class })
  @interface IGJBottom { }
\end{Verbatim}


\label{top-qualifier}

\paragraph{Top qualifier}
Your type hierarchy must have a top qualifier
--- a qualifier that is a (direct or indirect) supertype of every other
qualifier.
The default type for local variables is the top
qualifier (that type is then flow-sensitively
refined depending on what values are stored in the local variable).
If there is no single top qualifier, then there is no
unambiguous choice to make for local variables.

Furthermore, it is most convenient to users if the top qualifier is defined
by the type system.  An example of a type system that does
\emph{not} do that is the \<@Encrypted> type system of
Section~\ref{encrypted-example}.  It lacks its own explicit top qualifier and instead
uses \<@Unqualified>, which is shared across multiple type systems:

\begin{Verbatim}
  @TypeQualifier
  @SubtypeOf(Unqualified.class)
  @Target({ElementType.TYPE_USE, ElementType.TYPE_PARAMETER})
  public @interface Encrypted {}
\end{Verbatim}

% \noindent
% The interning type system of Section~\ref{interning-checker} also lacks a
% top qualifier; there is no \<@Uninterned> qualifier that is a supertype of 
% \<@\refclass{interning/quals}{Interned}>.

\noindent
It can be convenient to use \<@Unqualified> as the top type to avoid
having to define your own top type.  The disadvantage is that users lose
flexibility in expressing defaults:  there is no
way for a user to change the default qualifier.  A user could not specify
\<@DefaultQualifier(Unqualified.class)>,
because many type systems share \<@Unqualified> and it would apply to every
type system, which is unlikely to be desired.

It is best if a type system as an explicit qualifier for every
possible meaning.  For example,
the Nullness type system has both \<@\refclass{nullness/quals}{Nullable}> and
\<@\refclass{nullness/quals}{NonNull}>.  Because it has no built-in meaning
for unannotated types; a user may specify a default qualifier.  This
greater flexibility for the user is usually preferable.

There are reasons to not explicitly define the top qualifier, but to reuse
\<@Unqualified>.
The ability to omit the top qualifier is a convenience
when writing a type system, because it reduces the number of qualifiers
that must be defined; this is especially convenient when using the Basic
Checker (Section~\ref{basic-checker}).
% TODO: Examples of the following?
More importantly, omitting the top qualifier restricts the user in ways
that the type system designer may have intended.


\section{Type factory: Implicit annotations\label{writing-type-introduction}}

For some types and expressions, a qualifier should be treated as present
even if a programmer did not explicitly write it.  For example, every
literal (other than \<null>) has a \<@\refclass{nullness/quals}{NonNull}> type.

The implicit annotations may be specified declaratively and/or procedurally.


\subsection{Declaratively specifying implicit annotations\label{declarative-type-introduction}}

The \<@\refclass{quals}{ImplicitFor}> meta-annotation indicates implicit annotations.
When written on a qualifier, \refclass{quals}{ImplicitFor}
specifies the trees (AST nodes) and types for which the framework should
automatically add that qualifier.

In short, the types and trees can be
specified via any combination of five fields in \refclass{quals}{ImplicitFor}:

  \begin{itemize}
  \item \code{trees}: an array of
    \ahref{\TreeAPIBase{}/tree/Tree.Kind.html?is-external=true}{\code{com.sun.source.tree.Tree.Kind}}, e.g.,
    \code{NEW\_ARRAY} or \code{METHOD\_INVOCATION}
  \item \code{types}: an array of
    \refModelclass{type}{TypeKind}, e.g., \code{ARRAY}
    or \code{BOOLEAN}
  \item \code{treeClasses}: an array of class literals for classes
    implementing \refTreeclass{tree}{Tree}, e.g.,
    \code{LiteralTree.class} or \code{ExpressionTree.class}
  \item \code{typeClasses}: an array of class literals for classes
    implementing \code{javax.lang.model.type.TypeMirror}, e.g.,
    \code{javax.lang.model.type.PrimitiveType}.  Often you should use
    a subclass of \refclass{types}{AnnotatedTypeMirror}.
  \item \code{stringPatterns}: an array of regular expressions that will
    be matched against
    string literals, e.g., \code{"[01]+"} for a binary number.  Useful
    for annotations that indicate the format of a string.
  \end{itemize}

For example, consider the definitions of the \<@\refclass{nullness/quals}{NonNull}> and \<@\refclass{nullness/quals}{Nullable}>
type qualifiers:

%BEGIN LATEX
\begin{smaller}
%END LATEX
\begin{Verbatim}
  @TypeQualifier
  @SubtypeOf( { Nullable.class } )
  @ImplicitFor(
    types={TypeKind.PACKAGE},
    typeClasses={AnnotatedPrimitiveType.class},
    trees={
      Tree.Kind.NEW_CLASS,
      Tree.Kind.NEW_ARRAY,
      Tree.Kind.PLUS,
      // All literals except NULL_LITERAL:
      Tree.Kind.BOOLEAN_LITERAL, Tree.Kind.CHAR_LITERAL, Tree.Kind.DOUBLE_LITERAL, Tree.Kind.FLOAT_LITERAL,
      Tree.Kind.INT_LITERAL, Tree.Kind.LONG_LITERAL, Tree.Kind.STRING_LITERAL
    })
  @Target({ElementType.TYPE_USE, ElementType.TYPE_PARAMETER})
  public @interface NonNull {  }


  @TypeQualifier
  @SubtypeOf({})
  @ImplicitFor(trees={Tree.Kind.NULL_LITERAL})
  @Target({ElementType.TYPE_USE, ElementType.TYPE_PARAMETER})
  public @interface Nullable { }
\end{Verbatim}
%BEGIN LATEX
\end{smaller}
%END LATEX

For more details, see the Javadoc for the \refclass{quals}{ImplicitFor}
  annotation, and the Javadoc for the javac classes that are linked from
it.  You only need to understand a small amount about the javac AST, such
as the
\ahref{\TreeAPIBase{}/tree/Tree.Kind.html?is-external=true}{\code{Tree.Kind}}
and
\refModelclass{type}{TypeKind}
enums.  All the information you need is in the Javadoc, and
Section~\ref{javac-tips} can help you get started.


\subsection{Procedurally specifying implicit annotations}


The Checker Framework provides a representation of annotated types,
\refclass{types}{AnnotatedTypeMirror}, that extends the standard \<TypeMirror>
interface but integrates a representation of the annotations into a
type representation.  A checker's \emph{type factory} class, given an AST
node, returns the annotated type of that expression.  The Checker
Framework's abstract
\emph{base type factory} class, \refclass{types}{AnnotatedTypeFactory},
supplies a uniform, Tree-API-based interface
for querying the annotations on a program element, regardless of
whether that element is declared in a source file or in a class file.
It also handles default annotations, and it optionally performs
flow-sensitive local type inference.

\refclass{types}{AnnotatedTypeFactory} inserts the qualifiers that the programmer
explicitly inserted in the code.  Yet, certain constructs should be
treated as having a type qualifier even when the programmer has not
written one.  The type system designer may subclass
\refclass{types}{AnnotatedTypeFactory} and override
\code{annotateImplicit(Tree,AnnotatedTypeMirror)} and
\code{annotateImplicit(Element,AnnotatedTypeMirror)} to account for
such constructs.


\subsection{Flow-sensitive type qualifier inference}

The Checker Framework provides automatic type refinement as described
in Section~\ref{type-refinement}.

Class
\refclass{types}{BasicAnnotatedTypeFactory}
provides a 3 parameter constructor that allows subclasses to disable
flow inference.
By default the 2 parameter constructor performs flow inference.
To disable flow inference, call
\code{super(checker, root, false);}
in your subtype of
\refclass{types}{BasicAnnotatedTypeFactory}.


\section{Visitor: Type rules\label{extending-visitor}}

A type system's rules define which operations on values of a
particular type are forbidden.
These rules must be defined procedurally, not declaratively.

The Checker Framework provides a \textit{base visitor class},
\refclass{basetype}{BaseTypeVisitor}, that performs type-checking at each node of a
source file's AST\@.  It uses the visitor design pattern to traverse
Java syntax trees as provided by Oracle's
\ahref{http://types.cs.washington.edu/checker-framework/api/javac/tree//index.html}{Tree
API},
and it issues a warning whenever the type system is violated.

A checker's visitor overrides one method in the base visitor for each special
rule in the type qualifier system.  Most type-checkers
override only a few methods in \refclass{basetype}{BaseTypeVisitor}.  For example, the
visitor for the Nullness type system of Chapter~\ref{nullness-checker}
contains a single 4-line method that warns if an expression of nullable type
is dereferenced, as in:
\begin{Verbatim}
  myObject.hashCode();  // invalid dereference
\end{Verbatim}



By default, \refclass{basetype}{BaseTypeVisitor} performs subtyping checks that are
similar to Java subtype rules, but taking the type qualifiers into account.
\refclass{basetype}{BaseTypeVisitor} issues these errors:

\begin{itemize}

\item invalid assignment (type.incompatible) for an assignment from
  an expression type to an incompatible type.  The assignment may be a
  simple assignment, or pseudo-assignment like return expressions or
  argument passing in a method invocation

  In particular, in every assignment and pseudo-assignment, the
  left-hand side of the assignment is a supertype of (or the same type
  as) the right-hand side.  For example, this assignment is not
  permitted:

  \begin{Verbatim}
    @Nullable Object myObject;
    @NonNull Object myNonNullObject;
    ...
    myNonNullObject = myObject;  // invalid assignment
  \end{Verbatim}

\item invalid generic argument (generic.argument.invalid) when a type
  is bound to an incompatible generic type variable

\item invalid method invocation (method.invocation.invalid) when a
  method is invoked on an object whose type is incompatible with the
  method receiver type

\item invalid overriding parameter type (override.parameter.invalid)
  when a parameter in a method declaration is incompatible with that
  parameter in the overridden method's declaration

\item invalid overriding return type (override.return.invalid) when a
  parameter in a method declaration is incompatible with that
  parameter in the overridden method's declaration

\item invalid overriding receiver type (override.receiver.invalid)
  when a receiver in a method declaration is incompatible with that
  receiver in the overridden method's declaration

\end{itemize}


\subsection{AST traversal\label{ast-traversal}}

The Checker Framework needs to do its own traversal of the AST even though
it operates as an ordinary annotation processor~\cite{JSR269}.  Annotation
processors can utilize a visitor for Java code, but that visitor only
visits the public elements of Java code, such as classes, fields, methods,
and method arguments --- it does not visit code bodies or various other
locations.  The Checker Framework hardly uses the built-in visitor --- as
soon as the built-in visitor starts to visit a class, then the Checker
Framework's visitor takes over and visits all of the class's source code.

Because there is no standard API for the AST of Java code\footnote{Actually,
there is a standard API for Java ASTs --- JSR 198 (Extension API for
Integrated Development Environments)~\cite{JSR198}.  If tools were to
implement it
(which would just require writing wrappers or adapters), then the Checker
Framework and similar tools could be portable among different compilers and
IDEs.}, the Checker
Framework uses the javac implementation.  This is why the Checker Framework
is not deeply integrated with Eclipse, but runs as an external tool (see
Section~\ref{eclipse}).


\subsection{Avoid hardcoding\label{avoid-hardcoding}}

It may be tempting to write a type-checking rule for method invocation,
where your rule checks the name of the method being called and then treats
the method in a special way.  This is usually the wrong approach.  It
is better to write annotations, in a stub file
(Chapter~\ref{annotating-libraries}), and leave the work to the standard
type-checking rules.


\section{The checker class:  Compiler interface\label{writing-compiler-interface}}

A checker's entry point is a subclass of \refclass{basetype}{BaseTypeChecker}.  This entry
point, which we call the checker class, serves two
roles:  an interface to the compiler and a factory for constructing
type-system classes.

Because the Checker Framework provides reasonable defaults, oftentimes the
checker class has no work to do.  Here are the complete definitions of the
checker classes for the Interning and Nullness Checkers:

\begin{Verbatim}
  @TypeQualifiers({ Interned.class, PolyInterned.class })
  @SupportedLintOptions({"dotequals"})
  public final class InterningChecker extends BaseTypeChecker { }

  @TypeQualifiers({ Nullable.class, Raw.class, NonNull.class, PolyNull.class })
  @SupportedLintOptions({"flow", "cast", "cast:redundant"})
  public class NullnessChecker extends BaseTypeChecker { }
\end{Verbatim}


\urldef{\getSupportedTypeQualifiersURL}\url{http://types.cs.washington.edu/checker-framework/current/doc/checkers/basetype/BaseTypeChecker.html#getSupportedTypeQualifiers()}

The checker class must be annotated by
\code{@\refclass{quals}{TypeQualifiers}}, which lists the annotations
that make up the type hierarchy for this checker (including
polymorphic qualifiers), provided as an array of class literals.  Each
one is a type qualifier whose definition bears the
\code{@\refclass{quals}{TypeQualifier}} meta-annotation (or is
returned by the
\ahref{\getSupportedTypeQualifiersURL}{\<BaseTypeChecker\-.getSupportedTypeQualifiers>}
method).

\urldef{\reportURL}\url{http://types.cs.washington.edu/checker-framework/current/doc/checkers/source/SourceChecker.html#report(checkers.source.Result,%20java.lang.Object)}

The checker class bridges between the compiler and the rest of the checker.  It
invokes the type-rule check visitor on every Java source file being
compiled, and provides a simple API, \ahref{\reportURL}{\<report>}, to issue
errors using the compiler error reporting mechanism.

Also, the checker class follows the factory method pattern to
construct the concrete classes (e.g., visitor, factory) and annotation
hierarchy representation.  It is a convention that, for
a type system named Foo, the compiler
interface (checker), the visitor, and the annotated type factory are
named as \<FooChecker>, \<FooVisitor>, and \<FooAnnotatedTypeFactory>.
\refclass{basetype}{BaseTypeChecker} uses the convention to
reflectively construct the components.  Otherwise, the checker writer
must specify the component classes for construction.

A checker can customize the default error messages through a
\sunjavadoc{java/util/Properties.html}{Properties}-loadable text file named
\<messages.properties> that appears in the same directory as the checker class.
The property file keys are the strings passed to \ahref{\reportURL}{\<report>}
(like \code{type.incompatible}) and the values are the strings to be
printed (\code{"cannot assign ..."}).
The \<messages.properties> file only need to mention the new messages that
the checker defines.
It is also allowed to override messages defined in superclasses, but this
is rarely needed.
For more details about message keys, see Section~\refwithpageparen{compiler-message-keys}.


\subsection{Bundling multiple checkers}

To run a checker, a user supplies the \<-processor> command-line option.
There are two ways to run multiple related checkers as a unit.

\begin{enumerate}
\item
A user can pass
multiple \<-processor> command-line options, like:

\begin{Verbatim}
  javac -processor DistanceUnitChecker -processor SpeedUnitChecker ... files ...
\end{Verbatim}

\noindent
This is verbose, and it is also error-prone, since a user might omit one of
several related checkers that are designed to be run together.

\item
You can define an aggregate checker class that combines
multiple checkers.  Extend \refclass{source}{AggregateChecker} and override
the \<getSupportedTypeCheckers> method, like the following:

\begin{Verbatim}
  public class UnitCheckers extends AggregateChecker {
    protected Collection<Class<? extends SourceChecker>> getSupportedCheckers() {
      return Arrays.asList(DistanceUnitChecker.class, SpeedUnitChecker.class);
    }
  }
\end{Verbatim}

\noindent
Now, a user can pass a single \<-processor> argument on the command line:

\begin{Verbatim}
  javac -processor UnitCheckers ... files ...
\end{Verbatim}

\end{enumerate}



\subsection{Providing command-line options}

A checker can provide two kinds of command-line options:
boolean flags and
named string values (the standard annotation processor
options).

\subsubsection{Boolean flags}

To specify a simple boolean flag, add:

\begin{Verbatim}
@SupportedLintOptions({"flag"})
\end{Verbatim}

to your checker subclass.
The value of the flag can be queried using

\begin{Verbatim}
checker.getLintOption("flag", false)
\end{Verbatim}

The second argument sets the default value that should be returned.

To pass a flag on the command line, call javac as follows:

\begin{Verbatim}
javac -processor Mine -Alint=flag
\end{Verbatim}


\subsubsection{Named string values}

For more complicated options, one can use the standard annotation
processing \code{SupportedOptions} annotation on the checker, as in:

\begin{Verbatim}
@SupportedOptions({"info"})
\end{Verbatim}

The value of the option can be queried using

\begin{Verbatim}
env.getOptions().get("info")
\end{Verbatim}

where \code{env} is the current \code{ProcessingEnvironment}.

To pass an option on the command line, call javac as follows:

\begin{Verbatim}
javac -processor Mine -Ainfo=p1,p2
\end{Verbatim}

The value is returned as a single string and you have to perform the
required parsing of the option.




\section{Testing framework\label{testing-framework}}

The Checker Framework comes with a testing framework that is used for
testing the distributed checkers.
It is easy to use this testing framework to ensure correctness of your
checker!

You first need to provide a subclass of
\code{ParameterizedCheckerTest}
that determines the checker to use and all command-line options that
should be provided.
This class can be run as a JUnit test runner.
Note that you always need to use the
\code{-Anomsgtext} option to suppress the substitution of message keys
by human-readable values.
See the test setup classes in directory \code{tests/src/tests/} for examples.

Locate all your test cases in a subdirectory of the \code{tests}
directory.
The individual test cases are normal Java files that use stylized
comments to indicate expected error messages.
For example, consider this test case from the Nullness Checker:

\begin{Verbatim}
  //:: error: (dereference.of.nullable)
  s.toString();
\end{Verbatim}

An expected error message is introduced by the \code{//::} comment.
The next token is either \code{error:} or \code{warning:},
distinguishing what kind of message is expected.
Finally, the message key for the expected message is given.

Multiple expected messages can be given using the "//:: A :: B :: C"
syntax.

As an alternative, expected errors can be specified in a separate file
using the \code{.out} file extension.
These files are of the following format:

\begin{Verbatim}
:19: error: (dereference.of.nullable)
\end{Verbatim}

The number between the colons is the line number of the expected error
message.
This format is a lot harder to maintain and we suggest using the
in-line comment format.



\section{Debugging options\label{debugging-options}}

The Checker Framework provides debugging options that can be helpful when
writing a checker. These are provided via the standard \code{javac} ``\code{-A}''
switch, which is used to pass options to an annotation processor.

\begin{itemize}

\item \code{-Anomsgtext}: use message keys (such as ``\code{type.invalid}'')
rather than full message text when reporting errors or warnings

\item \code{-Adetailedmsgtext}: print a more detailed error message in addition
to the full message text when reporting errors or warnings

\item \code{-Ashowchecks}: print debugging information for each
pseudo-assignment check (as performed by \refclass{basetype}{BaseTypeVisitor}; see Section
\ref{extending-visitor} above)

\item \code{-Afilenames}: print the name of each file before type-checking it

\item \code{-AprintErrorStack}: print a stack trace together with
internal Checker Framework error messages

\item \code{-AprintAllQualifiers}: print all type qualifiers, including
qualifiers like \code{@Unqualified} which are usually not shown.
(Use the \code{@InvisibleQualifier} meta-annotation on a qualifier to hide it.)

\item \code{-Aignorejdkastub}: ignore the jdk.astub file in the checker
directory. Files passed through the \code{-Astubs} option are still processed. This is useful when
compiling the source code that is described by the stub file and experimenting with an
alternative stub file.

\end{itemize}

The following example demonstrates how these options are used:

%BEGIN LATEX
\begin{smaller}
%END LATEX
\begin{Verbatim}
$ javac -processor checkers.interning.InterningChecker \
    examples/InternedExampleWithWarnings.java -Ashowchecks -Anomsgtext -Afilenames

[InterningChecker] InterningExampleWithWarnings.java
 success (line  18): STRING_LITERAL "foo"
     actual: DECLARED @checkers.interning.quals.Interned java.lang.String
   expected: DECLARED @checkers.interning.quals.Interned java.lang.String
 success (line  19): NEW_CLASS new String("bar")
     actual: DECLARED java.lang.String
   expected: DECLARED java.lang.String
examples/InterningExampleWithWarnings.java:21: (not.interned)
    if (foo == bar)
            ^
 success (line  22): STRING_LITERAL "foo == bar"
     actual: DECLARED @checkers.interning.quals.Interned java.lang.String
   expected: DECLARED java.lang.String
1 error
\end{Verbatim}
%BEGIN LATEX
\end{smaller}
%END LATEX

You can use any standard debugger to observe the execution of your checker.
Set the execution main class to \code{com.sun.tools.javac.Main}, and insert
the JSR 308 javac.jar (resides in
\code{.../jsr308-langtools/dist/lib/javac.jar}).  If using an IDE, it is
recommended that you add \code{.../jsr308-langtools} as a project, so you
can step into its source code if needed.

You can also set up remote (or local) debugging using the following command as a template:

\begin{Verbatim}
java -jar $CHECKERS/binary/checkers.jar \
    -J-Xdebug -J-Xrunjdwp:transport=dt_socket,server=y,suspend=y,address=5005 \
    -processor checkers.nullness.NullnessChecker \
    src/sandbox/FileToCheck.java

\end{Verbatim}

% TODO: show example -AprintErrorStack usage. Update text above to
% refer to it.

% $ javac -processor checkers.fenum.FenumChecker IdentityArrayList.java 
% error: GraphQualifierHierarchy found an unqualified type.  Please ensure that your implicit rules cover all cases and/or use a @DefaulQualifierInHierarchy annotation.
% 1 error

% $ javac -processor checkers.fenum.FenumChecker -AprintErrorStack IdentityArrayList.java 
%% error: GraphQualifierHierarchy found an unqualified type.  Please ensure that your implicit rules cover all cases and/or use a @DefaulQualifierInHierarchy annotation.
%%   checkers.util.GraphQualifierHierarchy.checkAnnoInGraph(GraphQualifierHierarchy.java:253)
%%   checkers.util.GraphQualifierHierarchy.isSubtype(GraphQualifierHierarchy.java:243)
%%   checkers.fenum.FenumChecker$FenumQualifierHierarchy.isSubtype(FenumChecker.java:129)
%%   checkers.types.QualifierHierarchy.isSubtype(QualifierHierarchy.java:78)
%%   checkers.types.TypeHierarchy.isSubtypeImpl(TypeHierarchy.java:122)
%%   checkers.types.TypeHierarchy.isSubtype(TypeHierarchy.java:67)
%%   checkers.basetype.BaseTypeChecker.isSubtype(BaseTypeChecker.java:323)
%%   checkers.basetype.BaseTypeVisitor.commonAssignmentCheck(BaseTypeVisitor.java:608)
%%   checkers.basetype.BaseTypeVisitor.checkTypeArguments(BaseTypeVisitor.java:680)
%%   checkers.basetype.BaseTypeVisitor.visitMethodInvocation(BaseTypeVisitor.java:299)
%%   checkers.basetype.BaseTypeVisitor.visitMethodInvocation(BaseTypeVisitor.java:1)
%%   com.sun.tools.javac.tree.JCTree$JCMethodInvocation.accept(JCTree.java:1351)
%%   com.sun.source.util.TreePathScanner.scan(TreePathScanner.java:67)
%%   checkers.basetype.BaseTypeVisitor.scan(BaseTypeVisitor.java:122)
%%   checkers.basetype.BaseTypeVisitor.scan(BaseTypeVisitor.java:1)
%%   com.sun.source.util.TreeScanner.visitExpressionStatement(TreeScanner.java:241)
%%   com.sun.tools.javac.tree.JCTree$JCExpressionStatement.accept(JCTree.java:1176)
%%   com.sun.source.util.TreePathScanner.scan(TreePathScanner.java:67)
%%   checkers.basetype.BaseTypeVisitor.scan(BaseTypeVisitor.java:122)
%%   checkers.basetype.BaseTypeVisitor.scan(BaseTypeVisitor.java:1)
%%   com.sun.source.util.TreeScanner.scan(TreeScanner.java:90)
%%   com.sun.source.util.TreeScanner.visitBlock(TreeScanner.java:160)
%%   com.sun.tools.javac.tree.JCTree$JCBlock.accept(JCTree.java:793)
%%   com.sun.source.util.TreePathScanner.scan(TreePathScanner.java:67)
%%   checkers.basetype.BaseTypeVisitor.scan(BaseTypeVisitor.java:122)
%%   checkers.basetype.BaseTypeVisitor.scan(BaseTypeVisitor.java:1)
%%   com.sun.source.util.TreeScanner.scanAndReduce(TreeScanner.java:80)
%%   com.sun.source.util.TreeScanner.visitMethod(TreeScanner.java:143)
%%   checkers.basetype.BaseTypeVisitor.visitMethod(BaseTypeVisitor.java:218)
%%   checkers.basetype.BaseTypeVisitor.visitMethod(BaseTypeVisitor.java:1)
%%   com.sun.tools.javac.tree.JCTree$JCMethodDecl.accept(JCTree.java:693)
%%   com.sun.source.util.TreePathScanner.scan(TreePathScanner.java:67)
%%   checkers.basetype.BaseTypeVisitor.scan(BaseTypeVisitor.java:122)
%%   checkers.basetype.BaseTypeVisitor.scan(BaseTypeVisitor.java:1)
%%   com.sun.source.util.TreeScanner.scanAndReduce(TreeScanner.java:80)
%%   com.sun.source.util.TreeScanner.scan(TreeScanner.java:90)
%%   com.sun.source.util.TreeScanner.scanAndReduce(TreeScanner.java:98)
%%   com.sun.source.util.TreeScanner.visitClass(TreeScanner.java:132)
%%   checkers.basetype.BaseTypeVisitor.visitClass(BaseTypeVisitor.java:158)
%%   checkers.basetype.BaseTypeVisitor.visitClass(BaseTypeVisitor.java:1)
%%   com.sun.tools.javac.tree.JCTree$JCClassDecl.accept(JCTree.java:617)
%%   com.sun.source.util.TreePathScanner.scan(TreePathScanner.java:49)
%%   checkers.source.SourceChecker.typeProcess(SourceChecker.java:337)
%%   com.sun.source.util.AbstractTypeProcessor$AttributionTaskListener.finished(AbstractTypeProcessor.java:211)
%%   com.sun.tools.javac.main.JavaCompiler.flow(JavaCompiler.java:1272)
%%   com.sun.tools.javac.main.JavaCompiler.flow(JavaCompiler.java:1231)
%%   com.sun.tools.javac.main.JavaCompiler.compile2(JavaCompiler.java:885)
%%   com.sun.tools.javac.main.JavaCompiler.compile(JavaCompiler.java:844)
%%   com.sun.tools.javac.main.Main.compile(Main.java:419)
%%   com.sun.tools.javac.main.Main.compile(Main.java:333)
%%   com.sun.tools.javac.main.Main.compile(Main.java:324)
%%   com.sun.tools.javac.Main.compile(Main.java:76)
%%   com.sun.tools.javac.Main.main(Main.java:61)
%% 1 error



%% Not relevant to most readers.  Can go in a README file in our repository.
% \section{Putting your checker in the repository\label{writing-repository}}
%
% This section is relevant only if you wish to add your checker to the source code
% repository for the Checker Framework --- for example, to include your
% checker in the Checker Framework distribution.
%
% The checkers appear in directory \code{annotations/checkers/} of
% the \code{annotations} repository.  It contains the following relevant
% subdirectories:
% \begin{itemize}
% \item
%   \code{manual/}: Documentation for your checker goes here.
% \item
%   \code{src/checkers/\emph{annotation\_name}/}: Code for the checker,
%   in a directory that is a sibling of \code{quals/}, \code{nonnull/},
%   etc.
% \item
%   \code{jdk/\emph{annotation\_name}/}: Annotated ``skeleton class''
%   versions of the JDK and other libraries (see Section~\ref{skeleton}).
% \item
%   \code{tests/\emph{annotation\_name}/}: Inputs and outputs for the test
%   suite for the checker.  A single top-level test suite class goes in
%   \code{tests/src/tests/}.
% \end{itemize}


\section{javac implementation survival guide\label{javac-tips}}

A checker built using the Checker Framework makes use of a few interfaces
from the underlying compiler (Oracle's OpenJDK javac).
This section describes those interfaces.




\subsection{Checker access to compiler information}

The compiler uses and exposes three hierarchies to model the Java
source code and classfiles.


\subsubsection{Types - Java Language Model API\label{javac-types}}

A \refModelclass{type}{TypeMirror} represents a Java type.
% Java declaration, statement, or expression.

There is a \code{TypeMirror} interface to represent each type kind,
e.g., \code{PrimitiveType} for primitive types, \code{ExecutableType}
for method types, and \code{NullType} for the type of the \code{null} literal.

\code{TypeMirror} does not represent annotated types though.  Checkers
should use the Checker Framework types API,
\refclass{types}{AnnotatedTypeMirror}, instead.  \code{AnnotatedTypeMirror}
parallels the \code{TypeMirror} API, but also present the type annotations
associated with the type.

The Checker Framework and the checkers use the types API extensively.


\subsubsection{Elements - Java Language Model API\label{javac-elements}}

An \refModelclass{element}{Element} represents a potentially-public
declaration that can be accessed from elsewhere:  classes, interfaces, methods, constructors, and
fields.  \<Element> represents elements found in both source
code and bytecode.

There is an \code{Element} interface to represent each construct, e.g.,
\code{TypeElement} for class/interfaces, \code{ExecutableElement} for
methods/constructors, \code{VariableElement} for local variables and
method parameters.

If you need to operate on the declaration level, always use elements rather
than trees
% in same subsection, which is the limit of the numbering.
% (Section~\ref{javac-trees})
(see below).  This allows the code to work on
both source and bytecode elements.

Example: retrieve declaration annotations, check variable
modifiers (e.g., \code{strictfp}, \code{synchronized})


\subsubsection{Trees - Compiler Tree API\label{javac-trees}}

A \refTreeclass{tree}{Tree} represents a syntactic unit in the source code,
like a method declaration, statement, block, \<for> loop, etc. Trees only
represent source code to be compiled (or found in \code{-sourcepath});
no tree is available for classes read from bytecode.

There is a Tree interface for each Java source structure, e.g.,
\code{ClassTree} for class declaration, \code{MethodInvocationTree}
for a method invocation, and \code{ForEachTree} for an enhanced-for-loop
statement.

You should limit your use of trees. Checkers use Trees mainly to
traverse the source code and retrieve the types/elements corresponding to
them.  Then, the checker performs any needed checks on the types/elements instead.


\subsubsection{Using the APIs}

The three APIs use some common idioms and conventions; knowing them will
help you to create your checker.

\emph{Type-checking}:
Do not use \code{instanceof} to determine the class of the object,
because you cannot necessarily predict the run-time type of the object that
implements an interface.  Instead, use the \code{getKind()} method.  The
method returns \refModelclass{type}{TypeKind},
\refModelclass{element}{ElementKind}, and \refTreeclass{tree}{Tree.Kind}
for the three interfaces, respectively.

\emph{Visitors and Scanners}:
The compiler and the Checker Framework use the visitor pattern
extensively. For example, visitors are used to traverse the source tree
(\refclass{basetype}{BaseTypeVisitor} extends
\refTreeclass{util}{TreePathScanner}) and for type
checking (\refclass{types}{TreeAnnotator} implements
\refTreeclass{tree}{TreeVisitor}).

\emph{Utility classes}:
Some useful methods appear in a utility class.  The Oracle convention is that
the utility class for a \code{Foo} hierarchy is \code{Foos} (e.g.,
\refModelclass{util}{Types}, \refModelclass{util}{Elements}, and
\refTreeclass{util}{Trees}).  The Checker Framework uses a common
\code{Utils} suffix instead (e.g., \refclass{util}{TypesUtils},
\refclass{util}{TreeUtils}, \refclass{util}{ElementUtils}), with one
notable exception: \refclass{util}{AnnotatedTypes}.


\subsection{How a checker fits in the compiler as an annotation processor}

The Checker Framework builds on the Annotation Processing API
introduced in Java 6.  A type annotation processor is one that extends
\refclass{source}{AbstractTypeProcessor}; these get run on each class
source file after the compiler confirms that the class is valid Java code.

The most important methods of \refclass{source}{AbstractTypeProcessor}
are \code{typeProcess} and \code{getSupportedSourceVersion}. The former
class is where you would insert any sort of method call to walk the AST\@,
and the latter just returns a constant indicating that we are targeting
version 8 of the compiler. Implementing these two methods should be enough
for a basic plugin; see the Javadoc for the class for other methods that
you may find useful later on.

The Checker Framework uses Oracle's Tree API to access a program's AST\@.
The Tree API is specific to the Oracle OpenJDK, so the Checker Framework only
works with the OpenJDK javac, not with Eclipse's compiler ecj or with
\ahref{http://gcc.gnu.org/java/}{gcj}.  This also limits the tightness of
the integration of the Checker Framework into other IDEs such as \ahref{http://www.jetbrains.com/idea/}{IntelliJ IDEA}\@.
An implementation-neutral API would be preferable.
In the future, the Checker Framework
can be migrated to use the Java Model AST of JSR 198 (Extension API for
Integrated Development Environments)~\cite{JSR198}, which gives access to
the source code of a method.  But, at present no tools
implement JSR~198.  Also see Section~\ref{ast-traversal}.



\subsubsection{Learning more about javac}

Sun's javac compiler interfaces can be daunting to a
newcomer, and its documentation is a bit sparse. The Checker Framework
aims to abstract a lot of these complexities.
You do not have to understand the implementation of javac to
build powerful and useful checkers.
Beyond this document, 
other useful resources include the Java Infrastructure
Developer's guide at
\url{http://wiki.netbeans.org/Java_DevelopersGuide} and the compiler
mailing list archives at
\url{http://news.gmane.org/gmane.comp.java.openjdk.compiler.devel}
(subscribe at
\url{http://mail.openjdk.java.net/mailman/listinfo/compiler-dev}).



% LocalWords:  plugin javac's SourceChecker AbstractProcessor getMessages quals
% LocalWords:  getSourceVisitor SourceVisitor getFactory AnnotatedTypeFactory
% LocalWords:  SupportedAnnotationTypes SupportedSourceVersion TreePathScanner
% LocalWords:  TreeScanner visitAssignment AssignmentTree AnnotatedClassTypes
% LocalWords:  SubtypeChecker SubtypeVisitor NonNull isSubtype getClass nonnull
% LocalWords:  AnnotatedClassType isAnnotatedWith hasAnnotationAt TODO src jdk
% LocalWords:  processor NullnessChecker InterningChecker Nullness Nullable igj
% LocalWords:  AnnotatedTypeMirrors BaseTypeChecker BaseTypeVisitor basetype
% LocalWords:  Aqual Anqual java CharSequence getAnnotatedType UseLovely IGJ
% LocalWords:  AnnotatedTypeMirror LovelyChecker Anomsgtext Ashowchecks enums
% LocalWords:  Afilenames dereferenced SuppressWarnings declaratively SubtypeOf
% LocalWords:  TypeQualifier TypeHierarchy GraphQualifierHierarchy ReadOnly Foo
% LocalWords:  QualifierHierarchy QualifierRoot createQualifierHierarchy util
% LocalWords:  createTypeHierarchy QReadOnly ImplicitFor treeClasses TypeMirror
% LocalWords:  LiteralTree ExpressionTree typeClasses annotateImplicit nullable
% LocalWords:  TypeQualifiers getSupportedTypeQualifiers FooChecker nullness
% LocalWords:  FooVisitor FooAnnotatedTypeFactory basicstyle InterningVisitor
% LocalWords:  InterningAnnotatedTypeFactory QualifierDefaults TypeKind getKind
% LocalWords:  setAbsoluteDefaults PolymorphicQualifier TreeVisitor subnodes
% LocalWords:  SimpleTreeVisitor TreePath instanceof subinterfaces TypeElement
% LocalWords:  ExecutableElement PackageElement DeclaredType VariableElement
% LocalWords:  TypeParameterElement ElementVisitor javax getElementUtils NoType
% LocalWords:  ProcessingEnvironment ExecutableType MethodTree ArrayType Warski
% LocalWords:  MethodInvocationTree PrimitiveType BlockTree TypeVisitor blog
% LocalWords:  AnnotatedTypeVisitor SimpleAnnotatedTypeVisitor html langtools
% LocalWords:  AnnotatedTypeScanner bootclasspath asType stringPatterns Foos
% LocalWords:  DefaultQualifierInHierarchy invocable wildcards novariant Utils
% LocalWords:  AggregateChecker getSupportedTypeCheckers Uninterned sourcepath
% LocalWords:  DefaultQualifier bytecode NullType strictfp ClassTree TypesUtils
% LocalWords:  ForEachTree ElementKind TreeAnnotator TreeUtils ElementUtils ecj
% LocalWords:  AnnotatedTypes AbstractTypeProcessor gcj hardcoding jsr
% LocalWords:  typeProcess getSupportedSourceVersion
