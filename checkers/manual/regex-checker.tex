\htmlhr
\chapter{Regex checker for regular expression syntax\label{regex-checker}}

The Regex Checker prevents, at compile-time, use of syntactically invalid
regular expressions.

A regular expression, or regex, is a pattern for matching certain strings
of text.  In Java, a programmer writes a regular expression as a string.
At run time, the string is ``compiled'' into an efficient internal form
(\sunjavadoc{java/util/regex/Pattern.html}{Pattern}) that is used for
text-matching.

The syntax of regular expressions is complex, so it is easy to make a
mistake.  It is also easy to accidentally use a regex feature from another
language that is not supported by Java (see section ``Comparison to Perl
5'' in the \sunjavadoc{java/util/regex/Pattern.html}{Pattern} Javadoc).
Ordinarily, the programmer does not learn of these errors until run time.
The Regex checker warns about these problems at compile time.

To run the Regex Checker, supply the \code{-processor
  checkers.regex.RegexChecker} command-line option to javac.


\section{Regex annotations\label{regex-annotations}}

These qualifiers make up the Regex type system:

\begin{description}

\item[\<@\refclass{regex/quals}{Regex}>]
  indicates valid regular expression \code{String}s.

\item[\<@\refclass{regex/quals}{PolyRegex}>]
  indicates qualifier polymorphism. For a description of
  \<@\refclass{regex/quals}{PolyRegex}>,
  see Section~\ref{qualifier-polymorphism}.

\end{description}

\section{Annotating your code with \code{@Regex}\label{annotating-with-regex}}

\subsection{Helpful methods when using the Regex Checker\label{regex-methods}}

The Regex Checker provides a few utility methods that may be useful when
using the Regex Checker. Figure~\ref{fig:regex-util-example} gives an
example of the \code{RegexUtil} methods.

\begin{description}

\item[\refmethod{regex}{RegexUtil}{isRegex}{(String)}]
  is used to determine if a String is a valid regular expression. This is
  most useful for user-provided regular expressions (such as ones passed
  on the command-line.) As described in Section~\ref{regex-implicit-qualifiers}
  this method is usually not needed for regular expressions embedded in
  code.

\item[\refmethod{regex}{RegexUtil}{regexError}{(String)}]
  is used to get a detailed error message if a String is not a valid
  regular expression. If the String is a valid regular expression then
  this method will return null.

\item[\refmethod{regex}{RegexUtil}{regexException}{(String)}]
  is used to get the
  \sunjavadoc{java/util/regex/PatternSyntaxException.html}{PatternSyntaxException}
  that \sunjavadoc{java/util/regex/Pattern.html#compile(java.lang.String)}{Pattern.compile(String)}
  throws when compiling an invalid regular expression. This will
  return null if the String is a valid regular expression.

\item[\refmethod{regex}{RegexUtil}{asRegex}{(String)}]
  returns the argument as a \code{@Regex String} if it is a valid
  regular expression, otherwise it throws an Error. This method is
  mainly a workaround until the Regex Checker supports
  flow-sensitivity (see Section~\ref{type-refinement}) and should
  be used rarely once the Regex Checker supports flow-sensitivity.

\end{description}

\begin{figure}
%BEGIN LATEX
\begin{smaller}
%END LATEX
\begin{Verbatim}
String regex = getRegexFromUser();
if (!RegexUtil.isRegex(regex)) {
   throw new RuntimeException("Error parsing regex.", RegexUtil.regexException(regex));
   // or System.out.println("Error parsing regex: " + RegexUtil.regexError(regex));
}
regex = RegexUtil.asRegex(regex); // Only necessary until the Regex Checker supports flow-sensitivity.
Pattern p = Pattern.compile(regex);
\end{Verbatim}
%BEGIN LATEX
\end{smaller}
%END LATEX
\caption{Example of the \code{RegexUtil} methods.}
\label{fig:regex-util-example}
\end{figure}

\subsection{Implicit qualifiers\label{regex-implicit-qualifiers}}

As described in Section~\ref{effective-qualifier}, the Regex checker adds
implicit qualifiers, reducing the number of annotations that must appear
in your code. The checker implicitly adds the \code{Regex} qualifier to
any \code{String} literal that is a valid regex and to the \code{null}
literal.

% LocalWords:  Regex regex quals
