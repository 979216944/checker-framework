\htmlhr
\chapter{Regex checker for regular expression syntax\label{regex-checker}}

The Regex Checker prevents, at compile-time, use of syntactically invalid
regular expressions.

A regular expression, or regex, is a pattern for matching certain strings
of text.  In Java, a programmer writes a regular expression as a string.
At run time, the string is ``compiled'' into an efficient internal form
(\sunjavadoc{java/util/regex/Pattern.html}{Pattern}) that is used for
text-matching.

The syntax of regular expressions is complex, so it is easy to make a
mistake.  It is also easy to accidentally use a regex feature from another
language that is not supported by Java (see section ``Comparison to Perl
5'' in the \sunjavadoc{java/util/regex/Pattern.html}{Pattern} Javadoc).
Ordinarily, the programmer does not learn of these errors until run time.
The Regex checker warns about these problems at compile time.

To run the Regex Checker, supply the \code{-processor
  checkers.regex.RegexChecker} command-line option to javac.


\section{Regex annotations\label{regex-annotations}}

These qualifiers make up the Regex type system:

\begin{description}

\item[\<@\refclass{regex/quals}{Regex}>]
  indicates valid regular expression \code{String}s.

\item[\<@\refclass{regex/quals}{PolyRegex}>]
  indicates qualifier polymorphism. For a description of
  \<@\refclass{regex/quals}{PolyRegex}>,
  see Section~\ref{qualifier-polymorphism}.

\end{description}

\section{Annotating your code with \code{@Regex}\label{annotating-with-regex}}

\subsection{Implicit qualifiers\label{regex-implicit-qualifiers}}

As described in Section~\ref{effective-qualifier}, the Regex checker adds
implicit qualifiers, reducing the number of annotations that must appear
in your code. The checker implicitly adds the \code{Regex} qualifier to
any \code{String} literal that is a valid regex and to the \code{null}
literal.

% LocalWords:  Regex regex quals
