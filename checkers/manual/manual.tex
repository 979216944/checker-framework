\documentclass[10pt]{report}
\usepackage{pslatex}
\usepackage{fullpage}
\usepackage{graphicx}
\usepackage{hevea}
\usepackage{listings}
\usepackage{url}
\usepackage{alltt}
% Not supported by Hevea, so don't bother: \usepackage{minitoc}

\usepackage{relsize}
% \def\codesize{\smaller}
\def\codesize{\relax}           % for "pslatex"
%HEVEA \def\codesize{\relax}
\newcommand{\code}[1]{\ifmmode{\mbox{\codesize\ttfamily{#1}}}\else{\codesize\ttfamily #1}\fi}
% This can't handle a URL with an embedded "#" -- at least at UW CSE
\newcommand{\myurl}[1]{{\codesize\url{#1}}}
%HEVEA \def\myurl{\url}
\def\<#1>{\code{#1}}

\usepackage{fancyvrb}
%BEGIN LATEX
\RecustomVerbatimEnvironment{Verbatim}{Verbatim}{fontsize=\codesize}
%END LATEX

%HEVEA \footerfalse    % Disable hevea advertisement in footer

\newcommand{\htmlhr}{\relax}
%HEVEA \renewcommand{\htmlhr}{\@hr{}{}}

% Problem with using "\newcommand" or "\renewcommand": Hevea writes this
% into manual.image.tex, and invokes LaTeX on it.  Sometimes running "make"
% leads to an error as a result, sometimes not.  I don't know the pattern
% of the failures, though running "make clean" and then "make" seems to
% work.  So maybe there's a problem with an auxiliary file.  Solve it by
% always defining \discretionary, so that \renewcommand works.
%HEVEA \def\discretionary{\relax}\renewcommand{\discretionary}[3]{\relax}

%HEVEA \newstyle{.lstframe}{margin:auto;margin-bottom:2em}

% At least 80% of every float page must be taken up by
% floats; there will be no page with more than 20% white space.
\def\topfraction{.8}
\def\dbltopfraction{\topfraction}
\def\floatpagefraction{\topfraction}     % default .5
\def\dblfloatpagefraction{\topfraction}  % default .5
\def\textfraction{.2}


% Left and right curly braces and backslash, in tt font
\newcommand{\ttlcb}{\texttt{\char "7B}}
\newcommand{\ttrcb}{\texttt{\char "7D}}
\newcommand{\ttbs}{\texttt{\char "5C}}


\title{The Checker Framework: \\ Custom pluggable types for Java}
\author{% MIT Program Analysis Group \\
\url{http://types.cs.washington.edu/checker-framework/}}
\newcommand{\ReleaseInfo}{1.0.6 (24 Feb 2010)}
\date{Version \ReleaseInfo{}}

\begin{document}

\maketitle

% %BEGIN LATEX
% \tableofcontents
% %END LATEX

%BEGIN LATEX
  %% Bring items closer together in list environments
  % Prevent infinite loops
  \let\Itemize =\itemize
  \let\Enumerate =\enumerate
  \let\Description =\description
  % Zero the vertical spacing parameters
  \def\Nospacing{\itemsep=0pt\topsep=0pt\partopsep=0pt\parskip=0pt\parsep=0pt}
  % Redefine the environments in terms of the original values
  \renewenvironment{itemize}{\Itemize\Nospacing}{\endlist}
  \renewenvironment{enumerate}{\Enumerate\Nospacing}{\endlist}
  \renewenvironment{description}{\Description\Nospacing}{\endlist}

  % Add line between figure and text
  \makeatletter
  \def\topfigrule{\kern3\p@ \hrule \kern -3.4\p@} % the \hrule is .4pt high
  \def\botfigrule{\kern-3\p@ \hrule \kern 2.6\p@} % the \hrule is .4pt high
  \def\dblfigrule{\kern3\p@ \hrule \kern -3.4\p@} % the \hrule is .4pt high
  \makeatother
%END LATEX


% Reference to Checker Framework Javadoc for a class (not a method, etc.).
% Arg 1: directory under checkers/, including internal "/", but no leading
% or trailing "/".
% Arg 2: class name.
% In the printed version, only the base class name appears.
% In the HTML version, it's a link to the Javadoc.
\newcommand{\refclass}[2]{\ahref{doc/checkers/#1/#2.html}{\<#2>}}
% Reference to Checker Framework Javadoc for a method or field.).
% Arg 1: directory under checkers/, including internal "/", but no leading
% or trailing "/".
% Arg 2: class name.
% Arg 3: method name.
% Arg 4: fully-qualified arguments.  Example: "(T)"
% In the printed version, only "class.method" appears.
% In the HTML version, it's a link to the Javadoc.
\newcommand{\refmethod}[4]{\ahref{doc/checkers/#1/#2.html\##3#4}{\<#2.#3>}}
% Reference to Sun Javadoc.
% Arg 1: .html reference, without the .../api/ prefix
% Arg 2: What will appear in the formatted manual.
% Problem:  the "?is-external=true" must appear before any "#".  But why is
% it necessary at all?
% \newcommand{\sunjavadoc}[2]{\ahref{http://java.sun.com/javase/6/docs/api/#1?is-external=true}{\<#2>}}
\newcommand{\sunjavadoc}[2]{\ahref{http://java.sun.com/javase/6/docs/api/#1}{\<#2>}}

\setcounter{page}{2}

\newcommand{\refwithpage}[1]{\ref{#1}, page~\pageref{#1}}
%HEVEA \renewcommand{\refwithpage}[1]{\ref{#1}}
\newcommand{\refwithpageparen}[1]{\ref{#1} (page~\pageref{#1})}
%HEVEA \renewcommand{\refwithpageparen}[1]{\ref{#1}}
\newcommand{\chapterpageref}[1]{Chapter~\refwithpage{#1}}

\noindent
\textbf{For the impatient:}
Section~\refwithpageparen{installation}
describes how to \textbf{install and use} pluggable type-checkers.

% %BEGIN LATEX
% \medskip
% %END LATEX
% 
% \noindent
% You can also jump directly to the documentation for a particular checker:
%  Nullness checker (\chapterpageref{nullness-checker}),
%  Interning checker (\chapterpageref{interning-checker}),
%  IGJ immutability checker (\chapterpageref{igj-checker}),
%  Javari immutability checker (\chapterpageref{javari-checker}),
%  Lock checker (\chapterpageref{lock-checker}),
%  Tainting checker (\chapterpageref{tainting-checker}),
%  Linear (single-use) checker (\chapterpageref{linear-checker}),
%  Regex checker (\chapterpageref{regex-checker}),
%  Internationalization checker (\chapterpageref{i18n-checker}),
%  Basic checker (\chapterpageref{basic-checker}),
%  Typestate checker (\chapterpageref{typestate-checker}),
%  Units and dimensions checker (\chapterpageref{units-checker}).
% % Keep this list in sync with the list at the top of introduction.tex.


%HEVEA \setcounter{tocdepth}{1}
% Not supported by Hevea, so don't bother: \dominitoc
\tableofcontents
\newpage

\section{Introduction\label{introduction}}

The Checker Framework enhances Java's type system to make it more powerful
and useful.
This lets software developers detect and 
prevent errors in their Java programs.

The Checker Framework comes with 4 checkers for specific types of errors:

\begin{enumerate}

\item
  \ahrefloc{nullness-checker}{Nullness checker} for null pointer errors
  (see Section~\ref{nullness-checker})
\item
  \ahrefloc{interning-checker}{Interning checker} for errors in equality
  testing and interning (see Section~\ref{interning-checker})
\item
  \ahrefloc{igj-checker}{IGJ checker} for mutation errors (incorrect
  side effects), based on the IGJ type system (see
  Section~\ref{igj-checker})
\item
  \ahrefloc{javari-checker}{Javari checker} for mutation errors
  (incorrect side effects), based on the Javari type system (see
  Section~\ref{javari-checker})

\end{enumerate}

\noindent
These checkers are easy to use and are invoked as arguments to \<javac>.


The Checker Framework also enables you to write new checkers of your
own; see Sections~\ref{basic-checker} and~\ref{writing-a-checker}.


\subsection{How it works:  Pluggable types}

The Checker Framework supports adding
pluggable type systems to the Java language in a backward-compatible way.
Java's built-in typechecker finds and prevents many errors --- but it
doesn't find and prevent \emph{enough} errors.  The Checker Framework lets you
run an additional typechecker as a plug-in to the javac compiler.  Your
code stays completely backward-compatible:  your code compiles with any
Java compiler, it runs on any JVM, and your coworkers don't have to use the
enhanced type system if they don't want to.  You can check only part of
your program, and type inference tools exist to help you annotate your
code.


A type system designer uses the Checker Framework to define type qualifiers
and their semantics, and a
compiler plug-in (a ``checker'') enforces the semantics.  Programmers can
write the type qualifiers in their programs and use the plug-in to detect
or prevent errors.  The Checker Framework is useful both to programmers who
wish to write error-free code, and to type system designers who wish to
evaluate and deploy their type systems.



% This manual is organized as follows.
% \begin{itemize}
% \item Section~\ref{introduction} overviews the Checker Framework and
%   describes how to \ahrefloc{installation}{install} it (Section~\ref{installation}).
% \item Section~\ref{using-a-checker} describes how to \ahrefloc{using-a-checker}{use a checker}.
% \item 
%   The next sections are user manuals for the \ahrefloc{nullness-checker}{Nullness}
%   (Section~\ref{nullness-checker}), \ahrefloc{interning-checker}{Interning}
%   (Section~\ref{interning-checker}), \ahrefloc{javari-checker}{Javari} (Section~\ref{javari-checker}),
%   \ahrefloc{igj-checker}{IGJ} (Section~\ref{igj-checker}), and \ahrefloc{basic-checker}{Basic}
%   (Section~\ref{basic-checker}) checkers.
% \item Section~\ref{annotating-libraries} describes an approach for \ahrefloc{annotating-libraries}{annotating external
% libraries}.
% \item Section~\ref{writing-a-checker} describes how to
%   \ahrefloc{writing-a-checker}{write a new checker} using the Checker Framework.
% \end{itemize}






This document uses the terms ``checker'', ``checker plugin'',
``type-checking compiler plugin'', and ``annotation processor'' as
synonyms.


\subsection{Installation\label{installation}}

This section describes how to install the binary release of the Checker
Framework.  The binary release contains everything that you need, both to
run checkers and to write your own checkers.  As an alternative, the source
release (Section~\ref{install-source}) is useful if you wish to examine or
modify the implementation of checkers or of the framework itself.

% Not "\ahrefurl" because it looks bad in the printed manual.
\textbf{Requirement:} 
You must have \textbf{JDK 6} or later installed.  You can get JDK 6 from 
\ahref{http://java.sun.com/javase/downloads/index.jsp}{Sun}
or elsewhere.  If you are using Apple Mac OS X, you can either use
\ahref{http://developer.apple.com/java/}{Apple's implementation} or
\ahref{http://landonf.bikemonkey.org/static/soylatte/}{SoyLatte}.

For Unix/Linux/MacOS installation instructions, see Section~\ref{unix-installation}.
For Windows installation instructions, see Section~\ref{windows-installation}.



%%% *****
%%% UPDATE
%%% *****

%%% Note that much of this section is duplicated with the "Windows
%%% installation" section.  That is better for users, even though it is
%%% longer and makes the maintainers keep two versions in sync.
\subsubsection{Unix/Linux/MacOS installation\label{unix-installation}}

These instructions assume that you use the bash or sh shell.  If you use a
different shell, you may need to slightly adjust the commands.

\begin{enumerate}

\item
  Download the latest Checker Framework distribution
  % (\ahrefurl{http://types.cs.washington.edu/checker-framework/current/jsr308-checkers.zip})
  and unzip it.  You can put it anywhere you like; a standard place is in a
  new directory named \code{jsr308}.

\begin{Verbatim}
  export JSR308=$HOME/jsr308
  mkdir ${JSR308}
  cd ${JSR308}
  wget http://types.cs.washington.edu/checker-framework/current/jsr308-checkers.zip
  unzip jsr308-checkers.zip
\end{Verbatim}

\item
  The download includes an updated version of the javac compiler, called
  the ``Type Annotations compiler'' or ``JSR 308 compiler'', that will be
  shipped with Java 7.  In order to use the updated compiler when you type
  \code{javac}, add the directory \code{.../checkers/binary} to your path.

  Place the following commands in your \code{.bashrc} file (and also execute
  it on the command line, or log out and back in):
\begin{Verbatim}
  export JSR308=$HOME/jsr308
  export PATH=$JSR308/checkers/binary:${PATH}
\end{Verbatim}

% It is not necessary to add checkers.jar to your classpath, because the
% shipped compiler already does so.
%   export CLASSPATH=$JSR308/checkers/checkers.jar:${CLASSPATH}


\item
  Verify that the installation works.  From the command line, run:

\begin{Verbatim}
  javac -version
\end{Verbatim}

The output should be:

\begin{Verbatim}
  javac 1.7.0-jsr308-0.9.6
\end{Verbatim}

\end{enumerate}

That's all there is to it!  Now you are ready to start using the checkers.

Section~\ref{example-use} walks you through a simple example.  More detailed
instructions for using a checker appear in Section~\ref{using-a-checker}.


\subsubsection{Windows installation\label{windows-installation}}

\begin{enumerate}

\item
  Download the latest Checker Framework distribution
  % (\ahrefurl{http://types.cs.washington.edu/checker-framework/current/jsr308-checkers.zip})
  and unzip it to create a \<checkers> directory.  You can put it anywhere
  you like; a standard place is in a new directory under \<C:\ttbs{}Program
  Files>.

\begin{enumerate}
\item
  Save the file
  \ahrefurl{http://types.cs.washington.edu/checker-framework/current/jsr308-checkers.zip}
  to your Desktop.
\item
  Double-click the \<jsr308-checkers.zip> file on your computer.  Click on
  the \<checkers> directory, then Select \<Extract all files>, and use
  \<C:\ttbs{}Program Files> as the destination.  You will obtain a new
  \<C:\ttbs{}Program Files\ttbs{}checkers> folder.
\end{enumerate}

\item
  The download includes an updated version of the javac compiler, called
  the ``Type Annotations compiler'' or ``JSR 308 compiler'', that will be
  shipped with Java 7.  In order to use the updated compiler when you type
  \code{javac}, add the directory \<C:\ttbs{}Program
  Files\ttbs{}checkers\ttbs{}binary> to your path variable.  Also set a
  CHECKERS variable.

% Instructions stolen from http://www.webreference.com/js/tips/020429.html

To set an environment variable, you have two options:  make the change
temporarily or permanently.
\begin{itemize}
\item
To make the change \textbf{temporarily}, type at the command shell prompt:

\begin{alltt}
path = \emph{newdir};%PATH%
\end{alltt}

For example:

\begin{Verbatim}
path = C:\Program Files\checkers\binary;%PATH%
set CHECKERS = C:\Program Files\checkers
\end{Verbatim}

This is a temporary change that endures until the window is closed, and you
must re-do it every time you start a new command shell.

\item
To make the change \textbf{permanently},
Right-click the \<My Computer> icon and
select \<Properties>. Select the \<Advanced> tab and click the
\<Environment Variables> button. In the \<System Variables> pane, select
\<Path> from the list and click \<Edit>. In the \<Edit System Variable>
dialog box, move the cursor to the beginning of the string in the
\<Variable Value> field and type the full directory name followed by a
semicolon (\<;>).

% This is for the benefit of the Ant task.
Similarly, set the CHECKERS variable.

This is a permanent change that only needs to be done once ever.
\end{itemize}


% It is not necessary to add checkers.jar to your classpath, because the
% shipped compiler already does so.
%   export CLASSPATH=$JSR308/checkers/checkers.jar:${CLASSPATH}

\item
  Verify that the installation works.  From the command line, run:

\begin{Verbatim}
  javac -version
\end{Verbatim}

The output should be:

\begin{Verbatim}
  javac 1.7.0-jsr308-0.9.6
\end{Verbatim}

\end{enumerate}

That's all there is to it!  Now you are ready to start using the checkers.

Section~\ref{example-use} walks you through a simple example.  More detailed
instructions for using a checker appear in Section~\ref{using-a-checker}.



\subsection{Example use:  detecting a null pointer bug\label{example-use}}

To run a checker on a source file, just run javac as usual, passing the
\<-processor> flag.  For instance, if you usually run the compiler like
this:

\begin{Verbatim}
  javac Foo.java Bar.java
\end{Verbatim}

\noindent
then you will instead run it like this (where \<javac> is the JSR 308
compiler that is distributed with the Checker Framework):

\begin{alltt}
  javac -processor \textit{ProcessorName} Foo.java Bar.java
\end{alltt}

\noindent
(If you usually do your coding within an IDE, you will need to configure
the IDE to use the correct version of javac and to pass the command-line
argument.  See your IDE documentation for details.)

\begin{enumerate}
\item
  Let's consider this very simple Java class.  One local variable is
  annotated as \<NonNull>, indicating that \<ref> must be a reference to a
  non-null object.  Save the file as \<GetStarted.java>.

\begin{Verbatim}
import checkers.nullness.quals.*;

public class GetStarted {
    void sample() {
        @NonNull Object ref = new Object();
    }
}
\end{Verbatim}

\item
  Run the nullness checker on the class.  Either run this from the command line:

\begin{Verbatim}
  javac -processor checkers.nullness.NullnessChecker GetStarted.java
\end{Verbatim}

\noindent
or compile from within your IDE, which you have customized to use the JSR
308 compiler and to pass the extra arguments.

  The compilation should complete without any errors.

\item
  Let's introduce an error now.  Modify \<ref>'s assignment to:
\begin{Verbatim}
  @NonNull Object ref = null;
\end{Verbatim}

\item
  Run the nullness checker again, just as before.  This run should emit
  the following error:
\begin{Verbatim}
GetStarted.java:5: incompatible types.
found   : @Nullable <nulltype>
required: @NonNull Object
		@NonNull Object ref = null;
		                      ^
1 error
\end{Verbatim}

The type qualifiers (e.g. \<@NonNull>) are permitted anywhere
that would write a type, including generics and casts; see
Section~\ref{writing-annotations}.

\begin{alltt}
  \underline{@Interned} String intern() \ttlcb{} ... \ttrcb{}             // return value
  int compareTo(\underline{@NonNull} String other) \ttlcb{} ... \ttrcb{}  // parameter
  \underline{@NonNull} List<\underline{@Interned} String> messages;     // non-null list of interned Strings
\end{alltt}

\end{enumerate}


\htmlhr
\section{Using a checker\label{using-a-checker}}

Finding bugs with a checker plugin is a two-step process:

\begin{enumerate}

\item The programmer writes annotations, such as \code{@\refclass{nullness/quals}{NonNull}} and
  \code{@\refclass{interning/quals}{Interned}}, that specify additional information about Java types.
  (Or, the programmer uses an inference tool to automatically insert
  annotations in his code:  see Sections~\ref{nullness-inference} and~\ref{javari-inference}.)
  It is possible to annotate only part of your code:  see
  Section~\ref{unannotated-code}.

\item The checker reports whether the program contains any erroneous code
  --- that is, code that is inconsistent with the annotations.

\end{enumerate}



% The annotations have to be on your classpath even when you are not using
% the -processor, because of the existence of the import statement for
% the annotations.


\subsection{Writing annotations\label{writing-annotations}}

The syntax of type qualifier annotations in Java 7 is specified by
\ahref{http://types.cs.washington.edu/jsr308/}{JSR 308}~\cite{jsr308}.  Ordinary
Java permits annotations on declarations.  JSR 308 permits annotations
anywhere that you would write a type, including generics and casts.  You
can also write annotations to indicate type qualifiers for array levels and
receivers.  Here are a few examples:

\begin{alltt}
  \underline{@Interned} String intern() \ttlcb{} ... \ttrcb{}             // return value
  int compareTo(\underline{@NonNull} String other) \ttlcb{} ... \ttrcb{}  // parameter
  String toString() \underline{@ReadOnly} \ttlcb{} ... \ttrcb{}           // receiver ("this" parameter)
  \underline{@NonNull} List<\underline{@Interned} String> messages;     // generics:  non-null list of interned Strings
  \underline{@Interned} String \underline{@NonNull} [] messages;        // arrays:  non-null array of interned Strings
  myDate = (\underline{@ReadOnly} Date) readonlyObject;     // cast
\end{alltt}

You can also write the annotations within comments, as in
\code{List</*@NonNull*/ String>}.  The Type Annotations compiler, which is
distributed with the Checker Framework, will still process
the annotations.
However, your code will remain compilable by people who are not using the JSR
308 or Java 7 compiler.  For more details, see
Section~\ref{annotations-in-comments}.



\subsubsection{Distributing your annotated project\label{distributing}}

If your code contains any annotations (outside of comments, see Section~\ref{annotations-in-comments}), or any import
statements for the annotations, then your code has a dependency on the
annotation declarations.  You also will need to provide the annotation
declarations as well, if you decide to distribute your project.

For your convenience, inside the the checkers distribution \code{.zip} file
is a jar file,
\code{checkers-quals.jar}, that only contains the distributed qualifiers.
You may include the jar file in your distribution.

Your clients need to have the annotations jar in the classpath when
compiling your project.  When running it though, they most likely
don't require the annotations declarations (unless the annotation
classes are loaded via reflection, which would be unusual).


\subsection{Running a checker\label{running}}

To run a checker plugin, run the compiler \code{javac} as usual,
but pass the \code{-processor \emph{plugin\_class}} command-line
option.
(You might run the compiler from the command line as shown below, or your
IDE might run the javac command on your behalf, in which case see the IDE
documentation to learn how to customize it.)
Remember that you must be using the
Type Annotations version of \<javac>, which you already installed (see Section~\ref{installation}).

Two concrete examples (using the Nullness checker) are:

%BEGIN LATEX
\begin{smaller}
%END LATEX
\begin{Verbatim}
  javac -processor checkers.nullness.NullnessChecker MyFile.java
  javac -processor checkers.nullness.NullnessChecker -sourcepath checkers/jdk/nullness/src MyFile.java
\end{Verbatim}
%BEGIN LATEX
\end{smaller}
%END LATEX

\noindent
For a discussion of the \code{-sourcepath} argument, see
Section~\ref{skeleton-using}.

The checker is run only on the Java files specified on the command line (or
created by another annotation processor).
The checker does not analyze other classes (e.g., pre-compiled classes, or
classes whose source
code is available on the classpath), but it does check
the \emph{uses} of those classes in the source code being compiled.

The javac compiler halts compilation as soon as an error is found in a
source file.  You can pass \code{-Awarns} in the command-line to
treat checker errors as warnings.  This option allows you to see all
the type-checking errors at once, rather than just the errors in the first
file that contains errors.

You can always compile the code without the \code{-processor}
command-line option, but in that case no checking of the type
annotations is performed.

\subsubsection{Checker auto-discovery}

``Auto-discovery'' makes the \code{javac} compiler always run a checker
plugin, even if you do not explicitly pass the \code{-processor}
command-line option.  This can make your command line shorter, and ensures
that your code is checked even if you forget the command-line option.

To enable auto-discovery, place a configuration file named
\code{META-INF/services/javax.annotation.processing.Processor}
in your classpath.  The file contains the names of the checker plugins to
be used, listed one per line.  For instance, to run the Nullness and the
Interning checkers automatically, the configuration file should contain:

%BEGIN LATEX
\begin{smaller}
%END LATEX
\begin{Verbatim}
  checkers.nullness.NullnessChecker
  checkers.interning.InterningChecker
\end{Verbatim}
%BEGIN LATEX
\end{smaller}
%END LATEX

You can disable this auto-discovery mechanism by passing the
\code{-proc:none} command-line option to \<javac>.

%% Auto-discovering all the distributed checkers by default would be
%% problematic.  So, leave it up to the user to enable auto-discovery.
%%  1. We don't want to auto-discover both the Javari & IGJ type checkers,
%%     as then the user would see multiple, possibly contradictory, types
%%     of mutability diagnostics.
%%  2. The nullness and mutability checkers would issue lots of errors for
%%     unannotated code, and that would be irritating.



\subsubsection{Ant task\label{ant-task}}

If you use the \ahref{http://ant.apache.org/}{Ant} build tool to compile
your software, then you can add an Ant task that runs a checker.  We assume
that your Ant file already contains a compilation target that uses the
\code{javac} task.

First, set the \code{jsr308javac} property:

%BEGIN LATEX
\begin{smaller}
%END LATEX
\begin{Verbatim}
  <!-- Boilerplate to set jsr308javac property. Is there a better way? -->
  <property environment="env"/>
  <condition property="isUnix">
    <os family="unix" />
  </condition>
  <condition property="isWindows">
    <os family="windows" />
  </condition>
  <target name="init-jsr308javac-unix" if="isUnix">
    <property name="jsr308javac" value="${env.CHECKERS}/binary/javac" />
  </target>
  <target name="init-jsr308javac-windows" if="isWindows">
    <property name="jsr308javac" value="${env.CHECKERS}/binary/javac.bat" />
  </target>
\end{Verbatim}
%BEGIN LATEX
\end{smaller}
%END LATEX

\noindent
The \code{property} target makes environment variables (such as your home
directory) available to Ant.

Next, duplicate the compilation target, then modify it slightly as
indicated in this example, filling in each ellipsis (\ldots) from the
original compilation target:

%BEGIN LATEX
\begin{smaller}
%END LATEX
\begin{Verbatim}
  <target name="check-nullness"
          description="Check for nullness errors."
          depends="clean,...,init-jsr308javac-unix,init-jsr308javac-windows">
    <javac ...
           fork="yes"
           executable="${jsr308javac}">
      <compilerarg value="-version"/>
      <compilerarg line="-target 5"/>
      <compilerarg line="-processor checkers.nullness.NullnessChecker"/>
      <compilerarg line="-sourcepath ${env.CHECKERS}/jdk/nullness/src"/>
      <compilerarg value="-implicit:class"/>
      <classpath>
        <pathelement location="${env.annotations}/checkers/checkers.jar"/>
        ...
      </classpath>
      ...
    </javac>
  </target>
\end{Verbatim}
%BEGIN LATEX
\end{smaller}
%END LATEX

In the example, the target is named \code{check-nullness}, but you can
name it whatever you like.

The target assumes the existence of a \code{clean} target that removes all
\code{.class} files.  That is necessary because Ant's \code{javac} target
doesn't re-compile \code{.java} files for which a \code{.class} file
already exists.

The \code{executable} and \code{fork} fields of the \code{javac} task
ensure that an external javac program is called.  Otherwise, Ant will run
javac via a Java method call, and there is no guarantee that it will get
the JSR 308 version that is distributed with the Checker Framework.

The \code{-version} compiler argument is just for debugging; you may omit
it.

The \code{-target 5} compiler argument is optional, if you use Java 5 in
ordinary compilation when not performing pluggable type-checking.

The \code{-processor ...} compiler argument indicates which checker to
run.  You can supply additional arguments to the checker as well.

The \code{-implicit:class} compiler argument causes annotation processing
to be performed on implicitly compiled files.  (An implicitly compiled file
is one that was not specified on the command line, but for which the source
code is newer than the \code{.class} file.)  This is the default, but
supplying the argument explicitly suppresses a compiler warning.


\subsubsection{Maven plugin\label{maven-task}}

Adam Warski has written a Maven2 plugin that runs a checker.
The plugin is available at 
\myurl{http://www.warski.org/checkersplugin.html}.


\subsubsection{Intellij IDEA\label{intellij}}

IntelliJ IDEA (Maia release)
\ahref{http://blogs.jetbrains.com/idea/2009/07/type-annotations-jsr-308-support/}{supports}
the Type Annotations (JSR-308) syntax.

http://blogs.jetbrains.com/idea/2009/07/type-annotations-jsr-308-support/

\subsubsection{Eclipse\label{eclipse}}

There are two ways to run a checker from within the Eclipse IDE:  via Ant
or using an Eclipse plug-in.


\paragraph{Using an Ant task}

Add an Ant target as described in Section~\ref{ant-task}.  You can
run the Ant target by executing the following steps
(instructions copied from
\myurl{http://www.eclipse.org/documentation/?topic=/org.eclipse.platform.doc.user/gettingStarted/qs-84_run_ant.htm}):

\begin{enumerate}

\item
  Select \code{build.xml} in one of the navigation views and choose
  {\bf Run As $>$ Ant Build...} from its context menu.

\item
  A launch configuration dialog is opened on a launch configuration
  for this Ant buildfile.

\item
  In the {\bf Targets} tab, select the new ant task (e.g., check-interning).

\item
  Click {\bf Run}.

\item
  The Ant buildfile is run, and the output is sent to the Console view.

\end{enumerate}

\paragraph{Eclipse plug-in}

A prototype Eclipse plug-in for running a checker is available at
\myurl{http://types.cs.washington.edu/checker-framework/eclipse/}.  
The website contains instructions for installing and using the plug-in.
The plug-in is
experimental now, but some people have used it successfully (and we have fixed
all bugs that have been reported so far).


\subsubsection{tIDE}

tIDE, an open-source Java IDE, supports the Checker Framework.  See its
documentation at \myurl{http://tide.olympe-network.com/}.


\subsection{What the checker guarantees\label{checker-guarantees}}

A checker can guarantee that a particular property holds throughout the
code.  For example, the Nullness checker (Section~\ref{nullness-checker})
guarantees that every expression whose type is a \code{@\refclass{nullness/quals}{NonNull}} type never
evaluates to null.  The Interning checker (Section~\ref{interning-checker})
guarantees that every expression whose type is an \code{@\refclass{interning/quals}{Interned}} type
evaluates to an interned value.  The checker makes its guarantee by
examining every part of your program and verifying that no part of the
program violates the guarantee.

There are some limitations to the guarantee.

\begin{itemize}

\item
  Native methods and reflection can behave in a manner that is impossible
  for a compiler plugin to check.  Such constructs may violate the
  property being checked.  Similarly, deserialization and cloning can
  create objects that could not result from normal constructor calls, and
  that therefore may violate the property being checked.

\item 
  A compiler plugin can check only those parts of your program that you run
  it on. If you compile some parts of your program without the
  \code{-processor} switch or with the \code{-AskipClasses} property
  (in other words, without running the checker), or if you use the
  \code{@SuppressWarnings} annotation to suppress some errors or warnings,
  then there is no guarantee that the entire program satisfies the property
  being checked.  An analogous situation is using an external library that
  was compiled without being checked by the compiler plugin.

\item 
  Your code should pass the Java compiler without errors or warnings.  In
  particular, your code should use generic types, with no uses of raw types.
  Misuse of generics, including casting away generic types, can cause other
  errors to be missed.

\item
  \urldef{\jlsintersectiontypesurl}{\url}{http://java.sun.com/docs/books/jls/third_edition/html/typesValues.html#4.9}
  The Checker Framework does not yet support annotations on intersection
  types (see
  \ahref{\jlsintersectiontypesurl}{JLS \S4.9}).  As a result, checkers cannot provide guarantees about
  intersection types.

\item
  Specific checkers may have other limitations; see their documentation for
  details.

\end{itemize}

A checker can be useful in finding bugs or in verifying part of a
program, even if the checker is unable to verify the correctness of an
entire program.

If you find that a checker fails to issue a warning that it
should, then please report a bug (see Section~\ref{reporting-bugs}).


\subsection{Tips about writing annotations\label{tips-about-writing-annotations}}

\subsubsection{Annotations indicate normal behavior\label{annotate-normal-behavior}}

You should use annotations to indicate \emph{normal} behavior.  The
annotation indicate all the values that you \emph{want} to flow to
reference --- not every value that might possibly flow there if your
program has a bug.

Many methods are guaranteed to throw an exception if they are passed \code{null}
as an argument.  Examples include

\begin{Verbatim}
  java.lang.Double.valueOf(String)
  java.lang.String.contains(CharSequence)
  org.junit.Assert.assertNotNull(Object)
  com.google.common.base.Preconditions.checkNotNull(Object)
\end{Verbatim}

\code{@Nullable} might seem like a reasonable annotation for the parameter,
for two reasons.  First, \code{null} is a legal argument with a
well-defined semantics:  throw an exception.  Second, \code{@Nullable}
describes a possible program execution:  it might be possible for
\code{null} to flow there, if your program has a bug.

% (Checking for such a bug is the whole purpose of the \code{assertNotNull}
% and \code{checkNotNull} methods.)

However, it is never useful for a programmer to pass \code{null}.  It is
the programmer's intention that \code{null} never flows there.  If
\code{null} does flow there, the program will not continue normally.

Therefore, you should mark such parameters as \code{@NonNull}, indicating
the intended use of the method.  When you use the \code{@NonNull}
annotation, the checker is able to issue compile-time warnings about
possible run-time exceptions, which is its purpose.  Marking the parameter
as \code{@Nullable} would suppress such warnings, which is undesirable.

% (The note at
% http://google-collections.googlecode.com/svn/trunk/javadoc/com/google/common/base/Preconditions.html
% argues that the parameter could be marked as @Nullable, since it is
% possible for null to flow there at run time.  However, since that is an
% erroneous case, the annotation would be counterproductive rather than
% useful.)


\subsubsection{Subclasses must respect superclass annotations\label{annotations-are-a-contract}}

An annotation indicates a guarantee that a client can depend upon.  A subclass
is not permitted to \emph{weaken} the contract; for example,
if a method accepts \code{null} as an argument, then every overriding
definition must also accept \code{null}.
A subclass is permitted to \emph{strengthen} the contract; for example,
if a method does \emph{not} accept \code{null} as an argument, then an 
overriding definition is permitted to accept \code{null}.

As a bad example, consider an erroneous \code{@Nullable} annotation at
line 141 of \ahref{http://pauillac.inria.fr/~maranget/hevea/doc/manual018.html#toc22}{\code{com/google/common/collect/Multiset.java}}, version r78:

\begin{Verbatim}
101  public interface Multiset<E> extends Collection<E> {
...
122    /**
123     * Adds a number of occurrences of an element to this multiset.
...
129     * @param element the element to add occurrences of; may be {@code null} only
130     *     if explicitly allowed by the implementation
...
137     * @throws NullPointerException if {@code element} is null and this
138     *     implementation does not permit null elements. Note that if {@code
139     *     occurrences} is zero, the implementation may opt to return normally.
140     */
141    int add(@Nullable E element, int occurrences);
\end{Verbatim}

There exist implementations of Multiset that permit \code{null} elements,
and implementations of Multiset that do not permit \code{null} elements.  A
client with a variable \code{Multiset ms} does not know which variety of
Multiset \code{ms} refers to.  However, the \code{@Nullable} annotation
promises that \code{ms.add(null, 1)} is permissible.  (Recall from
Section~\ref{annotate-normal-behavior} that annotations should indicate
normal behavior.)

If parameter \code{element} on line 141 were to be annotated, the correct
annotation would be \code{@NonNull}.  Suppose a client has a reference to
same Multiset \code{ms}.  The only way the clienc can be sure not to throw an exception is to pass
only non-\code{null} elements to \code{ms.add()}.  A particular class
that implements Multiset could declare \code{add} to take a
\code{@Nullable} parameter.  That still satisfies the original contract.
It strengthens the contract by promising even more:  a client with such a
reference can pass any non-\code{null} value to \code{add()}, and may also
pass \code{null}.

\textbf{However}, the best annotation for line 141 is no annotation at all.
The reason is that each implementation of the Multiset interface should
specify its own nullness properties when it specifies the type parameter
for Multiset.  For example, two clients could be written as

\begin{Verbatim}
  class MyNullPermittingMultiset implements Multiset<@Nullable Object> { ... }
  class MyNullProhibitingMultiset implements Multiset<@NonNull Object> { ... }
\end{Verbatim}

\noindent
or, more generally, as

\begin{Verbatim}
  class MyNullPermittingMultiset<E extends @Nullable Object> implements Multiset<E> { ... }
  class MyNullProhibitingMultiset<E extends @NonNull Object> implements Multiset<E> { ... }
\end{Verbatim}

Then, the specification is more informative, and the Checker Framework is
able to do more precise checking, than if line 141 has an annotation.

It is a pleasant feature of the Checker Framework that in many cases, no
annotations at all are needed on type parameters such as \code{E} in MultiSet.


\subsubsection{When to use (and not use) type qualifiers\label{when-to-use-type-qualifiers}}

For some programming tasks, you can use either a Java subclass or a type
qualifier.  For instance, suppose that your code currently uses
\code{String} to represent an address.  You could create a new \code{Address}
class and refactor your code to use it, or you could create a
\code{@Address} annotation and apply it to some uses of \code{String} in
your code.  If both of these are truly possible, then it is probably more
foolproof to use the Java class.  We do not encourage you to use type
qualifiers as a poor substitute for classes.  However, sometimes type
qualifiers are a better choice.

Using a new class may your code incompatible with existing libraries or
clients.  Brian Goetz expands on this issues in an article on the
pseudo-typedef antipattern~\cite{Goetz2006:typedef}.  Even if compatibility
is not a concern, a code change may introduce bugs, whereas adding
annotations does not change the run-time behavior.  It is possible to add
annotations to existing code, including code you do not maintain or cannot
change.  It is possible to annotate primitive types without converting them
to wrappers, which would make the code both uglier and slower.

Type qualifiers can be applied to any type, including final classes that
cannot be subclassed.

Type qualifiers permit you to remove operations, with a compile-time
guarantee.  An example is mutating methods that are forbidden by immutable
types (see Sections~\ref{igj-checker} and~\ref{igj-checker}).  More
generally, type qualifiers permit creating a new supertype, not just a
subtype, of an existing Java type.

% This is the least important reason.
A final reason is efficiency.  Type qualifiers can be more
efficient, since there is no no run-time representation such as a wrapper
or a separate class, nor introduction of dynamic dispatch for methods that
could otherwise be statically dispatched.


\subsubsection{Annotations on constructor invocations\label{annotations-on-constructor-invocations}}

%% I want to get rid of this syntax.

In the checkers distributed with the Checker Framework, an annotation on a
constructor invocation is equivalent to a cast on a constructor result.
That is, the following two expressions have identical semantics:  one is
just shorthand for the other.

\begin{Verbatim}
  new @ReadOnly Date()
  (@ReadOnly Date) new Date()
\end{Verbatim}

However, you should rarely have to use this.  The Checker Framework will
determine the qualifier on the result, based on the ``return value''
annotation on the constructor definition.  The ``return value'' annotation
appears before the constructor name, for example:

\begin{Verbatim}
  class MyClass {
    @ReadOnly MyClass() { ... }
  }
\end{Verbatim}

In general, you should only use this syntax when you know that the cast is
guaranteed to succeed.  An example from the IGJ checker
(section~\ref{igj-checker}) is \<new @Immutable MyClass()> or \<new
@Mutable MyClass()>, where you know that every other reference to the class
is annotated \<@Readonly>.



\section{Handling warnings and legacy code\label{warnings-and-legacy}}


\subsection{Checking partially-annotated programs:  handling unannotated code\label{unannotated-code}}

Sometimes, you wish to type-check only part of your program.  
You might focus on the most mission-critical or error-prone part of your
code.  When you start to use a checker, you may not wish to annotate
your entire program right away.  You may not have source code (or
enough knowledge to annotate) the libraries that your program uses.

If annotated code uses unannotated code, then the checker may issue
warnings.  For example, the Nullness checker (Section~\ref{nullness-checker}) will
warn whenever an unannotated method result is used in a non-null context:

\begin{Verbatim}
  @NonNull myvar = unannotated_method();   // WARNING: unannotated_method may return a null value
\end{Verbatim}

If the call can return null, you should fix the bug in your program by
removing the \code{@\refclass{nullness/quals}{NonNull}} annotation in your own program.

If the library call never returns null,
there are several ways to eliminate the compiler warnings.
\begin{enumerate}
\item Annotate \code{unannotated\_method} in full.  This approach provides the
  the strongest guarantees, but may require you to annotate additional
  methods that \code{unannotated\_method} calls.  See
  Section~\ref{annotating-libraries} for a discussion of how to annotate
  libraries for which you have no source code.
\item Annotate only the signature of \code{unannotated\_method}, and
  suppress warnings in its body.  Two ways to suppress the warnings are via a
  \code{@SuppressWarnings} annotation or by not running the checker on that
  file (see Section~\ref{suppressing-warnings}).
\item Suppress all warnings related to uses of \code{unannotated\_method}
  via the \code{skipClasses} processor option
  (see Section~\ref{suppressing-warnings}).
  Since this can suppress more warnings than you may expect,
  it is usually better to annotate at least the method's signature.  If you
  choose the boundary between the annotated and unannotated code wisely,
  then you only have to annotate the signatures of a few classes/methods
  (e.g., the public interface to a library or package).
  
\end{enumerate}

Section~\ref{annotating-libraries} discusses adding annotations to
signatures when you do not have source code available.
Section~\ref{suppressing-warnings} discusses suppressing warnings.


If you annotate additional libraries, please share them with us so that we
can distribute the annotations with the Checker Framework; see
Section~\ref{reporting-bugs}.


\subsection{Suppressing warnings\label{suppressing-warnings}}

You may wish to suppress checker warnings because of unannotated libraries
or un-annotated portions of your own code, because of application
invariants that are beyond the capabilities of the type system, because of
checker limitations, because you are interested in only some of the
guarantees provided by a checker, or for other reasons.  You can suppress
warnings via
\begin{itemize}
\item
  the \code{@SuppressWarnings} annotation,
\item
  the \code{-AskipClasses} command-line option,
\item
  the javac \code{-Alint} command-line option, or
\item
  not using the \code{-processor} switch to \code{javac}.
\end{itemize}

You can suppress specific errors and warnings by use of the
\code{@SuppressWarnings("\emph{annotationname}")} annotation, for example
\code{@SuppressWarnings("interning")}.
This may be placed on program elements such as a class, method, or local
variable declaration.  It is good practice to suppress warnings in the
smallest possible scope.  For example, if a particular expression causes a
false positive warning, you should extract that expression into a local variable
and place a \code{@SuppressWarnings} annotation on the variable
declaration.
As another example, if you have annotated the signatures but not the bodies
of the methods in a class or package, put a \code{@SuppressWarnings}
annotation on the class declaration or on the package's
\code{package-info.java} file.

You can suppress all errors and warnings at all uses of a given class.
Set the \code{-AskipClasses} command-line option to a
regular expression that matches classes for which warnings and errors
should be suppressed.  For example, if you use
``{\codesize\verb|-AskipClasses=^java\.|}'' on the command line
(with appropriate quoting) when invoking
\code{javac}, then the checkers will suppress all warnings within those
classes, all warnings relating to invalid arguments, and all warnings
relating to incorrect use of the return value.

You can suppress an entire class of warnings via javac's \code{-Alint}
command-line option.  The \code{-Alint} option uses the same syntax as
javac's \code{-Xlint} option.
Following \code{-Alint=}, write a list of option
names.  If the option name is preceded by a hyphen (\code{-}), that
disables the option; otherwise it enables it.  For example:
\code{-Alint=-dotequals} causes the Interning checker
(Section~\ref{interning-checker}) not to output advice about when \code{a.equals(b)}
could be replaced by \code{a==b}.

You can also compile parts of your code without use of the
\code{-processor} switch to \code{javac}.  No checking is done during
such compilations.

Finally, some checkers have special rules.  For example, the Nullness
checker (Section~\ref{nullness-checker}) uses \code{assert} statements that contain
null checks to suppress warnings
(Section~\ref{suppressing-warnings-with-assertions}).



\subsection{Writing annotations in comments for backward compatibility\label{annotations-in-comments}}

Sometimes, your code needs to be compilable by people who are not
using the Type Annotations or Java 7 compiler.


\subsubsection{Annotations in comments}

A Java 4 compiler does not permit use of
annotations, and a Java 5 compiler only permits annotations on
declarations (but not on generic arguments, casts, method receiver, etc.).

For compatibility with all Java versions, you may write any annotation inside a
\code{/*}\ldots\code{*/} Java comment, as in \code{List</*@NonNull*/ String>}.
The Type Annotations compiler treats the code exactly as if you had not written the
\code{/*} and \code{*/}.
In other words, the Type Annotations compiler will recognize the
annotation, but your code will still compile with any other Java compiler.

(\textbf{Note:} This is a feature of the Type Annotations compiler that is
distributed along with the Checker Framework.  It is not supported by the
mainline OpenJDK compiler, which will ignore annotations written in
comments.  This is the only difference between the Type Annotations
compiler and the OpenJDK compiler.)
%   For more details
% about the differences, see file \code{README-jsr308.html} in the Type
% Annotations distribution.

In a single program, you may write some annotations in comments, and others
without comments.

By default, the compiler ignores any comment that contains spaces at the
beginning or end, or between the \code{@} and the annotation name.  
In other words, it reads \code{/*@NonNull*/} as an annotation but ignores
\code{/* @NonNull*/} or \code{/*@ NonNull*/} or \code{/*@NonNull */}.
This
feature enables backward compatibility with code that contains comments
that start with \code{@} but are not annotations.  (The
ESC/Java~\cite{FlanaganLLNSS02}, JML~\cite{LeavensBR2006:JML}, and
Splint~\cite{Evans96} tools all use ``\code{/*@}'' or ``\code{/*~@}'' as a
comment marker.)
Compiler flag
\code{-XDTA:spacesincomments} causes the compiler to parse annotation comments
even when they contain spaces.  You may need to use
\code{-XDTA:spacesincomments} if you use Eclipse's ``Source $>$ Correct
Indentation'' command, since it inserts space in comments.  But the
annotation comments are less readable with spaces, so you may wish to disable
inserting spaces:  in the Formatter preferences, in the Comments tab,
unselect the ``enable block comment formatting'' checkbox.


\subsubsection{Import statements\label{implicit-import-statements}}

When writing source code with annotations, it is more convenient to write a
short form such as \code{@NonNull} instead of
\code{@checkers.nullness.quals.NonNull}.  There are two ways to do this.

\begin{itemize}
\item
  Write an import statement like: \code{import checkers.nullness.quals.*;}

  A disadvantage of this is that everyone who compiles the code
  (even using a non-JSR-308 compiler) must have the annotation definitions
  (e.g., the \code{checkers.jar} or \code{checkers-quals.jar} file) on
  their classpath.  The reason is
  that a Java compiler issues an error if an imported package is not on the
  classpath.  See Section~\ref{distributing}.

\item
  \label{jsr308_imports}
  When you compile the code, set the shell environment variable
  \code{jsr308\_imports}.  This permits your code to compile whether or not
  the Type Annotations compiler is being used.

  In bash, you could write \code{export
    jsr308\_imports='checkers.nullness.quals.*'}, or prefix the \code{javac}
  command by \code{jsr308\_imports='checkers.nullness.quals.*'} .
  Alternately, you can set the system variable via the javac command line
  argument \code{-J-Djsr308\_imports="checkers.nullness.quals.*"}.

  You can specify multiple packages separated by the classpath separator
  (same as the file path separator:   \<;> for Windows, and \<:> for Unix
  and Mac.).  For example, to
  implicitly import the Nullness and Interning qualifiers,
  set \code{jsr308\_imports} to
  \code{checkers.nullness.quals.*:checkers.interning.quals.*}.  
\end{itemize}


\subsubsection{Migrating away from annotations in comments}

If your codebase currently uses annotations in comments, but you are
willing to use only compilers that support type annotations (such as any
Java 7 compiler), then you can remove the comment characters around your
annotations.  This Unix command will do so, for all Java files in the
current working directory or any subdirectory.

\begin{Verbatim}
   find . -type f -name '*.java' -print \
     | xargs grep -l -P '/\*\s*@([^ */]+)\s*\*/' \
     | xargs perl -pi.bak -e 's|/\*\s*@([^ */]+)\s*\*/|@\1|g'
\end{Verbatim}

You can customize this command:
\begin{itemize}
\item
To process comments with embedded spaces and asterisks, change 
two instances of ``\verb|[^ */]|'' to ``\verb|[^/]|''.
\item
To ignore comments with leading or trailing spaces, remove the four
instances of ``\verb|\s*|''.  
\item
  To not make backups, remove
``\verb|.bak|''.
\end{itemize}


If you are using implicit import statements
(Section~\ref{implicit-import-statements}), you may also need to introduce
explicit import statements into your code.


\section{Advanced type system features\label{advanced-type-system-features}}

You may wish to skim or skip this section on first reading.  After you have
used a checker for a little while and want to be able to express more
sophisticated and useful types, or to understand more about how the Checker
Framework works, you can return to it.


\subsection{Polymorphism and generics\label{polymorphism}}

\subsubsection{Generics (parametric polymorphism or type polymorphism)\label{generics}}

The Checker Framework fully supports 
qualified Java generic types (also known in the literature as ``parametric
polymorphism'').  Before running a checker, we recommend that you eliminate
raw types (e.g., \code{List} as opposed to \code{List<...>}) from your code.
Using generics helps prevent type errors just as using a pluggable
type-checker does.
% Should say why, or what are the consequences of violating this.

When instantiating a generic type, 
clients supply the qualifier along with the type argument, as in
\code{List<@NonNull String>}.

The declaration (that is, the implementation) of a generic class may use
the \code{extends} clause to restrict the types and qualifiers that may be
used for instantiating.  For example, given the declaration \code{class
  MyClass<T extends @NonNull Object> \ttlcb ...\ttrcb}, a client could use
\code{MyClass<@NonNull String>} but not \code{MyClass<@Nullable String>}.

\emph{Style note:}
When using the Nullness checker (Section~\ref{nullness-checker}),
programmers sometimes write \<extends @NonNull Object> even though it's the 
default.
The reason is that code with no extends clause, like 

\begin{Verbatim}
  class C<T> { ... }
\end{Verbatim}

typically means that class \<C> can be instantiated with any type argument at
all.  But in the Nullness type system, to permit all type arguments, to
obtain that effect one must write

\begin{Verbatim}
  class C<T extends @Nullable Object> { ... }
\end{Verbatim}


\paragraph{Type annotations on generic type variables}

A type annotation on a generic type variable overrides/ignores any type
qualifier (in the same type hierarchy) on the corresponding actual type
argument.  For example,
\code{@Nullable T} applies the type qualifier \code{@Nullable} to the
(unqualified) Java type of the type argument \code{T}.

Here is an example of applying a type annotation to a generic type
variable:

\begin{Verbatim}
  class MyClass2<T> {
    ...
    @Nullable T = null;
    ...
  }
\end{Verbatim}

\noindent
The type annotation does not restrict how \code{MyClass2} may be instantiated
(only the optional \code{extends} clause on the declaration of type
variable \code{T} would do so).  In other words, both 
\code{MyClass2<@NonNull String>} and \code{MyClass2<@Nullable String>} are
legal, and in both cases \code{@Nullable T} means \code{@Nullable String}.
In \code{MyClass2<@Interned String>}, 
\code{@Nullable T} means \code{@Nullable @Interned String}.

% Note that a type annotation on a generic type variable does not act like
% other type qualifiers.  In both cases the type annotation acts as a type
% constructor, but as noted above they act slightly differently.


% %% This isn't quite right because a type qualifier is itself a type
% %% constructor.
% More formally, a type annotation on a generic type variable acts as a type
% constructor rather than a type qualifier.  Another example of a type
% constructor is \code{[]}.  Just as \code{T[]} is not the same type as
% \code{T}, \code{@Nullable T} is not (necessarily) the same type as
% \code{T}.


\subsubsection{Qualifier polymorphism\label{qualifier-polymorphism}}

The Checker Framework also supports type \emph{qualifier} polymorphism for methods,
which permits a single method to have multiple different qualified type
signatures.

A polymorphic qualifier's definition is marked with
\<@\refclass{quals}{PolymorphicQualifier}>.  For example, 
\<@\refclass{nullness/quals}{PolyNull}> is a polymorphic type
qualifier for the Nullness type system:

\begin{Verbatim}
  @PolymorphicQualifier
  public @interface PolyNull { }
\end{Verbatim}

A method written using a polymorphic qualifier conceptually has multiple
versions, somewhat like a template in C++.  In each version, the
polymorphic qualifier has been replaced by another qualifier from the
hierarchy.  See the examples in Section~\ref{qualifier-polymorphism-examples}.

The method body must type-check with all signatures.  A method call is
type-correct if it type-checks under any signature.

Polymorphic qualifiers can be used within a method body.  They may not be
used on classes or fields.


\paragraph{Examples of using polymorphic qualifiers\label{qualifier-polymorphism-examples}}

As an example of the use of \<@PolyNull>, method \ahref{http://java.sun.com/javase/6/docs/api/java/lang/Class.html#cast(java.lang.Object)}{\<Class.cast>}
returns null if and only if its argument is \<null>:

\begin{Verbatim}
  @PolyNull T cast(@PolyNull Object obj) { ... }
\end{Verbatim}

\noindent
This is like writing:

\begin{Verbatim}
   @NonNull T cast( @NonNull Object obj) { ... }
  @Nullable T cast(@Nullable Object obj) { ... }
\end{Verbatim}

\noindent
except that the latter is not legal Java, since it defines two
methods with the same Java signature.


As another example, consider

\begin{Verbatim}
  @PolyNull T max(@PolyNull T x, @PolyNull T y);
\end{Verbatim}

\noindent
which is like writing

\begin{Verbatim}
   @NonNull T max( @NonNull T x,  @NonNull T y);
  @Nullable T max(@Nullable T x, @Nullable T y);
\end{Verbatim}

\noindent
One way of thinking about which one of the two \code{max} variants is
selected is that the nullness annotations of (the declared types of) both
arguments are \emph{unified} to a type that is a subtype of both.  If both
arguments are \code{@NonNull}, their unification is \<@NonNull>, and the
result is \<@NonNull>.  But if even one of the arguments is \<@Nullable>,
then the result is \<@Nullable>.


%% I can't think of a non-clumsy way to say this.
% Each method containing a polymorphic qualifier is (conceptually) expanded
% into multiple versions completely independently.

It does not make sense to write only a single instance of a polymorphic
qualifier in a method definition, as in

\begin{Verbatim}
  void m(@PolyNull Object obj)
\end{Verbatim}

\noindent
which expands to

\begin{Verbatim}
  void m(@NonNull Object obj)
  void m(@Nullable Object obj)
\end{Verbatim}

\noindent
which is no different than writing just

\begin{Verbatim}
  void m(@Nullable Object obj)
\end{Verbatim}

\noindent
The benefit of polymorphic qualifiers comes when one is used multiple times
in a method, since then each instance turns into the same type qualifier.
Most frequently, the polymorphic qualifier appears on both the return type
and at least one formal parameter.  It can also be useful to have
polymorphic qualifiers on (only) multiple formal parameters, especially if
the method side-effects one of its arguments.


%% It would be nice to give an example that isn't too contrived.


%% I don't see why this is necessarily true; one could define @PolyNull1
%% and @PolyNull2.  It's not so relevant to the manual anyway, and raising
%% the point just makes type system bigots criticize the Checker Framework.
% Qualifier polymorphism is limited to a single qualifier variable per method.


\subsection{Unused fields and dependent types}

Sometimes, the type of a field depends on the qualifier on the receiver.
The Checker Framework supports two varieties of such a field:  fields that
may not be used if the receiver has a given qualifier, and fields whose
qualifier changes based on the qualifier of the receiver.


\subsubsection{Unused fields\label{unused-fields}}

A Java subtype can have more fields than its supertype.  You can simulate
the same effect for type qualifiers:  a given field may not be accessed via
a reference with a supertype qualifier, but can be accessed via a reference
with a subtype qualifier.

This permits you to restrict use of a field to certain contexts.

The \code{@\refclass{quals}{Unused}} annotation
on a field declares that the field may not be accessed via a receiver of
the given qualified type (or any supertype).


\subsubsection{Dependent types\label{dependent-types}}

A variable has a \emph{dependent type} if its type depends on some other
value or type.
%  --- the type is dynamically, not statically, determined.
% (Type-safety can still be statically determined, though.)

The Checker Framework supports a form of dependent types, via the
\code{@\refclass{quals}{Dependent}} annotation.
This annotation changes the type of a field or variable, based on the
qualified type of the receiver (\code{this}).  This can be viewed as a more
expressive form of polymorphism (see Section~\ref{polymorphism}).  It can
also be seen as a way of linking the meanings of two type qualifier
hierarchies.

When the \code{@\refclass{quals}{Unused}} annotation is sufficient, you
should use it instead of \code{@Dependent}.


\subsubsection{Example\label{dependent-types-example}}

Suppose we have a class \code{Person} and a field \code{spouse} that is
non-\code{null} if the person is married.  We could declare this as

\begin{Verbatim}
  class Person {
    ...
    // non-null if this person is married
    @Nullable Person spouse;
    ...
  }
\end{Verbatim}

Now, suppose that we have defined the qualifier hierarchy in which 
\code{@Single} (meaning ``not married'') is a supertype of \code{@Married}.
A more informative declaration would be

\begin{Verbatim}
  class Person {
    ...
    @Nullable @Dependent(result=NonNull.class, when=Married.class) Person spouse;
    ...
  }
\end{Verbatim}

If a person is known to be \code{@Married}, the
\code{spouse} field is known to be non-\code{null}:

\begin{Verbatim}
  class Person {
    ...

    void celebrateWeddingAnniversary() @Married {
      System.out.println("Happy anniversary, "
                         + spouse.toString()); // no possible null pointer exception
    }

    ...
  }
\end{Verbatim}

Without the \code{@\refclass{quals}{Dependent}} annotation on the
declaration of the \code{spouse} variable, the Nullness Checker would
complain that \code{toString} was being invoked on a possibly-\code{null}
value.

An even better declaration is

\begin{Verbatim}
  class Person {
    ...
    @Unused(when=Single.class) @NonNull Person spouse;
    ...
  }
\end{Verbatim}

Then, if a person is known to be \code{@Married} (or more
appropriately non-\code{@Single}), the \code{spouse} field is known to
be non-\code{null}.  Also, if a person is known to be \code{@Single},
the \code{spouse} field may not be accessed:
 
\begin{Verbatim}
  @Single Person person = ...;
  Person spouse = person.spouse;  // invalid field access
  ...
\end{Verbatim}


\subsection{The effective qualifier on a type (defaults and inference)\label{effective-qualifier}}

A checker sometimes treats a type as having a slightly different qualifier
than what is written on the type --- especially if the programmer wrote no
qualifier at all.
Most readers can skip this section on first reading, because you will
probably find the system simply ``does what you mean'', without forcing
you to write too many qualifiers in your program.

  The following steps determine the effective
qualifier on a type --- the qualifier that the checkers treat as being present.

\begin{enumerate}
\item
  The type system adds implicit qualifiers.  Implicit qualifiers can be
  built into a type system (Section~\ref{writing-type-introduction}), in
  which case the type system's documentation should explain all of the type
  system's implicit qualifiers.  Or, a programmer may introduce an implicit
  annotations on each use of class $C$ by writing a qualifier on the
  declaration of class $C$.

\begin{itemize}
\item
  Example 1 (built-in):  In the Nullness type system,
  \<enum> values are never null, nor is a method receiver.
\item
  Example 2 (built-in):  In the Interning type system, string literals
  and \<enum> values are always interned.
\end{itemize}

\item
  If a type qualifier is present in the source code, that qualifier is used.

  If the type has an implicit qualifier, then it is an error to write an
  explicit qualifier that is equal to (redundant with) or a supertype of
  (weaker than) the implicit qualifier.  A programmer may strengthen
  (write a subtype of) an implicit qualifier, however.

\item
  If there is no implicit or explicit qualifier on a type, then a default
  qualifier may be applied; see Section~\ref{defaults}.  

  \smallskip

  At this point, every type has a qualifier.

\item
  The type system may refine a qualified type on a local variable --- that
  is, treat it as a subtype of how it was declared or defaulted.  This
  refinement is always sound and has the effect of eliminating false
  positive error messages.  See Section~\ref{type-refinement}.

  % Type
  % qualifier refinement is implemented by the \refclass{flow}{Flow} class.

\end{enumerate}



\subsubsection{Default qualifier for unannotated types\label{defaults}}

A type system designer, or an end-user programmer, can cause unannotated
references to be treated as if they had a default annotation.

There are several defaulting mechanisms, for convenience and flexibility.
When determining the default qualifier for a use of a type, the following
rules are used in order, until one applies.
\begin{itemize}
\item
  Use the innermost user-written \code{@DefaultQualifier}, as explained in
  this section.
\item
  Use the default specified by the type system designer
  (Section~\ref{typesystem-defaults}).
\item
  Use \code{@\refclass{quals}{Unqualified}}, which the framework
  inserts to avoid ambiguity and simplify the programming interface for
  type system designers.  Users do not have to worry about this detail.
\end{itemize}

% (Implementation detail:  setting defaults is implemented by the
% \refclass{util}{QualifierDefaults} class.)


The end-user programmer specifies a default qualifier by writing the \code{@\refclass{quals}{DefaultQualifier}}
annotation on a package, class, method, or variable declaration.  The
argument to \<@\refclass{quals}{DefaultQualifier}> is the fully qualified \code{String} name of an
annotation, and its optional second argument indicates where the default
applies.  If the second argument is omitted, the specified annotation is
the default in all locations.  See the Javadoc of \refclass{quals}{DefaultQualifier} for details.

For example, using the Nullness type system (Section~\ref{nullness-checker}):

\begin{Verbatim}
import checkers.quals.*;        // for DefaultQualifier[s]

@DefaultQualifier("checkers.nullness.quals.NonNull"),
class MyClass {

  public boolean compile(File myFile) { // myFile has type "@NonNull File"
    if (!myFile.exists())          // no warning: myFile is non-null
      return false;
    @Nullable File srcPath = ...;  // must annotate to specify "@Nullable File"
    ...
    if (srcPath.exists())          // warning: srcPath might be null
      ...
  }

  @DefaultQualifier("checkers.igj.quals.Nullable")
  public boolean isJavaFile(File myfile) {  // myFile has type "@Nullable File"
    ...
  }
}
\end{Verbatim}

If you wish to write multiple 
\<@\refclass{quals}{DefaultQualifier}> annotations at a single location,
use 
\<@\refclass{quals}{DefaultQualifiers}> instead.  For example:

\begin{Verbatim}
@DefaultQualifiers({
  @DefaultQualifier("checkers.nullness.quals.NonNull"),
  @DefaultQualifier("checkers.igj.quals.Mutable")
})
\end{Verbatim}


If \code{@DefaultQualifier}[\code{s}] is placed on a package (via the
\<package-info.java> file), then it applies to the given package \emph{and}
all subpackages.
% This is slightly at odds with Java's treatment of packages of different
% names as essentially unrelated, but is more intuitive and useful.

Recall that an annotation on a class definition indicates an implicit
qualifier (Section~\ref{effective-qualifier}) that can only be
strengthened, not weakened.  This can lead to unexpected results when if
the default qualifier applies to a class definition.  Thus, you may want to
put explicit qualifiers on class declarations (which prevents the default
from taking effect), or exclude class declarations from defaulting.


\paragraph{When a default qualifier may not be specified}

Sometimes, the meaning of an unannotated reference is determined by the
type system.  For example, in the Interning type system, each type is
either unqualified, or it has the \<@\refclass{interning/quals}{Interned}>
qualifier.  In such a case, specifying a default for unannotated types is
not sensible.

In other cases, the type hierarchy has an explicit qualifier for every
possible meaning.  For example, the Nullness type system has
\<@\refclass{nullness/quals}{Nullable}> types and
\<@\refclass{nullness/quals}{NonNull}> types.  It has no built-in meaning for
unannotated types; a user may specify a default qualifier.

Permitting users to specify defaults is a reason you may wish to make your
type hierarchy ``complete'', in the sense that there is a qualifier for
every location in the hierarchy.


\subsubsection{Automatic type refinement (flow-sensitive type qualifier inference)\label{type-refinement}}

In order to reduce the burden of annotating types in your program, the
checkers soundly treat certain variables and expressions as having a
subtype of their declared or defaulted (Section~\ref{defaults})
type.  This functionality
never introduces unsoundness or causes an error to be missed:  it merely
suppresses false positive warnings.

By default, all checkers, including new checkers that you write, can take
advantage of this functionality.  Most of the time, users don't have to
think about, and may not even notice, this feature of the framework.  The
checkers simply do the right thing even when a programmer forgets an
annotation on a local variable, or when a programmers writes an
unnecessarily general type in a declaration.

If you are curious or want more details about this feature, then read on.

As an example, the Nullness checker (Section~\ref{nullness-checker}) can automatically
determine that certain variables are non-null, even if they were explicitly
or by default annotated as nullable.
A variable or expression can be treated as \code{@\refclass{nullness/quals}{NonNull}}
from the time that it is either
assigned a non-null value or checked against null (e.g., via an assertion,
\code{if} statement, or being dereferenced), until it might be re-assigned (e.g.,
via an assignment that might affect this variable, or via a method call
that might affect this variable).

As with explicit annotations, the implicitly non-null types permit
dereferences and assignments to explicitly non-null types, without
compiler warnings.

Consider this code, along with comments indicating whether the
Nullness checker (Section~\ref{nullness-checker}) issues a warning.  Note that the same expression may yield a
warning or not depending on its context.

\begin{Verbatim}
  // Requires an argument of type @NonNull String
  void parse(@NonNull String toParse) { ... }

  // Argument does NOT have a @NonNull type
  void lex(String toLex) {
    parse(toLex);        // warning:  toLex might be null
    if (toLex != null) {
      parse(toLex);      // no warning:  toLex is known to be non-null
    }
    parse(toLex);        // warning:  toLex might be null
    toLex = new String(...);
    parse(toLex);        // no warning:  toLex is known to be non-null
  }
\end{Verbatim}

If you find instances where you think a value should be inferred to have
(or not have) a
given annotation, but the checker does not do so, please submit a bug
report (see Section~\ref{reporting-bugs}) that includes a small piece of
Java code that reproduces the problem.

% Flow-sensitive non-null inference has been implemented for the following
% varieties of expressions:
%
% \begin{itemize}
% \item null checks in if/else statements
% \item null checks in assert statements
% \item null checks that result in a return or thrown exception, or call System.exit
% \item assignments from new class/array expressions
% \end{itemize}
%
% \emph{Note:} The items in the above list exclude complex null checks, i.e., not
% of the form \code{x != null}. Support for these types of checks will be available in a
% future release.


% TODO:  Is NonNull inferred for any parameters or fields, or just for locals?

Type inference is never performed for method parameters of non-private
methods and for non-private fields, because unknown client code could use
them in arbitrary ways.  The inferred information is never written to the
\code{.class} file as user-written annotations are.

The inference indicates when a variable can be treated as having a subtype
of its declared type --- for instance, when an otherwise nullable type can be
treated as a \code{@\refclass{nullness/quals}{NonNull}} one.  The inference never treats a variable as
a supertype of its declared type (e.g., an expression of \code{@\refclass{nullness/quals}{NonNull}}
type is never inferred to be treated as possibly-null).



% LocalWords:  NonNull zipfile processor classfiles annotationname javac htoc
% LocalWords:  SuppressWarnings un skipClasses java plugins plugin TODO cp igj
% LocalWords:  nonnull javari langtools sourcepath classpath OpenJDK pre jsr
% LocalWords:  Djsr quals Alint javac's dotequals nullable supertype JLS Papi
% LocalWords:  deserialization Mahmood Telmo Correa changelog txt nullness ESC
% LocalWords:  Nullness Xspacesincomments unselect checkbox unsetting PolyNull
% LocalWords:  bashrc IDE xml buildfile PolymorphicQualifier enum API elts INF
% LocalWords:  typechecker proc discoverable Xlint util QualifierDefaults Foo
% LocalWords:  DefaultQualifier DefaultQualifiers SoyLatte GetStarted Formatter
% LocalWords:  Dcheckers Warski MyClass ProcessorName compareTo toString myDate
% LocalWords:  ReadOnly readonlyObject int XDTA spacesincomments newdir Awarns
% LocalWords:  subpackages bak tIDE Multiset NullPointerException AskipClasses
% LocalWords:  html

\htmlhr
\chapter{Nullness Checker\label{nullness-checker}}

If the Nullness Checker issues no warnings for a given program, then
running that program will never throw a null pointer exception.  This
guarantee enables a programmer to prevent errors from occurring when a
program is run.  See Section~\ref{nullness-checks} for more details about
the guarantee and what is checked.

The most important annotations supported by the Nullness Checker are 
\code{@\refclass{nullness/quals}{NonNull}} and 
\code{@\refclass{nullness/quals}{Nullable}}.
\code{@\refclass{nullness/quals}{NonNull}} is rarely written, because it is
the default.  All of the annotations are explained in
Section~\ref{nullness-annotations}.

To run the Nullness Checker, supply the \code{-processor
  checkers.nullness.NullnessChecker} command-line option to javac.  For
examples, see Section~\ref{nullness-example}.


\section{What the Nullness Checker checks\label{nullness-checks}}

The checker issues a warning in these cases:

\begin{enumerate}

\item
  When an expression of non-\code{@\refclass{nullness/quals}{NonNull}} type
  is dereferenced, because it might cause a null pointer exception.
  Dereferences occur not only when a field is accessed, but when an array
  is indexed, an exception is thrown, a lock is taken in a synchronized
  block, and more.  For a complete description of all checks performed by
  the Nullness Checker, see the Javadoc for
  \refclass{nullness}{NullnessVisitor}.

\item
  When an expression of \code{@\refclass{nullness/quals}{NonNull}} type
  might become null, because it
  is a misuse of the type:  the null value could flow to a dereference that
  the checker does not warn about.

  As a special case of an of \code{@\refclass{nullness/quals}{NonNull}}
  type becoming null, the checker also warns whenever a field of
  \code{@\refclass{nullness/quals}{NonNull}} type is not initialized in a
  constructor.  Also see the discussion of the \code{-Alint=uninitialized}
  command-line option below.

\end{enumerate}

This example illustrates the programming errors that the checker detects:

\begin{Verbatim}
  @Nullable Object   obj;  // might be null
  @NonNull  Object nnobj;  // never null
  ...
  obj.toString()         // checker warning:  dereference might cause null pointer exception
  nnobj = obj;           // checker warning:  nnobj may become null
  if (nnobj == null)     // checker warning:  redundant test
\end{Verbatim}

Parameter passing and return values are checked analogously to assignments.

The Nullness Checker also checks the correctness, and correct use, of
rawness annotations for checking initialization (see
Section~\ref{raw-partially-initialized}) and of map key annotations (see
Section~\ref{map-keys}).


The checker performs additional checks if certain \code{-Alint}
command-line options are provided.  (See
Section~\ref{alint} for more details about the \code{-Alint}
command-line option.)

\begin{enumerate}
\item
  \label{lint-nulltest}%
  If you supply the \code{-Alint=redundantNullComparison} command-line option, then the
  checker warns when a null check is performed against a value that is
  guaranteed to be non-null, as in \code{("m" == null)}.  Such a check is
  unnecessary and might indicate a programmer error or misunderstanding.
  The lint option is disabled by default because sometimes such checks are
  part of ordinary defensive programming.  

\item
  \label{lint-uninitialized}%
  If you supply the \code{-Alint=uninitialized} command-line option, then
  the checker warns if a constructor fails to initialize any field,
  including \code{@\refclass{nullness/quals}{Nullable}} types and primitive
  types.  Such a warning is unrelated to whether your code might throw a
  null pointer exception.  However, you might want to enable this warning
  because it is better code style to supply an explicit initializer, even
  if there is a default value such as \code{0} or \code{false}.
  This command-line option does not affect the Nullness Checker's tests
  that fields of \code{@\refclass{nullness/quals}{NonNull}} type are
  initialized --- such initialization is mandatory, not optional.

\end{enumerate}


\section{Nullness annotations\label{nullness-annotations}}

The Nullness Checker uses three separate type hierarchies:  one for nullness,
one for rawness (Section~\ref{raw-partially-initialized}),
and one for map keys (Section~\ref{map-keys})
The Nullness Checker has four varieties of annotations:  nullness
type qualifiers, nullness method annotations, rawness type qualifiers, and
map key type
qualifiers.

\subsection{Nullness qualifiers\label{nullness-qualifiers}}

The nullness hierarchy contains these qualifiers:

\begin{description}

\item[\code{@\refclass{nullness/quals}{Nullable}}]
  indicates a type that includes the null value.  For example, the type \code{Boolean}
  is nullable:  a variable of type \code{Boolean} always has one of the
  values \code{TRUE}, \code{FALSE}, or \code{null}.

\item[\code{@\refclass{nullness/quals}{NonNull}}]
  indicates a type that does not include the null value.  The type
  \code{boolean} is non-null; a variable of type \code{boolean} always has
  one of the values \code{true} or \code{false}.  The type \code{@NonNull
    Boolean} is also non-null:  a variable of type \code{@NonNull Boolean}
  always has one of the values \code{TRUE} or \code{FALSE} --- never
  \code{null}.  Dereferencing an expression of non-null type can never cause
  a null pointer exception.

  The \<@NonNull> annotation is rarely written in a program, because it is
  the default (see Section~\ref{null-defaults}).

\item[\code{@\refclass{nullness/quals}{PolyNull}}]
  indicates qualifier polymorphism.  For a description of
  \<@\refclass{nullness/quals}{PolyNull}>, see
  Section~\ref{qualifier-polymorphism}.

\item[\code{@\refclass{nullness/quals}{MonotonicNonNull}}]
  indicates a reference that may be \code{null}, but if it ever becomes
  non-\code{null}, then it never becomes \code{null} again.  This is
  appropriate for lazily-initialized fields, among other uses.  When the
  variable is read, its type is treated as
  \code{@\refclass{nullness/quals}{Nullable}}, but when the variable is
  assigned, its type is treated as
  \code{@\refclass{nullness/quals}{NonNull}}.

  Because the Nullness Checker works intraprocedurally (it analyzes one
  method at a time), when a \code{MonotonicNonNull} field is first read within a
  method, the field cannot be assumed to be non-null.  The benefit of
  MonotonicNonNull over Nullable is its different interaction with
  flow-sensitive type qualifier refinement (Section~\ref{type-refinement}).
  After a check of a MonotonicNonNull
  field, all subsequent accesses \emph{within that method} can be assumed
  to be NonNull, even after arbitrary external method calls that have
  access to the given field.

  It is permitted to initialize a MonotonicNonNull field to null, but the
  field may not be assigned to null anywhere else in the program.  If you
  supply the \<noInitForMonotonicNonNull> lint flag (for example, supply 
  \<-Alint=noInitForMonotonicNonNull> on the command line), then
  @MonotonicNonNull fields are not allowed to have initializers.


\end{description}

Figure~\ref{fig:nullness-hierarchy} shows part of the type hierarchy for the
Nullness type system.
(The annotations exist only at compile time; at run time, Java has no
multiple inheritance.)

\begin{figure}
\includeimage{nullness-and-raw}{2.5cm}
\caption{Partial type hierarchy for the Nullness type system.
Java's \<Object> is expressed as \<@Nullable Object>.  Programmers can omit
most type qualifiers, because the default annotation
(Section~\ref{null-defaults}) is usually correct.  Also shown is the
type hierarchy for rawness (Section~\ref{raw-partially-initialized}), which
indicates whether
initialization has completed.  The two type hierarchies are independent but
inter-related.  The Nullness Checker verifies both of these, as well as
another type hierarchy for map keys (Section~\ref{map-key-qualifiers}).}
\label{fig:nullness-hierarchy}
\end{figure}


\subsection{Nullness method annotations\label{nullness-method-annotations}}

The Nullness Checker supports several annotations that specify method
behavior.  These are declaration annotations, not type annotations:  they
apply to the method itself rather than to some particular type.

\begin{description}

\item[\code{@\refclass{nullness/quals}{RequiresNonNull}}]
  indicates a method precondition:  The annotated method expects the
  specified variables (typically field references) to be non-null when the
  method is invoked.

\item[\code{@\refclass{nullness/quals}{EnsuresNonNull}}]
\item[\code{@\refclass{nullness/quals}{EnsuresNonNullIf}}]
  indicates a method postcondition.  With \<@EnsuresNonNull>, the given
  expressions are non-null after the method returns; this is useful for a
  method that initializes a field, for example.  With
  \<@EnsuresNonNullIf>, if the annotated
  method returns the given boolean value (true or false), then the given
  expressions are non-null.  See Section~\ref{conditional-nullness} and the
  Javadoc for examples of their use.

\item[\code{@\refclass{nullness/quals}{AssertParametersNonNull}}]
  % Indicates a method precondition:  The annotated method expects all of
  % its parameters to be non-null.
  is used for suppressing warnings, in very rare cases.  See the Javadoc for
  details.

\end{description}


\subsection{Rawness qualifiers\label{rawness-qualifiers-overview}}

The Nullness Checker supports rawness annotations that indicate whether
an object is fully initialized --- that is, whether its fields have all
been assigned.

\begin{description}
\item[\code{@\refclass{nullness/quals}{Raw}}]
\item[\code{@\refclass{nullness/quals}{NonRaw}}]
\item[\code{@\refclass{nullness/quals}{PolyRaw}}]
\end{description}

Use of these annotations can help you to type-check more
code.  Figure~\ref{fig:nullness-hierarchy} shows its type hierarchy.  For
details, see Section~\ref{raw-partially-initialized}.


\subsection{Map key qualifiers\label{map-key-qualifiers}}

The Nullness Checker supports a map key annotation, \code{@\refclass{nullness/quals}{KeyFor}} that indicates whether
a value is a key for a given map --- that is, whether
\code{map.containsKey(value)} would evaluate to \code{true}.

\begin{description}
\item[\code{@\refclass{nullness/quals}{KeyFor}}]
\end{description}

Use of this annotation can help you to type-check more code.  For details,
see Section~\ref{map-keys}.


\section{Writing nullness annotations\label{writing-nullness-annotations}}

\subsection{Implicit qualifiers\label{nullness-implicit-qualifiers}}

As described in Section~\ref{effective-qualifier}, the Nullness Checker
adds implicit qualifiers, reducing the number of annotations that must
appear in your code.
For example, enum types are implicitly non-null, so you never need to write
\<@NonNull MyEnumType>.

For a complete description of all implicit nullness qualifiers, see the
Javadoc for \refclass{nullness}{NullnessAnnotatedTypeFactory}.



\subsection{Default annotation\label{null-defaults}}

Unannotated references are treated as if they had a default annotation,
using the NNEL (non-null except locals) rule described below.
A user may choose a different rule for defaults using the
\code{@\refclass{quals}{DefaultQualifier}} annotation; see
Section~\ref{defaults}.

%BEGIN LATEX
\begin{sloppy}
%END LATEX
Here are three possible default rules you may wish to use.  Other rules are
possible but are not as useful.
\begin{itemize}
\item
  \code{@\refclass{nullness/quals}{Nullable}}:  Unannotated types are regarded as possibly-null, or
  nullable.  This default is backward-compatible with Java, which permits
  any reference to be null.  You can activate this default by writing
  a \code{@DefaultQualifier(Nullable.class)} annotation on a
  % package/
  class or method
  % /variable
  declaration.
\item
  \code{@\refclass{nullness/quals}{NonNull}}:  Unannotated types are treated as non-null.
  % This may leads to fewer annotations written in your code.
  You can activate this
  default via the
  \code{@DefaultQualifier(NonNull.class)} annotation.
\item
  Non-null except locals (NNEL):  Unannotated types are treated as
  \code{@\refclass{nullness/quals}{NonNull}}, \emph{except} that the
  unannotated raw type of a local variable is treated as
  \code{@\refclass{nullness/quals}{Nullable}}.  (Any generic arguments to a
  local variable still default to
  \code{@\refclass{nullness/quals}{NonNull}}.)  This is the standard
  behavior.  You can explicitly activate this default via the
  \code{@DefaultQualifier(value=NonNull.class,
    locations=\discretionary{}{}{}\{DefaultLocation\discretionary{}{}{}.ALL\_EXCEPT\_LOCALS\})}
  annotation.

  The NNEL default leads to the smallest number of explicit annotations in
  your code~\cite{PapiACPE2008}.  It is what we recommend.  If you do not
  explicitly specify a different default, then NNEL is the default.
\end{itemize}
%BEGIN LATEX
\end{sloppy}
%END LATEX

\subsection{Conditional nullness\label{conditional-nullness}}

The Nullness Checker supports a form of conditional nullness types, via the
\code{@\refclass{nullness/quals}{EnsuresNonNullIf}} method annotations.
The annotation on a method declares that some expressions are non-null, if
the method returns true (false, respectively).

Consider \sunjavadoc{java/lang/Class.html}{java.lang.Class}.
Method
\sunjavadoc{java/lang/Class.html#getComponentType()}{Class.getComponentType()}
may return null, but it is specified to return a non-null value if
\sunjavadoc{java/lang/Class.html#isArray()}{Class.isArray()} is
true.
You could declare this relationship in the following way (this particular
example is already
done for you in the annotated JDK that comes with the Checker Framework):

\begin{Verbatim}
  class Class {
    @EnsuresNonNullIf(expression="getComponentType()", result=true)
    public native boolean isArray();

    public native @Nullable Class<?> getComponentType();
  }
\end{Verbatim}

A client that checks that a \code{Class} reference is indeed that of an array,
can then de-reference the result of \code{Class.getComponentType} safely
without any nullness check.  The Checker Framework source code itself
uses such a pattern:

\begin{Verbatim}
    if (clazz.isArray()) {
      // no possible null dereference on the following line
      TypeMirror componentType = typeFromClass(clazz.getComponentType());
      ...
    }
\end{Verbatim}

Another example is \sunjavadoc{java/util/Queue.html#peek()}{Queue.peek}
and \sunjavadoc{java/util/Queue.html#poll()}{Queue.poll}, which return
non-null if \sunjavadoc{java/util/Collections.html#isEmpty()}{isEmpty}
returns false.

The argument to \code{@EnsuresNonNullIf} is a Java expression, including method calls
(as shown above), method formal parameters, fields, etc.; for details, see
Section~\ref{java-expressions-as-arguments}.
More examples of the use of these annotations appear in the Javadoc for
\code{@\refclass{nullness/quals}{EnsuresNonNullIf}}.


\subsection{Nullness and arrays\label{nullness-arrays}}

The components of a newly created object of reference type are all
null. Only after initialization can the array actually be considered
to contain non-null components.
Therefore, the following is not allowed:

\begin{Verbatim}
  @NonNull Object [] oa = new @NonNull Object[10]; // error
\end{Verbatim}

Instead, one creates a nullable or lazy-nonnull array, initializes
each component, and then assigns the result to a non-null array:

\begin{Verbatim}
  @MonotonicNonNull Object [] temp = new @MonotonicNonNull Object[10];
  for (int i = 0; i < temp.length; ++i) {
    temp[i] = new Object();
  }
  @SuppressWarnings("nullness")
  @NonNull Object [] oa = temp;
\end{Verbatim}

Note that the checker is currently not powerful enough to ensure that
each array component was initialized. Therefore, the last assignment
needs to be trusted:  that is, a programmer must verify that it is safe,
then write a \<@SuppressWarnings> annotation.

% TODO: explain more aspects, give more examples.


\subsection{Inference of \code{@NonNull} and \code{@Nullable} annotations\label{nullness-inference}}

It can be tedious to write annotations in your code.  Tools exist that
can automatically infer annotations and insert them in your source code.
(This is different than type qualifier refinement for local variables
(Section~\ref{type-refinement}), which infers a more specific type for
local variables and uses them during type-checking but does not insert them
in your source code.  Type qualifier refinement is always enabled, no
matter how annotations on signatures got inserted in your source code.)

Your choice of tool depends on what default annotation (see
Section~\ref{null-defaults}) your code uses.  You only need one of these tools.

\begin{itemize}

\item
  Inference of \code{@\refclass{nullness/quals}{Nullable}}:
  %
  If your code uses the standard NNEL (non-null-except-locals) default or
  the \refclass{nullness/quals}{NonNull} default, then use the
  \ahref{http://groups.csail.mit.edu/pag/daikon/download/doc/daikon.html#AnnotateNullable}{AnnotateNullable}
  tool of the \ahref{http://pag.csail.mit.edu/daikon/}{Daikon invariant
    detector}.

\item
  Inference of \code{@\refclass{nullness/quals}{NonNull}}:
  %
  If your code uses the Nullable default, use one of these tools:
\begin{itemize}
\item
  \ahref{http://julia.scienze.univr.it/}{Julia analyzer},
\item
  \ahref{http://nit.gforge.inria.fr}{Nit: Nullability Inference Tool},
\item
  \ahref{http://jastadd.org/jastadd-tutorial-examples/non-null-types-for-java/}{Non-null
    checker and inferencer} of the \ahref{http://jastadd.org/}{JastAdd
    Extensible Compiler}.
\end{itemize}

\end{itemize}



\section{Suppressing nullness warnings\label{suppressing-warnings-nullness}}

The Checker Framework supplies several ways to suppress warnings, most
notably the \<@SuppressWarnings("nullness")> annotation (see
Section~\ref{suppressing-warnings}).  An example use is

\begin{Verbatim}
    // might return null
    @Nullable Object getObject(...) { ... }

    void myMethod() {
      // The programmer knows that this partucular call never returns null,
      // perhaps based on the arguments or the state of other objects.
      @SuppressWarnings("nullness")
      @NonNull Object o2 = getObject(...);
\end{Verbatim}


The Nullness Checker supports an additional warning suppression key,
\<nullness:generic.argument>.
Use of \<@SuppressWarnings("nullness:generic.argument")> causes the Nullness
Checker to suppress warnings related to misuse of generic type
arguments.  One use for this key is when a class is declared to take only
\<@NonNull> type arguments, but you want to instantiate the class with a
\<@Nullable> type argument, as in \code{List<@Nullable Object>}.  For a more
complete explanation of this example, see
Section~\refwithpage{faq-list-map-nonnull-typeargs}.

The Nullness Checker also permits you to use assertions or method calls to
suppress warnings; see below.

% TODO: check whether the SuppressWarnings keys are correct.


\subsection{Suppressing warnings with assertions and method calls\label{suppressing-warnings-with-assertions}}

Occasionally, it is inconvenient or
verbose to use the \<@SuppressWarnings> annotation.  For example, Java does
not permit annotations such as \<@SuppressWarnings> to appear on statements.
In such cases, you may be able to use the \<@AssumeAssertion> comment in
an \<assert> statement (see Section~\ref{assumeassertion}).

For situations when all of these approaches are inconvenient,
the Nullness Checker provides two additional ways to suppress warnings:
via the \<castNonNull> method and the
\<@AssertParametersNonNull> annotation.  These are
appropriate when the Nullness Checker issues a warning, but the programmer
knows for sure that the warning is a false positive, because the value
cannot ever be null at run time.

%% TODO:
%% This command got lost/removed during a previous merge
%% -- however, it is still referenced in the text!?!
\newcommand{\nullnessSuppressionString}{nullness}

\begin{enumerate}
\item
  Use an assertion.  If the string ``\<\nullnessSuppressionString>''
  appears in the message body, then the Nullness Checker treats the
  assertion as suppressing a warning and assumes that the assertion always
  succeeds.  For example, the checker assumes that no null pointer
  exception can occur in code such as
\begin{Verbatim}
  assert x != null : "@SuppressWarnings(nullness)";
  ... x.f ...
\end{Verbatim}

  If the string ``\<\nullnessSuppressionString>'' does not appear in the
  assertion message, then the Nullness Checker treats the assertion as being
  used for defensive programming, and it warns if the method might throw a
  nullness-related exception.

  A downside of putting the string in the assertion message is that if the
  assertion ever fails, then a user might see the string and be confused.
  But the string should only be used if the programmer has reasoned that
  the assertion can never fail.

% (Another way of stating the Nullness Checker's use of assertions is as an
% additional caveat to the guarantees provided by a checker
% (Section~\ref{checker-guarantees}).  The Nullness Checker prevents null
% pointer errors in your code under the assumption that assertions are
% enabled, and it does not guarantee that all of your assertions succeed.)


% TODO:
% There is a new lint option "flow:inferFromAsserts" that can be used
% to always infer information from assert statements, even without the
% suppression key.
% Discuss how this influences the guarantees from the checker.
% This is a feature of the framework, not only the Nullness Checker.
% Move the discussion somewhere else, but keep something here?
% Maybe warnings or advanced-features.

% TODO: Also discuss the whole inference over conditions somewhere.

\item
  Use the \refmethod{nullness}{NullnessUtils}{castNonNull}{(T)} method.

The Nullness
 Checker considers both the return value, and also the argument, to
 be non-null after the method call.  Therefore, the
 \<castNonNull> method can be used either as a cast expression or
 as a statement.  The Nullness Checker issues no warnings in any of
the following code:

\begin{Verbatim}
  // one way to use as a cast:
  @NonNull String s = castNonNull(possiblyNull1);

  // another way to use as a cast:
  castNonNull(possiblyNull2).toString();

  // one way to use as a statement:
  castNonNull(possiblyNull3);
  possiblyNull3.toString();`
\end{Verbatim}

  The method also throws \<AssertionError> if Java assertions are enabled and
  the argument is \<null>.  However, it is not intended for general defensive
  programming; see Section~\ref{defensive-programming}.

  A potential disadvantage of using the \<castNonNull> method is that your
  code becomes dependent on the Checker Framework at run time as well as at
  compile time.  You can avoid this by copying the implementation of
  \<castNonNull> into your own code, and possibly renaming it if you do not
  like the name.  Be sure to retain the documentation that indicates that
  your copy is intended for use only to suppress warnings and not for
  defensive programming.  See Section~\ref{defensive-programming} for an
  explanation of the distinction.

\item
  Use the \code{@\refclass{nullness/quals}{AssertParametersNonNull}}
  annotation.  It is used on \<castNonNull>, and may be used on other
  methods with the same semantics; it should probably never be used in any
  other situation.

\end{enumerate}


\subsection{Suppressing warnings on nullness-checking routines and defensive programming\label{defensive-programming}}

%% Work this in
% As explained in Section~\ref{annotate-normal-behavior}, annotations should
% indicate normal behavior that will not cause an exception.
%
% TODO: discuss how to write your own, and why the default doesn't have
% assert or checking methods suppress warnings.


One way to suppress warnings in the Nullness Checker is to use
method \code{castNonNull}.
(Section~\ref{suppressing-warnings-with-assertions} gives other techniques.)

This section explains why the Nullness Checker introduces a new method
rather than re-using the \<assert> statement (as in
\<assert x != null>) or an existing method such as:

\begin{Verbatim}
  org.junit.Assert.assertNotNull(Object)
  com.google.common.base.Preconditions.checkNotNull(Object)
\end{Verbatim}

In each case, the assertion or method indicates an application invariant --- a
fact that should always be true.  There are two distinct reasons a
programmer may have written the invariant, depending on whether the
programmer is 100\% sure that the application invariant holds.

\begin{enumerate}
\item
  A programmer might write it as \textbf{defensive programming}.  This causes
  the program to throw an exception, which is useful for debugging because
  it gives an earlier run-time indication of the error.
  A programmer would use an assertion in this way if the programmer is not
  100\% sure that the application invariant holds.

  % , or even to document what the program
  % is intended to do.

\item
  A programmer might write it to \textbf{suppress} false positive
  \textbf{warning messages} from a checker.  A programmer would use an
  assertion this way if the programmer is 100\% sure that the application
  invariant holds, and the reference can never be null at run time.

\end{enumerate}

With assertions and existing methods like JUnit's \<assertNotNull>, there
is no way of knowing the programmer's intent in using the method.
Different programmers or codebases may use them in different ways.
Guessing wrong would make the Nullness Checker less useful, because it
would either miss real errors or issue warnings where there is no real
error.  Also, different checking tools issue different false warnings that
need to be suppressed, so warning suppression needs to be customized for
each tool rather than inferred from general-purpose code.


As an example of using assertions for defensive programming, some style
guides suggest using assertions or method calls to indicate nullness.  A
programmer might write

\begin{Verbatim}
    String s = ...
    assert s != null;    // or:  assertNotNull(s);   or: checkNotNull(s);
    ... Double.valueOf(s) ...
\end{Verbatim}

A programming error might cause \<s> to be null, in which case the code
would throw an exception at run time.
If the assertion caused the Nullness Checker to assume that \<s> is not
\<null>, then the Nullness Checker would issue no warning for this code.
That would be undesirable, because the whole purpose of the Nullness
Checker is to give a compile-time warning about possible run-time
exceptions.  Furthermore, if the programmer uses assertions for defensive
programming systematically throughout the codebase, then many useful
Nullness Checker warnings would be suppressed.


Because it is important to distinguish between the two uses of assertions
(defensive programming vs.~suppressing warnings), the Checker Framework
introduces the \refmethod{nullness}{NullnessUtils}{castNonNull}{(T)} method.
Unlike existing assertions and
methods, \<castNonNull> is intended only to suppress false warnings that are
issued by the Nullness Checker, not for defensive programming.

If you know that a particular codebase uses
% the \<assert> statement or
a nullness-checking method not for defensive programming but to indicate
facts that are guaranteed to be true (that is, these assertions will never
fail at run time), then you can cause the Nullness Checker to suppress
warnings related to them, just as it does for \<castNonNull>.
Annotate its definition just as
\refmethod{nullness}{NullnessUtils}{castNonNull}{(T)} is annotated (see the
source code for the Checker Framework).
% TODO:
% For an assert statement, XXXXX.
Also, be sure to document the intention in the method's Javadoc, so that
programmers do not
accidentally misuse it for defensive programming.


If you are annotating a codebase that already contains precondition checks,
such as:

\begin{Verbatim}
  public String get(String key, String def) {
    checkNotNull(key, "key"); //NOI18N
    ...
  }
\end{Verbatim}

\noindent
then you should mark the appropriate parameter as \<@NonNull> (which is the
default).  This will prevent the checker from issuing a warning about the
\<checkNotNull> call.


\section{\code{@Raw} annotation for partially-initialized objects\label{raw-partially-initialized}}

An object is
\emph{raw} from the time that its constructor starts until its constructor
finishes.  This is relevant to the Nullness Checker because while the
constructor is executing --- that is, before initialization completes ---
a \<@NonNull>
field may be observed to be null, until that field is set.  In
particular, the Nullness Checker issues a warning for code like this:

\begin{Verbatim}
  public class MyClass {
    private @NonNull Object f;
    public MyClass(int x, int y) {
      // Error because constructor contains no assignment to this.f.
      // By the time the constructor exits, f must be initialized to a non-null value.
    }
    public MyClass(int x) {
      // Error because this.f is accessed before f is initialized.
      // At the beginning of the constructor's execution, accessing this.f
      // yields null, even though field f has a non-null type.
      this.f.toString();
    }
    public MyClass(int x, int y, int z) {
      m();
    }
    public void m() {
      // Error because this.f is accessed before f is initialized,
      // even though the access is not in a constructor.
      // When m is called from the constructor, accessing f yields null,
      // even though field f has a non-null type.
      this.f.toString();
    }
\end{Verbatim}

\noindent
In general, code can depend that field \<f> is not \<null>, because the
field is declared with a \code{@\refclass{nullness/quals}{NonNull}} type.
However, this guarantee does not hold for a partially-initialized object.

The Nullness Checker uses the \code{@\refclass{nullness/quals}{Raw}} annotation to indicate that an object
is not yet fully initialized --- that is, not all its \<@NonNull> fields have been
assigned.  Rawness is mostly relevant within the constructor, or for
references to \code{this} that escape the constructor (say, by being stored
in a field or passed to a method before initialization is complete).  
Use of rawness annotations is rare in most code.

The most common use for the \<@Raw> annotation is for a helper routine that
is called by constructor.  For example:

\begin{Verbatim}
  class MyClass {
    Object field1;
    Object field2;
    Object field3;

    public MyClass(String arg1) {
      this.field1 = arg1;
      init_other_fields();
    }

    // A helper routine that initializes all the fields other than field1
    @EnsuresNonNull({"field2", "field3"})
    private void init_other_fields(@Raw MyClass this) {
      field2 = new Object();
      field3 = new Object();
    }
  }
\end{Verbatim}

For compatibility with Java 7 and earlier, you can write the receiver
parameter in comments (see Section~\ref{annotations-in-comments}):
\begin{Verbatim}
    private void init_other_fields(/*>>> @Raw MyClass this*/) {
\end{Verbatim}

% Most readers can
% skip this section on first reading; you can return to it once you have
% mastered the rest of the Nullness Checker.

\subsection{Rawness qualifiers\label{rawness-qualifiers}}

The rawness hierarchy is shown in Figure~\ref{fig:nullness-hierarchy}.
The rawness hierarchy contains these qualifiers:

\begin{description}

\item[\code{@\refclass{nullness/quals}{Raw}}]
  indicates a type that may contain a partially-initialized object.  In a
  partially-initialized object, fields that are annotated as
  \code{@\refclass{nullness/quals}{NonNull}} may be null because the field
  has not yet been assigned.  Within the constructor,
  \code{this} has \code{@\refclass{nullness/quals}{Raw}} type until all
  the \code{@NonNull} fields have been assigned.
  A partially-initialized object (\code{this} in a constructor) may be
  passed to a helper method or stored in a variable; if so, the method
  receiver, or the field, would have to be annotated as \<@Raw>.

% Cut this?
\item[\code{@\refclass{nullness/quals}{NonRaw}}]
  indicates a type that contains a fully-initialized object.  \code{NonRaw}
  is the default, so there is little need for a programmer to write this
  explicitly.

\item[\code{@\refclass{nullness/quals}{PolyRaw}}]
  indicates qualifier polymorphism over rawness (see
  Section~\ref{qualifier-polymorphism}).

\end{description}

% However, if the constructor makes
% a method call (passing \code{this} as a parameter or the receiver), then
% the called method could observe the object in an illegal state.

If a reference has
\code{@Raw} type, then all of its \code{@NonNull} fields are treated as
\code{@\refclass{nullness/quals}{MonotonicNonNull}}:  when read, they are
treated as being \code{@\refclass{nullness/quals}{Nullable}}, but when
written, they are treated as being
\code{@\refclass{nullness/quals}{NonNull}}.


The rawness hierarchy is orthogonal to the nullness hierarchy.  It
is legal for a reference to be \<@NonNull @Raw>, \<@Nullable @Raw>,
\<@NonNull @NonRaw>, or \<@Nullable @NonRaw>.  The nullness hierarchy tells
you about the reference itself:  might the reference be null?  The rawness
hierarchy tells you about the \<@NonNull> fields in the referred-to object:
might those fields be temporarily null in contravention of their
type annotation?
% It's a figure rather than appearing inline so as not to span page breaks
% in the printed manual.
Figure~\ref{fig:rawness-examples} contains some examples.

\begin{figure}
\begin{tabular}{l|l|l}
Declarations & Expression & Expression's nullness type, or checker error \\ \hline
\begin{minipage}{1.5in}
\begin{Verbatim}
class C {
  @NonNull Object f;
  @Nullable Object g;
  ...
}
\end{Verbatim}
\end{minipage} & & \\ \cline{2-3}
\<@NonNull @NonRaw C a;>
& \<a> & \<@NonNull> \\
& \<a.f> & \<@NonNull> \\
& \<a.g> & \<@Nullable> \\ \cline{2-3}
\<@NonNull @Raw C b;>
& \<b> & \<@NonNull> \\
& \<b.f> & \<@MonotonicNonNull> \\
& \<b.g> & \<@Nullable> \\ \cline{2-3}
\<@Nullable @NonRaw C c;>
& \<c> & \<@Nullable> \\
& \<c.f> & error: deref of nullable \\
& \<c.g> & error: deref of nullable \\ \cline{2-3}
\<@Nullable @Raw C d;>
& \<d> & \<@Nullable> \\
& \<d.f> & error: deref of nullable \\
& \<d.g> & error: deref of nullable \\
\end{tabular}
\caption{Examples of the interaction between nullness and rawness.
  Declarations are shown at the left for reference, but the focus of the
  table is the expressions and their nullness type or error.}
\label{fig:rawness-examples}
\end{figure}


% Does our implementation handle static fields soundly?  NO!  See issue
% #105.  Maybe document this?


\subsection{How an object becomes non-raw\label{becoming-non-raw}}

Within the constructor,
\code{this} starts out with \code{@\refclass{nullness/quals}{Raw}} type.
As soon as all of the \code{@\refclass{nullness/quals}{NonNull}} fields
have been initialized, then \code{this} is treated as non-raw.
% TODO:  (See
% Section~\ref{becoming-non-raw-clarification} for a slight clarification of
% this rule.)

The Nullness Checker issues an error if the constructor fails to initialize
any \code{@NonNull} field.  This ensures that the object is in a legal (non-raw)
state by the time that the constructor exits.
\urldef{\jlsdefiniteassignmenturl}{\url}{http://docs.oracle.com/javase/specs/jls/se7/html/jls-16.html}
\urldef{\jlsfinalvariablesurl}{\url}{http://docs.oracle.com/javase/specs/jls/se7/html/jls-4.html#jls-4.12.4}
This is different than Java's test for definite assignment (see
\ahref{\jlsdefiniteassignmenturl}{JLS ch.16}),
% , which requires that local
% variables (and blank \code{final} fields) must be assigned.  Java does not
% require that non-\code{final} fields be assigned, since
which does not apply to fields (except blank final ones, defined in
\ahref{\jlsdefiniteassignmenturl}{JLS \S 4.12.4}) because fields
have a default value of null.


% and can only be passed to methods when the corresponding parameter is
% annotated with \code{@\refclass{nullness/quals}{Raw}}.  Similar
% restrictions apply to assigning \code{this} to a field.

All \code{@NonNull} fields must either have a
default in the field declaration, or be assigned in the constructor or in a
helper method that the constructor calls.  If
your code initializes (some) fields in a helper method, you will need to
annotate the helper method with an annotation such as
\code{@\refclass{nullness/quals}{EnsuresNonNull}(\{"field1", "field2"\})}
for all the fields that the helper method assigns.
It's a bit odd, but you use that same annotation, \code{@EnsuresNonNull},
to indicate that a primitive field has its value set in a helper method,
which is relevant when you supply the \code{-Alint=uninitialized}
command-line option (see Section~\ref{lint-uninitialized}).

% TODO:
% We need an
%   @EnsuresInitialized("b")
% that is analogous to
%   @EnsuresNonNull("b")
% when we are dealing with a field of primitive type.
% But, for now just use the same annotation, @EnsuresNonNull, for both purposes.


\subsection{Partial initialization\label{partial-initialization}}

So far, we have discussed rawness as if it is an all-or-nothing property:
an object is fully raw until initialization completes, and then it is no
longer raw.  The full truth is a bit more complex:  during the
initialization process, an object can be partially initialized, and as the
object's superclass constructors complete, its rawness changes.  The
Nullness Checker lets you express such properties when necessary.

Consider a simple example:

\begin{Verbatim}
class A {
  Object a;
  A() {
    a = new Object();
  }
}
class B extends A {
  Object b;
  B() {
    super();
    b = new Object();
  }
}
\end{Verbatim}

Consider what happens during execution of \<new B()>.

\begin{enumerate}
\item \<B>'s constructor begins to execute.  At this point, neither the
  fields of \<A> nor those of \<B> have been initialized yet.
\item \<B>'s constructor calls \<A>'s constructor, which begins to execute.
  No fields of \<A> nor of \<B> have been initialized yet.
\item \<A>'s constructor completes.  Now, all the fields of \<A> have been
  initialized, and their invariants (such as that field \<a> is non-null) can be
  depended on.  However, because \<B>'s constructor has not yet completed
  executing, the object being constructed is not yet fully initialized.
  When treated as an \<A> (e.g., if only the \<A> fields are accessed), the
  object is initialized (non-raw), but when treated as a \<B>, the object
  is still raw.
\item \<B>'s constructor completes.  The object is fully initialized
  (non-raw), if \<B>'s constructor was invoked via a \<new B()>
  expression.  On the other hand, if there was a \<class C extends B \{
  ... \}>, and \<B>'s constructor had been invoked from that, then the
  object currently under construction would \emph{not} be fully initialized
  --- it would only be initialized when treated as an \<A> or a \<B>, but
  not when treated as a \<C>.
\end{enumerate}

At any moment during initialization, the superclasses of a given class
can be divided into those that have completed initialization and those that
have not yet completed initialization.  More precisely, at any moment there
is a point in the class hierarchy such that all the classes above that
point are fully initialized, and all those below it are not yet
initialized.  As initialization proceeds, this dividing line between the
initialized and raw classes moves down the type hierarchy.

The Nullness Checker lets you indicate where the dividing line is between
the initialized and non-initialized classes.
You have two equivalent ways to indicate the dividing line:  \<@Raw>
indicates the first class \emph{below} the dividing line, or
\<@NonRaw(\emph{classliteral})> indicates the first class \emph{above} the
dividing line.

When you write \code{@\refclass{nullness/quals}{Raw} MyClass x;}, that
means that variable \<x> is initialized for all superclasses of \<MyClass>,
and (possibly) uninitialized for \<MyClass> and all subclasses.

When you write \code{@\refclass{nullness/quals}{NonRaw(Foo.class)} MyClass
  x;}, that means that variable \<x> is initialized for \<Foo> and all its
superclasses, and (possibly) uninitialized for all subclasses of \<Foo>.

If \<A> is a direct superclass of \<B> (as in the example above), then 
\<@Raw A x;> and \<@NonRaw(B.class) A x;> are equivalent declarations.
Neither one is the same as \<@NonRaw A x;>, which indicates that, whatever
the actual class of the object that \<x> refers to, that object is fully
initialized.  Since \<@NonRaw> (with no argument) is the default, you will
rarely see it written.

\label{becoming-non-raw-clarification}

We can now state a clarification of Section~\ref{becoming-non-raw}'s rule
for an object becoming non-raw.
As soon as all of the \code{@\refclass{nullness/quals}{NonNull}} fields
have been initialized, then \code{this} is treated as
\code{@\refclass{nullness/quals}{NonRaw}(\emph{typeofthis})}, rather than
treated as simply 
\code{@\refclass{nullness/quals}{NonRaw}}.

The example above lists 4 moments during construction.  At those moments,
the type of the object being constructed is:

\begin{enumerate}
\item
  \<@Raw Object>
\item
  \<@Raw Object>
\item
  \<@NonRaw(A.class) A>
  %% Not quite equivalent because the Java (non-qualified) type differs
  % ; equivalently, \<@Raw B>
\item
  \<@NonRaw(B.class) B>
\end{enumerate}

\paragraph{Example}

As another example, consider the following 12 declarations:

\begin{Verbatim}
    @Raw Object rO;
    @NonRaw(Object.class) Object nroO;
    Object o;

    @Raw A rA;
    @NonRaw(Object.class) A nroA;  // same as "@Raw A"
    @NonRaw(A.class) A nraA;
    A a;

    @NonRaw(Object.class) B nroB;
    @Raw B rB;
    @NonRaw(A.class) B nraB;  // same as "@Raw B"
    @NonRaw(B.class) B nrbB;
    B b;
\end{Verbatim}

In the following table, the type in cell C1 is a supertype of the type in
cell C2 if:  C1 is at least as high and at least as far left in the table
as C2 is.  For example, \<nraA>'s type is a supertype of those of \<rB>,
\<nraB>, \<nrbB>, \<a>, and \<b>.  (The empty cells on the top row are real
types, but are not expressible.  The other empty cells are not interesting
types.)

\noindent
\begin{tabular}{|c|c|c|}

\hline
    \<@Raw Object rO;>
&
& 
\\
\hline

    \<@NonRaw(Object.class) Object nroO;>
&
\begin{minipage}{2in}
\begin{Verbatim}
@Raw A rA;
@NonRaw(Object.class) A nroA;
\end{Verbatim}
\end{minipage}
&
    \<@NonRaw(Object.class) B nroB;>
\\
\hline

&
    \<@NonRaw(A.class) A nraA;>
&
\begin{minipage}{1.75in}
\begin{Verbatim}
@Raw B rB;
@NonRaw(A.class) B nraB;
\end{Verbatim}
\end{minipage}
\\
\hline

&
&
    \<@NonRaw(B.class) B nrbB;>
\\
\hline

    \<Object o;>
&
    \<A a;>
&
    \<B b;>
\\
\hline
\end{tabular}



% \urldef{\jlsconstructorbodyurl}{\url}{http://docs.oracle.com/javase/specs/jls/se7/html/jls-8.html#jls-8.8.7}
% (Recall that the superclass constructor is called on the first line, or is
% inserted automatically by the compiler before the first line, see
% \ahref{\jlsconstructorbodyurl}{JLS \S8.8.7}.)



\subsection{More details about rawness checking\label{rawness-checking}}


\paragraph{Suppressing warnings}

You can suppress warnings related to partially-initialized objects with
\<@SuppressWarnings("rawness")>.  Do not confuse this with the unrelated
\<@SuppressWarnings("rawtypes")> annotation for non-instantiated generic types!


\paragraph{Checking initialization of all fields, not just \code{@NonNull} ones}

When the \code{-Alint=uninitialized} command-line option is provided, then
an object is considered raw until \emph{all} its fields are assigned, not
just the \code{@NonNull} ones.  See Section~\ref{lint-uninitialized}.


\paragraph{Use of method annotations}

A method with a raw receiver often assumes that a few fields (but not all
of them) are non-null, and sometimes sets some more fields to non-null
values.  To express these concepts, use the
\code{@\refclass{nullness/quals}{RequiresNonNull}},
\code{@\refclass{nullness/quals}{EnsuresNonNull}}, and
\code{@\refclass{nullness/quals}{EnsuresNonNullIf}} method annotations;
see Section~\ref{nullness-method-annotations}.


% Should we change the terminology?
\paragraph{The terminology ``raw''}

The name ``raw'' comes from a research paper that proposed this
approach~\cite{FahndrichL2003}.
A better name might have been ``not yet initialized'' or ``partially
initialized'', but the term ``raw'' is now well-known.
The \code{@\refclass{nullness/quals}{Raw}}
annotation has nothing to do with the raw types of Java Generics.


\section{Map key annotations\label{map-keys}}

Java's
\sunjavadoc{java/util/Map.html#get(java.lang.Object)}{\code{Map.get}}
method always has the possibility to return null, if the key is not in the
map.

A call \<mymap.get(mykey)> returns non-null if two conditions are satisfied:
\begin{enumerate}
\item \<mymap>'s values are all non-null; that is, \<mymap> was
  declared as \code{Map<\emph{KeyType}, @NonNull \emph{ValueType}>}.  Note
  that \<@NonNull> is the default type, so it need not be written explicitly.
\item \<mykey> is a key in \<mymap>; that is, \<mymap.containsKey(mykey)>
  returns \<true>.  You express this fact to the Nullness Checker by
  declaring \<mykey> as \<@KeyFor("mymap") \emph{KeyType} mykey>.  For a
  local variable, the \<@KeyFor("mymap")> type qualifier can generally be
  inferred.
\end{enumerate}
\noindent
If either of these two conditions is violated, then \<mymap.get(mykey)> has
the possibility of returning null.


  Thus, to guarantee that the value returned from \code{Map.get} is
non-null, it is necessary that the map contains only non-null values,
\emph{and} the key is in the map.
The \code{@\refclass{nullness/quals}{KeyFor}} annotation states the latter
property.

If a type is annotated as \code{@KeyFor("m")}, then any value v with that type
is a key in Map \<m>.  Another way of saying this is that the expression
\code{m.containsKey(v)} evaluates to true.

You usually do not have to write \code{@KeyFor} explicitly, because the
checker infers it based on usage patterns, such as calls to
\code{containsKey} or iteration over a map's
\sunjavadoc{java/util/Map.html#keySet()}{\textrm{key set}}.

One usage pattern where you \emph{do} have to write \<@KeyFor> is for a
user-managed collection that is a subset of the key set:

\begin{Verbatim}
Map<String, Object> m;
Set<@KeyFor("m") String> matchingKeys; // keys that match some criterion
for (@KeyFor("m") String k : matchingKeys) {
  ... m.get(k) ...  // known to be non-null
}
\end{Verbatim}

As with any annotation, use of the \<@KeyFor> annotation may force you to
slightly refactor your code.  For example, this would be illegal:

\begin{Verbatim}
  Map<K,V> m;
  Collection<@KeyFor("m") K> coll;
  coll.add(x);   // compiler error, because the @KeyFor annotation is violated
  m.put(x, ...);
\end{Verbatim}

\noindent
but this would be OK (no compiler error):

\begin{Verbatim}
  Collection<@KeyFor("m") K> coll;
  m.put(x, ...);
  coll.add(x);
\end{Verbatim}


Because the \<@KeyFor> type hierarchy is independent from the nullness and
rawness hierarchies, it uses a different warning suppression key.
You can suppress warnings related to map keys with
\<@SuppressWarnings("keyfor")>.


\section{Additional details\label{nullness-additional-details}}

The Nullness Checker does some special checks in certain circumstances, in
order to soundly reduce the number of warnings that it produces.

For example, a call to 
\sunjavadoc{java/lang/System.html#getProperty(java.lang.String)}{System.getProperty(String)}
can return null in general, but it will not return null if the argument is
one of the built-in-keys listed in the documentation of 
\sunjavadoc{java/lang/System.html#getProperties()}{System.getProperties()}.
The Nullness Checker is aware of this fact, so you do not have to suppress
a warning for a call like \<System.getProperty("line.separator")>.  The
warning is still issued for code like this:

\begin{Verbatim}
  final String s = "line.separator";
  nonNullvar = System.getProperty(s);
\end{Verbatim}

\noindent
though that case could be handled as well, if desired.
(Suppression of the warning is, strictly speaking, not sound, because a
library that your code calls, or your code itself, could perversely change
the system properties; the Nullness Checker assumes this bizarre coding
pattern does not happen.)


\section{Examples\label{nullness-example}}

\subsection{Tiny examples\label{nullness-tiny-examples}}

To try the Nullness Checker on a source file that uses the \code{@\refclass{nullness/quals}{NonNull}} qualifier,
use the following command (where \code{javac} is the JSR 308 compiler that
is distributed with the Checker Framework):

\begin{Verbatim}
  javac -processor checkers.nullness.NullnessChecker examples/NullnessExample.java
\end{Verbatim}

\noindent
Compilation will complete without warnings.

To see the checker warn about incorrect usage of annotations (and therefore the
possibility of a null pointer exception at run time), use the following command:

\begin{Verbatim}
  javac -processor checkers.nullness.NullnessChecker examples/NullnessExampleWithWarnings.java
\end{Verbatim}

\noindent
The compiler will issue three warnings regarding violation of the semantics of
\code{@\refclass{nullness/quals}{NonNull}}.
% in the \code{NonNullExampleWithWarnings.java} file.


\subsection{Annotated library\label{nullness-annotated-library}}

Some libraries that are annotated with nullness qualifiers are:

\begin{itemize}
\item
The Nullness Checker itself.

\item
The
\ahref{http://code.google.com/p/plume-lib/}{Plume-lib library}.
Run the command \code{make check-nullness}.


\item
The
\ahref{http://groups.csail.mit.edu/pag/daikon/}{Daikon invariant detector}.
Run the command \code{make check-nullness}.

% \item
% The annotation scene library.
% To run the Nullness Checker on the annotation scene library,
% % TODO: how does one do this?
% first download the scene library suite (which includes build
% dependencies for the scene library as well as its source code) and extract it
% into your Checker Framework installation. The checker can then be run on the annotation
% scene library with Apache Ant using the following commands:
%
% \begin{Verbatim}
%   cd checkers
%   ant -f scene-lib-test.xml
% \end{Verbatim}
%
% % \noindent
% % where \code{checkers} is the location of the Checker Framework installation.
%
% You can view the annotated source code, which contains \code{@\refclass{nullness/quals}{NonNull}} annotations, in
% the
% %BEGIN LATEX
% \begin{smaller}
% %END LATEX
% \code{checkers/scene-lib-test/src/annotations/}
% %BEGIN LATEX
% \end{smaller}
% %END LATEX
% directory.

\end{itemize}


\section{Tips for getting started\label{nullness-getting-started}}

Here are some tips about getting started using the Nullness Checker on a
legacy codebase.  For more generic advice (not specific to the Nullness
Checker), see Section~\ref{get-started-with-legacy-code}.

Your goal is to add \code{@\refclass{nullness/quals}{Nullable}} annotations
to the types of any variables that can be null.  (The default is to assume
that a variable is non-null unless it has a \code{@Nullable} annotation.)
Then, you will run the Nullness Checker.  Each of its errors indicates
either a possible null pointer exception, or a wrong/missing annotation.
When there are no more warnings from the checker, you are done!

We recommend that you start by searching the code for occurrences of
\code{null} in the following locations; when you find one, write the
corresponding annotation:

\begin{itemize}
\item
  in Javadoc:  add \code{@Nullable} annotations to method signatures (parameters and return types).
% Search for "\*.*\bnull\b"
\item
  \code{return null}:  add a \code{@Nullable} annotation to the return type
  of the given method.
% Search for "return null" and "return.*? null" and "return.*: null"
\item
  \code{\emph{param} == null}:  when a formal parameter is compared to
  \code{null}, then in most cases you can add a \code{@Nullable} annotation
  to the formal parameter's type
\item
  \code{\emph{TypeName} \emph{field} = null;}:  when a field is initialized to
  \code{null} in its declaration, then it needs either a
  \code{@\refclass{nullness/quals}{Nullable}} or a
  \code{@\refclass{nullness/quals}{MonotonicNonNull}} annotation.  If the field
  is always set to a non-null value in the constructor, then you can just
  change the declaration to \code{\emph{Type} \emph{field};}, without an
  initializer, and write no type annotation (because the default is
  \<@NonNull>).
\item
  declarations of \<contains>, \<containsKey>, \<containsValue>, \<equals>,
  \<get>, \<indexOf>, \<lastIndexOf>, and \<remove> (with \<Object> as the
  argument type):
  change the argument type to \<@Nullable Object>; for \<remove>, also change
  the return type to \<@Nullable Object>.
% Emacs code for the argument types:
% ;;NOT: (tags-query-replace " apply(Object " " apply(/*@Nullable*/ Object ")
% (tags-query-replace " \\(get\\|equals\\|remove\\|contains\\|containsValue\\|containsKey\\|indexOf\\|lastIndexOf\\)(Object " " \\1(/*@Nullable*/ Object ")

\end{itemize}

\noindent
You should ignore all other occurrences of \code{null} within a method
body.  In particular, you (almost) never need to annotate local variables.

Only after this step should you run \code{ant} to invoke
the Nullness Checker.  The reason is that it is quicker to search for
places to change than to repeatedly run the checker and fix the errors it
tells you about, one at a time.

Here are some other tips:
\begin{itemize}
\item
    In any file where you write an annotation such as \code{@Nullable},
    don't forget to add \code{import checkers.nullness.quals.*;}.
\item
    To indicate an array that can be null, write, for example: \code{int
      @Nullable []}. \\
    By contrast, \code{@Nullable Object []} means a non-null array that
    contains possibly-null objects.
\item
    If you know that a particular variable is definitely not null, but the
    Nullness Checker cannot figure it out, then you can tell it by writing
    an assertion (see Section~\ref{suppressing-warnings}):
\begin{Verbatim}
assert var != null : "@SuppressWarnings(nullness)";
\end{Verbatim}
\item
    To indicate that a routine returns the same value every time it is
    called, use \code{@\refclass{dataflow/quals}{Pure}} (see Section~\ref{nullness-method-annotations}).
\item
    To indicate a method precondition (a contract stating the conditions
    under which a client is allowed to call it), you can use annotations
    such as \code{@\refclass{nullness/quals}{RequiresNonNull}} (see Section~\ref{nullness-method-annotations}.
\end{itemize}



\section{Other tools for nullness checking\label{nullness-related-work}}

\newcommand{\linktoNonNull}{\code{\refclass{nullness/quals}{NonNull}}}
\newcommand{\linktoNullable}{\code{\refclass{nullness/quals}{Nullable}}}

The Checker Framework's nullness annotations are similar to annotations used
in IntelliJ IDEA, FindBugs, JML, the JSR 305 proposal, NetBeans, and other tools.  Also
see Section~\ref{other-tools} for a comparison to other tools.

You might prefer to use the Checker Framework because it has a more
powerful analysis that can warn you about more null pointer errors in your
code.

If your code is already annotated with a different nullness
annotation, you can reuse that effort.  The Checker Framework comes with
cleanroom re-implementations of annotations from other tools.  It treats
them exactly as if you had written the corresponding annotation from the
Nullness Checker, as described in Figure~\ref{fig:nullness-refactoring}.


% These lists should be kept in sync with NullnessAnnotatedTypeFactory.java .
\begin{figure}
\begin{center}
% The ~ around the text makes things look better in Hevea (and not terrible
% in LaTeX).
\begin{tabular}{ll}
\begin{tabular}{|l|}
\hline
 ~com.sun.istack.NotNull~ \\ \hline
 ~edu.umd.cs.findbugs.annotations.NonNull~ \\ \hline
 ~javax.annotation.Nonnull~ \\ \hline
 ~javax.validation.constraints.NotNull~ \\ \hline
 ~org.eclipse.jdt.annotation.NonNull~ \\ \hline
 ~org.jetbrains.annotations.NotNull~ \\ \hline
 ~org.netbeans.api.annotations.common.NonNull~ \\ \hline
 ~org.jmlspecs.annotation.NonNull~ \\ \hline
\end{tabular}
&
$\Rightarrow$
~checkers.nullness.quals.NonNull~
\\
\
\\
\begin{tabular}{|l|l|}
\hline
 ~com.sun.istack.Nullable~ \\ \hline
 ~edu.umd.cs.findbugs.annotations.Nullable~ \\ \hline
 ~edu.umd.cs.findbugs.annotations.CheckForNull~ \\ \hline
 ~edu.umd.cs.findbugs.annotations.UnknownNullness~ \\ \hline
 ~javax.annotation.Nullable~ \\ \hline
 ~javax.annotation.CheckForNull~ \\ \hline
 ~org.eclipse.jdt.annotation.Nullable~ \\ \hline
 ~org.jetbrains.annotations.Nullable~ \\ \hline
 ~org.netbeans.api.annotations.common.CheckForNull~ \\ \hline
 ~org.netbeans.api.annotations.common.NullAllowed~ \\ \hline
 ~org.netbeans.api.annotations.common.NullUnknown~ \\ \hline
 ~org.jmlspecs.annotation.Nullable~ \\ \hline
\end{tabular}
&
$\Rightarrow$
~checkers.nullness.quals.Nullable~
\end{tabular}
\end{center}
%BEGIN LATEX
\vspace{-1.5\baselineskip}
%END LATEX
\caption{Correspondence betwee other nullness annotations and the
  Checker Framework's annotations.}
\label{fig:nullness-refactoring}
\end{figure}

%% Removed, because it's tedious and should be obvious to a decent programmer.
% Your IDE may be able to do that for you.  Alternately, do the following:
% \begin{enumerate}
% \item
%   replace \<@Nonnull> by \<@NonNull> (note capitalization difference)
% \item
%   replace \<@CheckForNull> by \<@Nullable>
% \item
%   replace \<@UnknownNullness> by \<@Nullable>
% \item
%   convert each single-type import statement (without a ``\<*>'' character)
%    according to the table above.
% \item
%   convert each on-demand import statements, such as ``\<import
%    edu.umd.cs.findbugs.annotations.*;>''.
% \begin{itemize}
%    \item
%   One approach is to change it into a set of single-type imports,
%       then convert the relevant ones.
%    \item
%   Another approach is to change it according to the table above, then
%       try to compile and re-introduce the single-type imports as necessary.
% \end{itemize}
%    These approaches let you continue to use other annotations in the
%    \<edu.umd.cs.findbugs.annotations> package, even though you are not using
%    its nullness annotations.
% \end{enumerate}


Alternately, the Checker Framework can process those other annotations (as
well as its own, if they also appear in your program).  The Checker
Framework has its own definition of the annotations on the left side of
Figure~\ref{fig:nullness-refactoring}, so that they can be used as type
qualifiers.  The Checker Framework interprets them according to the right
side of Figure~\ref{fig:nullness-refactoring}.

The Checker Framework may issue more or fewer errors than another tool.
This is expected, since each tool uses a different analysis.  Remember that
the Checker Framework aims at soundness:  it aims to never miss a possible
null dereference, while at the same time limiting false reports.  Also,
note FindBugs's non-standard meaning for \<@Nullable>
(Section~\ref{findbugs-nullable}).

Because some of the names are the same (\<NonNull>, \<Nullable>), you can
import at most one of the annotations with
conflicting names; the other(s) must be written out fully rather than
imported.

Note that some older tools interpret array and vararg declarations
inconsistently with the Java specification.  For example, they might
interpret \<@NonNull Object []> as ``non-null array of objects'', rather
than as ``array of non-null objects'' which is the correct Java
interpretation.  Such an interpretation is unfortunate and confusing.  See
Section~\ref{faq-array-syntax-meaning} for some more details about this
issue.


\subsection{Which tool is right for you?\label{choosing-nullness-tool}}

Different tools are appropriate in different circumstances.  Here is a
brief comparison with FindBugs, but similar points apply to other tools.

The Checker Framework has a more powerful nullness analysis; FindBugs misses
some real
errors.  However, FindBugs does not require you to annotate your code as
thoroughly as the Checker Framework does.  Depending on the importance of
your code, you may desire:  no nullness checking, the cursory checking of
FindBugs, or the thorough checking of the Checker Framework.  You might
even want to ensure that both tools run, for example if your coworkers or
some other organization are still using FindBugs.  If you know that you
will eventually want to use the Checker Framework, there is no point using
FindBugs first; it is easier to go straight to using the Checker Framework.

FindBugs can find other errors in addition to nullness errors; here
we focus on its nullness checks.  Even if you use FindBugs for its other
features, you may want to use the Checker Framework for analyses that can
be expressed as pluggable type-checking, such as detecting nullness errors.

Regardless of whether you wish to use the FindBugs nullness analysis, you
may continue running all of the other FindBugs analyses at the same time as
the Checker Framework; there are no interactions among them.

If FindBugs (or any other tool) discovers a nullness error that the Checker
Framework does not, please report it to us (see
Section~\ref{reporting-bugs}) so that we can enhance the Checker Framework.



\subsection{Incompatibility note about FindBugs \tt{@Nullable}\label{findbugs-nullable}}

FindBugs has a non-standard definition of \<@Nullable>.  FindBugs's treatment is not
documented in its own
\ahref{http://findbugs.sourceforge.net/api/edu/umd/cs/findbugs/annotations/Nullable.html}{Javadoc};
it is different from the definition of \<@Nullable> in every other tool for
nullness analysis; it means the same thing as \<@NonNull> when applied to a
formal parameter; and it invariably surprises programmers.  Thus, FindBugs's
\<@Nullable> is detrimental rather than useful as documentation.
In practice, your best bet is to not rely on FindBugs for nullness analysis,
even if you find FindBugs useful for other purposes.

You can skip the rest of this section unless you wish to learn more details.

FindBugs suppresses all warnings at uses of a \<@Nullable> variable.
(You have to use \<@CheckForNull> to
indicate a nullable variable that FindBugs should check.)  For example:

\begin{Verbatim}
     // declare getObject() to possibly return null
     @Nullable Object getObject() { ... }

     void myMethod() {
       @Nullable Object o = getObject();
       // FindBugs issues no warning about calling toString on a possibly-null reference!
       o.toString();
     }
\end{Verbatim}

\noindent
The Checker Framework does not emulate this non-standard behavior of
FindBugs, even if the code uses FindBugs annotations.

With FindBugs, you annotate a declaration, which suppresses checking at
\emph{all} client uses, even the places that you want to check.
It is better to suppress warnings at only the specific client uses
where the value is known to be non-null; the Checker Framework supports
this, if you write \<@SuppressWarnings> at the client uses.
The Checker Framework also supports suppressing checking at all client uses,
by writing a \<@SuppressWarnings> annotation at the declaration site.
Thus, the Checker Framework supports both use cases, whereas FindBugs
supports only one and gives the programmer less flexibility.

In general, the Checker Framework will issue more warnings than FindBugs,
and some of them may be about real bugs in your program.
See Section~\ref{suppressing-warnings-nullness} for information about
suppressing nullness warnings.

(FindBugs made a poor choice of names.  The choice of names should make a
clear distinction between annotations that specify whether a reference is
null, and annotations that suppress false warnings.  The choice of names
should also have been consistent for other tools, and intuitively clear to
programmers.  The FindBugs choices make the FindBugs annotations less
helpful to people, and much less useful for other tools.  As a separate
issue, the FindBugs
analysis is also very imprecise.  For type-related analyses, it is best to
stay away from the FindBugs nullness annotations, and use a more capable
tool like the Checker Framework.)



% As background, here is an explanation of the (sometimes surprising)
% semantics of the FindBugs nullness annotations.
%
%  * edu.umd.cs.findbugs.annotations.NonNull     javax.annotation.Nonnull
%    These mean the obvious thing:   the reference is never null.
%
%  * edu.umd.cs.findbugs.annotations.Nullable     javax.annotation.Nullable
%    This means that the value may be null, but that *all warnings should be
%    suppressed* regarding its use.  In other words, the value is really
%    nullable, but clients should treat it as non-null.  For example:
%
%      // declare getObject() to possibly return null
%      @Nullable Object getObject() { ... }
%
%      // FindBugs issues no warning about calling toString on a possibly-null reference
%      getObject().toString();
%
%    In the Checker Framework, this corresponds to declaring the method
%    return value as @Nullable, then suppressing warnings at client uses
%    that are known to be non-null.  An easy way to suppress the warning
%    is by adding an assert statement about the return value.
%
%    (Alternately, you could declare the method return value as @NonNull, and
%    suppress warnings within the method definition where it returns null,
%    but this approach is not recommended because the @NonNull annotation on
%    the return value would be misleading, and warnings should be suppressed
%    at particular sites where they are known to be unnecessary, not at all
%    call sites whatsoever.)
%
%  * edu.umd.cs.findbugs.annotations.CheckForNull      javax.annotation.CheckForNull
%    This means that the value may be null.  To avoid a NullPointerException,
%    every client should check nullness before dereferencing the value.
%    In the Checker Framework, this corresponds to @Nullable.



% LocalWords:  NonNull plugin quals un NonNullExampleWithWarnings java ahndrich
% LocalWords:  NotNull IntelliJ FindBugs Nullable TODO Alint nullable NNEL JSR
% LocalWords:  DefaultLocation Nullness PolyNull nullness AnnotateNullable JLS
% LocalWords:  Daikon JastAdd javac DefaultQualifier boolean MyEnumType NonRaw
% LocalWords:  NullnessAnnotatedTypeFactory NullnessVisitor MonotonicNonNull
% LocalWords:  inferencer Nonnull CheckForNull UnknownNullness rawtypes de ch
% LocalWords:  castNonNull NullnessUtils assertNotNull codebases checkNotNull
% LocalWords:  Nullability typeargs nulltest EnsuresNonNullIf listFiles faq
% LocalWords:  isDirectory AssertionError intraprocedurally SuppressWarnings rB
% LocalWords:  FindBugs's getObject RequiresNonNull EnsuresNonNull KeyFor
% LocalWords:  AssertParametersNonNull nonnull EnsuresNonNull ReadOnly arg
% LocalWords:  keySet getField keyfor param TypeName containsValue indexOf nraA
% LocalWords:  lastIndexOf deref getProperty getProperties classliteral MyClass
% LocalWords:  typeofthis nraB nrbB rO nroO nroB 5cm IGJ containsKey enum
% LocalWords:  JUnit's 5in field1 field2 superclasses Foo C1 C2 2in 75in PolyRaw

\htmlhr
\chapter{Interning Checker\label{interning-checker}}

\urldef{\jlsboxingurl}{\url}{http://docs.oracle.com/javase/specs/jls/se7/html/jls-5.html#jls-5.1.7}

If the Interning Checker issues no errors for a given program, then all
reference equality tests (i.e., all uses of ``\code{==}'') are proper;
that is,
\code{==} is not misused where \code{equals()} should have been used instead.

Interning is a design pattern in which the same object is used whenever two
different objects would be considered equal.  Interning is also known as
canonicalization or hash-consing, and it is related to the flyweight design
pattern.
Interning has two benefits:  it can save memory, and it can speed up testing for
equality by permitting use of \code{==}.

The Interning Checker prevents two types of errors in your code.  First, 
\code{==} should be used
only on interned values; using \code{==} on
non-interned values can result in subtle bugs.  For example:

\begin{Verbatim}
  Integer x = new Integer(22);
  Integer y = new Integer(22);
  System.out.println(x == y);  // prints false!
\end{Verbatim}

\noindent
The Interning Checker helps programmers to prevent such bugs.
Second, 
the Interning Checker also helps to prevent performance problems that result
from failure to use interning.
(See Section~\ref{checker-guarantees} for caveats to the checker's guarantees.)

Interning is such an important design pattern that Java builds it in for
these types: \<String>, \<Boolean>, \<Byte>, \<Character>, \<Integer>,
\<Short>.  Every string literal in the program is guaranteed to be interned
(\ahref{\url{http://docs.oracle.com/javase/specs/jls/se7/html/jls-3.html#jls-3.10.5}}{JLS
  \S3.10.5}), and the
\sunjavadoc{java/lang/String.html#intern()}{String.intern()} method
performs interning for strings that are computed at run time.
The \<valueOf> methods in wrapper classes always (\<Boolean>, \<Byte>) or
sometimes (\<Character>, \<Integer>, \<Short>) return an interned result
(\ahref{\jlsboxingurl}{JLS \S5.1.7}).
Users can also write their own interning methods for other types.

It is a proper optimization to use \code{==}, rather than \code{equals()},
whenever the comparison is guaranteed to produce the same result --- that
is, whenever the comparison is never provided with two different objects
for which \code{equals()} would return true.  Here are three reasons that
this property could hold:

\begin{enumerate}
\item
  Interning.  A factory method ensures that, globally, no two different
  interned objects are \code{equals()} to one another.  (In some cases
  other, non-interned objects of the class might be \code{equals()} to one
  another; in other cases, every object of the class is interned.)
  Interned objects should always be immutable.
\item
  Global control flow.  The program's control flow is such that the
  constructor for class $C$ is called a limited number of times, and with
  specific values that ensure the results are not \code{equals()} to one
  another.  Objects of class $C$ can always be compared with \code{==}.
  Such objects may be mutable or immutable.
\item
  Local control flow.  Even though not all objects of the given type may be
  compared with \code{==}, the specific objects that can reach a given
  comparison may be.  For example, suppose that an array contains no
  duplicates.  Then testing to find the index of a given element that is
  known to be in the array can use \code{==}.
\end{enumerate}

To eliminate Interning Checker errors, you will need to annotate the
declarations of any expression used as an argument to \code{==}.
Thus, the Interning Checker
could also have been called the Reference Equality Checker.  In the
future, the checker will include annotations that target the non-interning
cases above, but for now you need to use \<@Interned>, \<@UsesObjectEquals>
(which handles a surprising number of cases), and/or
\<@SuppressWarnings>.

To run the Interning Checker, supply the
\code{-processor org.checkerframework.checker.interning.InterningChecker}
command-line option to javac.  For examples, see Section~\ref{interning-example}.


\section{Interning annotations\label{interning-annotations}}

These qualifiers are part of the Interning type system:

\begin{description}

\item[\refqualclass{checker/interning/quals}{Interned}]
  indicates a type that includes only interned values (no non-interned
  values).

\item[\refqualclass{checker/interning/quals}{PolyInterned}]
  indicates qualifier polymorphism.  For a description of
  \refqualclass{checker/interning/quals}{PolyInterned}, see
  Section~\ref{qualifier-polymorphism}.

\item[\refqualclass{checker/interning/quals}{UsesObjectEquals}]
  is a class (not type) annotation that indicates that this class's
  \<equals> method is the same as that of \<Object>.  In other words,
  neither this class nor any of its superclasses overrides the \<equals>
  method.  Since \<Object.equals> uses reference equality, this means that
  for such a class, \<==> and \<equals> are equivalent, and so the
  Interning Checker does not issue errors or warnings for either one.

\end{description}


\section{Annotating your code with \code{@Interned}\label{annotating-with-interned}}

\begin{figure}
\includeimage{interning}{2.5cm}
\caption{Type hierarchy for the Interning type system.}
\label{fig-interning-hierarchy}
\end{figure}

In order to perform checking, you must annotate your code with the \refqualclass{checker/interning/quals}{Interned}
type annotation, which indicates a type for the canonical representation of an
object:

\begin{Verbatim}
            String s1 = ...;  // type is (uninterned) "String"
  @Interned String s2 = ...;  // Java type is "String", but checker treats it as "@Interned String"
\end{Verbatim}

The type system enforced by the checker plugin ensures that only interned
values can be assigned to \code{s2}.

To specify that \emph{all} objects of a given type are interned, annotate the
class declaration:

\begin{Verbatim}
  public @Interned class MyInternedClass { ... }
\end{Verbatim}

This is equivalent to annotating every use of \code{MyInternedClass}, in a
declaration or elsewhere.  For example, \code{enum} classes are implicitly
so annotated.


\subsection{Implicit qualifiers\label{interning-implicit-qualifiers}}

As described in Section~\ref{effective-qualifier}, the Interning Checker
adds implicit qualifiers, reducing the number of annotations that must
appear in your code.
For example, String literals and the null literal are always considered interned, and
object creation expressions (using \code{new}) are never considered
\refqualclass{checker/interning/quals}{Interned} unless they are annotated as such, as in

%BEGIN LATEX
\begin{smaller}
%END LATEX
\begin{Verbatim}
@Interned Double internedDoubleZero = new @Interned Double(0); // canonical representation for Double zero
\end{Verbatim}
%BEGIN LATEX
\end{smaller}
%END LATEX

For a complete description of all implicit interning qualifiers, see the
Javadoc for \refclass{checker/interning}{InterningAnnotatedTypeFactory}.


\section{What the Interning Checker checks\label{interning-checks}}

Objects of an \refqualclass{checker/interning/quals}{Interned} type may be safely compared using the ``\code{==}''
operator.

The checker issues an error in two cases:

\begin{enumerate}

\item
  When a reference (in)equality operator (``\code{==}'' or ``\code{!=}'')
  has an operand of non-\refqualclass{checker/interning/quals}{Interned} type.

\item
  When a non-\refqualclass{checker/interning/quals}{Interned} type is used where an \refqualclass{checker/interning/quals}{Interned} type
  is expected.

\end{enumerate}

This example shows both sorts of problems:

\begin{Verbatim}
            Date  date;
  @Interned Date idate;
  ...
  if (date == idate) { ... }  // error: reference equality test is unsafe
  idate = date;               // error: idate's referent may no longer be interned
\end{Verbatim}

\label{lint-dotequals}

The checker also issues a warning when \code{.equals} is used where
\code{==} could be safely used.  You can disable this behavior via the
javac \code{-Alint} command-line option, like so: \code{-Alint=-dotequals}.

For a complete description of all checks performed by
  the checker, see the Javadoc for
  \refclass{checker/interning}{InterningVisitor}.

\label{checking-class}
You can also restrict which types the checker should examine and type-check,
using the \code{-Acheckclass} option.  For example, to find only the
interning errors related to uses of \code{String}, you can pass
\code{-Acheckclass=java.lang.String}.  The Interning Checker always checks all
subclasses and superclasses of the given class.


\subsection{Limitations of the Interning Checker\label{interning-limitations}}

% There is no point to linking to the Javadoc for the valueOf methods,
% which don't discuss interning.

The Interning Checker conservatively assumes that the \<Character>, \<Integer>,
and \<Short> \<valueOf> methods return a non-interned value.  In fact, these
methods sometimes return an interned value and sometimes a non-interned
value, depending on the run-time argument (\ahref{\jlsboxingurl}{JLS
\S5.1.7}).  If you know that the run-time argument to \<valueOf> implies that
the result is interned, then you will need to suppress an error.  (An
alternative would be to enhance the Interning Checker to estimate the upper
and lower bounds on char, int, and short values so that it can more
precisely determine whether the result of a given \<valueOf> call is
interned.)



\section{Examples\label{interning-example}}

To try the Interning Checker on a source file that uses the \refqualclass{checker/interning/quals}{Interned} qualifier,
use the following command (where \code{javac} is the JSR 308 compiler that
is distributed with the Checker Framework):

\begin{Verbatim}
  javac -processor org.checkerframework.checker.interning.InterningChecker examples/InterningExample.java
\end{Verbatim}

\noindent
Compilation will complete without errors or warnings.

To see the checker warn about incorrect usage of annotations, use the following
command:

\begin{Verbatim}
  javac -processor org.checkerframework.checker.interning.InterningChecker examples/InterningExampleWithWarnings.java
\end{Verbatim}

\noindent
The compiler will issue an error regarding violation of the semantics of
\refqualclass{checker/interning/quals}{Interned}.
% in the \code{InterningExampleWithWarnings.java} file.


The Daikon invariant detector
(\myurl{http://plse.cs.washington.edu/daikon/}) is also annotated with
\refqualclass{checker/interning/quals}{Interned}.  From directory \code{java},
run \code{make check-interning}.



\section{Other interning annotations\label{other-interning-annotations}}

The Checker Framework's interning annotations are similar to annotations used
elsewhere.

If your code is already annotated with a different interning
annotation, you can reuse that effort.  The Checker Framework comes with
cleanroom re-implementations of annotations from other tools.  It treats
them exactly as if you had written the corresponding annotation from the
Interning Checker, as described in Figure~\ref{fig-interning-refactoring}.


% These lists should be kept in sync with InterningAnnotatedTypeFactory.java .
\begin{figure}
\begin{center}
% The ~ around the text makes things look better in Hevea (and not terrible
% in LaTeX).
\begin{tabular}{ll}
\begin{tabular}{|l|}
\hline
 ~com.sun.istack.Interned~ \\ \hline
\end{tabular}
&
$\Rightarrow$
~checkers.interning.quals.Interned~
\end{tabular}
\end{center}
%BEGIN LATEX
\vspace{-1.5\baselineskip}
%END LATEX
\caption{Correspondence between other interning annotations and the
  Checker Framework's annotations.}
\label{fig-interning-refactoring}
\end{figure}

Alternately, the Checker Framework can process those other annotations (as
well as its own, if they also appear in your program).  The Checker
Framework has its own definition of the annotations on the left side of
Figure~\ref{fig-interning-refactoring}, so that they can be used as type
qualifiers.  The Checker Framework interprets them according to the right
side of Figure~\ref{fig-interning-refactoring}.



% LocalWords:  plugin MyInternedClass enum InterningExampleWithWarnings java
% LocalWords:  PolyInterned Alint dotequals quals InterningAnnotatedTypeFactory
% LocalWords:  javac InterningVisitor JLS Acheckclass UsesObjectEquals 5cm
%  LocalWords:  consing valueOf superclasses s2 cleanroom canonicalization
%  LocalWords:  5cm

\htmlhr
\chapter{IGJ immutability checker\label{igj-checker}}

IGJ is a Java language extension that helps programmers to avoid mutation errors
(unintended side effects).
If the IGJ checker issues no warnings for a given program, then that program
will never change objects that should not be changed.  This guarantee
enables a programmer to detect and prevent mutation-related errors.
(See Section~\ref{checker-guarantees} for caveats to the guarantee.)

To run the IGJ Checker, supply the \code{-processor checkers.igj.IGJChecker}
command-line option to javac.  For examples, see Section~\ref{igj-example}.


\section{IGJ and Mutability\label{igj-and-mutability}}

IGJ~\cite{ZibinPAAKE2007} permits a
programmer to express that a particular object should never be modified via any
reference (object immutability), or that a reference should never be used to
modify its referent (reference immutability). Once a programmer has expressed
these facts, an automatic checker analyzes the code to either locate mutability
bugs or to guarantee that the code contains no such bugs.

\begin{figure}
\includeimage{igj}{3.5cm}
\caption{Type hierarchy for three of IGJ's type qualifiers.}
\label{fig:igj-hierarchy}
\end{figure}

To learn more details of the IGJ language and type system, please see the
ESEC/FSE 2007 paper ``\ahref{http://www.cs.washington.edu/homes/mernst/pubs/immutability-generics-fse2007-abstract.html}{Object and reference immutability using Java
generics}''~\cite{ZibinPAAKE2007}.
The IGJ checker supports Annotation IGJ (Section~\ref{annotation-igj-dialect}),
which is a slightly different dialect
of IGJ than that described in the ESEC/FSE paper.


\section{IGJ Annotations\label{igj-annotations}}

Each object is either immutable (it can never be modified) or mutable (it
can be modified).  The following qualifiers are part of the IGJ type system.

\begin{description}

\item[\code{@\refclass{igj/quals}{Immutable}}]
  An immutable reference always refers to an immutable object.  Neither the
  reference, nor any aliasing reference, may modify the object.

\item[\code{@\refclass{igj/quals}{Mutable}}]
  A mutable reference refers to a mutable object.  The reference, or some
  aliasing mutable reference, may modify the object.

\item[\code{@\refclass{igj/quals}{ReadOnly}}]
  A readonly reference cannot be used to modify its referent.  The referent
  may be an immutable or a mutable object.  In other words, it is possible
  for the referent to change via an aliasing mutable reference, even though
  the referent cannot be changed via the readonly reference.

\item[\code{@\refclass{igj/quals}{Assignable}}]
  The annotated field may be re-assigned regardless of the
  immutability of the enclosing class or object instance.

\item[\code{@\refclass{igj/quals}{AssignsFields}}]
  is similar to \<@Mutable>, but permits only limited mutation ---
  assignment of fields --- and is intended for use by constructor helper
  methods.

\item[\code{@\refclass{igj/quals}{I}}]
  simulates mutability overloading or the template behavior of generics.
  It can be applied to classes, methods, and parameters.  See
  Section~\ref{igj-templating}.

\end{description}

For additional details, see~\cite{ZibinPAAKE2007}.


\section{What the IGJ checker checks\label{igj-checks}}

The IGJ checker issues an error whenever mutation happens through a
readonly reference, when fields of a readonly reference which are not
explicitly marked with \code{@\refclass{igj/quals}{Assignable}} are
reassigned, or when a readonly reference is assigned to a mutable
variable.  The checker also emits a warning when casts increase the
mutability access of a reference.

% There is no visitor to reference!
% For a complete description of all checks performed by
% the checker, see the Javadoc for \refclass{igj}{IGJVisitor}.


\section{Implicit and default qualifiers\label{igj-implicit-qualifiers}}

As described in Section~\ref{effective-qualifier}, the IGJ checker
adds implicit qualifiers, reducing the number of annotations that must
appear in your code.
% For example, ...

For a complete description of all implicit IGJ qualifiers, see the
Javadoc for \refclass{igj}{IGJAnnotatedTypeFactory}.

The default annotation (for types that are unannotated and not given an
implicit qualifier) is as follows:
\begin{itemize}
\item
  \code{@Mutable} for almost all references.  This is backward-compatible
  with Java, since Java permits any reference to be mutated.
\item
  \code{@Readonly} for local variables.  This qualifier may be refined by
  flow-sensitive local type refinement (see Section~\ref{type-refinement}).
\item
  \code{@Readonly} for type parameter and wildcard bounds.  For example,

\begin{Verbatim}
  interface List<T extends Object> { ... }
\end{Verbatim}

\noindent
is defaulted to

\begin{Verbatim}
  interface List<T extends @Readonly Object> { ... }
\end{Verbatim}

This default is not backward-compatible --- that is, you may have to
explicitly add \code{@Mutable} annotations to some type parameter bounds in
order to make unannotated Java code type-check under IGJ\@.  However, this
reduces the number of annotations you must write overall (since most
variables of generic type are in fact not modified), and permits more
client code to type-check (otherwise a client could not write
\code{List<@Readonly Date>}).

\end{itemize}



\section{Annotation IGJ Dialect\label{annotation-igj-dialect}}

The IGJ checker supports the Annotation IGJ dialect of IGJ\@.  The syntax of
Annotation IGJ is based on type annotations.

The syntax of the original IGJ
dialect~\cite{ZibinPAAKE2007} was based on Java 5's generics and annotation mechanisms. The original
IGJ dialect was not backward-compatible with Java (either syntactically or
semantically). The dialect of IGJ checked by the IGJ checker corrects these
problems.

The differences between the Annotation IGJ dialect and the original IGJ dialect
are as follows.

\subsection{Semantic Changes}

\begin{itemize}

\item
  Annotation IGJ does not permit covariant changes in generic type
  arguments, for backward compatibility with Java.  In ordinary Java, types
  with different generic type arguments, such as \code{Vector<Integer>} and
  \code{Vector<Number>}, have no subtype relationship, even if the
  arguments (\code{Integer} and \code{Number}) do. The original IGJ dialect
  changed the Java subtyping rules to permit safely varying a type argument
  covariantly in certain circumstances. For example,

\begin{Verbatim}
  Vector<Mutable, Integer>  <:  Vector<ReadOnly, Integer>
                            <:  Vector<ReadOnly, Number>
                            <:  Vector<ReadOnly, Object>
\end{Verbatim}

is valid in IGJ, but in Annotation IGJ, only

\begin{Verbatim}
  @Mutable Vector<Integer>  <:  @ReadOnly Vector<Integer>
\end{Verbatim}

holds and the other two subtype relations do not hold

\begin{Verbatim}
  @ReadOnly Vector<Integer> </:  @ReadOnly Vector<Number>
                            </:  @ReadOnly Vector<Object>
\end{Verbatim}


\item
  Annotation IGJ supports array immutability. The original IGJ dialect did
  not permit the (im)mutability of array elements to be specified, because
  the generics syntax used by the original IGJ dialect cannot be applied to
  array elements.

\end{itemize}

\subsection{Syntax Changes}

\begin{itemize}

\item  Immutability is specified through
  \ahref{http://types.cs.washington.edu/jsr308/}{type annotations}~\cite{JSR308-2008-09-12} (Section~\ref{igj-annotations}),
not through a combination of generics and annotations.  Use of type
annotations makes Annotation IGJ backward compatible with Java syntax.

\item Templating over Immutability: The annotation \code{@\refclass{igj/quals}{I}(\emph{id})} is used to template
over immutability.  See Section~\ref{igj-templating}.

\end{itemize}


\subsection{Templating Over Immutability: \code{@I}\label{igj-templating}}

\code{@\refclass{igj/quals}{I}} is a template annotation over IGJ Immutability annotations. It acts
similarly to type variables in Java's generic types, and the name
\code{@\refclass{igj/quals}{I}} mimics the standard \code{<I>} type variable name used in code
written in the original IGJ dialect.  The annotation value string is used
to distinguish between multiple instances of \code{@\refclass{igj/quals}{I}} --- in the
generics-based original dialect, these would be expressed as two type
variables \code{<I>} and \code{<J>}.

\paragraph{Usage on classes\label{igj-usage-on-classes}}

A class declaration annotated with \code{@\refclass{igj/quals}{I}} can then be
used with any IGJ Immutability annotation.  The actual immutability that
\code{@\refclass{igj/quals}{I}} is resolved to dictates the immutability type for all the non-static
appearances of \code{@\refclass{igj/quals}{I}} with the same value as the class declaration.

  Example:
\begin{Verbatim}
    @I
    public class FileDescriptor {
       private @Immutable Date creationData;
       private @I Date lastModData;

       public @I Date getLastModDate(@ReadOnly FileDescriptor this) { }
    }

    ...
    void useFileDescriptor() {
       @Mutable FileDescriptor file =
                         new @Mutable FileDescriptor(...);
       ...
       @Mutable Data date = file.getLastModDate();

    }
\end{Verbatim}

In the last example, \code{@\refclass{igj/quals}{I}} was resolved to \code{@\refclass{igj/quals}{Mutable}} for the instance file.

\paragraph{Usage on methods\label{igj-usage-on-methods}}

For example, it could be used for method parameters, return values, and the
actual IGJ immutability value would be resolved based on the method invocation.

For example, the below method \code{getMidpoint} returns a \code{Point} with the same
immutability type as the passed parameters if \code{p1} and \code{p2} match
in immutability, otherwise \code{@\refclass{igj/quals}{I}} is resolved to \code{@\refclass{igj/quals}{ReadOnly}}:

\begin{Verbatim}
  static @I Point getMidpoint(@I Point p1, @I Point p2) { ... }
\end{Verbatim}

The \code{@\refclass{igj/quals}{I}} annotation value distinguishes between \code{@\refclass{igj/quals}{I}}
declarations.  So, the below method \code{findUnion} returns a collection of the same
immutability type as the \emph{first} collection parameter:

\begin{Verbatim}
  static <E> @I("First") Collection<E> findUnion(@I("First") Collection<E> col1,
                                                 @I("Second") Collection<E> col2) { ... }
\end{Verbatim}


\section{Iterators and their abstract state\label{igj-library-annotations}}

This section explains why the receiver of \<Iterator.next()> is annotated
as \<@ReadOnly>.

An iterator conceptually has two pieces of state:
\begin{enumerate}
\item
  the underlying collection
\item
  an index into that collection (indicating the next object to be returned)
\end{enumerate}

We choose to exclude the index from the abstract state of the iterator.
That is, a change to the index does not count as a mutation of the
iterator itself.

Changes to the underlying collection are more important and interesting,
and unintentional changes are much more likely to lead to important
errors.  Therefore, this choice about the iterator's abstract state
appears to be more useful than other choices.  For example, if the
iterator's abstract state included both the underlying collection and
the index, then there would be no way to express, or check, that
\<Iterator.next> does not change the underlying collection.


\section{Examples\label{igj-example}}

To try the IGJ checker on a source file that uses the IGJ qualifier, use
the following command (where \code{javac} is the JSR 308 compiler that
is distributed with the Checker Framework).

\begin{Verbatim}
  javac -processor checkers.igj.IGJChecker examples/IGJExample.java
\end{Verbatim}

The IGJ checker itself is also annotated with IGJ annotations.


% LocalWords:  plugin ReadOnly AssignsFields im templating getMidpoint cp TODO
% LocalWords:  findUnion igj IGJ's quals ESEC readonly covariant
% LocalWords:  NullnessAnnotatedTypeFactory IGJAnnotatedTypeFactory

\htmlhr
\chapter{Javari immutability checker\label{javari-checker}}

Javari~\cite{TschantzE2005,QuinonezTE2008} is a Java language extension that helps programmers to avoid mutation
errors that result from unintended side effects.
If the Javari checker issues no warnings for a given program, then that
program will never change objects that should not be changed.  This
guarantee enables a programmer to detect and prevent mutation-related
errors.  (See Section~\ref{checker-guarantees} for caveats to the guarantee.)
The Javari webpage (\myurl{http://types.cs.washington.edu/javari/}) contains
papers that explain the Javari language and type system.
By contrast to those papers, the Javari checker uses an annotation-based
dialect of the Javari language.

The Javarifier tool infers Javari types for an existing program; see
Section~\ref{javari-inference}.

Also consider the IGJ checker (Chapter~\ref{igj-checker}).  The IGJ type
system is more expressive than that of Javari, and the IGJ checker is a bit
more robust.  However, IGJ lacks a type inference tool such as Javarifier.

To run the Javari Checker, supply the \code{-processor
  checkers.javari.JavariChecker} command-line option to javac.  For
examples, see Section~\ref{javari-examples}.



\begin{figure}
\includeimage{javari}{2.5cm}
\caption{Type hierarchy for Javari's ReadOnly type qualifier.}
\label{fig:javari-hierarchy}
\end{figure}


\section{Javari annotations\label{javary-annotations}}

The following six annotations make up the Javari type system.

\begin{description}

\item[\code{@\refclass{javari/quals}{ReadOnly}}]
  indicates a type that provides only read-only access.  A reference of
  this type may not be used to modify its referent, but aliasing references
  to that object might change it.

\item[\code{@\refclass{javari/quals}{Mutable}}]
  indicates a mutable type.
  
\item[\code{@\refclass{javari/quals}{Assignable}}]
  is a field annotation, not a type qualifier.  It indicates that the given
  field may always be assigned, no matter what the type of the reference
  used to access the field.
  
\item[\code{@\refclass{javari/quals}{QReadOnly}}]
  corresponds to Javari's ``\code{?\ readonly}'' for wildcard types.  An
  example of its use is \code{List<@QReadOnly Date>}.  It allows only the
  operations which are allowed for both readonly and mutable types.

\item[\code{@\refclass{javari/quals}{PolyRead}}]
  (previously named \code{@RoMaybe}) specifies polymorphism over
  mutability; it simulates mutability overloading.  It can be applied to
  methods and parameters.  See Section~\ref{qualifier-polymorphism} and the
  \code{@\refclass{javari/quals}{PolyRead}} Javadoc for more details.

\item[\code{@\refclass{javari/quals}{ThisMutable}}]
  means that the mutability of the field is the same as that of the
  reference that contains it.  \code{@ThisMutable} is the default on
  fields, and does not make sense to write elsewhere.  Therefore,
  \code{@ThisMutable} should never appear in a program.

\end{description}


\section{Writing Javari annotations\label{writing-javari-annotations}}


\subsection{Implicit qualifiers\label{javari-implicit-qualifiers}}

As described in Section~\ref{effective-qualifier}, the Javari checker
adds implicit qualifiers, reducing the number of annotations that must
appear in your code.
% For example, ...

For a complete description of all implicit Javari qualifiers, see the
Javadoc for \refclass{javari}{JavariAnnotatedTypeFactory}.


\subsection{Inference of Javari annotations\label{javari-inference}}

It can be tedious to write annotations in your code.  The Javarifier tool
(\myurl{http://types.cs.washington.edu/javari/javarifier/}) infers 
Javari types for an existing program.  It 
automatically inserts Javari annotations in your Java program or
in \code{.class} files.

This has two benefits:  it relieves the programmer of the tedium of writing
annotations (though the programmer can always refine the inferred
annotations), and it annotates libraries, permitting checking of programs
that use those libraries.



\section{What the Javari checker checks\label{javari-checks}}

The checker issues an error whenever mutation happens through a readonly
reference, when fields of a readonly reference which are not explicitly
marked with \code{@\refclass{javari/quals}{Assignable}} are reassigned, or
when a readonly expression is assigned to a mutable variable.  The checker
also emits a warning when casts increase the mutability access of a
reference.

% There is no Javadoc as of 2/2009.
% For a complete description of all checks performed by
% the Nullness checker, see the Javadoc for
% \refclass{javari}{JavariVisitor}.


\section{Iterators and their abstract state\label{javari-library-annotations}}

For an explanation of why the receiver of \<Iterator.next()> is annotated
as \<@ReadOnly>, see Section~\ref{igj-library-annotations}.


\section{Examples\label{javari-examples}}

To try the Javari checker on a source file that uses the Javari
qualifier, use the following command (where \code{javac} is the JSR 308
compiler  that
is distributed with the Checker Framework).  Alternately, you may
specify just one of the test files.

\begin{Verbatim}
  javac -processor checkers.javari.JavariChecker tests/javari/*.java
\end{Verbatim}

\noindent
The compiler should issue the errors and warnings (if any) specified in the
\code{.out} files with same name.

To run the test suite for the Javari checker, use \code{ant javari-tests}.

The Javari checker itself is also annotated with Javari annotations.


% LocalWords:  PolyRead javari cp plugin ReadOnly QReadOnly romaybe Javarifier
% LocalWords:  readonly wildcard Javadoc javac RoMaybe quals IGJ
% LocalWords:  JavariAnnotatedTypeFactory

\htmlhr
\chapter{Lock Checker\label{lock-checker}}

The Lock Checker prevents certain concurrency errors by enforcing a
locking discipline.  A locking discipline indicates which locks must be held
when a given operation occurs.  You express the locking discipline by
declaring a variable's type to have the qualifier
\<\refqualclass{checker/lock/qual}{GuardedBy}{\small("\emph{lockexpr}")}>.
This indicates that the variable's value may
be dereferenced only if the given lock is held.


To run the Lock Checker, supply the
\code{-processor org.checkerframework.checker.lock.LockChecker}
command-line option to javac.  The \<-AconcurrentSemantics>
command-line option is always enabled for the Lock Checker (see Section~\ref{faq-concurrency}).


\section{What the Lock Checker guarantees\label{lock-guarantees}}

The Lock Checker gives the following guarantee.
Suppose that expression $e$ has type
\<\refqualclass{checker/lock/qual}{GuardedBy}(\ttlcb"x", "y.z"\ttrcb)>.
Then the value computed for $e$ is only dereferenced by a thread when the
thread holds locks \<x> and \<y.z>.
Dereferencing a value is reading or writing one of its fields.
The guarantee about $e$'s value
holds not only if the expression $e$ is dereferenced
directly, but also if the value was first copied into a variable,
returned as the
result of a method call, etc.
Copying a reference is always
permitted by the Lock Checker, regardless of which locks are held.

A lock is held if it has been acquired but not yet released.
Java has two types of locks.
A monitor lock is acquired upon entry to a \<synchronized> method or block,
and is released on exit from that method or block.
%  (More precisely,
%  the current thread locks the monitor associated with the value of
%  \emph{E}; see \href{https://docs.oracle.com/javase/specs/jls/se8/html/jls-17.html#jls-17.1}{JLS \S17.1}.)
An explicit lock is acquired by a method call such as
\sunjavadoc{java/util/concurrent/locks/Lock.html\#lock--}{Lock.lock()},
and is released by another method call such as
\sunjavadoc{java/util/concurrent/locks/Lock.html\#unlock--}{Lock.unlock()}.
The Lock Checker enforces that any expression whose type implements
\sunjavadoc{java/util/concurrent/locks/Lock.html}{Lock} is used as an
explicit lock, and all other expressions are used as monitor locks.
% The class that implements the Lock interface could itself use the
% current object as a monitor lock.  This doesn't seem like it
% needs to be mentioned here, though.

Ensuring that your program obeys its locking discipline is an easy and
effective way to eliminate a common and important class of errors.
If the Lock Checker issues no warnings, then your program obeys its locking discipline.
However, your program might still have other types of concurrency errors.
%
For example, you might have specified an inadequate locking discipline
because you forgot some \refqualclass{checker/lock/qual}{GuardedBy}
annotations.
%
Your program might release and
re-acquire the lock, when correctness requires it to hold it throughout a
computation.
%
And, there are other concurrency errors that cannot, or
should not, be solved with locks.

\section{Lock annotations\label{lock-annotations}}

This section describes the lock annotations you can write on types and methods.


\subsection{Type qualifiers\label{lock-type-qualifiers}}

\begin{description}

\item[\refqualclass{checker/lock/qual}{GuardedBy}{\small(\emph{exprSet})}]
  If a variable \<x> has type \<@GuardedBy("\emph{expr}")>, then a thread may
  dereference the value referred to by \<x> only when the thread holds the
  lock that \emph{expr} currently evaluates to.

  The \<@GuardedBy> annotation can list multiple expressions, as in
  \<@GuardedBy(\ttlcb"\emph{expr1}", "\emph{expr2}"\ttrcb)>, in which case
  the dereference is
  permitted only if the thread holds all the locks.

  Section~\ref{java-expressions-as-arguments} explains which
  expressions the Lock Checker is able to analyze as lock expressions.
  These include \code{<self>}, i.e. the value of the annotated reference
  (non-primitive) variable.  For example, \code{@GuardedBy("<self>") Object o}
  indicates that the value referenced by \<o> is guarded by the intrinsic
  (monitor) lock of the value referenced by \<o>.

  \<@GuardedBy(\{\})>, which means the value is always allowed to be
  dereferenced, is the default type qualifier that is used for all locations
  where the programmer does not
  write an explicit locking type qualifier (except all CLIMB-to-top locations
  other than upper bounds and exception parameters --- see Section~\ref{climb-to-top}).
  (Section~\ref{lock-checker-default-qualifier} discusses this choice.)
  It is also the conservative
  default type qualifier for method parameters in unannotated libraries
  (see \chapterpageref{annotating-libraries}).

\item[\refqualclass{checker/lock/qual}{GuardedByUnknown}]
  If a variable \<x> has type \code{@GuardedByUnknown}, then
  it is not known which locks protect \<x>'s value.  Those locks might
  even be out of scope (inaccessible) and therefore unable to be written
  in the annotation.
  The practical consequence is that
  the value referred to by \<x> can never be dereferenced.

  Any value can be assigned to a variable of type
  \code{@GuardedByUnknown}.  In particular, if it is written on a
  formal parameter, then any value,
  including one whose locks are not currently held,
  may be passed as an argument.

  \<@GuardedByUnknown> is the conservative
  default type qualifier for method receivers in unannotated libraries
  (see \chapterpageref{annotating-libraries}).

\item[\refqualclass{checker/lock/qual}{GuardedByBottom}]
  If a variable \<x> has type \code{@GuardedByBottom}, then
  the value referred to by \<x> is \code{null} and can never
  be dereferenced.

\end{description}

\begin{figure}
\includeimage{lock-guardedby}{3cm}
\caption{The subtyping relationship of the Lock Checker's qualifiers.
\code{@GuardedBy(\{\})} is the default type qualifier for unannotated
types (except all CLIMB-to-top locations other than upper bounds and exception
parameters --- see Section~\ref{climb-to-top}).
}
\label{fig-lock-guardedby-hierarchy}
\end{figure}

Figure~\ref{fig-lock-guardedby-hierarchy} shows the type hierarchy of these
qualifiers.
All \code{@GuardedBy} annotations are incomparable:
if \emph{exprSet1} $\neq$ \emph{exprSet2}, then \code{@GuardedBy(\emph{exprSet1})} and
\code{@GuardedBy(\emph{exprSet2})} are siblings in the type hierarchy.
You might expect that
\<@GuardedBy(\{"x", "y"\}) T> is a subtype of \<@GuardedBy(\{"x"\}) T>.  The
first type requires two locks to be held, and the second requires only one
lock to be held and so could be used in any situation where both locks are
held.  The type system conservatively prohibits this in order to prevent
type-checking loopholes that would result from aliasing and side effects
--- that is, from having two mutable references, of different types, to the
same data. See
Section~\ref{lock-guardedby-invariant-subtyping} for an example
of a problem that would occur if this rule were relaxed.


\paragraph{Polymorphic type qualifiers\label{lock-polymorphic-type-qualifiers}}

%\refqualclass{checker/lock/qual}{GuardSatisfied}{\small(\emph{index})}
%and
%\refqualclass{checker/interning/qual}{PolyGuardedBy}
%indicates qualifier polymorphism.  For a description of qualifier
%polymorphism, see Section~\ref{qualifier-polymorphism}.

\begin{description}

\item[\refqualclass{checker/lock/qual}{GuardSatisfied}{\small(\emph{index})}]
  If a variable \<x> has type \code{@GuardSatisfied}, then all
  lock expressions for \<x>'s value are held.

  As with other qualifier-polymorphism annotations
  (Section~\ref{qualifier-polymorphism}), the \emph{index} argument
  indicates when two values are guarded by the same (unknown) set of locks.

  \code{@GuardSatisfied} is only allowed in method signatures:  on
  formal parameters (including the receiver) and return types.
  It may not be written on fields.  Also, it is a limitation of the
  current design that \code{@GuardSatisfied} may not be written on
  array elements or on local variables.

  A return type can only be annotated with \<@GuardSatisfied(index)>,
  not \<@GuardSatisfied>.

  See Section~\ref{lock-checker-polymorphism-example}
  for an example of a use of \code{@GuardSatisfied}.

%\item[\refqualclass{checker/interning/qual}{PolyGuardedBy}]
%  It is unknown what the guards are or whether they are held.
%  An expression whose type is \code{@PolyGuardedBy}
%  cannot be dereferenced.

\end{description}


\subsection{Declaration annotations\label{lock-declaration-annotations}}

The Lock Checker supports several annotations that specify method behavior.
These are declaration annotations, not type annotations: they apply to the
method itself rather than to some particular type.

\paragraph{Method pre-conditions and post-conditions\label{lock-method-pre-post-conditions}}

\begin{sloppypar}
\begin{description}
\item[\refqualclass{checker/lock/qual}{Holding}\small{(String[] locks)}]
  All the given lock expressions
  are held at the method call site.

\item[\refqualclass{checker/lock/qual}{EnsuresLockHeld}\small{(String[] locks)}]
  The given lock
  expressions are
  locked upon method return if the method
  terminates successfully.  This is useful for annotating a
  method that acquires a lock such as
  \sunjavadoc{java/util/concurrent/locks/ReentrantLock.html\#lock--}{ReentrantLock.lock()}.

\item[\refqualclass{checker/lock/qual}{EnsuresLockHeldIf}\small{(String[] locks, boolean result)}]
  If the annotated method returns the given
  boolean value (true or false), the given lock
  expressions are locked upon method return if the method
  terminates successfully.
  This is useful for annotating a
  method that conditionally acquires a lock.
  See Section~\ref{ensureslockheld-examples} for examples.

\end{description}

\paragraph{Side effect specifications\label{lock-side-effect-specifications}}

\begin{description}

\item[\refqualclass{checker/lock/qual}{LockingFree}]
  The method does not acquire or release locks,
  directly or indirectly.  The method is not \<synchronized>, it contains
  no \<synchronized> blocks, it contains no calls to \<lock> or \<unlock>
  methods, and it contains no calls to methods that are not themselves \<@LockingFree>.

  Since
  \code{@SideEffectFree} implies \code{@LockingFree}, if both are applicable
  then you only need to write \code{@SideEffectFree}.

\item[\refqualclass{checker/lock/qual}{ReleasesNoLocks}]
  The method maintains a strictly nondecreasing lock hold count on the
  current thread for any locks that were held prior
  to the method call.  The method might acquire locks but then release
  them, or might acquire locks but not release them (in which case it should
  also be annotated with
  \refqualclass{checker/lock/qual}{EnsuresLockHeld} or
  \refqualclass{checker/lock/qual}{EnsuresLockHeldIf}).

  This is the default for methods being type-checked that have no \<@LockingFree>,
  \<@MayReleaseLocks>, \code{@SideEffectFree}, or \code{@Pure}
  annotation.

\item[\refqualclass{checker/lock/qual}{MayReleaseLocks}]
  The method may release locks that were held prior to the method being called.
  You can write this when you are certain the method releases locks, or
  when you don't know whether the method releases locks.
  This is the conservative default for methods in unannotated libraries (see \chapterpageref{annotating-libraries}).

\end{description}
\end{sloppypar}


\section{Type-checking rules\label{lock-type-checking-rules}}

In addition to the standard subtyping rules enforcing the subtyping relationship
described in Figure~\ref{fig-lock-guardedby-hierarchy}, the Lock Checker enforces
the following additional rules.


\subsection{Polymorphic qualifiers\label{lock-type-checking-rules-polymorphic-qualifiers}}

\begin{description}

\item[\code{@GuardSatisfied}]

  The overall rules for polymorphic qualifiers are given in
  Section~\ref{qualifier-polymorphism}.

  Here are additional constraints for (pseudo-)assignments:

  \begin{itemize}
  \item
    If the left-hand side has type \<@GuardSatisfied> (with or without an index),
    then all locks mentioned in the right-hand side's \<@GuardedBy> type
    must be currently held.
  \item
    A formal parameter with type qualifier \<@GuardSatisfied> without an
    index cannot be assigned to.
  \item
    If the left-hand side is a formal parameter with type
    \<@GuardSatisfied(\emph{index})>, the right-hand-side must have
    identical \<@GuardSatisfied(\emph{index})> type.
  \end{itemize}

  If a formal parameter type is
  annotated with \<@GuardSatisfied> without an index, then that formal parameter
  type is unrelated to every other type in the \<@GuardedBy> hierarchy,
  including other occurrences of \<@GuardSatisfied> without an index.

  \<@GuardSatisfied> may not be used on formal parameters, receivers, or
  return types of a method annotated with \<@MayReleaseLocks>.
\end{description}

\subsection{Dereferences\label{lock-type-checking-rules-dereferences}}

\begin{description}

\item[\code{@GuardedBy}]
  An expression of type \<@GuardedBy(\emph{eset})> may be dereferenced only
  if all locks in \emph{eset} are held.

\item[\code{@GuardSatisfied}]
  An expression of type \<@GuardSatisfied> may be dereferenced.

\item[Not \code{@GuardedBy} or \code{@GuardSatisfied}]
  An expression whose type is not annotated with \code{@GuardedBy} or
  \code{@GuardSatisfied} may not be dereferenced.
%  In particular, an expression of type \code{@PolyGuardedBy} may not be dereferenced.

\end{description}

\subsection{Primitive types, boxed primitive types, and Strings\label{lock-type-checking-rules-primitives}}

Primitive types, boxed primitive types (such as \<java.lang.Integer>), and type \<java.lang.String>
are implicitly annotated with \<@GuardedBy(\{\})>.
It is an error for the programmer to annotate any of these types with an annotation from
the \<@GuardedBy> type hierarchy, including \<@GuardedBy(\{\})>.

%  Primitive values are not guarded.  Instead, the variables that store them are.
%  Therefore, for reads, writes and other operations on primitive values, the Lock Checker requires that
%  the appropriate locks be held, but it does not enforce any other rules.
%  In particular, it does not require the annotations
%  in the types involved in the operation (including assignments and
%  pseudo-assignments) to match.  For example, given:
%  \begin{verbatim}
%  ReentrantLock lock1, lock2;
%  @GuardedBy("lock1") int a;
%  @GuardedBy("lock2") int b;
%  @GuardedBy({}) int c;
%  ...
%  lock1.lock();
%  lock2.lock();
%  a = b;
%  a = c;
%  a = b + c;
%  \end{verbatim}
%  The expressions \code{a = b}, \code{a = c}, and \code{a = b + c}
%  all type check, whereas none of them would type check if \code{a},
%  \code{b} and \code{c} were not primitives.

\subsection{Overriding\label{lock-type-checking-rules-overriding}}

\begin{description}

\item[Overriding methods annotated with \code{@Holding}]
  If class $B$ overrides method $m$ from class $A$, then the expressions in
  $B$'s \<@Holding>
  annotation must be a subset of or equal to that of $A$'s \<@Holding>
  annotation..

\item[Overriding methods annotated with side effect annotations]
  If class $B$ overrides method $m$ from class $A$, then
  the side effect annotation on $B$'s declaration of $m$
  must be at least as strong as that in $A$'s declaration of $m$.
  From weakest to strongest, the side effect annotations
  processed by the Lock Checker are:
\begin{verbatim}
  @MayReleaseLocks
  @ReleasesNoLocks
  @LockingFree
  @SideEffectFree
  @Pure
\end{verbatim}

\end{description}

\subsection{Side effects\label{lock-type-checking-rules-polymorphic-side-effects}}

\begin{description}

\item[Releasing explicit locks]
  Any method that releases an explicit lock must be annotated
  with \code{@MayReleaseLocks}.
  The Lock Checker issues a warning if it encounters a method declaration
  annotated with \code{@MayReleaseLocks} and having a formal parameter
  or receiver annotated with \code{@GuardSatisfied}.  This is because
  the Lock Checker cannot guarantee that the guard will be satisfied
  throughout the body of a method if that method may release a lock.

\item[No side effects on lock expressions]
  If expression \emph{expr} is used to acquire a lock, then
  \emph{expr} must evaluate to the same value, starting from when
  \emph{expr} is used to acquire a lock until \emph{expr} is used to
  release the lock.
  An expression is used to acquire a lock if it is the receiver at a
  call site of a \<synchronized> method, is the expression in a
  \<synchronized> block, or is the argument to a \<lock> method.

\item[Locks are released after possible side effects]
% These are standard dataflow analysis rules, but are worth
% repeating here due to how important they are for the day-to-day
% use of the Lock Checker.  I believe this would be the single
% largest source of confusion amongst Lock Checker users if this
% were not stated explicitly.
  After a call to a method annotated with \code{@LockingFree},
  \code{@ReleasesNoLocks}, \code{@SideEffectFree}, or \code{@Pure},
  the Lock Checker's estimate of held locks
  after a method call is the same as that prior to the method call.
  After a call to a method annotated with \code{@MayReleaseLocks},
  the estimate of held locks is conservatively reset to the empty set,
  except for those locks specified to be held after the call
  by an \code{@EnsuresLockHeld} or \code{@EnsuresLockHeldIf}
  annotation on the method.  Assignments to variables also
  cause the estimate of held locks to be conservatively reduced
  to a smaller set if the Checker Framework determines that the
  assignment might have side-effected a lock expression.
  For more information on side effects, please refer to
  Section~\ref{type-refinement-purity}.

\end{description}


\section{Examples\label{lock-examples}}

The Lock Checker guarantees that a value that was computed from an expression of \code{@GuardedBy} type is
dereferenced only when the current thread holds all the expressions in the
\code{@GuardedBy} annotation.

\subsection{Examples of @GuardedBy\label{lock-examples-guardedby}}

The following example demonstrates the basic
type-checking rules.

\begin{Verbatim}
class MyClass {
  final ReentrantLock lock; // Initialized in the constructor

  @GuardedBy("lock") Object x = new Object();
  @GuardedBy("lock") Object y = x; // OK, since dereferences of y will require "lock" to be held.
  @GuardedBy({}) Object z = x; // ILLEGAL since dereferences of z don't require "lock" to be held.
  @GuardedBy("lock") Object myMethod() { // myMethod is implicitly annotated with @ReleasesNoLocks.
     return x; // OK because the return type is annotated with @GuardedBy("lock")
  }

  [...]

  void exampleMethod() {
     x.toString(); // ILLEGAL because the lock is not known to be held
     y.toString(); // ILLEGAL because the lock is not known to be held
     myMethod().toString(); // ILLEGAL because the lock is not known to be held
     lock.lock();
     x.toString();  // OK: the lock is known to be held
     y.toString();  // OK: the lock is known to be held, and toString() is annotated with @SideEffectFree.
     myMethod().toString(); // OK: the lock is known to be held, since myMethod
                            // is implicitly annotated with @ReleasesNoLocks.
  }
}
\end{Verbatim}

Note that the expression \code{new Object()} is inferred to have type \code{@GuardedBy("lock")}
because it is immediately assigned to a newly-declared
variable having type annotation \code{@GuardedBy("lock")}.  You could
explicitly write \code{new @GuardedBy("lock") Object()} but it is not
required.

The following example demonstrates that using \code{<self>} as a lock expression
allows a guarded value to be dereferenced even when the original
variable name the value was originally assigned to falls out of scope.

\begin{Verbatim}
class MyClass {
  private final @GuardedBy("<self>") Object x = new Object();
  void method() {
    x.toString(); // ILLEGAL because x is not known to be held.
    synchronized(x) {
      x.toString(); // OK: x is known to be held.
    }
  }

  public @GuardedBy("<self>") Object get_x() {
    return x; // OK, since the return type is @GuardedBy("<self>").
  }
}

class MyOtherClass {
  void method() {
    MyClass m = new MyClass();
    final @GuardedBy("<self>") Object o = m.get_x();
    o.toString(); // ILLEGAL because o is not known to be held.
    synchronized(o) {
      o.toString(); // OK: o is known to be held.
    }
  }
}
\end{Verbatim}


\subsection{@GuardedBy(\{``a'', ``b''\}) is not a subtype of @GuardedBy(\{``a''\})\label{lock-guardedby-invariant-subtyping}}


\textbf{@GuardedBy(exprSet)}

The following example demonstrates the reason the Lock Checker enforces the
following rule:  if \emph{exprSet1} $\neq$ \emph{exprSet2}, then
\code{@GuardedBy(\emph{exprSet1})} and \code{@GuardedBy(\emph{exprSet2})} are siblings in the type
hierarchy.

\begin{Verbatim}
class MyClass {
    final Object lockA = new Object();
    final Object lockB = new Object();
    @GuardedBy("lockA") Object x = new Object();
    @GuardedBy({"lockA", "lockB"}) Object y = new Object();
    void myMethod() {
        y = x;      // ILLEGAL; if legal, later statement x.toString() would cause trouble
        synchronized(lockA) {
          x.toString();  // dereferences y's value without holding lock lockB
        }
    }
}
\end{Verbatim}


If the Lock Checker permitted the assignment
\code{y = x;}, then the undesired dereference would be possible.


\subsection{Examples of @Holding\label{lock-examples-holding}}

The following example shows the interaction between \<@GuardedBy> and
\<@Holding>:

\begin{Verbatim}
  void helper1(@GuardedBy("myLock") Object a) {
    a.toString(); // ILLEGAL: the lock is not held
    synchronized(myLock) {
      a.toString();  // OK: the lock is held
    }
  }
  @Holding("myLock")
  void helper2(@GuardedBy("myLock") Object b) {
    b.toString(); // OK: the lock is held
  }
  void helper3(@GuardedBy("myLock") Object d) {
    d.toString(); // ILLEGAL: the lock is not held
  }
  void myMethod2(@GuardedBy("myLock") Object e) {
    helper1(e);  // OK to pass to another routine without holding the lock
                 // (but helper1's body has an error)
    e.toString(); // ILLEGAL: the lock is not held
    synchronized (myLock) {
      helper2(e); // OK: the lock is held
      helper3(e); // OK, but helper3's body has an error
    }
  }
\end{Verbatim}


\subsection{Examples of @EnsuresLockHeld and @EnsuresLockHeldIf\label{ensureslockheld-examples}}

\code{@EnsuresLockHeld} and \code{@EnsuresLockHeldIf} are primarily intended
for annotating JDK locking methods, as in:

\begin{Verbatim}
package java.util.concurrent.locks;

class ReentrantLock {

    @EnsuresLockHeld("this")
    public void lock();

    @EnsuresLockHeldIf (expression="this", result=true)
    public boolean tryLock();

    ...
}
\end{Verbatim}

They can also be used to annotate user methods, particularly for
higher-level lock constructs such as a Monitor, as in this simplified example:

\begin{Verbatim}
public class Monitor {

    private final ReentrantLock lock; // Initialized in the constructor

    ...

    @EnsuresLockHeld("lock")
    public void enter() {
       lock.lock();
    }

    ...
}
\end{Verbatim}

\subsection{Example of @LockingFree, @ReleasesNoLocks, and @MayReleaseLocks\label{lock-lockingfree-example}}

\code{@LockingFree} is useful when a method does not make any use of synchronization
or locks but causes other side effects (hence \code{@SideEffectFree} is not appropriate).
\code{@SideEffectFree} implies \code{@LockingFree}, therefore if both are applicable,
you should only write \code{@SideEffectFree}. \code{@ReleasesNoLocks} has a weaker guarantee
than \code{@LockingFree}, and \code{@MayReleaseLocks} provides no guarantees.

\begin{verbatim}
private Object myField;
private final ReentrantLock lock; // Initialized in the constructor
private @GuardedBy("lock") Object x; // Initialized in the constructor

[...]

// This method does not use locks or synchronization, but it cannot
// be annotated as @SideEffectFree since it alters myField.
@LockingFree
void myMethod() {
  myField = new Object();
}

@SideEffectFree
int mySideEffectFreeMethod() {
  return 0;
}

@MayReleaseLocks
void myUnlockingMethod() {
  lock.unlock();
}

@ReleasesNoLocks
void myLockingMethod() {
  lock.lock();
}

@MayReleaseLocks
void clientMethod() {
  if (lock.tryLock()) {
    x.toString(); // OK: the lock is held
    myMethod();
    x.toString(); // OK: the lock is still held since myMethod is locking-free
    mySideEffectFreeMethod();
    x.toString(); // OK: the lock is still held since mySideEffectFreeMethod is side-effect-free
    myUnlockingMethod();
    x.toString(); // ILLEGAL: myUnlockingMethod may have released a lock
  }
  if (lock.tryLock()) {
    x.toString(); // OK: the lock is held
    myLockingMethod();
    x.toString(); // OK: the lock is held
  }
  if (lock.isHeldByCurrentThread()) {
    x.toString(); // OK: the lock is known to be held
  }
}
\end{verbatim}


\subsection{Polymorphism and method formal parameters with unknown guards\label{lock-checker-polymorphism-example}}

The polymorphic \code{@GuardSatisfied} type annotation allows a method body
to dereference the method's formal parameters even if the
\code{@GuardedBy} annotations on the actual parameters are unknown at
the method declaration site.

The declaration of
\sunjavadoc{java/lang/StringBuffer.html\#append-java.lang.String-}{StringBuffer.append(String str)}
is annotated as:

\begin{verbatim}
@LockingFree
public @GuardSatisfied(1) StringBuffer append(@GuardSatisfied(1) StringBuffer this,
                                              @GuardSatisfied(2) String str)
\end{verbatim}

The method manipulates the values of its arguments, so all their locks must
be held.  However, the declaration does not know what those are and they
might not even be in scope at the declaration.  Therefore, the declaration
cannot use \<@GuardedBy> and must use \<@GuardSatisfied>.  The arguments to
\<@GuardSatisfied> indicate that the receiver and result (which are the
same value) are guarded by the same (unknown, possibly empty) set of locks,
and the \<str> parameter may be guarded by a different set of locks.

The \code{@LockingFree} annotation indicates that
this method makes no use of
locks or synchronization.

Given these annotations on \<append>, the following code type-checks:

\begin{verbatim}
final ReentrantLock lock1, lock2; // Initialized in the constructor
@GuardedBy("lock1") StringBuffer filename;
@GuardedBy("lock2") StringBuffer extension;
...
lock1.lock();
lock2.lock();
filename = filename.append(extension);
\end{verbatim}
% The 'filename = ' assignment is unnecessary in the example
% and is not good Java style, but it illustrates the type-checking against the
% return value of the call to append.




\section{More locking details\label{lock-details}}

This section gives some details that are helpful for understanding how Java
locking and the Lock Checker works.

\subsection{Two types of locking:  monitor locks and explicit locks\label{lock-two-types}}

Java provides two types of locking:  monitor locks and explicit locks.

\begin{itemize}
\item
  A \<synchronized(\emph{E})> block acquires the lock on the value of
  \emph{E}; similarly, a method declared using the \<synchronized> method
  modifier acquires the lock on the method receiver when called.
  (More precisely,
  the current thread locks the monitor associated with the value of
  \emph{E}; see \href{https://docs.oracle.com/javase/specs/jls/se8/html/jls-17.html#jls-17.1}{JLS \S17.1}.)
  The lock is automatically released when execution exits the block or the
  method body, respectively.
  We use the term ``monitor lock'' for a lock acquired using a
  \<synchronized>  block or \<synchronized> method modifier.
\item A method call, such as
  \sunjavadoc{java/util/concurrent/locks/Lock.html\#lock--}{Lock.lock()},
  acquires a lock that implements the
  \sunjavadoc{java/util/concurrent/locks/Lock.html}{Lock}
  interface.
  The lock is released by another method call, such as
  \sunjavadoc{java/util/concurrent/locks/Lock.html\#unlock--}{Lock.unlock()}.
  We use the term ``explicit lock'' for a lock expression acquired in this
  way.
\end{itemize}

You should not mix the two varieties of locking, and the Lock Checker
enforces this.  To prevent an object from being used both as a monitor and
an explicit lock, the Lock Checker issues a warning if a
\<synchronized(\emph{E})> block's expression \<\emph{E}> has a type that
implements \sunjavadoc{java/util/concurrent/locks/Lock.html}{Lock}.
% The Lock Checker does not keep track of which locks are monitors
% and which are explicit, so this check is necessary for the Lock Checker to
% function correctly, and it also alerts the programmer of a code smell.


\subsection{Held locks and held expressions; aliasing\label{lock-aliasing}}

Whereas Java locking is defined in terms of values, Java programs are
written in terms of expressions.
We say that a lock expression is held if the value to which the expression
currently evaluates is held.

The Lock Checker conservatively estimates the expressions that are held at
each point in a program.
The Lock Checker does not track aliasing
(different expressions that evaluate to the same value); it only considers
the exact expression used to acquire a lock to be held.  After any statement
that might side-effect a held expression or a lock expression, the Lock
Checker conservatively considers the expression to be no longer held.

Section~\ref{java-expressions-as-arguments} explains which Java
expressions the Lock Checker is able to analyze as lock expressions.


The \code{@LockHeld} and \code{@LockPossiblyHeld} type qualifiers are used internally by the Lock Checker
and should never be written by the programmer.
If you
see a warning mentioning \code{@LockHeld} or \code{@LockPossiblyHeld},
please contact the Checker Framework developers as it is likely to
indicate a bug in the Checker Framework.


\subsection{Run-time checks for locking\label{lock-runtime-checks}}

When you perform a run-time check for locking, such as
\<if (explicitLock.isHeldByCurrentThread())\{...\}> or
\<if (Thread.holdsLock(monitorLock))\{...\}>,
then the Lock Checker considers the lock expression to be held
within the scope of the test.  For more details, see
Section~\ref{type-refinement}.
% Note that the java.util.concurrent.locks.Lock interface does not include
% a run-time test, but ReentrantLock does.


\subsection{Discussion of default qualifier\label{lock-checker-default-qualifier}}

The default qualifier for unannotated types is \<@GuardedBy(\{\})>.
This default forces you to write explicit \<@GuardSatisfied> in method
signatures in the common case that clients ensure that all locks are held.

It might seem that \<@GuardSatisfied> would be a better default for
method signatures, but such a default would require even more annotations.
The reason is that \<@GuardSatisfied> cannot be used on fields.  If
\<@GuardedBy(\{\})> is the default for fields but \<@GuardSatisfied> is the
default for parameters and return types, then getters, setters, and many
other types of methods do not type-check without explicit lock qualifiers.


\subsection{Discussion of \<@Holding>\label{lock-checker-holding}}

A programmer might choose to use the \code{@Holding} method annotation in
two different ways:  to specify correctness constraints for a
synchronization protocol, or to summarize intended usage.  Both of these
approaches are useful, and the Lock Checker supports both.

\paragraph{Synchronization protocol\label{lock-checker-holding-synchronization-protocol}}

  \code{@Holding} can specify a synchronization protocol that
  is not expressible as locks over the parameters to a method.  For example, a global lock
  or a lock on a different object might need to be held.  By requiring locks to be
  held, you can create protocol primitives without giving up
  the benefits of the annotations and checking of them.

\paragraph{Method summary that simplifies reasoning\label{lock-checker-holding-method-summary}}

  \code{@Holding} can be a method summary that simplifies reasoning.  In
  this case, the \code{@Holding} doesn't necessarily introduce a new
  correctness constraint; the program might be correct even if the lock
  were not already acquired.

  Rather, here \code{@Holding} expresses a fact about execution:  when
  execution reaches this point, the following locks are known to be already held.  This
  fact enables people and tools to reason intra- rather than
  inter-procedurally.

  In Java, it is always legal to re-acquire a lock that is already held,
  and the re-acquisition always works.  Thus, whenever you write

\begin{Verbatim}
  @Holding("myLock")
  void myMethod() {
    ...
  }
\end{Verbatim}

\noindent
it would be equivalent, from the point of view of which locks are held
during the body, to write

\begin{Verbatim}
  void myMethod() {
    synchronized (myLock) {   // no-op:  re-acquire a lock that is already held
      ...
    }
  }
\end{Verbatim}


It is better to write a \code{@Holding} annotation rather than writing the
extra synchronized block.  Here are reasons:

\begin{itemize}
\item
  The annotation documents the fact that the lock is intended to already be
  held;  that is, the method's contract requires that the lock be held when
  the method is called.
\item
  The Lock Checker enforces that the lock is held when the method is
  called, rather than masking a programmer error by silently re-acquiring
  the lock.
\item
  The version with a synchronized statement can deadlock if, due to a programmer error,
  the lock is not already held.  The Lock Checker prevents this type of
  error.
\item
  The annotation has no run-time overhead.  The lock re-acquisition
  consumes time, even if it succeeds.
\end{itemize}


\section{Other lock annotations\label{lock-other-annotations}}

The Checker Framework's lock annotations are similar to annotations used
elsewhere.

If your code is already annotated with a different lock
annotation, the Checker Framework can type-check your code.
It treats annotations from other tools
exactly as if you had written the corresponding annotation from the
Lock Checker, as described in Figure~\ref{fig-lock-refactoring}.


% These lists should be kept in sync with LockAnnotatedTypeFactory.java .
\begin{figure}
\begin{center}
% The ~ around the text makes things look better in Hevea (and not terrible
% in LaTeX).

\begin{tabular}{ll}
\begin{tabular}{|l|}
\hline
 ~net.jcip.annotations.GuardedBy~ \\ \hline
 ~javax.annotation.concurrent.GuardedBy~ \\ \hline
\end{tabular}
&
$\Rightarrow$
%HEVEA ~org.checkerframework.checker.lock.qual.GuardedBy (for fields) or \ldots Holding (for methods)~
%BEGIN LATEX
\begin{tabular}{l}
 ~org.checkerframework.checker.lock.qual.GuardedBy (for fields), or~ \\
 ~org.checkerframework.checker.lock.qual.Holding (for methods)~
\end{tabular}
%END LATEX
\end{tabular}
\end{center}
%BEGIN LATEX
\vspace{-1.5\baselineskip}
%END LATEX
\caption{Correspondence between other lock annotations and the
  Checker Framework's annotations.}
\label{fig-lock-refactoring}
\end{figure}


\subsection{Relationship to annotations in \emph{Java Concurrency in Practice}\label{lock-jcip-annotations}}

The book \href{http://jcip.net/}{\emph{Java Concurrency in Practice}}~\cite{Goetz2006} defines a
\href{http://jcip.net.s3-website-us-east-1.amazonaws.com/annotations/doc/net/jcip/annotations/GuardedBy.html}{\code{@GuardedBy}} annotation that is the inspiration for ours.  The book's
\code{@GuardedBy} serves two related but distinct purposes:

\begin{itemize}
\item
  When applied to a field, it means that the given lock must be held when
  accessing the field.  The lock acquisition and the field access may occur
  arbitrarily far in the future.
\item
  When applied to a method, it means that the given lock must be held by
  the caller at the time that the method is called --- in other words, at
  the time that execution passes the \code{@GuardedBy} annotation.
\end{itemize}

The Lock Checker renames the method annotation to
\refqualclass{checker/lock/qual}{Holding}, and it generalizes the
\refqualclass{checker/lock/qual}{GuardedBy} annotation into a type annotation
that can apply not just to a field but to an arbitrary type (including the
type of a parameter, return value, local variable, generic type parameter,
etc.).  Another important distinction is that the Lock Checker's
annotations express and enforce a locking discipline over values, just like
the JLS expresses Java's locking semantics; by contrast, JCIP's annotations
express a locking discipline that protects variable names and does not
prevent race conditions.
  This makes the annotations more expressive and also more amenable
to automated checking.  It also accommodates the distinct
meanings of the two annotations, and resolves ambiguity when \<@GuardedBy>
is written in a location that might apply to either the method or the
return type.

(The JCIP book gives some rationales for reusing the annotation name for
two purposes.  One rationale is
that there are fewer annotations to learn.  Another rationale is
that both variables and methods are ``members'' that can be ``accessed''
and \code{@GuardedBy} creates preconditions for doing so.
Variables can be accessed by reading or writing them (putfield, getfield),
and methods can be accessed by calling them (invokevirtual,
invokeinterface).  This informal intuition is
inappropriate for a tool that requires precise semantics.)

% It would not work to retain the name \code{@GuardedBy} but put it on the
% receiver; an annotation on the receiver indicates what lock must be held
% when it is accessed in the future, not what must have already been held
% when the method was called.


\section{Possible extensions\label{lock-extensions}}

The Lock Checker validates some uses of locks, but not all.  It would be
possible to enrich it with additional annotations.  This would increase the
programmer annotation burden, but would provide additional guarantees.

Lock ordering:  Specify that one lock must be acquired before or after
another, or specify a global ordering for all locks.  This would prevent
deadlock.

Not-holding:  Specify that a method must not be called if any of the listed
locks are held.

These features are supported by
\href{http://clang.llvm.org/docs/ThreadSafetyAnalysis.html}{Clang's
  thread-safety analysis}.


% LocalWords:  quals GuardedBy JCIP putfield getfield invokevirtual
% LocalWords:  invokeinterface threadsafety Clang's GuardedByUnknown
%  LocalWords:  api 5cm lockexpr Dereferencing exprSet expr expr1 expr2
%  LocalWords:  GuardedByBottom exprSet1 exprSet2 GuardSatisfied 3cm pre
%  LocalWords:  PolyGuardedBy EnsuresLockHeld ReentrantLock boolean eset
%  LocalWords:  EnsuresLockHeldIf LockingFree ReleasesNoLocks str lock1
%  LocalWords:  MayReleaseLocks GuardedByName lock2 jls JLS LockHeld intra
%  LocalWords:  LockPossiblyHeld explicitLock isHeldByCurrentThread JCIP's
%  LocalWords:  holdsLock monitorLock cleanroom

\htmlhr
\chapter{Tainting checker\label{tainting-checker}}

The tainting checker prevents certain kinds of trust errors.
A \emph{tainted}, or untrusted, value is one that comes from an arbitrary,
possibly malicious source, such as user input or unvalidated data.
In certain parts of your application, using a tainted value can compromise
the application's integrity, causing it to crash, corrupt data, leak
private data, etc.

% Ought to have many more examples

For example, a user-supplied pointer, handle, or map key should be
validated before being dereferenced.
As another example, a user-supplied string should not be concatenated into a
SQL query, lest the program be subject to a 
\ahref{http://en.wikipedia.org/wiki/Sql_injection}{SQL injection} attack.
A location in your program where malicious data could do damage is
called a \emph{sensitive sink}.

A program must ``sanitize'' or ``untaint'' an untrusted value before using
it at a sensitive sink.  There are two general ways to untaint a value:
by checking
that it is innocuous/legal (e.g., it contains no characters that can be
interpreted as SQL commands when pasted into a string context), or by
transforming the value to be legal (e.g., quoting all the characters that
can be interpreted as SQL commands).  A correct program must use one of
these two techniques so that tainted values never flow to a sensitive sink.
The Tainting Checker ensures that your program does so.

If the Tainting Checker issues no warning for a given program, then no
tainted value ever flows to a sensitive sink.  However, your program is not
necessarily free from all trust errors.  As a simple example, you might
have forgotten to annotate a sensitive sink as requiring an untainted type,
or you might have forgotten to annotate untrusted data as having a tainted
type.


\section{Tainting annotations\label{tainting-annotations}}

% TODO: add both qualifiers explicitly, and then describe their relationship.

The Tainting type system uses the following annotations:
\begin{itemize}
\item
  \code{@\refclass{tainting/quals}{Untainted}} indicates
  a type that includes only untainted, trusted values.
\item
  \code{@\refclass{tainting/quals}{Tainted}} indicates
  a type that may include only tainted, untrusted values.
  \code{@Tainted} is a subtype of \code{@Untainted}.
\item
  \code{@\refclass{tainting/quals}{PolyTainted}} is a qualifier that is
  polymorphic over tainting (see Section~\ref{qualifier-polymorphism}).
\end{itemize}


\section{Tips on writing \code{@Untainted} annotations\label{writing-untainted}}

Most programs are designed with a boundary that surrounds sensitive
computations, separating them from untrusted values.  Outside this
boundary, the program may manipulate malicious values, but no malicious
values ever pass the boundary to be operated upon by sensitive
computations.

In some programs, the area outside the boundary is very small:  values are
sanitized as soon as they are received from an external source.  In other
programs, the area inside the boundary is very large:  values are sanitized
only immediately before being used at a sensitive sink.  Either approach
can work, so long as every possibly-tainted value is sanitized before it
reaches a sensitive sink.

Once you determine the boundary, annotating your program is easy:  put
\code{@Tainted} outside the boundary, \code{@Untainted} inside, and
\code{@SuppressWarnings("tainting")} at the validation or
sanitization routines that are used at the boundary.
% (Or, the Tainting Checker may indicate to you that the boundary
% does not exist or has holes through which tainted values can pass.)

The Tainting Checker's standard default qualifier is \code{@Tainted} (see
Section~\ref{defaults} for overriding this default).  This is the safest
default, and the one that should be used for all code outside the boundary
(for example, code that reads user input).  You can set the default
qualifier to \code{@Untainted} in code that may contain sensitive sinks.

The Tainting Checker does not know the intended semantics of your program,
so it cannot warn you if you mis-annotate a sensitive sink as taking
\code{@Tainted} data, or if you mis-annotate external data as
\code{@Untainted}.  So long as you correctly annotate the sensitive sinks
and the places that untrusted data is read, the Tainting Checker will
ensure that all your other annotations are correct and that no undesired
information flows exist.

As an example, suppose that you wish to prevent SQL injection attacks.  You
would start by annotating the
\sunjavadoc{java/sql/Statement.html}{Statement} class to indicate that the
\code{execute} operations may only operate on untainted queries
(Chapter~\ref{annotating-libraries} describes how to annotate external
libraries):

\begin{Verbatim}
  public boolean execute(@Untainted String sql) throws SQLException;
  public boolean executeUpdate(@Untainted String sql) throws SQLException; 
\end{Verbatim}


\section{\code{@Tainted} and \code{@Untainted} can be used for many purposes\label{tainting-many-uses}}

The \code{@Tainted} and \code{@Untainted} annotations have only minimal
built-in semantics.  In fact, the Tainting Checker provides only a small
amount of functionality beyond the Basic Checker
(Section~\ref{basic-checker}).  This lack of hard-coded behavior means that
the annotations can serve many different purposes.  Here are just a few
examples:

\begin{itemize}
\item
  Prevent SQL injection attacks:  \code{@Tainted} is external input,
  \code{@Untainted} has been checked for SQL syntax.
\item
  Prevent cross-site scripting attacks:  \code{@Tainted} is external input,
  \code{@Untainted} has been checked for JavaScript syntax.
\item
  Prevent information leakage:  \code{@Tainted} is secret data, 
  \code{@Untainted} may be displayed to a user.
\end{itemize}

In each case, you need to annotate the appropriate untainting/sanitization
routines.  This is similar to the \code{@Encrypted} annotation
(Section~\ref{encrypted-example}), where the cryptographic functions are
beyond the reasoning abilities of the type system.  In each case, the type
system verifies most of your code, and the \code{@SuppressWarnings}
annotations indicate the few places where human attention is needed.


If you want more specialized semantics, or you want to annotate multiple
types of tainting in a single program, then you can copy the definition of
the Tainting Checker to create a new annotation and checker with a more
specific name and semantics.  See Chapter~\ref{writing-a-checker} for more
details.


% LocalWords:  quals untaint PolyTainted mis untainting sanitization

\htmlhr
\chapter{Linear Checker for preventing aliasing\label{linear-checker}}

The Linear Checker implements type-checking for a linear type system.  A
linear type system prevents aliasing:  there is only one (usable) reference
to a given object at any time.  Once a reference appears on the right-hand
side of an assignment, it may not be used any more.  The same rule applies
for pseudo-assignments such as procedure argument-passing (including as the
receiver) or return.

One way of thinking about this is that a reference can only be used once,
after which it is ``used up''.  This property is checked statically at
compile time.  The single-use property only applies to use in an
assignment, which makes a new reference to the object; ordinary field
dereferencing does not use up a reference.

By forbidding aliasing, a linear type system can prevent problems such as
unexpected modification (by an alias), or ineffectual modification (after a
reference has already been passed to, and used by, other code).


% TODO: need better motivation.


To run the Linear Checker, supply the
\code{-processor org.checkerframework.checker.Linear.LinearChecker}
command-line option to javac.


Figure~\ref{fig-linear-example} gives an example of the Linear Checker's rules.

\begin{figure}
%BEGIN LATEX
\begin{smaller}
%END LATEX
\begin{Verbatim}
class Pair {
  Object a;
  Object b;
  public String toString() {
    return "<" + String.valueOf(a) + "," + String.valueOf(b) + ">";
  }
}

void print(@Linear Object arg) {
  System.out.println(arg);
}

@Linear Pair printAndReturn(@Linear Pair arg) {
  System.out.println(arg.a);
  System.out.println(arg.b);      // OK: field dereferencing does not use up the reference arg
  return arg;
}

@Linear Object m(Object o, @Linear Pair lp) {
  @Linear Object lo2 = o;         // ERROR: aliases may exist
  @Linear Pair lp3 = lp;          
  @Linear Pair lp4 = lp;          // ERROR: reference lp was already used
  lp3.a;                            
  lp3.b;                          // OK: field dereferencing does not use up the reference
  print(lp3);
  print(lp3);                     // ERROR: reference lp3 was already used
  lp3.a;                          // ERROR: reference lp3 was already used
  @Linear Pair lp4 = new Pair(...);
  lp4.toString();
  lp4.toString();                 // ERROR: reference lp4 was already used
  lp4 = new Pair();               // OK to reassign to a used-up reference
  // If you need a value back after passing it to a procedure, that
  // procedure must return it to you.
  lp4 = printAndReturn(lp4);
  if (...) {
    print(lp4);
  }
  if (...) {
    return lp4;                   // ERROR: reference lp4 may have been used
  } else {
    return new Object();
  }
}
\end{Verbatim}
%BEGIN LATEX
\end{smaller}
%END LATEX
\caption{Example of Linear Checker rules.}
\label{fig-linear-example}
\end{figure}


\section{Linear annotations\label{linear-annotations}}

The linear type system uses one user-visible annotation:
\refqualclass{checker/linear/quals}{Linear}.  The annotation indicates
a type for which each value may only have a single reference ---
equivalently, may only be used once on the right-hand side of an
assignment.

The full qualifier hierarchy for the linear type system includes three
types:
\begin{itemize}
\item
\code{@UsedUp} is the type of references whose object has been assigned to
another reference.  The reference may not be used in any way, including
having its fields dereferenced, being tested for equality with \<==>, or
being assigned to another reference.  Users never need to write this
qualifier.
\item
\code{@Linear} is the type of references that have no aliases, and that may
be dereferenced at most once in the future.  The type of \<new \emph{T}()> is
\code{@Linear \emph{T}} (the analysis does not account for the slim
possibility that an alias to \<this> escapes the constructor).
\item
\code{@NonLinear} is the type of references that may be dereferenced, and
aliases made, as many times as desired.  This is the default, so users only
need to write \code{@NonLinear} if they change the default.
\end{itemize}

% TODO: Should draw a picture rather than leaving this as text.

\noindent
\code{@UsedUp} is a supertype of \code{@NonLinear}, which is a
supertype of \code{@Linear}.

This hierarchy makes an assignment like

\begin{Verbatim}
  @Linear Object l = new Object();
  @NonLinear Object nl = l;
  @NonLinear Object nl2 = nl;
\end{Verbatim}

\noindent
legal.  In other words, the fact that an object is referenced by a
\<@Linear> type means that there is only one usable reference to it \emph{now},
not that there will \emph{never} be multiple usable references to it.
(The latter guarantee would be possible to enforce, but it is not what the
Linear Checker does.)


\section{Limitations\label{linear-limitations}}

The \<@Linear> annotation is supported and checked only on method
parameters (including the receiver), return types, and local variables.
Supporting \<@Linear> on fields would require a sophisticated alias
analysis or type system, and is future work.
% One could imagine "this-linear" for handling fields.

No annotated libraries are provided for linear types.  Most libraries would
not be able to use linear types in their purest form.  For example, you
cannot put a linearly-typed object in a hash table, because hash table
insertion calls \<hashCode>; \<hashCode> uses up the reference and does not
return the object, even though it does not retain any pointers to the
object.  For similar reasons, a collection of linearly-typed objects could
not be sorted or searched.

Our lightweight implementation is intended for use in the parts of your
program where errors relating to aliasing and object reuse are most likely.
You can use manual reasoning (and possibly an unchecked cast or warning
suppression) when objects enter or exit those portions of your program, or
when that portion of your program uses an unannotated library.


% LocalWords:  quals UsedUp hashCode NonLinear

\htmlhr
\chapter{Regex checker\label{regex-checker}}

The Regex Checker prevents, at compile-time, use of syntactically invalid
regular expressions.

A regular expression, or regex, is a pattern for matching certain strings
of text.  In Java, a programmer writes a regular expression as a string.
At run time, the string is ``compiled'' into an efficient internal form
(\sunjavadoc{java/util/regex/Pattern.html}{Pattern}) that is used for
text-matching.

The syntax of regular expressions is complex, so it is easy to make a
mistake.  It is also easy to accidentally use a regex feature from another
language that is not supported by Java (see section ``Comparison to Perl
5'' in the \sunjavadoc{java/util/regex/Pattern.html}{Pattern} Javadoc).
Ordinarily, the programmer does not learn of these errors until run time.
The Regex checker warns about these problems at compile time.


\section{Regex annotations\label{regex-annotations}}

The Regex Checker uses one annotation only:
\code{@\refclass{regex/quals}{ValidRegex}}, to indicate valid regular
expression \code{String}s.

The checker implicitly adds the \code{ValidRegex} qualifier to any
\code{String} literal that is a valid regex.


\section{Running the Regex Checker\label{regex-running}}

The Regex Checker can be invoked by running the following command:

\begin{Verbatim}
  javac -processor checkers.regex.RegexChecker MyFile.java ...
\end{Verbatim}


% LocalWords:  Regex regex quals

\htmlhr
\chapter{Internationalization checker\label{i18n-checker}}

The Internationalization Checker verifies that your code is properly
internationalized.  Internationalization is the process of adapting
software to different languages and locales.  Internationalization is
sometimes called localization (though the terms are not identical), and is
sometimes called i18n (because the word starts with ``i'', ends with ``n'',
and has 18 characters in between).

The checker focuses on one aspect of localization:  user-visible strings
should be presented in the user's own language, such as English, French, or
German.  This is achieved by looking up keys in a localization resource,
which maps keys to user-visible strings.  For instance, one version of a
resource might map \code{"CANCEL\_STRING"} to
\code{"Cancel"}, and another version of the same resource might map
\code{"CANCEL\_STRING"} to \code{"Abbrechen"}.

There are other aspects to localization, such as formatting of dates (3/5
vs.~5/3 for March 5), that the checker does not check.

The Internationalization Checker verifies these two properties:

\begin{enumerate}

\item
  Any user-visible text should be obtained from a localization resource.
  For example, \code{String} literals should not be output to the user.

\item
  When looking up keys in a localization resource, the key should exist in
  that resource.  This check catches incorrect or misspelled localization
  keys.

\end{enumerate}


\section{Internationalization annotations\label{i18n-annotations}}

The Internationalization Checker supports two annotations:

\begin{enumerate}
\item \code{@\refclass{i18n/quals}{Localized}}: indicates that the qualified
\code{String} is a message that has been localized and/or formatted with
respect to the used locale.

\item \code{@\refclass{i18n/quals}{LocalizableKey}}: indicates that the
qualified \code{String} or \code{Object} is a valid key found in the
localization resource.
\end{enumerate}

You may need to add the \code{@Localized} annotation to more methods in the
JDK or other libraries, or in your own code.


\section{Running the Internationalization Checker\label{i18n-running}}

The Internationalization Checker can be invoked by running the following
command:

\begin{Verbatim}
  javac -processor checkers.i18n.I18nChecker -Abundlename=MyResource MyFile.java ...
\end{Verbatim}

You must specify the localization resource, which maps keys to user-visible
strings.  The checker supports two types of localization resource:
ResourceBundle or property file.  You should specify just one of the
following two command-line options:

\begin{enumerate}

\item \code{-Abundlename=\emph{resource\_name}}

  \emph{resource\_name} is the name of the resource to be used with
  \sunjavadoc{java/util/ResourceBundle.html#getBundle(java.lang.String, java.util.Locale, java.lang.ClassLoader)}{ResourceBundle.getBundle()}.
  The checker uses the default \code{Locale} and \code{ClassLoader} in the
  compilation system.
  (For a tutorial about \code{ResourceBundle}s, see
  \myurl{http://java.sun.com/developer/technicalArticles/Intl/ResourceBundles/}.)

\item \code{-Apropfile=\emph{prop\_file}}

  \emph{prop\_file} is the name of a properties file that maps
  localization keys to localized message.  The file format is described in
  the Javadoc for 
  \sunjavadoc{java/util/Properties.html#load(java.io.Reader)}{Properties.load()}.

\end{enumerate}

\htmlhr
\chapter{Basic checker\label{basic-checker}}

The Basic checker enforces only subtyping rules.  It operates over
annotations specified by a user on the command line.  Thus, users can
create a simple type checker without writing any code beyond definitions of
the type qualifier annotations.

The Basic checker can accommodate all of the type system enhancements that
can be declaratively specified (see Chapter~\ref{writing-a-checker}).
This includes type introduction rules (implicit
annotations, e.g., literals are implicitly considered \code{@\refclass{nullness/quals}{NonNull}}) via
the \code{@\refclass{quals}{ImplicitFor}} meta-annotation, and other features such as
flow-sensitive type qualifier inference (Section~\ref{type-refinement}) and
qualifier polymorphism (Section~\ref{qualifier-polymorphism}).

The Basic checker is also useful to type system designers who wish to
experiment with a checker before writing code; the Basic checker
demonstrates the functionality that a checker inherits from the Checker
Framework.

If you need typestate analysis, then you can extend a typestate checker,
much as you would extend the Basic Checker if you do not need typestate
analysis.  For more details (including a definition of ``typestate''), see
Chapter~\ref{typestate-checker}.

For type systems that require special checks (e.g., warning about
dereferences of possibly-null values), you will need to write code and
extend the framework as discussed in Chapter~\ref{writing-a-checker}.


\section{Using the Basic checker\label{basic-using}}

The Basic checker is used in the same way as other checkers (using the
\code{-processor checkers.basic.BasicChecker} option; see Chapter~\ref{using-a-checker}), except that it
requires an additional annotation processor argument via the standard
``\code{-A}'' switch:

\begin{itemize}

\item
\code{-Aquals}: this option specifies a comma-no-space-separated list of
the fully-qualified class
names of the annotations used as qualifiers in the custom type system.
%
It serves the same purpose as the \code{@\refclass{quals}{TypeQualifiers}}
annotation used by other checkers (see section
\ref{writing-compiler-interface}).

The annotations listed in \code{-Aquals} must be accessible to
the compiler during compilation in the classpath.  In other words, they must
already be compiled before you run the Basic checker with \code{javac}; it
is not sufficient to supply their source files on the command line.

\end{itemize}

To suppress a warning issued by the basic checker, use a 
\code{@\sunjavadoc{java/lang/SuppressWarnings.html}{SuppressWarnings}}
annotation, with the argument being the unqualified, uncapitalized name of
any of the annotations passed to \code{-Aquals}.  This will suppress all
warnings, regardless of which of the annotations is involved in the
warning.  (As a matter of style, you should choose one of the annotations
as your \code{@SuppressWarnings} key and stick with it for that entire type
hierarchy.)


\section{Basic checker example\label{basic-example}\label{encrypted-example}}

Consider a hypothetical \code{Encrypted} type qualifier, which denotes that the
representation of an object (such as a \code{String}, \code{CharSequence}, or
\code{byte[]}) is encrypted. To use the Basic checker for the \code{Encrypted}
type system, follow three steps.

\begin{enumerate}
\item
 Define an annotation for the \code{Encrypted} qualifier:

\begin{Verbatim}
package myquals;

import checkers.quals.*;

/**
 * Denotes that the representation of an object is encrypted.
 * ...
 */
@TypeQualifier
@SubtypeOf(Unqualified.class)
@Target({ElementType.TYPE_PARAMETER, ElementType.TYPE_USE})
public @interface Encrypted {}
\end{Verbatim}

Don't forget to compile this class:

\begin{Verbatim}
$ javac myquals/Encrypted.java
\end{Verbatim}

The resulting \<.class> file should either be on your classpath, or on the
processor path (set via the \<-processorpath> command-line option to javac).

\item 
  Write \code{@Encrypted} annotations in your program (YourProgram.java):

\begin{Verbatim}
import myquals.Encrypted;

...

public @Encrypted String encrypt(String text) {
    // ...
}

// Only send encrypted data!
public void sendOverInternet(@Encrypted String msg) {
    // ...
}

void sendText() {
    // ...
    @Encrypted String ciphertext = encrypt(plaintext);
    sendOverInternet(ciphertext);
    // ...
}

void sendPassword() {
    String password = getUserPassword();
    sendOverInternet(password);
}
\end{Verbatim}

You may also need to add \code{@SuppressWarnings} annotations to the
\code{encrypt} and \code{decrypt} methods.  Analyzing them is beyond the
capability of any realistic type system.

\item
  Invoke the compiler with the Basic checker, specifying the
  \code{@Encrypted} annotation using the \code{-Aquals} option.
  You should add the \code{Encrypted} classfile to the processor classpath:

\begin{Verbatim}
$ javac -processorpath myqualspath -processor checkers.basic.BasicChecker \
        -Aquals=myquals.Encrypted YourProgram.java

YourProgram.java:42: incompatible types.
found   : java.lang.String
required: @myquals.Encrypted java.lang.String
    sendOverInternet(password);
                     ^
\end{Verbatim}

\end{enumerate}


\htmlhr
\chapter{Typestate checker\label{typestate-checker}}

In a regular type system, a variable has the same type throughout its
scope.
In a typestate system, a variable's type can change as operations
are performed on it.

The most common example of typestate is for a \<File> object.  Assume a file
can be in two states, \<@Open> and \<@Closed>.  Calling the \<close()> method
changes the file's state.  Any subsequent attempt to read, write, or close
the file will lead to a run-time error.  It would be better for the type
system to warn about such problems, or guarantee their absence, at compile
time.

Just as you can extend the Basic Checker to create a type checker, you can
extend a typestate checker to create a type checker that supports typestate
analysis.  Two extensible typestate analyses that build on the Checker
Framework are available.  One is by Adam Warski:
\myurl{http://www.warski.org/typestate.html}.
The other is by Daniel Wand:
\myurl{http://typestate.ewand.de/}.


\section{Comparison to flow-sensitive type refinement\label{typestate-vs-type-refinement}}

The Checker Framework's flow-sensitive type refinement
(Section~\ref{type-refinement}) implements a form of typestate analysis.
For example, after code that tests a variable against null, the Nullness
Checker (Chapter~\ref{nullness-checker}) treats the variable's type as
\<@NonNull \emph{T}>, for some \<\emph{T}>\@.

For many type systems, flow-sensitive type refinement is sufficient.  But
sometimes, you need full typestate analysis.  This section compares the
two.  (Dependent types and unused variables
(Section~\ref{unused-fields-and-dependent-types}) also have similarities
with typestate analysis and can occasionally substitute for it.  For
brevity, this discussion omits them.)

A typestate analysis is easier for a user to create or extend.
Flow-sensitive type refinement is built into the Checker Framework and is
optionally extended by each checker.  Modifying the rules requires writing
Java code in your checker.  By contrast, it is possible to write a simple
typestate checker declaratively, by writing annotations on the methods
(such as \<close()>) that change a reference's typestate.

A typestate analysis can change a reference's type to something that is not
consistent with its original definition.  For example, suppose that a
programmer decides that the \<@Open> and \<@Closed> qualifiers are
incomparable --- neither is a subtype of the other.  A typestate analysis
can specify that the \<close()> operation converts an \<@Open File> into a
\<@Closed File>.  By contrast, flow-sensitive type refinement can only give
a new type that is a subtype of the declared type --- for flow-sensitive
type refinement to be effective, \<@Closed> would need to be a child of
\<@Open> in the qualifier hierarchy (and \<close()> would need to be
treated specially by the checker).


\htmlhr
\chapter{External checkers\label{external-checkers}}

The checker framework has been used to build other checkers that are not
distributed together with the framework.

If you want a reference to your checker included in this chapter,
send us a link and short description of your checker, 


\section{Units and dimensions checker\label{units-checker}}

A checker for units and dimensions is available at
\url{http://www.lexspoon.org/expannots/}.


\section{Thread locality checker}

Loci, a checker for thread locality, is available at
\url{http://www.it.uu.se/research/upmarc/loci/}.
Developer resources are available at the project page
\url{http://java.net/projects/loci/}.


% In a mail from Amanj Mahmud <amanjpro@gmail.com> on 28.03.2011:

% The plugin name: 
% ``Loci: A Pluggable Type Checker for Expressing Thread Locality in
% Java''

% Project homepage: http://www.it.uu.se/research/upmarc/loci

% Project's developer's page: http://java.net/projects/loci


\section{Safety-Critical Java checker}

A checker for Safety-Critical Java (SCJ, JSR 302) is available at
\url{http://sss.cs.purdue.edu/projects/oscj/checker/checker.html}.
Developer resources are available at the project page
\url{http://code.google.com/p/scj-jsr302/}.


% In a mail from Aleš Plšek <aplsek@gmail.com> on 29.03.2011:

% Name: SCJ Checker
% WWW: http://sss.cs.purdue.edu/projects/oscj/checker/checker.html
% Source-Code Repository: http://code.google.com/p/scj-jsr302/

% Description: The SCJ Checker implements verification of a set of
% annotations defined by the Safety-Critical Java standard (JSR-302).
% The checker mainly focuses on proving memory safety of Java programs
% that use a region-based memory management.

% Publications: Static checking of safety critical Java annotations:
% http://portal.acm.org/citation.cfm?doid=1850771.1850792



% LocalWords:  TODO ImplicitFor Aquals TypeQualifiers sourcepath java NonNull
% LocalWords:  CharSequence classpath nullness quals SuppressWarnings classfile
% LocalWords:  uncapitalized processorpath Warski

\htmlhr
\chapter{Advanced type system features\label{advanced-type-system-features}}

This section describes features that are automatically supported by every
checker written with the Checker Framework.
You may wish to skim or skip this section on first reading.  After you have
used a checker for a little while and want to be able to express more
sophisticated and useful types, or to understand more about how the Checker
Framework works, you can return to it.


\section{Polymorphism and generics\label{polymorphism}}

\subsection{Generics (parametric polymorphism or type polymorphism)\label{generics}}

The Checker Framework fully supports
type-qualified Java generic types (also known in the research literature as ``parametric
polymorphism'').  Before running any checker, we recommend that you eliminate
raw types from your code (e.g., your code should use \code{List<...>} as
opposed to \code{List}).
Using generics helps prevent type errors just as using a pluggable
type-checker does.
% Should say why, or what are the consequences of violating this.

When instantiating a generic type,
clients supply the qualifier along with the type argument, as in
\code{List<@NonNull String>}.


\paragraph{Restricting instantiation of a generic class}

There are two ways to restrict the type qualifiers that may be used on
the actual type argument when instantiating a generic class.

The first technique is the standard Java approach of using the
\code{extends} or \code{super} clause to supply an upper or lower bound.
For example:

\begin{Verbatim}
  MyClass<T extends @NonNull Object> { ... }

  MyClass<@NonNull String> m1;       // OK
  MyClass<@Nullable String> m2;      // error
\end{Verbatim}

The second technique is to write a type annotation on the declaration of a
generic type parameter, which specifies the exact annotation that is
required on the actual type argument, rather than just a bound.  For example:

\begin{Verbatim}
  class MyClassNN<@NonNull T> { ... }
  class MyClassNble<@Nullable T> { ... }

  MyClassNN<@NonNull Number> v1;     // OK
  MyClassNN<@Nullable Number> v2;    // error
  MyClassNble<@NonNull Number> v4;   // error
  MyClassNble<@Nullable Number> v3;  // OK
\end{Verbatim}

A way to view a type annotation on a generic type parameter declaration is
as syntactic sugar for the annotation on both the \<extends> and the
\<super> clauses of the declaration.  For example, these two declarations
have the same effect:

\begin{Verbatim}
  class MyClassNN<@NonNull T> { ... }
  class MyClassNN<T extends @NonNull Object super @NonNull void> { ... }
\end{Verbatim}

\noindent
except that the latter is not legal Java syntax.  The syntactic sugar is
necessary because of two limitations in Java syntax:  it is illegal to
specify both the upper and the
lower bound, and it is impossible to specify a type annotation for a lower
bound without also specifying a type (use of \<void> is illegal).

If a type parameter declaration is annotated with \code{@A}, and a bound is
also given, then the annotation applies everywhere that there is no
explicit annotation.  For example, the following pairs of declarations are
identical.

\begin{Verbatim}
  class MyClassNN<@A T> { ... }
  class MyClassNN<T extends @A Object super @A void> { ... }

  class MyClassNN<@A T extends Number> { ... }
  class MyClassNN<T extends @A Number super @A void> { ... }

  class MyClassNN<@A T extends @B Number> { ... }
  class MyClassNN<T extends @B Number super @A void> { ... }

  class MyClassNN<@A T super Number> { ... }
  class MyClassNN<T extends @A Object super @A Number> { ... }

  class MyClassNN<@A T super @B Number> { ... }
  class MyClassNN<T extends @A Object super @B Number> { ... }
\end{Verbatim}

We can see from the above that almost all of the following types mean
different things:

\begin{Verbatim}
  class MyList1<@Nullable T> { ... }
  class MyList2<@NonNull T> { ... }
  class MyList3<T extends @Nullable Object> { ... }
  class MyList4<T extends @NonNull Object> { ... } // same as MyList2
\end{Verbatim}

One way to express the difference is by comparing what expressions are
legal in the implementation of the list --- that is, what expressions may
appear in the ellipsis, such as inside a method's body.  Suppose each class
has, in the ellipsis, these declarations:

\begin{Verbatim}
  T t;
  @Nullable T nble;
  @NonNull T nn;
  void add(T arg) { ... }
  T get(int i) { ... }
\end{Verbatim}

\noindent
Then the following expressions would be legal, inside a given implementation.

\begin{tabular}{|l|c|c|c|c|} \hline
                        & MyList1 & MyList2 & MyList3 & MyList4 \\ \hline
  t = null;             & OK      & error   & error   & error   \\ \hline
  nble = null;          & OK      & OK      & OK      & OK      \\ \hline
  nn = null;            & error   & error   & error   & error   \\ \hline
  t = this.get(0);      & OK      & OK      & OK      & OK      \\ \hline
  nble = this.get(0);   & OK      & OK      & OK      & OK      \\ \hline
  nn = this.get(0);     & error   & OK      & error   & OK      \\ \hline
  this.add(t);          & OK      & OK      & OK      & OK      \\ \hline
  this.add(nble);       & OK      & error   & error   & error   \\ \hline
  this.add(nn);         & OK      & OK      & OK      & OK      \\ \hline
\end{tabular}

(This complete example appears as file
\<checker-framework/checkers/tests/nullness/GenericsExample.java>.)

%% This text is not very helpful.
% The
% implementation of \code{MyList2} may only place non-null objects in the
% list and may assume that retrieved elements are non-null.  The
% implementation of \code{MyList3} is similar in that it may only place
% non-null objects in the list, because it might be instantiated as, say,
% \code{MyList3<@NonNull Date>}.  When retrieving elements from the list,
% the implementation of \code{MyList3} must account for the fact that
% elements of \code{MyList3} may be null, because it might be instantiated
% as, say, \code{MyList3<@Nullable Date>}.
The differences are more
significant when the qualifier hierarchy is more complicated than just
\<@Nullable> and \<@NonNull>.


\paragraph{Defaults for bounds}
Ordinarily, a type parameter declaration with no extends clause means the
type parameter can be instantiated with any type argument at all.  For
example:

\begin{Verbatim}
  class C<T> { ... }
  class C<T extends Object> { ... }  // identical to previous line
\end{Verbatim}

\noindent
However, instantiation may be restricted if a default qualifier is in
effect (see Section~\ref{defaults}).  For example, the Nullness checker
(Chapter~\ref{nullness-checker}) uses a (configurable) default of
\<@NonNull> (see Section~\ref{null-defaults}).  That means that either
declaration above is interpreted as

\begin{Verbatim}
  class C<T extends @NonNull Object> { ... }
\end{Verbatim}

\noindent
and an instantiation such as \code{C<@Nullable Number>} is illegal.
In such a case, to permit all type arguments, the programmer would write

\begin{Verbatim}
  class C<T extends @Nullable Object> { ... }
\end{Verbatim}


It is possible to set the default qualifier for upper bounds separately
from other default qualifiers, by writing an annotation such as
\<@DefaultQualifier(value="Nullable", locations={DefaultLocation.UPPER\_BOUNDS})>.


\paragraph{Type annotations on a use of a generic type variable}

A type annotation on a generic type variable overrides/ignores any type
qualifier (in the same type hierarchy) on the corresponding actual type
argument.  For example, suppose that \code{T} is a formal type parameter.
Then using \code{@Nullable T} within the scope of \code{T} applies the type
qualifier \code{@Nullable} to the (unqualified) Java type of \code{T}.

Here is an example of applying a type annotation to a generic type
variable:

\begin{Verbatim}
  class MyClass2<T> {
    ...
    @Nullable T = null;
    ...
  }
\end{Verbatim}

\noindent
The type annotation does not restrict how \code{MyClass2} may be instantiated
(only the optional \code{extends} clause on the declaration of type
variable \code{T} would do so).  In other words, both
\code{MyClass2<@NonNull String>} and \code{MyClass2<@Nullable String>} are
legal, and in both cases \code{@Nullable T} means \code{@Nullable String}.
In \code{MyClass2<@Interned String>},
\code{@Nullable T} means \code{@Nullable @Interned String}.

% Note that a type annotation on a generic type variable does not act like
% other type qualifiers.  In both cases the type annotation acts as a type
% constructor, but as noted above they act slightly differently.


% %% This isn't quite right because a type qualifier is itself a type
% %% constructor.
% More formally, a type annotation on a generic type variable acts as a type
% constructor rather than a type qualifier.  Another example of a type
% constructor is \code{[]}.  Just as \code{T[]} is not the same type as
% \code{T}, \code{@Nullable T} is not (necessarily) the same type as
% \code{T}.


\subsection{Qualifier polymorphism\label{qualifier-polymorphism}}

The Checker Framework also supports type \emph{qualifier} polymorphism for
methods, which permits a single method to have multiple different qualified
type signatures.

To \emph{define} a polymorphic qualifier, mark the definition with
\<@\refclass{quals}{PolymorphicQualifier}>.  For example,
\<@\refclass{nullness/quals}{PolyNull}> is a polymorphic type
qualifier for the Nullness type system:

\begin{Verbatim}
  @PolymorphicQualifier
  @Target(ElementType.TYPE_USE)
  public @interface PolyNull { }
\end{Verbatim}

To \emph{use} a polymorphic qualifier, just write it on a type.
For example, you can write \<@PolyNull> anywhere that you would write
\<@NonNull> or \<@Nullable>.

A method written using a polymorphic qualifier conceptually has multiple
versions, somewhat like a template in C++ or the generics feature of Java.
In each version, each instance of the polymorphic qualifier has been
replaced by the same other qualifier from the hierarchy.  See the examples
below in Section~\ref{qualifier-polymorphism-examples}.

The method body must type-check with all signatures.  A method call is
type-correct if it type-checks under any one of the signatures.  If a call
matches multiple signatures, then the compiler uses the most specific
matching signature for the purpose of type-checking.  This is just like
Java's rule for resolving overriding methods, though there is no effect on
run-time dispatch or behavior.

Polymorphic qualifiers can be used on a method signature or body.
They may not be used on classes or fields.

%% I don't see why this is necessarily true; one could define @PolyNull1
%% and @PolyNull2.  It's not so relevant to the manual anyway, and raising
%% the point just makes type system bigots criticize the Checker Framework.
% Qualifier polymorphism is limited to a single qualifier variable per method.


\paragraph{Examples of using polymorphic qualifiers\label{qualifier-polymorphism-examples}}

As an example of the use of \<@PolyNull>, method \sunjavadoc{java/lang/Class.html#cast(java.lang.Object)}{Class.cast}
returns null if and only if its argument is \<null>:

\begin{Verbatim}
  @PolyNull T cast(@PolyNull Object obj) { ... }
\end{Verbatim}

\noindent
This is like writing:

\begin{Verbatim}
   @NonNull T cast( @NonNull Object obj) { ... }
  @Nullable T cast(@Nullable Object obj) { ... }
\end{Verbatim}

\noindent
except that the latter is not legal Java, since it defines two
methods with the same Java signature.


As another example, consider

\begin{Verbatim}
  @PolyNull T max(@PolyNull T x, @PolyNull T y);
\end{Verbatim}

\noindent
which is like writing

\begin{Verbatim}
   @NonNull T max( @NonNull T x,  @NonNull T y);
  @Nullable T max(@Nullable T x, @Nullable T y);
\end{Verbatim}

\noindent
Another way of thinking about which one of the two \code{max} variants is
selected is that the nullness annotations of (the declared types of) both
arguments are \emph{unified} to a type that is a supertype of both, also
known as the \emph{least upper bound} or lub.  If both
arguments are \code{@NonNull}, their unification (lub) is \<@NonNull>, and the
method return type is \<@NonNull>.  But if even one of the arguments is \<@Nullable>,
then the unification (lub) is \<@Nullable>, and so is the return type.


\paragraph{Use multiple polymorphic qualifiers in a method signature\label{qualifier-polymorphism-multiple-qualifiers}}

%% I can't think of a non-clumsy way to say this.
% Each method containing a polymorphic qualifier is (conceptually) expanded
% into multiple versions completely independently.

Usually, it does not make sense to write only a single instance of a polymorphic
qualifier in a method definition:  if you write one instance of (say)
\<@PolyNull>, then you should use at least two.  (An exception is a
polymorphic qualifier on an array element type; this section ignores that
case, but see below for further details.)

For example, there is no point to writing

\begin{Verbatim}
  void m(@PolyNull Object obj)
\end{Verbatim}

\noindent
which expands to

\begin{Verbatim}
  void m(@NonNull Object obj)
  void m(@Nullable Object obj)
\end{Verbatim}

This is no different (in terms of which calls to the method will
type-check) than writing just

\begin{Verbatim}
  void m(@Nullable Object obj)
\end{Verbatim}

The benefit of polymorphic qualifiers comes when one is used multiple times
in a method, since then each instance turns into the same type qualifier.
Most frequently, the polymorphic qualifier appears on at least one formal
parameter and also on the return type.  It can also be useful to have
polymorphic qualifiers on (only) multiple formal parameters, especially if
the method side-effects one of its arguments.
For example, consider

\begin{Verbatim}
void moveBetweenStacks(Stack<@PolyNull Object> s1, Stack<@PolyNull Object> s2) {
  s1.push(s2.pop());
}
\end{Verbatim}

\noindent
In this example, if it is acceptable to rewrite your code to use Java
generics, the code can be even cleaner:

\begin{Verbatim}
<T> void moveBetweenStacks(Stack<T> s1, Stack<T> s2) {
  s1.push(s2.pop());
}
\end{Verbatim}


%% It would be nice to give an example that isn't too contrived.


\paragraph{Using a single polymorphic qualifier on an element type\label{qualifier-polymorphism-element-types}}

There is an exception to the general rule that a polymorphic qualifier
should be used multiple times in a signature.  It can make sense to use a
polymorphic qualifier just once, if it is on an array or generic element
type.

For example, consider a routine that returns the first index, in an array
or collection, of a given element:

\begin{Verbatim}
  public static int indexOf(@PolyNull Object[] a, Object elt) { ... }

  public static int indexOf(Collection<@PolyNull Object> a, Object elt) { ... }
\end{Verbatim}

If \<@PolyNull> were replaced with either \<@Nullable> or \<@NonNull>, then
some safe client calls would be rejected.

Of course, it would be better style to use a generic method, as in either
of these signatures (and likewise for the \<Collection> version):

\begin{Verbatim}
 public static <T> int indexOf(T[] a, /*@Nullable*/ Object elt) { ... }
 public static <T> int indexOf(T[] a, T elt) { ... }
\end{Verbatim}

In conclusion, use of a single polymorphic qualifier may be necessary in
legacy code, but can be avoided by use of better code style.


\section{Unused fields and dependent types\label{unused-fields-and-dependent-types}}

Sometimes, the type of a field depends on the qualifier on the receiver.
The Checker Framework supports two varieties of such a field:  a field that
may not be used if the receiver has a given qualifier, and a fields whose
qualifier changes based on the qualifier of the receiver.
(Also see the discussion of typestate checkers, in
Chapter~\ref{typestate-checker}.)


\subsection{Unused fields\label{unused-fields}}

A Java subtype can have more fields than its supertype.  You can simulate
the same effect for type qualifiers:  a given field may not be accessed via
a reference with a supertype qualifier, but can be accessed via a reference
with a subtype qualifier.

This permits you to restrict use of a field to certain contexts.

The \code{@\refclass{quals}{Unused}} annotation
on a field declares that the field may not be accessed via a receiver of
the given qualified type (or any supertype).


\subsection{Dependent types\label{dependent-types}}

A variable has a \emph{dependent type} if its type depends on some other
value or type.
%  --- the type is dynamically, not statically, determined.
% (Type-safety can still be statically determined, though.)

The Checker Framework supports a form of dependent types, via the
\code{@\refclass{quals}{Dependent}} annotation.
This annotation changes the type of a field or variable, based on the
qualified type of the receiver (\code{this}).  This can be viewed as a more
expressive form of polymorphism (see Section~\ref{polymorphism}).  It can
also be seen as a way of linking the meanings of two type qualifier
hierarchies.

When the \code{@\refclass{quals}{Unused}} annotation is sufficient, you
should use it instead of \code{@Dependent}.


\subsection{Example\label{dependent-types-example}}

Suppose we have a class \code{Person} and a field \code{spouse} that is
non-\code{null} if the person is married.  We could declare this as

\begin{Verbatim}
  class Person {
    ...
    // non-null if this person is married
    @Nullable Person spouse;
    ...
  }
\end{Verbatim}

Now, suppose that we have defined the qualifier hierarchy in which
\code{@Single} (meaning ``not married'') is a supertype of \code{@Married}.
A more informative declaration for \<Person> would be

\begin{Verbatim}
  class Person {
    ...
    @Nullable @Dependent(result=NonNull.class, when=Married.class) Person spouse;
    ...
  }
\end{Verbatim}

If a person is known to be \code{@Married}, the
\code{spouse} field is known to be non-\code{null}:

\begin{Verbatim}
  class Person {
    ...

    void celebrateWeddingAnniversary() @Married {
      System.out.println("Happy anniversary, "
                         + spouse.toString()); // no possible null pointer exception
    }

    ...
  }
\end{Verbatim}

\noindent
Without the \code{@\refclass{quals}{Dependent}} annotation on the
declaration of the \code{spouse} variable, the Nullness Checker would
complain that \code{toString} was invoked on a possibly-\code{null}
value.

An even better declaration is

\begin{Verbatim}
  class Person {
    ...
    @Unused(when=Single.class) @NonNull Person spouse;
    ...
  }
\end{Verbatim}

Then, if a person is known to be \code{@Married} (or more
appropriately non-\code{@Single}), the \code{spouse} field is known to
be non-\code{null}.  Also, if a person is known to be \code{@Single},
the \code{spouse} field may not be accessed:

\begin{Verbatim}
  @Single Person person = ...;
  Person spouse = person.spouse;  // invalid field access
  ...
\end{Verbatim}


\section{The effective qualifier on a type (defaults and inference)\label{effective-qualifier}}

A checker sometimes treats a type as having a slightly different qualifier
than what is written on the type --- especially if the programmer wrote no
qualifier at all.
Most readers can skip this section on first reading, because you will
probably find the system simply ``does what you mean'', without forcing
you to write too many qualifiers in your program.
In particular, qualifiers in method bodies are extremely rare.

  The following steps determine the effective
qualifier on a type --- the qualifier that the checkers treat as being present.

\begin{enumerate}
\item
  The type system adds implicit qualifiers.  Implicit qualifiers can be
  built into a type system (Section~\ref{writing-type-introduction}), in
  which case the type system's documentation should explain all of the type
  system's implicit qualifiers.  Or, a programmer may introduce an implicit
  annotation on each use of class $C$ by writing a qualifier on the
  declaration of class $C$.

\begin{itemize}
\item
  Example 1 (built-in):  In the Nullness type system,
  \<enum> values are never null, nor is a method receiver.
\item
  Example 2 (built-in):  In the Interning type system, string literals
  and \<enum> values are always interned.
\end{itemize}

\item
  If a type qualifier is present in the source code, that qualifier is used.

  If the type has an implicit qualifier, then it is an error to write an
  explicit qualifier that is equal to (redundant with) or a supertype of
  (weaker than) the implicit qualifier.  A programmer may strengthen
  (write a subtype of) an implicit qualifier, however.

\item
  If there is no implicit or explicit qualifier on a type, then a default
  qualifier may be applied; see Section~\ref{defaults}.

  \smallskip

  At this point, every type has a qualifier.

\item
  The type system may refine a qualified type on a local variable --- that
  is, treat it as a subtype of how it was declared or defaulted.  This
  refinement is always sound and has the effect of eliminating false
  positive error messages.  See Section~\ref{type-refinement}.

  % Type
  % qualifier refinement is implemented by the \refclass{flow}{Flow} class.

\end{enumerate}



\subsection{Default qualifier for unannotated types\label{defaults}}

A type system designer, or an end-user programmer, can cause unannotated
references to be treated as if they had a default annotation.

There are several defaulting mechanisms, for convenience and flexibility.
When determining the default qualifier for a use of a type, the following
rules are used in order, until one applies.
\begin{itemize}
\item
  Use the innermost user-written \code{@DefaultQualifier}, as explained in
  this section.
\item
  Use the default specified by the type system designer
  (Section~\ref{typesystem-defaults}).
\item
  Use \code{@\refclass{quals}{Unqualified}}, which the framework
  inserts to avoid ambiguity and simplify the programming interface for
  type system designers.  Users do not have to worry about this detail,
  but type system implementers can rely on the fact that some
  qualifier is present.
\end{itemize}

% (Implementation detail:  setting defaults is implemented by the
% \refclass{util}{QualifierDefaults} class.)


The end-user programmer specifies a default qualifier by writing the \code{@\refclass{quals}{DefaultQualifier}}
annotation on a package, class, method, or variable declaration.  The
argument to \<@\refclass{quals}{DefaultQualifier}> is the \code{String}
name of an annotation.  It may be a short name like \code{"NonNull"}, if an
appropriate import statement exists.  Otherwise, it should be
fully-qualified, like \code{"checkers.nullness.quals.NonNull"}.
The optional second argument indicates where the default
applies.  If the second argument is omitted, the specified annotation is
the default in all locations.  See the Javadoc of \refclass{quals}{DefaultQualifier} for details.

For example, using the Nullness type system (Chapter~\ref{nullness-checker}):

\begin{Verbatim}
import checkers.quals.*;        // for DefaultQualifier[s]
import checkers.nullness.quals.NonNull;

@DefaultQualifier("NonNull"),
class MyClass {

  public boolean compile(File myFile) { // myFile has type "@NonNull File"
    if (!myFile.exists())          // no warning: myFile is non-null
      return false;
    @Nullable File srcPath = ...;  // must annotate to specify "@Nullable File"
    ...
    if (srcPath.exists())          // warning: srcPath might be null
      ...
  }

  @DefaultQualifier("Mutable")
  public boolean isJavaFile(File myfile) {  // myFile has type "@Mutable File"
    ...
  }
}
\end{Verbatim}

If you wish to write multiple
\<@\refclass{quals}{DefaultQualifier}> annotations at a single location,
use
\<@\refclass{quals}{DefaultQualifiers}> instead.  For example:

\begin{Verbatim}
@DefaultQualifiers({
  @DefaultQualifier("NonNull"),
  @DefaultQualifier("Mutable")
})
\end{Verbatim}


If \code{@DefaultQualifier}[\code{s}] is placed on a package (via the
\<package-info.java> file), then it applies to the given package \emph{and}
all subpackages.
% This is slightly at odds with Java's treatment of packages of different
% names as essentially unrelated, but is more intuitive and useful.

Recall that an annotation on a class definition indicates an implicit
qualifier (Section~\ref{effective-qualifier}) that can only be
strengthened, not weakened.  This can lead to unexpected results if
the default qualifier applies to a class definition.  Thus, you may want to
put explicit qualifiers on class declarations (which prevents the default
from taking effect), or exclude class declarations from defaulting.


%% Don't even bother to bring this up; it will just sow confusion without
%% being helpful.
% For some type systems, a user may not specify a default qualifier, or doing
% so prevents giving any other qualifier to any reference.  This is a
% consequence of the design of the type system; see
% Section~\ref{bottom-qualifier}.


When a programmer omits an \<extends> clause at a declaration of a type
parameter, the default still applies to the implicit upper bound.  For
example, consider these two declarations:

\begin{Verbatim}
  class C<T> { ... }
  class C<T extends Object> { ... }  // identical to previous line
\end{Verbatim}

\noindent
The two declarations are treated identically by Java, and the default
qualifier applies to the \<Object> upper bound whether it is implicit or
explicit.  (The @NonNull default annotation applies only to the upper bound
in the \<extends> clause, not to the lower bound in the inexpressible
implicit \<super void> clause.)


\subsection{Automatic type refinement (flow-sensitive type qualifier inference)\label{type-refinement}}

In order to reduce the burden of annotating types in your program, the
checkers soundly treat certain variables and expressions as having a
subtype of their declared or defaulted (Section~\ref{defaults})
type.  This functionality
never introduces unsoundness or causes an error to be missed:  it merely
suppresses false positive warnings.

By default, all checkers, including new checkers that you write, can take
advantage of this functionality.  Most of the time, users don't have to
think about, and may not even notice, this feature of the framework.  The
checkers simply do the right thing even when a programmer forgets an
annotation on a local variable, or when a programmers writes an
unnecessarily general type in a declaration.

If you are curious or want more details about this feature, then read on.

As an example, the Nullness checker (Chapter~\ref{nullness-checker}) can automatically
determine that certain variables are non-null, even if they were explicitly
or by default annotated as nullable.
The checker treats a variable or expression as \code{@\refclass{nullness/quals}{NonNull}}
\begin{itemize}
\item
starting at the time that it is either
assigned a non-null value or checked against null (e.g., via an assertion,
\code{if} statement, or being dereferenced)
\item
until it might be re-assigned (e.g.,
via an assignment that might affect this variable, or via a method call
that might affect this variable).
\end{itemize}

As with explicit annotations, the implicitly non-null types permit
dereferences and assignments to non-null types, without
compiler warnings.

Consider this code, along with comments indicating whether the
Nullness checker (Chapter~\ref{nullness-checker}) issues a warning.  Note that the same expression may yield a
warning or not depending on its context.

\begin{Verbatim}
  // Requires an argument of type @NonNull String
  void parse(@NonNull String toParse) { ... }

  // Argument does NOT have a @NonNull type
  void lex(@Nullable String toLex) {
    parse(toLex);        // warning:  toLex might be null
    if (toLex != null) {
      parse(toLex);      // no warning:  toLex is known to be non-null
    }
    parse(toLex);        // warning:  toLex might be null
    toLex = new String(...);
    parse(toLex);        // no warning:  toLex is known to be non-null
  }
\end{Verbatim}

If you find examples where you think a value should be inferred to have
(or not have) a
given annotation, but the checker does not do so, please submit a bug
report (see Section~\ref{reporting-bugs}) that includes a small piece of
Java code that reproduces the problem.

% Flow-sensitive non-null inference has been implemented for the following
% varieties of expressions:
%
% \begin{itemize}
% \item null checks in if/else statements
% \item null checks in assert statements
% \item null checks that result in a return or thrown exception, or call System.exit
% \item assignments from new class/array expressions
% \end{itemize}
%
% \emph{Note:} The items in the above list exclude complex null checks, i.e., not
% of the form \code{x != null}. Support for these types of checks will be available in a
% future release.


% TODO:  Is NonNull inferred for any parameters or fields, or just for locals?

Type inference is never performed for method parameters of non-private
methods and for non-private fields, because unknown client code could use
them in arbitrary ways.  The inferred information is never written to the
\code{.class} file as user-written annotations are.

The inference indicates when a variable can be treated as having a subtype
of its declared type --- for instance, when an otherwise nullable type can be
treated as a \code{@\refclass{nullness/quals}{NonNull}} one.  The inference never treats a variable as
a supertype of its declared type (e.g., an expression of \code{@\refclass{nullness/quals}{NonNull}}
type is never inferred to be treated as possibly-null).

\subsection{Fields and flow sensitivity analysis}

Flow sensitivity analysis infers the type of fields in some restricted cases:

\begin{itemize}

\item
A final initialized field:
Type inference is performed for final fields that are initialized to a
compile-time constant at the declaration site; so the type of \code{protocol}
is \code{@NonNull String} in the following declaration:

\begin{Verbatim}
    public final String protocol = "https";
\end{Verbatim}

Please note that such inferred type may leak to the public interface of the
class.  To override such behavior, you can explicitly insert the desired
annotation, e.g.

\begin{Verbatim}
    public final @Nullable String protocol = "https";
\end{Verbatim}

\item
Within method bodies:
Type inference is performed for fields in the context of method bodies,
like local variables, but method invocations invalidate any inferred
information.  Consider the following example, where \code{name} is a nullable
field:

\begin{Verbatim}
class DBObject {
  @Nullable Date updatedAt;

  void update() {
    if (updatedAt == null)
        updatedAt = new Date();
    // updatedAt is nonnull
    log("Updating object at " + updatedAt.getTime());

    persistData();
    // updatedAt is nullable again
    log.debug("Saved object updated at " + updatedAt.getTime()); // invalid!
  }
}
\end{Verbatim}

Here the call to \code{persistData()} invalidates the inferred non-null type
of \code{updatedAt}.

When methods do not modify any object state or have any identity side-effects
(e.g. \code{log()} method here), you can annotate these methods as
\code{Pure}.  Annotating them as \code{Pure}, would cause the flow analyzer to
carry the inferred types across the method invocation boundary.

\end{itemize}


\subsection{Inherited defaults}

In certain situations, it would be convenient for an annotation on a
superclass member to be automatically inherited by subclasses that override
it.  This feature would reduce both annotation effort and program
comprehensibility.  In general, a program is read more often than it is
edited/annotated, so the Checker Framework does not currently support this
feature.  Here are more detailed justifications:

\begin{itemize}

\item
  Currently, a user can determine the annotation on a parameter or return
  value by looking at a single file.  If annotations could be inherited
  from supertypes, then a user would have to examine all supertypes to
  understand the meaning of an unannotated type in a given file.

\item
  Different annotations might be inherited from a supertype and an
  interface, or from two interfaces.  Presumably, the subtype's annotations
  would be stronger than either (the greatest lower bound in the type
  system), or an error would be thrown if no such annotations existed.

\end{itemize}

If these issues can be resolved, then the feature may be added in the
future.  Or, it may be added optionally, and each type-checker
implementation can enable it if desired.


\section{Inexpressible types\label{inexpressible-types}}

The Type Annotations syntax~\cite{jsr308} is designed to be easy to read.  As a result,
there are types that it cannot express.  An example is the type of
\<Collection.toArray()>, which returns an array of objects, where the
objects have the same annotation as the elements of the receiver.

A possible annotation would be

\begin{Verbatim}
public @Polynull Object [] toArray() ArrayList<@PolyNull E> { ... }
\end{Verbatim}

\noindent
except that this is illegal syntax:  ``\code{ArrayList<@PolyNull E>}'' is
not legal in the receiver position.  (This is a motivation for
\ahref{http://types.cs.washington.edu/jsr308/specification/java-annotation-design.html#receiver-type-parameter-annotations}{extending}
the Type Annotations syntax.)

The annotated libraries (Section~\ref{annotating-libraries}) contain a less-precise annotation for
\code{toArray}.  The Nullness Checker special-cases \code{toArray} to
act as if it had the above annotation.  The cases that
are currently being handled are described in
\refclass{nullness}{CollectionToArrayHeuristics}.
This approach would be possible for other checkers and other methods as the
need arises.


% LocalWords:  MyClass quals PolymorphicQualifier DefaultQualifier subpackages
% LocalWords:  DefaultQualifiers actuals toArray CollectionToArrayHeuristics nn
% LocalWords:  MyList Nullness DefaultLocation nullness PolyNull util java TODO
% LocalWords:  QualifierDefaults nullable lub persistData updatedAt nble

\htmlhr
\chapter{Handling warnings and legacy code\label{warnings-and-legacy}}

Section~\ref{get-started-with-legacy-code} describes a methodology for
applying annotations to legacy code.  This chapter tells you what to do if,
for some reason, you cannot change your code in such a way as to eliminate
a checker warning.

Also recall that you can convert checker errors into warnings via the
\code{-Awarns} command-line option; see Section~\ref{running}.


\section{Checking partially-annotated programs:  handling unannotated code\label{unannotated-code}}

Sometimes, you wish to type-check only part of your program.
You might focus on the most mission-critical or error-prone part of your
code.  When you start to use a checker, you may not wish to annotate
your entire program right away.
% Not having source code is *not* a reason.
You may not have
enough knowledge to annotate poorly-documented libraries that your program uses.

If annotated code uses unannotated code, then the checker may issue
warnings.  For example, the Nullness checker (Chapter~\ref{nullness-checker}) will
warn whenever an unannotated method result is used in a non-null context:

\begin{Verbatim}
  @NonNull myvar = unannotated_method();   // WARNING: unannotated_method may return null
\end{Verbatim}

If the call \emph{can} return null, you should fix the bug in your program by
removing the \code{@\refclass{nullness/quals}{NonNull}} annotation in your own program.

If the library call \emph{never} returns null,
there are several ways to eliminate the compiler warnings.
\begin{enumerate}
\item Annotate \code{unannotated\_method} in full.  This approach provides
  the strongest guarantees, but may require you to annotate additional
  methods that \code{unannotated\_method} calls.  See
  Chapter~\ref{annotating-libraries} for a discussion of how to annotate
  libraries for which you have no source code.
\item Annotate only the signature of \code{unannotated\_method}, and
  suppress warnings in its body.  Two ways to suppress the warnings are via a
  \code{@SuppressWarnings} annotation or by not running the checker on that
  file (see Section~\ref{suppressing-warnings}).
\item Suppress all warnings related to uses of \code{unannotated\_method}
  via the \code{skipUses} processor option
  (see Section~\ref{suppressing-warnings}).
  Since this can suppress more warnings than you may expect,
  it is usually better to annotate at least the method's signature.  If you
  choose the boundary between the annotated and unannotated code wisely,
  then you only have to annotate the signatures of a limited number of
  classes/methods
  (e.g., the public interface to a library or package).

\end{enumerate}

Chapter~\ref{annotating-libraries} discusses adding annotations to
signatures when you do not have source code available.
Section~\ref{suppressing-warnings} discusses suppressing warnings.


If you annotate a third-party library, please share it with us so that we
can distribute the annotations with the Checker Framework; see
Section~\ref{reporting-bugs}.


\section{Suppressing warnings\label{suppressing-warnings}}

You may wish to suppress checker warnings because of unannotated libraries
or un-annotated portions of your own code, because of application
invariants that are beyond the capabilities of the type system, because of
checker limitations, because you are interested in only some of the
guarantees provided by a checker, or for other reasons.  You can suppress
warnings via
\begin{enumerate}
\item
  the \code{@SuppressWarnings} annotation,
\item
  the \code{-AskipUses} command-line option,
\item
  the \code{-Alint} command-line option,
\item
  not using the \code{-processor} command-line option, or
\item
  checker-specific mechanisms.
\end{enumerate}

\noindent
We now explain these mechanisms in turn.

% See the @SuppressWarningsKey annotation and the getSuppressWarningsKey method.

\subsection{\code{@SuppressWarnings} annotation\label{suppresswarnings-annotation}}

You can suppress specific errors and warnings by use of the
\code{@SuppressWarnings("\emph{checkername}")} annotation, for example
\code{@SuppressWarnings("interning")} or \code{@SuppressWarnings("nullness")}.
The argument \emph{checkername} is in lower case and is derived from the
way you invoke the checker; for example, if you invoke a checker as
\code{javac -processor MyNiftyChecker ...}, then you would suppress its
error messages with \code{@SuppressWarnings("mynifty")}.  (An exception is
the Basic Checker, for which you use the annotation name; see
Section~\ref{basic-using}).

A \code{@\sunjavadoc{java/lang/SuppressWarnings.html}{SuppressWarnings}}
annotation may be placed on program elements such as a local
variable declaration, a method, or a class.  It suppresses all warnings
related to the given checker, for that program element.

For instance, one common use is
to suppress warnings at a cast that you know is safe.  Here is an example
that uses the Tainting Checker (Section~\ref{tainting-checker}):

\begin{Verbatim}
  @SuppressWarnings("tainting")
  String myvar = (@Untainted String) expr;  // expr has type: @Tainted String
\end{Verbatim}

It is good practice to suppress warnings in the
smallest possible scope.  For example, if a particular expression causes a
false positive warning, you should extract that expression into a local variable
and place a \code{@SuppressWarnings} annotation on the variable
declaration.
As another example, if you have annotated the signatures but not the bodies
of the methods in a class or package, put a \code{@SuppressWarnings}
annotation on the class declaration or on the package's
\code{package-info.java} file.

\label{compiler-message-keys}

Another good practice is to use the most specific possible argument to
\code{@SuppressWarnings}.  The string can be of the form \emph{checkername} or
or \emph{checkername:messagekey}.  The \emph{checkername} part is as
described above.  The \emph{messagekey} part suppresses only
errors/warnings relating to the given message key.  For example,
\code{cast.unsafe} is the key for warnings about an unsafe cast, and
\code{cast.redundant} to the key for warnings about a redundant cast.

Thus, the above example could have been written as:

\begin{Verbatim}
  @SuppressWarnings("tainting")              // suppresses all tainting-related warnings
  @SuppressWarnings("tainting:cast.unsafe")  // suppresses tainting warnings about unsafe casts
  @SuppressWarnings("tainting:cast")         // suppresses tainting warnings about casts 
\end{Verbatim}

\noindent
For a list of the message keys, 
see the \code{messages.properties} files in 
%BEGIN LATEX
\\
%END LATEX
\code{checker-framework/checkers/src/checkers/\emph{checkername}/messages.properties}.
Each checker is built on the \code{basetype} checker and inherits its
properties.  Thus, to find all the error keys for a checker, you usually
need to examine its own \code{messages.properties} file and that of
\code{basetype}.

If a checker produces a warning/error and you want to determine its message
key, you can re-run the checker, passing the the \code{-Anomsgtext}
command-line option (Section~\ref{debugging-options}).



\subsection{\code{-AskipUses} command-line option\label{askipuses}}

You can suppress all errors and warnings at all \emph{uses} of a given
class (but the class itself is still type-checked).
Set the \code{-AskipUses} command-line option to a
regular expression that matches classes for which warnings and errors
should be suppressed.  For example, if you use
``{\codesize\verb|-AskipUses=^java\.|}'' on the command line
(with appropriate quoting) when invoking
\code{javac}, then the checkers will suppress all warnings within
classes whose fully-qualified name starts with \codesize\verb|java.|, all
warnings relating to invalid arguments, and all warnings relating to incorrect
use of the return value.

To suppress all errors and warnings related to multiple classes, you can use
the regular expression alternative operator ``\code{|}'', as in
``{\codesize\verb+-AskipUses="java\.lang\.|java\.util\."+}'' to suppress
all warnings related to classes belong to the \code{java.lang} or
\code{java.util} packages.

\subsection{\code{-Alint} command-line option\label{alint}}

\label{lint-options}

The \code{-Alint} option enables or disables optional checks, analogously to
javac's \code{-Xlint} option.
Each of the distributed checkers supports at least the following lint options:

\begin{itemize}

\item
  \code{cast:unsafe} (default: on) warn about unsafe casts that are not
  checked at run time, as in \code{((@NonNull String) myref)}.  Such casts
  are generally not necessary when flow-sensitive local type refinement is
  enabled.

\item
  \code{cast:redundant} (default: on) warn about redundant
  casts that are guaranteed to succeed at run time,
  as in \code{((@NonNull String) "m")}.  Such casts are not necessary,
  because the target expression of the cast already has the given type
  qualifier.

\item
  \code{cast} Enable or disable all cast-related warnings.

\item
  \code{all} Enable or disable all lint warnings, including
  checker-specific ones if any.  Examples include \code{nulltest} for the
  Nullness Checker (see Section~\ref{lint-nulltest}) and \<dotequals> for
  the Interning Checker (see Section~\ref{lint-dotequals}).  This option
  does not enable/disable the checker's standard checks, just its optional
  ones.

\item
  \code{none} The inverse of \<all>:  disable or enable all lint warnings,
  including checker-specific ones if any.

\end{itemize}

% This syntax is different from -Xlint that uses a colon instead of an
% equals sign, because javac forces the use of the equals sign.

\noindent
To activate a lint option, write \code{-Alint=} followed by a
comma-delimited list of check names.  If the option is preceded by a
hyphen (\code{-}), the warning is disabled.  For example, to disable all
lint options except redundant casts, you can pass
\code{-Alint=-all,cast:redundant} on the command line.

\subsection{No \code{-processor} command-line option\label{no-processor}}

You can also compile parts of your code without use of the
\code{-processor} switch to \code{javac}.  No checking is done during
such compilations.

\subsection{Checker-specific mechanisms\label{checker-specific-suppression}}

Finally, some checkers have special rules.  For example, the Nullness
checker (Chapter~\ref{nullness-checker}) uses \code{assert} statements that contain
null checks, and the special \<castNonNull> method, to suppress warnings
(Section~\ref{suppressing-warnings-with-assertions}).
This manual also explains special mechanisms for
suppressing warnings issued by the Fenum checker
(Section~\ref{fenum-suppressing}) and the Units checker
(Section~\ref{units-suppressing}).


\section{Backward compatibility with earlier versions of Java\label{backward-compatibility}}

Sometimes, your code needs to be compiled by people who are not using a
compiler that supports type annotations.
Sections~\ref{annotations-in-comments}--\ref{uncommenting-annotations}
discuss this situation, which you can handle by writing annotations in
comments.

In other cases, your code needs to be run by people who are not using a Java~8
JVM\@.  Section~\ref{java5-class-files} discusses this situation, which
you can handle by passing the \code{-target 5} command-line argument.

(\textbf{Note:} These are features of the Type Annotations compiler that is
distributed along with the Checker Framework.  They are \emph{not}
supported by the mainline OpenJDK compiler.  These features are the key
difference between the Type Annotations compiler and the OpenJDK compiler
on which it is built.)

%   For more details
% about the differences, see file \code{README-jsr308.html} in the Type
% Annotations distribution.


\subsection{Annotations in comments\label{annotations-in-comments}}

A Java 4 compiler does not permit use of
annotations, and a Java 5 compiler only permits annotations on
declarations (but not on generic arguments, casts, \<extends> clauses, method receiver, etc.).

So that your code can be compiled by any Java compiler (for any version of
the Java language), you may write any annotation inside a
\code{/*}\ldots\code{*/} Java comment, as in \code{List</*@NonNull*/ String>}.
The Type Annotations compiler treats the code exactly as if you had not written the
\code{/*} and \code{*/}.
In other words, the Type Annotations compiler will recognize the
annotation, but your code will still compile with any other Java compiler.

This feature only works if you provide no \code{-source} command-line
argument to \code{javac}, or if the \code{-source} argument is \code{1.8}
or \code{8}.

In a single program, you may write some annotations in comments, and others
without comments.

By default, the compiler ignores any comment that contains spaces at the
beginning or end, or between the \code{@} and the annotation name.
In other words, it reads \code{/*@NonNull*/} as an annotation but ignores
\code{/* @NonNull*/} or \code{/*@ NonNull*/} or \code{/*@NonNull */}.
This
feature enables backward compatibility with code that contains comments
that start with \code{@} but are not annotations.  (The
ESC/Java~\cite{FlanaganLLNSS02}, JML~\cite{LeavensBR2006:JML}, and
Splint~\cite{Evans96} tools all use ``\code{/*@}'' or ``\code{/*~@}'' as a
comment marker.)
Compiler flag
\code{-XDTA:spacesincomments} causes the compiler to parse annotation comments
even when they contain spaces.  You may need to use
\code{-XDTA:spacesincomments} if you use Eclipse's ``Source $>$ Correct
Indentation'' command, since it inserts space in comments.  But the
annotation comments are less readable with spaces, so you may wish to disable
inserting spaces:  in the Formatter preferences, in the Comments tab,
unselect the ``enable block comment formatting'' checkbox.

There is no way to turn off the annotations in comments feature.  If you
don't want this feature, you can use a standard Java 8 compiler that
supports type annotations but not annotations in comments.  If your code
already contains comments of the form \</*@...*/> that look like type
annotations, and you want the Type Annotations compiler not to try to
interpret them, then you can add spaces to the comments.


\subsection{Implicit import statements\label{implicit-import-statements}}

When writing source code with annotations, it is more convenient to write a
short form such as \code{@NonNull} instead of
\code{@checkers.nullness.quals.NonNull}.

The traditional way to do this is to write an import statement like
``\code{import checkers.nullness.quals.*;}''.  This works, but everyone who
compiles the code (no matter what compiler they use, and even if the
annotations are in comments) must have the annotation definitions (e.g.,
the \code{checkers.jar} or \code{checkers-quals.jar} file) on their
classpath.  The reason is that a Java compiler issues an error if an
imported package is not on the classpath.  See Section~\ref{distributing}.

\label{jsr308_imports}

An alternative is to set the shell environment variable
\code{jsr308\_imports} when you compile the code.
The Type Annotations compiler treats this as if the given packages were
imported, but other compilers
ignore the
\code{jsr308\_imports} environment variable --- they do not need it, since
they do not support annotations in comments.  Thus, your code can compile
whether or not the Type Annotations compiler is being used.

You can specify multiple packages separated by the classpath separator
(same as the file path separator:  \<;> for Windows, and \<:> for Unix and
Mac).  For example, to implicitly import the Nullness and Interning
qualifiers, set \code{jsr308\_imports} to
\code{checkers.nullness.quals.*:checkers.interning.quals.*}.

If you issue the javac command from the command line or in a Makefile, you
may need to add quotes, to prevent your shell from expanding the \code{*}
character.
In bash, you could write \code{export
  jsr308\_imports='checkers.nullness.quals.*'}, or prefix the \code{javac}
command by \code{jsr308\_imports='checkers.nullness.quals.*'} .
Alternately, you can set the environment variable via the javac
command-line argument \code{-J-Djsr308\_imports='checkers.nullness.quals.*'}.
If you supply the \code{-J-Djsr308\_imports} argument via an Ant buildfile,
you do not need the extra quoting.


\subsection{Migrating away from annotations in comments\label{uncommenting-annotations}}

Suppose that your codebase currently uses annotations in comments, but you
wish to remove the comment characters around your annotations, because in
the future you will use only compilers that support type annotations.
This Unix command removes
the comment characters, for all Java files in the current
working directory or any subdirectory.

\begin{Verbatim}
   find . -type f -name '*.java' -print \
     | xargs grep -l -P '/\*\s*@([^ */]+)\s*\*/' \
     | xargs perl -pi.bak -e 's|/\*\s*@([^ */]+)\s*\*/|@\1|g'
\end{Verbatim}

You can customize this command:
\begin{itemize}
\item
To process comments with embedded spaces and asterisks, change
two instances of ``\verb|[^ */]|'' to ``\verb|[^/]|''.
\item
To ignore comments with leading or trailing spaces, remove the four
instances of ``\verb|\s*|''.
\item
  To not make backups, remove ``\verb|.bak|''.
\end{itemize}


If your code used implicit import statements
(Section~\ref{implicit-import-statements}), then after uncommenting the
annotations, you may also need to introduce
explicit import statements into your code.


\subsection{Annotations in Java 5 \code{.class} files\label{java5-class-files}}

If you supply the \code{-target 5} command-line argument along with no
\code{-source} argument (or \code{-source 8}, which is equivalent), then the
Type Annotations compiler creates a \code{.class} file that can be run on a
Java 5 JVM, but that contains the type annotations.  (It does not matter
whether the type annotations were written in comments or not.)  The fact
that the \code{.class} file contains the type annotations is useful when
type-checking client code.  If you try to type-check client code against a
library that lacks type annotations, then spurious warnings can result.
So, use of \code{-target 5} gives backward compatibility with earlier JVMs
while still permitting pluggable type-checking.

Ordinary Java compilers do not let you use a \code{-target} command-line
argument with a value less than the \code{-source} argument.

Use of the \code{-source 5} command-line argument produces a \code{.class}
file that does not contain type annotations.  One reason you might want to
periodically compile with the \code{-source 5} argument is to ensure that
your code does not use any Java 8 features other than type annotations in
comments.


% LocalWords:  quals skipUses un AskipUses Alint annotationname javac's Awarns
% LocalWords:  Xlint dotequals castNonNull XDTA spacesincomments Formatter jsr
% LocalWords:  unselect checkbox classpath Djsr bak Nullness nullness java lang
% LocalWords:  checkername util myref nulltest html ESC buildfile mynifty
% LocalWords:  MyNiftyChecker messagekey basetype uncommenting Anomsgtext

\htmlhr
\chapter{Annotating libraries\label{annotating-libraries}}

When annotated code uses an unannotated library, a checker may issue warnings.
As described in Section~\ref{unannotated-code}, the best way to correct
this problem is to add annotations to the library.  (Alternately, you can instead
suppress all warnings related to an unannotated library by use of the 
\code{-AskipClasses} command-line option; see
Section~\ref{suppressing-warnings}.).  If you have source code for the
library, you can easily add the annotations.
This section tells you
how to add annotations to a library for which you have no source code,
because the library is distributed only in binary (\code{.class} or
\code{.jar}) form.  This section is also useful if you do not wish to edit the
library's source code.

The Checker Framework distribution contains annotations
for popular libraries, such as the JDK\@.
If you annotate additional libraries, please share them with us so that we
can distribute the annotations with the Checker Framework; see
Section~\ref{reporting-bugs}.


You can determine the correct annotations for a library either
automatically by running an inference tool, or manually by reading the
documentation.  Presently, type inference tools are available for the
Nullness (Section~\ref{nullness-inference}) and Javari
(Section~\ref{javari-inference}) type systems.

You can make the annotations known to the JSR 308 compiler (and thus
to the checkers) in two ways.

\begin{itemize}

\item You can use the stub class generation tool to create a ``stub
  file'' containing classes with no method bodies,
  and annotate the stub classes
  file.  Then, you can supply the stub files to the checker when
  compiling/checking your program.
  Section~\ref{stub} describes how to use the stub class generation
  tools.

\item You can annotate the compiled
  \code{.jar} or \code{.class} files using the annotation file utilities
  (\myurl{http://types.cs.washington.edu/annotation-file-utilities/}).
  First, express the annotations textually as an annotation index file, and
  then the tools insert them in the compiled library class files.
  See the Annotation File Utilities documentation for full details.

\end{itemize}


\section{Using stub classes\label{stub}\label{stub-creating-and-using}}

A stub file contains ``stub classes'' that contain annotated signatures.  A
checker uses those annotated signatures at compile time, instead of or in
addition to annotations that appear in the library.

Section~\ref{stub-creating} describes how to create stub classes.
Section~\ref{stub-using} describes how to use stub classes.
These sections illustrate stub classes via the example of creating a \code{@\refclass{interning/quals}{Interned}}-annotated
version of \code{java.lang.String}.  (You don't need to repeat these steps,
since such a stub class is already included in the Checker Framework
distribution; see file \code{checkers/src/checkers/interning/jdk.astub}, which
is reproduced in Section~\ref{stub-format}.)

% First, you must install the skeleton class generator
% (Section~\ref{skeleton-installing}).

\subsection{Creating a stub file\label{stub-creating}}

\begin{enumerate}

\item
  Create a stub file by running the stub class generator.  (\<checkers.jar>
  must be on your classpath.)

\begin{Verbatim}
  cd nullness-stub
  java checkers.util.stub.StubGenerator java.lang.String > String.astub
\end{Verbatim}

  Supply it with the fully-qualified name of the class for which you wish to
  generate a stub class.  The stub class generator prints the
  stub class to standard out, so you may wish to redirect its output to a
  file.

\item
  Add import statements for the annotations.  So you would need to
add the following import statement at the beginning of the file:

\begin{Verbatim}
  import checkers.interning.quals.Interned;
\end{Verbatim}

\item
  Add annotations to the stub class.  For example, you might annotate
  the \ahref{http://java.sun.com/javase/6/docs/api/java/lang/String.html#intern()}{\<String.intern()>} method as follows:

\begin{Verbatim}
  @Interned String intern();
\end{Verbatim}

  You may also remove irrelevant parts of the stub file; see
  Section~\ref{stub-format}.

\end{enumerate}


\subsection{Using a stub file\label{stub-using}}

  When you run \code{javac} with a given checker/processor, you can specify
  a list of the stub files or directories using
  \code{-Astubs=\emph{file\_or\_path\_name}}.  The stub path entries
  are delimited by
  \<File.pathSeparator> (`\<:>' for Linux and Mac, `\<;>' for Windows).
  When you supply a stub directory, the checker only considers the enclosed
  stub files whose names end with \code{.astub}.

  The \code{-Astubs} argument causes the Checker Framework to read annotations
  from annotated stub classes in preference to the unannotated original
  library classes.

%BEGIN LATEX
\begin{smaller}
%END LATEX
\begin{Verbatim}
  javac -processor checkers.interning.InterningChecker -Astubs=String.astub:stubs MyFile.java MyOtherFile.java ...
\end{Verbatim}
%BEGIN LATEX
\end{smaller}
%END LATEX


\subsection{Stub file format\label{stub-format}}

The stub file format is designed for simplicity, readability, and
compactness.  It reads like a Java file but contains only the
necessary information for type checking.

As an illustration, the stub file for the Interning type system
(Chapter~\ref{interning-checker}) is as follows.  This file appears as
\code{checkers/src/checkers/interning/jdk.astub} in the Checker Framework
distribution.

\begin{Verbatim}
  import checkers.interning.quals.Interned;

  package java.lang;

  // All instances of Class are interned.
  @Interned class Class<T> { }

  class String {
    // The only interning-related method in the JDK.
    @Interned String intern();
  }
\end{Verbatim}


You can use a regular Java file as a stub file.  Every valid Java file is a
valid stub file.  However, you can omit
information that is not relevant to pluggable type-checking; this makes the
stub file smaller and easier for people to read and write.  You can also
put annotated signatures for multiple classes in a single stub file.


The stub file format is allowed to differ from Java source code in the
following ways:
\begin{description}

\item{\textbf{Method bodies:}}
  The stub class does not require method bodies for classes; any method
  body may be replaced by a semicolon (\code{;}), as in an interface or
  abstract method declaration.

\item{\textbf{Method declarations:}}
  You only have to specify the methods that you need to annotate.
  Any method declaration may be omitted, in which case the checker reads
  its annotations from the library.  (If you are using a stub class, then
  typically the library's version is unannotated.)

\item{\textbf{Declaration specifiers:}}
  Declaration specifiers (e.g., \<public>, \<final>, \<volatile>)
  may be omitted, since they have nothing to do with types.

\item{\textbf{Import statements:}}
  The only required import statements are the ones to import type
  annotations.  Such imports must be at the beginning of the
  file.  Other import statements are optional.

\item{\textbf{Multiple classes and packages:}}
  The stub file format permits having multiple classes and packages.
  The packages are separated by a package statement:
  \<package my.package;>.  Each package declaration may occur only once; in
  other words, all classes from a package must appear together.

\end{description}


\subsection{Known problems}


The Checker Framework stub file reader has several limitations:

\begin{itemize}
\item
  % Still a problem as of 9/2/2009.
  It does not handle \code{enum}s.
\item
  % Still a problem as of 9/2/2009.
  It only handles type annotations, not declaration annotations (e.g. IGJ's \<Assignable>).
\end{itemize}


\subsection{Style tips for stub files}

Every Java file is a stub file.  If you have access to the Java file, then
it is usually best to use the Java file as the stub file, without removing
any of the parts that the stub file format permits you to.  Just add
annotations to the full source code.  This approach retains the original
documentation and source code, making it easier for a programmer to
double-check the annotations.  It also enables creation of diffs, easing
the process of upgrading when a library adds new methods.  And, the
annotations are in a format that the library maintainers can even
incorporate.

The downside of this approach is that the stub files are larger.  This can
slow down parsing.  Furthermore, a programmer must search the stub file
for a given method rather than just skimming one or two pages of signatures.

If you do not have access to the library source code, then you can create a
stub file from the Javadoc or the class file, and then annotate it.



% Label "skeleton" is for old links from the Javarifier manual, to prevent
% them from being broken links.

\section{Using skeleton files (distributed annotated JDKs)\label{skeleton-using}\label{skeleton}}

The Checker Framework distribution contains 
annotated JDKs at the path \<checkers/jdk/[checker-name]/src>.
These are in another format called
``skeleton classes''.  We are currently working on converting the skeleton
files into stub files.

% Skeleton classes are inferior to stub classes for two reasons.  First,
% skeleton files must be on the classpath during compilation but must
% \emph{not} be on the classpath during execution; this is inconvenient and
% error-prone.  Second, the skeleton files contain incorrect values for
% certain static final fields.  These incorrect values can lead to
% run-time problems unless the Java code is re-compiled without the skeleton
% classes after type-checking is complete.


\begin{enumerate}

\item
  When you run \code{javac}, add a \code{-sourcepath} argument to indicate
  where to find the skeleton classes.
  Supply \code{-sourcepath} in addition to whatever other arguments you
  usually use, including \code{-classpath}.

  The \code{-sourcepath} argument causes the compiler to read annotations
  from annotated skeleton classes in preference to the unannotated original
  library classes.  However, the compiler will use the originals on the
  classpath if no file is available on the sourcepath.

%BEGIN LATEX
\begin{smaller}
%END LATEX
\begin{Verbatim}
  javac -processor checkers.nullness.NullnessChecker -sourcepath checkers/jdk/nullness/src my_source_files
\end{Verbatim}
%BEGIN LATEX
\end{smaller}
%END LATEX

\item
  Run the compiled code as usual.  Do \emph{not} include the skeleton files
  on the classpath.  If a skeleton method is called instead of the true
  library method, then your program will throw a \code{RuntimeException}.

\end{enumerate}


% \section{Installing the skeleton class generator\label{skeleton-installing}}
%
% Source code for the skeleton class generator tool is included in the
% Checker Framework
% distribution, but because the tool has additional dependencies, the provided
% build script does not build the tool by default.
%
% Follow these steps to install the skeleton class generator:
%
% \begin{enumerate}
%
% \item
%   Install the annotation file utilities, using the instructions at
%   \myurl{http://types.cs.washington.edu/annotation-file-utilities/}.
%   Per those instructions, the \code{annotation-file-utilities.jar} file
%   should be on your classpath.
%
% % TODO This item should become optional; tell people to install the AFU in
% % the right place.
% \item
%   Update the \code{build.properties} file in the Checker Framework distribution so
%   that the \code{annotation-utils.lib} property specifies the location of
%   the \code{annotation-file-utilities.jar} library.
%
% \item
%   Build the skeleton class generator tool by running \code{ant
%     skeleton-util dist} in the \code{checkers} directory.  This updates the
%   \code{checkers.jar} file to contain the skeleton class generator.
%   \code{checkers.jar} should already be on your classpath (see
%   Section~\ref{installation}).
%
% \end{enumerate}


% LocalWords:  plugin utils util dist RuntimeException NonNull TODO AFU enum
% LocalWords:  sourcepath Nullness javac classpath src quals pathSeparator JDKs
% LocalWords:  IGJ's jdk Astubs skipClasses astub AskipClasses toArray
% LocalWords:  CollectionToArrayHeuristics

\htmlhr
\chapter{How to create a new checker\label{creating-a-checker}}
\label{writing-a-checker} % for old links; don't use any more!

\newcommand{\TreeAPIBase}{https://docs.oracle.com/javase/8/docs/jdk/api/javac/tree/com/sun/source}
\newcommand{\refTreeclass}[2]{\href{\TreeAPIBase{}/#1/#2.html?is-external=true}{\<#2>}}
\newcommand{\ModelAPIBase}{http://docs.oracle.com/javase/8/docs/api/javax/lang/model}
\newcommand{\refModelclass}[2]{\href{\ModelAPIBase{}/#1/#2.html?is-external=true}{\<#2>}}

This chapter describes how to create a checker
--- a type-checking compiler plugin that detects bugs or verifies their
absence.  After a programmer annotates a program,
the checker plugin verifies that the code is consistent
with the annotations.
If you only want to \emph{use} a checker, you do not need to read this
chapter.


Writing a simple checker is easy!  For example, here is a complete, useful
type-checker:

\begin{Verbatim}
import java.lang.annotation.Target;
import java.lang.annotation.ElementType;
import org.checkerframework.framework.qual.SubtypeOf;
import org.checkerframework.framework.qual.Unqualified;

@SubtypeOf(Unqualified.class)
@Target({ElementType.TYPE_USE, ElementType.TYPE_PARAMETER})
public @interface Encrypted {}
\end{Verbatim}

This checker is so short because it builds on the Subtyping Checker
(Chapter~\ref{subtyping-checker}).
See Section~\ref{subtyping-example} for more details about this particular checker.
When you wish to create a new checker, it is often easiest to begin by
building it declaratively on top of the Subtyping Checker, and then return to
this chapter when you need more expressiveness or power than the Subtyping
Checker affords.

Three choices for creating your own checker are:
\begin{itemize}
\item
  Customizing an existing checker.
  Checkers that are designed for extension include
  the Subtyping Checker (\chapterpageref{subtyping-checker}),
  the Fake Enumeration Checker (\chapterpageref{fenum-checker}),
  the Units Checker (\chapterpageref{units-checker}),
  and a typestate checker (\chapterpageref{typestate-checker}).
\item
  Follow the instructions in this chapter to create a checker from scratch.
  This enables creation of checkers that are more powerful than customizing
  an existing checker.
\item
  Copy and then modify a different existing checker --- whether
  one distributed with the Checker Framework or a third-party one.
  This can be fast, or you can get tangled up if you don't fully understand
  the subtleties of the existing checker that you are modifying.
  Oftentimes following the instructions in this chapter is easier.
  (If you are going to copy a checker, one good choice to copy and modify
  is the Regex Checker (\chapterpageref{regex-checker}).  A bad choice is
  the Nullness Checker (\chapterpageref{nullness-checker}),
  which is more sophisticated than anything you want to start out building.)
\end{itemize}

You do not need all of the details in this chapter, at least at first.
In addition to reading this chapter of the manual, you may find it helpful
to examine the implementations of the checkers that are distributed with
the Checker Framework.
The Javadoc documentation of the framework and the checkers is in the
distribution and is also available online at
\myurl{https://checkerframework.org/api/}.

If you write a new checker and wish to advertise it to the world, let us
know so we can mention it in \chapterpageref{third-party-checkers}
or even include it in the Checker Framework distribution.


\section{How checkers build on the Checker Framework\label{creating-tool-relationships}}

This table shows the relationship among tools that the Checker Framework
builds on or that are built on the Checker Framework.
All of the tools support the Java 8 type annotation syntax.
You use the Checker Framework to build pluggable type systems, and the
Annotation File Utilities to manipulate \code{.java} and \code{.class} files.

\newlength{\bw}
\setlength{\bw}{.5in}

%% Strictly speaking, "Subtyping Checker" should sit on top of Checker
%% Framework and below all the specific checkers.  But omit it for simplicity.

% Unfortunately, Hevea inserts a horizontal line between every pair of rows
% regardless of whether there is a \hline or \cline.  So, make paragraphs.
\begin{center}
\begin{tabular}{|p{\bw}|p{\bw}|p{\bw}|p{\bw}|p{.4\bw}|p{\bw}|p{1.5\bw}|p{1\bw}|}
\cline{1-4} \cline{6-6}
\centering Subtyping \par Checker &
\centering Nullness \par Checker &
\centering Mutation \par Checker &
\centering Tainting \par Checker &
\centering \ldots &
\centering Your \par Checker &
\multicolumn{2}{c}{}
\\ \hline
\multicolumn{6}{|p{6\bw}|}{\centering Base Checker \par (enforces subtyping rules)} &
\centering Type \par inference &
% Adding "\centering" here causes a LaTeX alignment error
Other \par tools
\\ \hline
\multicolumn{6}{|p{6\bw}|}{\centering Checker Framework \par (enables creation of pluggable type-checkers)} &
\multicolumn{2}{p{3\bw}|}{\centering \href{https://checkerframework.org/annotation-file-utilities/}{Annotation File Utilities} \par (\code{.java} $\leftrightarrow$ \code{.class} files)}
\\ \hline
\multicolumn{8}{|p{8.5\bw}|}{\centering
  \href{https://checkerframework.org/jsr308/}{Type Annotations} syntax
  and classfile format (``JSR 308'') \par \centering (no built-in semantics)} \\ \hline
\end{tabular}
\end{center}


The Base Checker
(more precisely, the \refclass{common/basetype}{BaseTypeChecker})
enforces the standard subtyping rules.
The Subtyping Checker is a simple use of the Base Checker that supports
providing type qualifiers on the command line.
You usually want to build your checker on the Base Checker.


\section{The parts of a checker\label{creating-parts-of-a-checker}}

The Checker Framework provides abstract base classes (default
implementations), and a specific checker overrides as little or as much of
the default implementations as necessary.
To simplify checker implementations, by default the Checker Framework
automatically discovers the parts of a checker by looking for specific files.
Thus, checker implementations follow a very formulaic structure.
To illustrate, a checker for MyProp must be laid out as follows:
%
\begin{Verbatim}
myPackage/
  | qual/                               type qualifiers
  | MyPropVisitor.java                  [optional] type rules
  | MyPropAnnotatedTypeFactory.java     [optional] type introduction and dataflow rules
  | MyPropChecker.java                  [optional] interface to the compiler
\end{Verbatim}
%
Note that \<MyPropChecker.java> is required unless you are building on the
Subtyping Checker.

Sections~\ref{creating-typequals}--\ref{creating-compiler-interface} describe
the individual components of a type system as written using the Checker
Framework:

\begin{description}

\item{\ref{creating-typequals}}
  \textbf{Type qualifiers and hierarchy.}  You define the annotations for
  the type system and the subtyping relationships among qualified types
  (for instance, that \<@NonNull Object> is a subtype of \<@Nullable
  Object>).

\item{\ref{creating-extending-visitor}}
  \textbf{Type rules.}  You specify the type system semantics (type
  rules), violation of which yields a type error.  A type system has two types of
  rules.
\begin{itemize}
\item
  Subtyping rules related to the type hierarchy, such as that every
  assignment
  % and pseudo-assignment
  satisfies a subtyping relationship.
  Your checker automatically inherits these subtyping rules from the Base
  Checker (Chapter~\ref{subtyping-checker}), so there is nothing for you to do.
\item
  Additional rules that are specific to your particular checker.  For
  example, in the Nullness type system, only references with a
  \refqualclass{checker/nullness/qual}{NonNull} type may be dereferenced.  You
  write these additional rules yourself.
\end{itemize}

\item{\ref{creating-type-introduction}}
  \textbf{Type introduction rules.}  For some types and
  expressions, a qualifier should be treated as implicitly present even if a
  programmer did not explicitly write it.  For example, in the Nullness
  type system every literal
  other than \<null> has a \refqualclass{checker/nullness/qual}{NonNull} type;
  examples of literals include \code{"some string"} and \<java.util.Date.class>.

\item{\ref{creating-dataflow}}
  \textbf{Dataflow rules.}  These optional rules enhance flow-sensitive
  type qualifier inference (local variable type inference).

\item{\ref{creating-compiler-interface}}
  \textbf{Interface to the compiler.}  The compiler interface indicates
  which annotations are part of the type system, which command-line options
  and \<@SuppressWarnings> annotations the checker recognizes, etc.
\end{description}




\section{Compiling and using a custom checker\label{creating-compiling}}

You can place your checker's source files wherever you like.
\begin{itemize}
\item
  Forking the Checker Framework repository and putting your files in
  parallel to existing checker implementations is a particularly convenient
  choice.  You must be able to compile the Checker Framework
  (Section~\ref{build-source}).
\item
  \begin{sloppypar}
  If you put your checker elsewhere, then when you compile your checker,
  the classpath must include \<\$CHECKERFRAMEWORK/checker/dist/checker.jar>
  and \<\$CHECKERFRAMEWORK/checker/dist/javac.jar>.
  \end{sloppypar}
\end{itemize}

% You may also wish to consult Section~\ref{creating-testing-framework} for
% information on testing a checker and
% Section~\ref{creating-debugging-options} for information on debugging a
% checker.

Once your custom checker is written, using it is very similar to using a
built-in checker (Section~\ref{running}):
simply pass the fully-qualified name of your \<BaseTypeChecker>
subclass to the \<-processor> command-line option:
\begin{alltt}
  javac \textbackslash
    -processor \textit{mypackage.MyPropChecker} \textbackslash
    SourceFile.java
\end{alltt}
Note that your custom checker's
compiled files must be on the Java classpath.
Invoking a custom checker that builds on
the Subtyping Checker is slightly different (Section~\ref{subtyping-using}).



\subsection{Tips for creating a checker\label{creating-tips}}

To make your job easier, we recommend that you build your type-checker
incrementally, testing at each phase rather than trying to build the whole
thing at once.

Here is a good way to proceed.

\begin{enumerate}
\item
  Before you start coding, first write the user manual.  The manual
  explains the type system, what it guarantees, how to use it, etc., from
  the point of view of a user.  Writing the manual will help you flesh out
  your goals and the concepts, which are easier to understand and change in
  text than in an implementation.
  Section~\ref{creating-documenting-a-checker} gives a suggested structure
  for the manual chapter, which will help you avoid omitting any parts.
  Get feedback from someone else at this point to ensure that your manual
  is comprehensible.

  Once you have designed and documented the parts of your type system, you
  may also want to choose some case studies and ``play computer'', manually
  type-checking the code according to the rules you defined.
  During manual checking, ask
  yourself what reasoning you applied, what information you needed, and
  whether your written-down rules were sufficient.

\item
  Implement the type qualifiers and hierarchy
  (Section~\ref{creating-typequals}).

  Write simple test cases that consist of only assignments,
  to test your type hierarchy.  For instance, if
  your type hierarchy consists of a supertype \<@UnknownSign> and a subtype
  \<@NonNegative>, then you could write a test case such as:

\begin{Verbatim}
  void foo(@UnknownSign int us, @NonNegative int nn) {
    @UnknownSign int a = us;
    @UnknownSign int b = nn;
    //:: error: assignment.type.incompatible
    @NonNegative int c = us;  // expected error on this line
    @NonNegative int d = nn;
  }
\end{Verbatim}

  Type-check your test files using the Subtyping Checker
  (\chapterpageref{subtyping-checker}).

\item
  Write the checker class itself
  (Section~\ref{creating-compiler-interface}).

  Ensure that you can still type-check your test files and that the results
  are the same.  You will not use the Subtyping Checker any more; you will
  call the checker directly, as in

\begin{Verbatim}
  javac -processor mypackage.MyChecker File1.java File2.java ...
\end{Verbatim}

\item
  If your checker source code is in a clone of the Checker Framework
  repository, integrate your checker with the Checker Framework's Ant
  targets for testing (Section~\ref{creating-testing-framework}).  This
  will make it much more convenient to run tests, and to ensure that they
  are passing, as your work proceeds.

\item
  Annotate parts of the JDK, if relevant
  (Section~\ref{creating-a-checker-annotated-jdk}).

  Write test cases for at least some of the annotated JDK methods to ensure
  that the annotations are being properly read by your checker.

\item
  Implement type rules, if any (Section~\ref{creating-extending-visitor}).
  (Some type systems need JDK annotations but don't have any additional
  type rules.)

  Before implementing type rules (or any other code in your type-checker),
  you are recommended to read the javadoc for the utility routines in the
  \<org.checkerframework.javacutil> package, especially
  \refclass{javacutil}{AnnotationUtils},
  \refclass{javacutil}{ElementUtils},
  \refclass{javacutil}{TreeUtils},
  \refclass{javacutil}{TypeAnnotationUtils}, and
  \refclass{javacutil}{TypesUtils}.  Familiarity with
  them will help you to know how to access needed information and avoid
  reimplementing existing functionality.

  Write simple test cases to test the type rules, and ensure that the
  type-checker behaves as expected on those test files.
  For example, if your type system forbids indexing an array by a
  possibly-negative value, then you would write a test case such as:

\begin{Verbatim}
  void foo(String[] myarray, @UnknownSign int us, @NonNegative int nn) {
    myarray[us];  // expected error on this line
    myarray[nn];
  }
\end{Verbatim}

\item
  Implement type introduction rules, if any (Section~\ref{creating-type-introduction}).

  Test your type introduction rules.
  For example, if your type system sets the qualifier for manifest literal
  integers and for array lengths, you would write a test case like the following:

\begin{Verbatim}
  void foo(String[] myarray) {
    @NonNegative nn1 = -1;  // expected error on this line
    @NonNegative nn2 = 0;
    @NonNegative nn3 = 1;
    @NonNegative nn4 = myarray.length;
  }
\end{Verbatim}

\item
  Optionally, implement dataflow refinement rules
  (Section~\ref{creating-dataflow}).

  Test them if you wrote any.
  For instance, if after an arithmetic comparison, your type system infers
  which expressions are now known to be non-negative, you could write a
  test case such as:

\begin{Verbatim}
  void foo(@UnknownSign int us, @NonNegative int nn) {
    @NonNegative nn2;
    nn2 = us;  // expected error on this line
    if (us > j) {
      nn2 = us;
    }
    if (us >= j) {
      nn2 = us;
    }
    if (j < us) {
      nn2 = us;
    }
    if (j <= us) {
      nn2 = us;
    }
    nn = us;  // expected error on this line
  }
\end{Verbatim}

\end{enumerate}




\section{Annotations: Type qualifiers and hierarchy\label{creating-typequals}}

A type system designer specifies the qualifiers in the type system (Section~\ref{creating-define-type-qualifiers})
and
the type hierarchy that relates them.
The type hierarchy --- the subtyping relationships among the qualifiers ---
can be defined either
declaratively via meta-annotations (Section~\ref{creating-declarative-hierarchy}), or procedurally through
subclassing \refclass{framework/type}{QualifierHierarchy} or
\refclass{framework/type}{TypeHierarchy} (Section~\ref{creating-procedural-hierarchy}).


\subsection{Defining the type qualifiers\label{creating-define-type-qualifiers}}

%% True, but seems irrelevant here, so it detracts from the message.
% Each qualifier restricts the values that
% a type can represent.  For example \<@NonNull String> type can only
% represent non-null values, indicating that the variable may not hold
% \<null> values.

Type qualifiers are defined as Java annotations.  In Java, an
annotation is defined using the Java \code{@interface} keyword.
Here is how to define a two-qualifier hierarchy:

\begin{Verbatim}
package mypackage.qual;
import java.lang.annotation.Documented;
import java.lang.annotation.ElementType;
import java.lang.annotation.Retention;
import java.lang.annotation.RetentionPolicy;
import java.lang.annotation.Target;
import org.checkerframework.framework.qual.DefaultQualifierInHierarchy;
import org.checkerframework.framework.qual.SubtypeOf;
/**
 * The run-time value of the integer is unknown.
 *
 * @checker_framework.manual #nonnegative-checker Non-Negative Checker
 */
@Documented
@Retention(RetentionPolicy.RUNTIME)
@Target({ElementType.TYPE_USE, ElementType.TYPE_PARAMETER})
@SubtypeOf({})
@DefaultQualifierInHierarchy
public @interface UnknownSign {}


package mypackage.qual;
import java.lang.annotation.Documented;
import java.lang.annotation.ElementType;
import java.lang.annotation.Retention;
import java.lang.annotation.RetentionPolicy;
import java.lang.annotation.Target;
import org.checkerframework.framework.qual.ImplicitFor;
import org.checkerframework.framework.qual.LiteralKind;
import org.checkerframework.framework.qual.SubtypeOf;
/**
 * Indicates that the value is greater than or equal to zero.
 *
 * @checker_framework.manual #nonnegative-checker Non-Negative Checker
 */
@Documented
@Retention(RetentionPolicy.RUNTIME)
@Target({ElementType.TYPE_USE, ElementType.TYPE_PARAMETER})
@SubtypeOf({UnknownSign.class})
@ImplicitFor(literals = LiteralKind.NULL)
public @interface NonNegative {}
\end{Verbatim}

The \refqualclass{framework/qual}{SubtypeOf} meta-annotation
indicates the parent in the type hierarchy.

The \sunjavadocanno{java/lang/annotation/Target.html}{Target}
meta-annotation indicates where the annotation
may be written. All type qualifiers that users can write in source code should
have the value \<ElementType.TYPE\_USE> and optionally with the additional value
of \<ElementType.TYPE\_PARAMETER>, but no other \<ElementType> values.
%% This feels like clutter that distracts from the main point of the section.
% (Terminological note:  a \emph{meta-annotation} is an annotation that
% is written on an annotation definition, such as
% \refqualclass{framework/qual}{SubtypeOf} and
% \sunjavadocanno{java/lang/annotation/Target.html}{Target}.)

The annotations should be placed within a directory called \<qual>, and this
directory should be placed in the same directory as your Checker's source file.

For example, the Nullness Checker's source file is located at
\<.../nullness/NullnessChecker.java>. The \<NonNull> qualifier is located in the
directory \<.../nullness/qual/>.

The Checker Framework automatically treats any annotation that
is declared in the qual package as a type qualifier.
(See Section \ref{creating-indicating-supported-annotations} for more details.)

% \noindent
% The \<@Target({ElementType.TYPE\_USE})> meta-annotation
% distinguishes it from an ordinary
% annotation that applies to a declaration (e.g., \<@Deprecated> or
% \<@Override>).
% The framework ignores any annotation whose
% declaration does not bear the \<@Target({ElementType.TYPE\_USE})>
% meta-annotation (with minor
% exceptions, such as \<@SuppressWarnings>).

Your type system should include a top qualifier and a bottom qualifier
(Section~\ref{creating-bottom-and-top-qualifier}).
In most cases, the bottom qualifier should be meta-annotated with
\<@ImplicitFor(literals=LiteralKind.NULL)>.

You should also define a
polymorphic qualifier \<@Poly\emph{MyTypeSystem}>
(Section~\ref{qualifier-polymorphism}).


\subsection{Declaratively defining the qualifier hierarchy\label{creating-declarative-hierarchy}}

Declaratively, the type system designer uses two meta-annotations (written
on the declaration of qualifier annotations) to specify the qualifier
hierarchy.

\begin{itemize}

\item \refqualclass{framework/qual}{SubtypeOf} denotes that a qualifier is a subtype of
  another qualifier or qualifiers, specified as an array of class
  literals.  For example, for any type $T$,
  \refqualclass{checker/nullness/qual}{NonNull} $T$ is a subtype of \refqualclass{checker/nullness/qual}{Nullable} $T$:

  \begin{Verbatim}
    @Target({ElementType.TYPE_USE, ElementType.TYPE_PARAMETER})
    @SubtypeOf( { Nullable.class } )
    public @interface NonNull { }
  \end{Verbatim}

  % (The actual definition of \refclass{checker/nullness/qual}{NonNull} is slightly more complex.)


  %% True, but a distraction.  Move to Javadoc?
  % (It would be more natural to use Java subtyping among the qualifier
  % annotations, but Java forbids annotations from subtyping one another.)
  %
  \refqualclass{framework/qual}{SubtypeOf} accepts multiple annotation classes as an argument,
  permitting the type hierarchy to be an arbitrary DAG\@.

% TODO: describe multiple type hierarchies
% TODO: describe multiple polymorphic qualifiers and PolyAll
% TODO: the code consistently uses "top" for type qualifiers and
%       "root" for ASTs, in particular for CompilationUnitTrees.

  All type qualifiers, except for polymorphic qualifiers (see below and
  also Section~\ref{qualifier-polymorphism}), need to be
  properly annotated with \refclass{framework/qual}{SubtypeOf}.

  The top qualifier is annotated with
  \<@SubtypeOf( \{ \} )>.  The top qualifier is the qualifier that is
  a supertype of all other qualifiers.  For example, \refqualclass{checker/nullness/qual}{Nullable}
  is the top qualifier of the Nullness type system, hence is defined as:

  \begin{Verbatim}
    @Target({ElementType.TYPE_USE, ElementType.TYPE_PARAMETER})
    @SubtypeOf( { } )
    public @interface Nullable { }
  \end{Verbatim}

  \begin{sloppypar}
  If the top qualifier of the hierarchy is the unqualified type, then its children
  will use \code{@SubtypeOf(Unqualified.class)}, but no \code{@SubtypeOf(\{\})} annotation on the top qualifier is necessary.  For an example, see the
  \<Encrypted> type system of Section~\ref{encrypted-example}.
  \end{sloppypar}

\item \refqualclass{framework/qual}{PolymorphicQualifier} denotes that a qualifier is a
  polymorphic qualifier.  For example:

  \begin{Verbatim}
    @Target({ElementType.TYPE_USE, ElementType.TYPE_PARAMETER})
    @PolymorphicQualifier
    public @interface PolyNull { }
  \end{Verbatim}

  For a description of polymorphic qualifiers, see
  Section~\ref{qualifier-polymorphism}.  A polymorphic qualifier needs
  no \refqualclass{framework/qual}{SubtypeOf} meta-annotation and need not be
  mentioned in any other \refqualclass{framework/qual}{SubtypeOf}
  meta-annotation.

\end{itemize}

The declarative and procedural mechanisms for specifying the hierarchy can
be used together.  In particular, when using the \refqualclass{framework/qual}{SubtypeOf}
meta-annotation, further customizations may be
performed procedurally (Section~\ref{creating-procedural-hierarchy})
by overriding the \refmethodterse{framework/util}{GraphQualifierHierarchy}{isSubtype}{-java.util.Collection-java.util.Collection-} method in the checker class
(Section~\ref{creating-compiler-interface}).
However, the declarative mechanism is sufficient for most type systems.


\subsection{Procedurally defining the qualifier hierarchy\label{creating-procedural-hierarchy}}

While the declarative syntax suffices for many cases, more complex
type hierarchies can be expressed by overriding, in your subclass of \refclass{common/basetype}{BaseTypeVisitor},
either \refmethodterse{framework/type}{AnnotatedTypeFactory}{createQualifierHierarchy}{--} or \refmethodterse{framework/type}{AnnotatedTypeFactory}{createTypeHierarchy}{--} (typically
only one of these needs to be overridden).
For more details, see the Javadoc of those methods and of the classes
\refclass{framework/type}{QualifierHierarchy} and \refclass{framework/type}{TypeHierarchy}.

The \refclass{framework/type}{QualifierHierarchy} class represents the qualifier hierarchy (not the
type hierarchy).  A type-system designer may subclass
\refclass{framework/type}{QualifierHierarchy} to express customized qualifier
relationships (e.g., relationships based on annotation
arguments).

The \refclass{framework/type}{TypeHierarchy} class represents the type hierarchy ---
that is, relationships between
annotated types, rather than merely type qualifiers, e.g., \<@NonNull
Date> is a subtype of \<@Nullable Date>.  The default \refclass{framework/type}{TypeHierarchy} uses
\refclass{framework/type}{QualifierHierarchy} to determine all subtyping relationships.
The default \refclass{framework/type}{TypeHierarchy} handles
generic type arguments, array components, type variables, and
wildcards in a similar manner to the Java standard subtype
relationship but with taking qualifiers into consideration.  Some type
systems may need to override that behavior.  For instance, the Java
Language Specification specifies that two generic types are subtypes only
if their type arguments are identical:  for example,
\code{List<Date>} is not a subtype of \code{List<Object>}, or of any other
generic \code{List}.
(In the technical jargon, the generic arguments are ``invariant'' or ``novariant''.)


\subsection{Defining a default annotation\label{creating-typesystem-defaults}}

A type system applies a default qualifier where the user has not written a
qualifier (and no implicit qualifier is applicable), as explained in
Section~\ref{defaults}.

The type system designer may specify a default annotation declaratively,
using the \refqualclass{framework/qual}{DefaultQualifierInHierarchy}
meta-annotation.
Note that the default will apply to any source code that the checker reads,
including stub libraries, but will not apply to compiled \code{.class}
files that the checker reads.

\begin{sloppypar}
Alternately, the type system designer may specify a default procedurally,
by overriding the
\refmethod{framework/type}{GenericAnnotatedTypeFactory}{addCheckedCodeDefaults}{-org.checkerframework.framework.util.defaults.QualifierDefaults-}
method.  You may do this even if you have declaratively defined the
qualifier hierarchy.
\end{sloppypar}

If the default qualifier in the type hierarchy requires a value, there are
ways for the type system designer to specify a default value both
declaratively and procedurally, as well.  To do so declaratively, append
the string \<default \emph{value}> where \emph{value} is the actual value
you want to be the default, after the declaration of the value in the
qualifier file.  For instance, \code{int value() default 0;} would make
\code{value} default to zero. Alternatively, the procedural method
described above can be used.


\subsection{Relevant Java types\label{creating-relevant-java-types}}

A checker can use the \refqualclass{framework/qual}{RelevantJavaTypes}
annotation on the checker class to specify which Java types are relevant to the
checker.  All irrelevant types without explicit annotations are defaulted to
the top annotation.


\subsection{Do not re-use type qualifiers\label{creating-do-not-re-use-type-qualifiers}}

Every annotation should belong to only one type system.  No annotation
should be used by multiple type systems.  This is true even of annotations
that are internal to the type system and are not intended to be written by
the programmer.

Suppose that you have two type systems that both use the same type
qualifier \<@A>.  In a client program, a use of type \<T> may require type
qualifier \<@A> for one type system but a different qualifier for the other
type system.  There is no annotation that a programmer can write to make
the program type-check under both type systems.

This also applies to type qualifiers that a programmer does not write,
because the compiler outputs \<.class> files that contain an explicit type
qualifier on every type --- a defaulted or inferred type qualifier if the
programmer didn't write a type qualifier explicitly.


\subsection{Completeness of the type hierarchy\label{creating-bottom-and-top-qualifier}}

When you define a type system, its type hierarchy must be a
complete lattice --- that is, there must be a top type that is a
supertype of all other types, and there must be a bottom type that is a
subtype of all other types.
Furthermore, it is best if the top type and bottom type are defined
explicitly for the type system, rather than (say) reusing a qualifier from the
Checker Framework such as \<@Unqualified>.

It is possible that a single type-checker checks multiple type hierarchies.
An example is the Nullness Checker, which has three separate type
hierarchies, one each for
nullness, initialization, and map keys.  In this case, each type hierarchy
would have its own top qualifier and its own bottom qualifier; they don't
all have to share a single top qualifier or a single bottom qualifier.


\paragraph{Bottom qualifier\label{creating-bottom-qualifier}}
Your type hierarchy must have a bottom qualifier
--- a qualifier that is a (direct or indirect) subtype of every other
qualifier.

Your type system must give \<null> the bottom type.
(The only exception
is if the type system has special treatment for \<null> values, as the
Nullness Checker does.)
A way to do this is to meta-annotate your bottom type with
\<@ImplicitFor(literals=LiteralKind.NULL)>.
This legal code
will not type-check unless \<null> has the bottom type:
\begin{Verbatim}
<T> T f() {
    return null;
}
\end{Verbatim}

% \begin{sloppypar}
% You don't necessarily have to define a new bottom qualifier.  You can
% use \<org.checkerframework.framework.qual.Bottom> if your type system does not already have an
% appropriate bottom qualifier.
% \end{sloppypar}

Some type systems have a special bottom type that is used \emph{only} for
the \code{null} value, and for dead code and other erroneous situations.
In this case, users should only write the bottom qualifier on explicit
bounds.  In this case, the definition of the bottom qualifier should be
meta-annotated with:

% import java.lang.annotation.ElementType;
% import java.lang.annotation.Target;
% import org.checkerframework.framework.qual.TargetLocations;
% import org.checkerframework.framework.qual.TypeUseLocation;
%
\begin{Verbatim}
@Target({ElementType.TYPE_USE, ElementType.TYPE_PARAMETER})
@TargetLocations({TypeUseLocation.EXPLICIT_LOWER_BOUND, TypeUseLocation.EXPLICIT_UPPER_BOUND})
\end{Verbatim}

Furthermore, by convention the name of such a qualifier ends with ``\<Bottom>''.

The hierarchy shown in Figure~\ref{fig-initialization-hierarchy} lacks
a bottom qualifier, but the actual implementation does contain a (non-user-visible) bottom qualifier.


\paragraph{Top qualifier\label{creating-top-qualifier}}
Your type hierarchy must have a top qualifier
--- a qualifier that is a (direct or indirect) supertype of every other
qualifier.
Here is the reason.
The default type for local variables is the top
qualifier (that type is then flow-sensitively
refined depending on what values are stored in the local variable).
If there is no single top qualifier, then there is no
unambiguous choice to make for local variables.

Furthermore, it is most convenient to users if the top qualifier is defined
by the type system.  It is possible to use the framework's
\<@Unqualified> as the top type, but this is poor practice.
Users lose
flexibility in expressing defaults:  there is no
way for a user to change the default qualifier for just that type system.
If a user specifies
\<@DefaultQualifier(Unqualified.class)>,
then the default would apply to every
type system that uses \<@Unqualified>, which is unlikely to be desired.


\subsection{Annotations whose argument is a Java expression (dependent type annotations)\label{dependent-types}}
\label{expression-annotations}

Sometimes, an annotation needs to refer to a Java expression.
Section~\ref{java-expressions-as-arguments} gives examples of such
annotations and also explains what Java expressions can and cannot be
referred to.

This section explains how to implement a dependent type annotation.

A ``dependent type annotation''
must have one attribute, \<value>, that is an
array of strings.  The Checker Framework verifies that the annotation's
arguments are valid expressions according to the rules of
Section~\ref{java-expressions-as-arguments}.  If
the expression is not valid, an error is issued and the string in the
annotation is changed to indicate that the expression is not valid.

The Checker Framework standardizes the expression strings.  For example, a
field \<f> can be referred to as either ``f'' or ``this.f''.  If the
programmer writes ``f'', the Checker Framework changes it to
``this.f'', as if the programmer had written that.
An advantage of this canonicalization is that
that comparisons, such as \<isSubtype>, can be implemented as string comparisons.

The Checker Framework viewpoint-adapts type annotations on method, constructor,
and field declarations at uses for those methods.  For example, given the
following class

\begin{Verbatim}
class MyClass {
   Object field = ...;
   @Anno("this.field") Object field2 = ...;
}
\end{Verbatim}
and assuming the variable \<myClass> is of type \<MyClass>, then the type of
\<myClass.field> is viewpoint-adapted to \<@Anno("myClass.field")>.

To use this built-in functionality, add a \refqualclass{framework/qual}{JavaExpression} annotation
to any annotation element that should be interpreted as a Java expression.  The type of the
element must be an array of Strings.  If your checker requires special handling of Java expressions,
your checker implementation should override
\refmethod{framework/type}{GenericAnnotatedTypeFactory}{createDependentTypesHelper}{--}
to return a subclass of \<DependentTypesHelper>.

Given a specific expression in the program (of type Tree or Node), a
checker may need to obtain its canonical string representation.  This
enables the checker to create an dependent type annotation that refers to
it, or to compare to the string expression of an existing expression
annotation.
To obtain the string, first create a
\refclass{dataflow/analysis}{FlowExpressions.Receiver} object by calling
\refmethodanchortext{dataflow/analysis}{FlowExpressions}{internalReprOf}{-org.checkerframework.javacutil.AnnotationProvider-com.sun.source.tree.ExpressionTree-}{internalReprOf(AnnotationProvider,
ExpressionTree)} or
\refmethodanchortext{dataflow/analysis}{FlowExpressions}{internalReprOf}{-org.checkerframework.javacutil.AnnotationProvider-org.checkerframework.dataflow.cfg.node.Node-}{internalReprOf(AnnotationProvider,
Node)}.
Then, call \<toString()> on the \<FlowExpressions.Receiver> object.


\section{Visitor: Type rules\label{creating-extending-visitor}}

A type system's rules define which operations on values of a
particular type are forbidden.
These rules must be defined procedurally, not declaratively.
Put them in a file \<\emph{MyChecker}Visitor.java> that extends
\refclass{common/basetype}{BaseTypeVisitor}.

BaseTypeVisitor performs type-checking at each node of a
source file's AST\@.  It uses the visitor design pattern to traverse
Java syntax trees as provided by Oracle's
\href{http://docs.oracle.com/javase/8/docs/jdk/api/javac/tree/}{Tree
API},
and it issues a warning (by calling
\refmethod{framework/source}{SourceChecker}{report}
{-org.checkerframework.framework.source.Result-java.lang.Object-})
whenever the type system is violated.

Most type-checkers
override only a few methods in \refclass{common/basetype}{BaseTypeVisitor}.
A checker's visitor overrides one method in the base visitor for each special
rule in the type qualifier system.
The the last line of the overridden version is
\<return super.visit\emph{TreeType}(node, p);>
so that, if the method didn't raise any error,
the superclass version can perform standard checks.


By default, \refclass{common/basetype}{BaseTypeVisitor} performs subtyping checks that are
similar to Java subtype rules, but taking the type qualifiers into account.
\refclass{common/basetype}{BaseTypeVisitor} issues these errors:

\begin{itemize}

\item invalid assignment (type.incompatible) for an assignment from
  an expression type to an incompatible type.  The assignment may be a
  simple assignment, or pseudo-assignment like return expressions or
  argument passing in a method invocation

  In particular, in every assignment and pseudo-assignment, the
  left-hand side of the assignment is a supertype of (or the same type
  as) the right-hand side.  For example, this assignment is not
  permitted:

  \begin{Verbatim}
    @Nullable Object myObject;
    @NonNull Object myNonNullObject;
    ...
    myNonNullObject = myObject;  // invalid assignment
  \end{Verbatim}

\item invalid generic argument (type.argument.type.incompatible) when a type
  is bound to an incompatible generic type variable

\item invalid method invocation (method.invocation.invalid) when a
  method is invoked on an object whose type is incompatible with the
  method receiver type

\item invalid overriding parameter type (override.parameter.invalid)
  when a parameter in a method declaration is incompatible with that
  parameter in the overridden method's declaration

\item invalid overriding return type (override.return.invalid) when a
  parameter in a method declaration is incompatible with that
  parameter in the overridden method's declaration

\item invalid overriding receiver type (override.receiver.invalid)
  when a receiver in a method declaration is incompatible with that
  receiver in the overridden method's declaration

\end{itemize}


\subsection{AST traversal\label{creating-ast-traversal}}

The Checker Framework needs to do its own traversal of the AST even though
it operates as an ordinary annotation processor~\cite{JSR269}.  Java
provides a visitor for Java code that is intended to be used by annotation
processors, but that visitor only
visits the public elements of Java code, such as classes, fields, methods,
and method arguments --- it does not visit code bodies or various other
locations.  The Checker Framework hardly uses the built-in visitor --- as
soon as the built-in visitor starts to visit a class, then the Checker
Framework's visitor takes over and visits all of the class's source code.

Because there is no standard API for the AST of Java
code\footnote{Actually, there is a standard API for Java ASTs --- JSR 198
  (Extension API for Integrated Development Environments)~\cite{JSR198}.
  If tools were to implement it (which would just require writing wrappers
  or adapters), then the Checker Framework and similar tools could be
  portable among different compilers and IDEs.}, the Checker Framework uses
the javac implementation.  This is why the Checker Framework is not deeply
integrated with Eclipse or IntelliJ IDEA, but runs as an external tool (see
Section~\ref{eclipse}).


\subsection{Avoid hardcoding\label{creating-avoid-hardcoding}}

It may be tempting to write a type-checking rule for method invocation,
where your rule checks the name of the method being called and then treats
the method in a special way.  This is sometimes necessary but is usually
the wrong approach.  It is better to write annotations, in a stub file
(Chapter~\ref{annotating-libraries}), and leave the work to the standard
type-checking rules.


\section{Type factory: Implicit annotations (type introduction rules)\label{creating-type-introduction}}

For some types and expressions, a qualifier should be treated as present
even if a programmer did not explicitly write it.  For example, every
literal (other than \<null>) has a \refqualclass{checker/nullness/qual}{NonNull} type.
Section~\ref{effective-qualifier} explains the meaning of implicit
qualifiers, such as that they cannot be overridden.

The implicit annotations may be specified declaratively
(Section~\ref{creating-declarative-type-introduction}) and/or procedurally
(Section~\ref{creating-procedurally-specifying-implicit-annotations}).
It is easiest to specify them declaratively, when the declarative method is
sufficiently expressive.


\subsection{Declaratively specifying implicit annotations\label{creating-declarative-type-introduction}}

The \refqualclass{framework/qual}{ImplicitFor} meta-annotation indicates
implicit annotations.
When written on a qualifier, \refclass{framework/qual}{ImplicitFor}
specifies the trees (AST nodes) and types for which the framework should
automatically add that qualifier.

In short, the types and trees can be
specified via any combination of four fields in \refclass{framework/qual}{ImplicitFor}.

For example, consider the definitions of the
\refqualclass{checker/nullness/qual}{Nullable} and
\refqualclass{checker/nullness/qual}{NonNull}
type qualifiers:

%BEGIN LATEX
\begin{smaller}
%END LATEX
\begin{Verbatim}
  @SubtypeOf({})
  @ImplicitFor(literals = { LiteralKind.NULL },
               typeNames = { java.lang.Void.class })
  @Target({ElementType.TYPE_USE, ElementType.TYPE_PARAMETER})
  public @interface Nullable { }

  @SubtypeOf( { Nullable.class } )
  @ImplicitFor(types = { TypeKind.PACKAGE,
                         TypeKind.INT, TypeKind.BOOLEAN, TypeKind.CHAR,
                         TypeKind.DOUBLE, TypeKind.FLOAT, TypeKind.LONG,
                         TypeKind.SHORT, TypeKind.BYTE },
      // All literals except NULL_LITERAL:
      literals = { LiteralKind.BOOLEAN, LiteralKind.CHAR, LiteralKind.DOUBLE,
                   LiteralKind.FLOAT, LiteralKind.INT, LiteralKind.LONG,
                   LiteralKind.STRING })
  @Target({ElementType.TYPE_USE, ElementType.TYPE_PARAMETER})
  public @interface NonNull {  }
\end{Verbatim}
%BEGIN LATEX
\end{smaller}
%END LATEX

For more details, see the Javadoc for the \refclass{framework/qual}{ImplicitFor}
annotation, and the Javadoc for the javac classes that are linked from
it.  You only need to understand a small amount about the javac AST, such
as the
\href{\TreeAPIBase{}/tree/Tree.Kind.html?is-external=true}{\code{Tree.Kind}}
and
\refModelclass{type}{TypeKind}
enums.  All the information you need is in the Javadoc, and
Section~\ref{creating-javac-tips} can help you get started.


\subsection{Procedurally specifying implicit annotations\label{creating-procedurally-specifying-implicit-annotations}}

If the \<@ImplicitFor> annotation is not sufficiently expressive, then you
can write code to set implicit annotations.  To do so, create a subclass of
\refclass{framework/type}{AnnotatedTypeFactory} and override its
\<addComputedTypeAnnotations> methods.

\<AnnotatedTypeFactory>, when given a program
expression, returns the expression's type.  This should include not only
the qualifiers that the programmer explicitly wrote in the source code, but
also default and implicit annotations, and flow-sensitive local type
inference (see Section~\ref{effective-qualifier} for explanations of these
concepts).

To add implicit annotations, you should override
\refmethodanchortext{framework/type}{AnnotatedTypeFactory}{addComputedTypeAnnotations}{-com.sun.source.tree.Tree-org.checkerframework.framework.type.AnnotatedTypeMirror-}{addComputedTypeAnnotations(Tree,AnnotatedTypeMirror)}
(or
\refmethodanchortext{framework/type}{GenericAnnotatedTypeFactory}{addComputedTypeAnnotations}{-com.sun.source.tree.Tree-org.checkerframework.framework.type.AnnotatedTypeMirror-boolean-}{addComputedTypeAnnotations(Tree,AnnotatedTypeMirror,boolean)}
if extending \code{GenericAnnotatedTypeFactory})
and
\refmethodanchortext{framework/type}{AnnotatedTypeFactory}{addComputedTypeAnnotations}{-javax.lang.model.element.Element-org.checkerframework.framework.type.AnnotatedTypeMirror-}{addComputedTypeAnnotations(Element,AnnotatedTypeMirror)}.
The methods operate on \refclass{framework/type}{AnnotatedTypeMirror},
which is the Checker Framework's representation of an annotated type.



\section{Dataflow: enhancing flow-sensitive type qualifier inference\label{creating-dataflow}}

By default, every checker performs flow-sensitive type refinement, also known as
local type inference, as described
in Section~\ref{type-refinement}.

This section of the manual explains how to enhance the Checker Framework's
built-in flow-sensitive type refinement.
Most commonly, you will inform the Checker Framework about a run-time test
that gives information about the type qualifiers in your type system.
Section~\refwithpageparen{type-refinement-runtime-tests} gives examples of
type systems with and without run-time tests.

The steps to customizing type refinement are:
\begin{enumerate}
\item{\S\ref{creating-dataflow-determine-expressions}}
  Determine which expressions will be refined.
\item{\S\ref{creating-dataflow-create-classes}}
  Create required class and configure its use.
\item{\S\ref{creating-dataflow-override-methods}}
  Override methods that handle \refclass{dataflow/cfg/node}{Node}s of interest.
\item{\S\ref{creating-dataflow-implement-refinement}}
  Implement the refinement.
\end{enumerate}

The Regex Checker's dataflow customization for the
\refmethod{checker/regex}{RegexUtil}{asRegex}{-java.lang.String-}
run-time check is used as an example throughout the steps.

If needed, you can find more details about the implementation of
flow-sensitive type refinement, and the control flow graph (CFG) data
structure that it uses, in the
\href{https://checkerframework.org/manual/checker-framework-dataflow-manual.pdf}{Dataflow
  Manual}.


\subsection{Determine expressions to refine the types of\label{creating-dataflow-determine-expressions}}

A run-time check or run-time
operation involves multiple expressions (arguments, results).
Determine which expression the customization will refine.  This is
usually specific to the type system and run-time test.
There is no code to write in this step; you are merely determining
the design of your type refinement.

For the program operation \code{op(a,b)}, you can refine
the types in either or both of the following ways:
\begin{enumerate}
\item Change the result type of the entire expression \code{op(a,b)}.

As an example (and as the running example of implementing a dataflow
refinement), the \code{RegexUtil.asRegex} method is declared as:

%BEGIN LATEX
\begin{smaller}
%END LATEX
\begin{Verbatim}
  @Regex(0) String asRegex(String s, int groups) { ... }
\end{Verbatim}
%BEGIN LATEX
\end{smaller}
%END LATEX

\noindent
This annotation is sound and conservative:  it says that an expression such
as \code{RegexUtil.asRegex(myString, myInt)} has type \code{@Regex(0)
  String}.  However, this annotation is imprecise.  When the \code{group}
argument is known at compile time, a better estimate can be given.  For
example, \code{RegexUtil.asRegex(myString, 2)} has type \code{@Regex(2)
  String}.

\item Change the type of some other expression, such as \code{a} or \code{b}.

As an example, consider an equality test in the Nullness type system:

\begin{Verbatim}
  @Nullable String s;
    if (s != null) {
      ...
    } else {
      ...
    }
\end{Verbatim}

The type of \<s != null> is always \<@NonNull boolean>.  However, in the
true branch, the type of \<s> can be refined to \<@NonNull String>.

\end{enumerate}

If you are refining the types of arguments or the result of a method call,
then you may be able to implement your flow-sensitive refinement rules by
just writing \refqualclass{framework/qual}{EnsuresQualifier} and/or
\refqualclass{framework/qual}{EnsuresQualifierIf} annotations.
When this is possible, it is the best approach.

Sections~\ref{creating-dataflow-create-classes}--\ref{creating-dataflow-implement-refinement}
explain how to create a transfer class when the
\refqualclass{framework/qual}{EnsuresQualifier} and
\refqualclass{framework/qual}{EnsuresQualifierIf} annotations are insufficient.


\subsection{Create required class\label{creating-dataflow-create-classes}}

In the same directory as \<\emph{MyChecker}Checker.java>, create a class
named \<\emph{MyChecker}Transfer> that extends
\refclass{framework/flow}{CFTransfer}.

Leave the class body empty for now.  Your class will add functionality by
overriding methods of CFTransfer, which performs the default Checker
Framework type refinement.

As an example, the Regex Checker's extended
\refclass{framework/flow}{CFTransfer} is
\refclass{checker/regex}{RegexTransfer}.

(If you disregard the instructions above and choose a different name or a
different directory for your \<\emph{MyChecker}Transfer> class, you will
also need to override the \<createFlowTransferFunction> method in your type
factory to return to return a new instance of the class.)

(As a reminder, use of \refqualclass{framework/qual}{EnsuresQualifier} and
\refqualclass{framework/qual}{EnsuresQualifierIf} may obviate the need for
a transfer class.)

%% More extended directions about what do to if the name is non-standard.
% If the checker's extended \refclass{framework/flow}{CFTransfer}
% starts with the name of the type system, then the type factory will use the
% transfer class without further configuration. For example, if the checker
% class is \<FooChecker>, then if the transfer class is \<FooTransfer>, then it is
% not necessary to configure the type factory
% to use \<FooTransfer>.  If some other naming convention is used, then
% to configure your checker's type factory to use the new extended
% \refclass{framework/flow}{CFTransfer}, override the
% \code{createFlowTransferFunction} method in your type factory to return a new instance
% of the extended \refclass{framework/flow}{CFTransfer}.
%
% %BEGIN LATEX
% \begin{smaller}
% %END LATEX
% \begin{Verbatim}
%   @Override
%   public CFTransfer createFlowTransferFunction(
%           CFAbstractAnalysis<CFValue, CFStore, CFTransfer> analysis) {
%       return new RegexTransfer((CFAnalysis) analysis);
%   }
% \end{Verbatim}
% %BEGIN LATEX
% \end{smaller}
% %END LATEX

%% The text below is true, but not required.
%\item \textbf{Create a class that extends
%    \refclass{framework/flow}{CFAbstractAnalysis} and uses the extended
%    \refclass{framework/flow}{CFAbstractTransfer}}
%
%  \begin{sloppypar}
%  \refclass{framework/flow}{CFAbstractTransfer} and its superclass,
%  \refclass{dataflow/analysis}{Analysis}, are the central coordinating classes
%  in the Checker Framework's dataflow algorithm. The
%  \code{createTransferFunction} method must be overridden in an extended
%  \refclass{framework/flow}{CFAbstractTransfer} to return a new instance of the
%  extended \refclass{framework/flow}{CFAbstractTransfer}.
%  \end{sloppypar}
%
%  \begin{sloppypar}
%  The Regex Checker's extended \refclass{framework/flow}{CFAbstractAnalysis} is
%  \refclass{checker/regex/classic}{RegexAnalysis}, which overrides the
%  \code{createTransferFunction} to return a new
%  \refclass{checker/regex/classic}{RegexTransfer} instance:
%  \end{sloppypar}
%
%%BEGIN LATEX
%\begin{smaller}
%%END LATEX
%\begin{Verbatim}
%  @Override
%  public RegexTransfer createTransferFunction() {
%      return new RegexTransfer(this);
%  }
%\end{Verbatim}
%%BEGIN LATEX
%\end{smaller}
%%END LATEX
%
%\item \textbf{Configure the checker's type factory to use the extended
%    \refclass{framework/flow}{CFAbstractAnalysis}}
%
%\begin{sloppypar}
%To configure your checker's type factory to use the new extended
%\refclass{framework/flow}{CFAbstractAnalysis}, override the
%\code{createFlowAnalysis} method in your type factory to return a new instance
%of the extended \refclass{framework/flow}{CFAbstractAnalysis}.
%\end{sloppypar}
%
%%BEGIN LATEX
%\begin{smaller}
%%END LATEX
%\begin{Verbatim}
%  @Override
%  protected RegexAnalysis createFlowAnalysis(
%          List<Pair<VariableElement, CFValue>> fieldValues) {
%
%      return new RegexAnalysis(checker, this, fieldValues);
%  }
%\end{Verbatim}
%%BEGIN LATEX
%\end{smaller}
%%END LATEX


\subsection{Override methods that handle Nodes of interest\label{creating-dataflow-override-methods}}

Decide what source code syntax is relevant to the the run-time checks or
run-time operations you are trying to support.  The CFG (control flow
graph) represents source code as \refclass{dataflow/cfg/node}{Node}, a
node in the abstract syntax tree of the program being checked (see
\href{#creating-dataflow-representation}{``Program representation''} below).

In your extended \refclass{framework/flow}{CFTransfer}
override the visitor method that handles the \refclass{dataflow/cfg/node}{Node}s
relevant to your run-time check or run-time operation.
Leave the body of the overriding method empty for now.

For example, the Regex Checker refines the type of a run-time test method
call.  A method call is represented by a
\refclass{dataflow/cfg/node}{MethodInvocationNode}.  Therefore,
\refclass{checker/regex}{RegexTransfer} overrides the
\code{visitMethodInvocation} method:

%BEGIN LATEX
\begin{smaller}
%END LATEX
\begin{Verbatim}
  public TransferResult<CFValue, CFStore> visitMethodInvocation(
    MethodInvocationNode n, TransferInput<CFValue, CFStore> in)  { ... }
\end{Verbatim}
%BEGIN LATEX
\end{smaller}
%END LATEX


\subsubsection{Program representation\label{creating-dataflow-representation}}

% A \refclass{dataflow/cfg/node}{Node} generally maps one-to-one with a
% \refTreeclass{tree}{Tree}. When dataflow processes a method, it translates
% \refTreeclass{tree}{Tree}s into \refclass{dataflow/cfg/node}{Node}s and then
% calls the appropriate visit method on
% \refclass{framework/flow}{CFAbstractTransfer} which then performs the dataflow
% analysis for the passed in \refclass{dataflow/cfg/node}{Node}.

The \refclass{dataflow/cfg/node}{Node} subclasses can be found in the
\code{org.checkerframework.dataflow.cfg.node} package.  Some examples are
\refclass{dataflow/cfg/node}{EqualToNode},
\refclass{dataflow/cfg/node}{LeftShiftNode},
\refclass{dataflow/cfg/node}{VariableDeclarationNode}.

A \refclass{dataflow/cfg/node}{Node}
is basically equivalent to a javac compiler \refTreeclass{tree}{Tree}.

See Section~\ref{creating-javac-tips} for more information about \refTreeclass{tree}{Tree}s.
As an example, the statement \<String a = "";>, is represented as this
abstract syntax tree:
\begin{Verbatim}
VariableTree:
  name: "a"
  type:
    IdentifierTree
      name: String
  initializer:
    LiteralTree
      value: ""
\end{Verbatim}



\subsection{Implement the refinement\label{creating-dataflow-implement-refinement}}

\begin{sloppypar}
Each visitor method in \refclass{framework/flow}{CFAbstractTransfer}
returns a \refclass{dataflow/analysis}{TransferResult}.  A
\refclass{dataflow/analysis}{TransferResult} represents the
refined information that is known after an operation.  It has two
components:  the result type for the \refclass{dataflow/cfg/node}{Node}
being evaluated, and a map from expressions in scope to estimates of their
types (a \refclass{dataflow/analysis}{Store}).  Each of these components is
relevant to one of the two cases in
Section~\ref{creating-dataflow-determine-expressions}:
\end{sloppypar}

\begin{enumerate}
\item
\begin{sloppypar}
Changing the \refclass{dataflow/analysis}{TransferResult}'s result type changes
the type that is returned by the \refclass{framework/type}{AnnotatedTypeFactory}
for the tree corresponding to the \refclass{dataflow/cfg/node}{Node} that was
visited.  (Remember that \refclass{common/basetype}{BaseTypeVisitor} uses the
\refclass{framework/type}{AnnotatedTypeFactory} to look up the type of a
\refTreeclass{tree}{Tree}, and then performs checks on types of one or more
\refTreeclass{tree}{Tree}s.)
\end{sloppypar}

For example, When \refclass{checker/regex}{RegexTransfer} evaluates a
\code{RegexUtils.asRegex} invocation, it updates the
\refclass{dataflow/analysis}{TransferResult}'s result type. This changes the
type of the \code{RegexUtils.asRegex} invocation when its
\refTreeclass{tree}{Tree} is looked up by the
\refclass{framework/type}{AnnotatedTypeFactory}.  See below for details.

\item
Updating the \refclass{dataflow/analysis}{Store} treats an expression as
having a refined type for the remainder of the method or conditional block. For
example, when the Nullness Checker's dataflow evaluates \code{myvar != null}, it
updates the \refclass{dataflow/analysis}{Store} to specify that the variable
\code{myvar} should be treated as having type \code{@NonNull} for the rest of the
then conditional block.  Not all kinds of expressions can be refined; currently
method return values, local variables, fields, and array values can be stored in
the \refclass{dataflow/analysis}{Store}.  Other kinds of expressions, like
binary expressions or casts, cannot be stored in the
\refclass{dataflow/analysis}{Store}.

\end{enumerate}


\begin{sloppypar}
The rest of this section details implementing the visitor method
\code{RegexTransfer.visitMethodInvocation} for the \code{RegexUtil.asRegex}
run-time test.  You can find other examples of visitor methods in
\refclass{checker/lock}{LockTransfer} and
\refclass{checker/formatter}{FormatterTransfer}.
\end{sloppypar}



\begin{enumerate}
\item \textbf{Determine if the visited \refclass{dataflow/cfg/node}{Node} is of
    interest}

A visitor method is invoked for all
instances of a given \refclass{dataflow/cfg/node}{Node} kind in the
program.
The visitor must inspect the
\refclass{dataflow/cfg/node}{Node} to determine if it is an
instance of the desired run-time test or operation.  For example,
\code{visitMethodInvocation} is called when dataflow processes any method
invocation, but the \refclass{checker/regex}{RegexTransfer} should only refine
the result of \code{RegexUtils.asRegex} invocations:

%BEGIN LATEX
\begin{smaller}
%END LATEX
\begin{Verbatim}
  @Override
  public TransferResult<CFValue, CFStore> visitMethodInvocation(...)
    ...
    MethodAccessNode target = n.getTarget();
    ExecutableElement method = target.getMethod();
    Node receiver = target.getReceiver();
    if (receiver instanceof ClassNameNode) {
      String receiverName = ((ClassNameNode) receiver).getElement().toString();

      // Is this a call to method RegexUtil.isRegex(s, groups)?
      if (receiverName.equals("RegexUtil")
          && ElementUtils.matchesElement(method,
                 null, "isRegex", String.class, int.class)) {
            ...
\end{Verbatim}
%BEGIN LATEX
\end{smaller}
%END LATEX

\item \textbf{Determine the refined type}

Sometimes the refined type is dependent on the parts of the operation,
such as arguments passed to it.

For example, the refined type of \code{RegexUtils.asRegex} is dependent on the
integer argument to the method call. The \refclass{checker/regex}{RegexTransfer}
uses this argument to build the resulting type \code{@Regex(\emph{i})}, where \code{\emph{i}}
is the value of the integer argument.  For simplicity the below code only uses
the value of the integer argument if the argument was an integer literal.  It
could be extended to use the value of the argument if it was any compile-time
constant or was inferred at compile time by another analysis, such as the
Constant Value Checker (\chapterpageref{constant-value-checker}).

%BEGIN LATEX
\begin{smaller}
%END LATEX
\begin{Verbatim}
  AnnotationMirror regexAnnotation;
  Node count = n.getArgument(1);
  if (count instanceof IntegerLiteralNode) {
    // argument is a literal integer
    IntegerLiteralNode iln = (IntegerLiteralNode) count;
    Integer groupCount = iln.getValue();
    regexAnnotation = factory.createRegexAnnotation(groupCount);
  } else {
    // argument is not a literal integer; fall back to @Regex(), which is the same as @Regex(0)
    regexAnnotation = AnnotationUtils.fromClass(factory.getElementUtils(), Regex.class);
  }
\end{Verbatim}
%BEGIN LATEX
\end{smaller}
%END LATEX


\item \textbf{Return a \refclass{dataflow/analysis}{TransferResult} with the
    refined types}

Recall that the type of an expression is refined by modifying the
\refclass{dataflow/analysis}{TransferResult} returned by a visitor method.
Since the \refclass{checker/regex}{RegexTransfer} is updating the type of
the run-time test itself, it will update the result type and not the
\refclass{dataflow/analysis}{Store}.

A \refclass{framework/flow}{CFValue} is created to hold the type inferred.
\refclass{framework/flow}{CFValue} is a wrapper class for values being inferred
by dataflow:
%BEGIN LATEX
\begin{smaller}
%END LATEX
\begin{Verbatim}
  CFValue newResultValue = analysis.createSingleAnnotationValue(regexAnnotation,
      result.getResultValue().getType().getUnderlyingType());
\end{Verbatim}
%BEGIN LATEX
\end{smaller}
%END LATEX

Then, RegexTransfer's \code{visitMethodInvocation} creates and returns a
\refclass{dataflow/analysis}{TransferResult} using \code{newResultValue} as the
result type.

%BEGIN LATEX
\begin{smaller}
%END LATEX
\begin{Verbatim}
  return new RegularTransferResult<>(newResultValue, result.getRegularStore());
\end{Verbatim}
%BEGIN LATEX
\end{smaller}
%END LATEX

As a result of this code, when the Regex Checker encounters a
\code{RegexUtils.asRegex} method call, the checker will refine the return
type of the method if it can determine the value of the integer parameter
at compile time.

\end{enumerate}


\subsection{Disabling flow-sensitive inference\label{creating-dataflow-disable}}

In the uncommon case that you wish to disable the Checker Framework's
built-in flow inference in your checker (this is different than choosing
not to extend it as described in Section~\ref{creating-dataflow}), put the
following two lines at the beginning of the constructor for your subtype of
\refclass{common/basetype}{BaseAnnotatedTypeFactory}:

\begin{Verbatim}
        // disable flow inference
        super(checker, false);
\end{Verbatim}


\section{The checker class:  Compiler interface\label{creating-compiler-interface}}

A checker's entry point is a subclass of
\refclass{framework/source}{SourceChecker}, and is usually a direct subclass
of either \refclass{common/basetype}{BaseTypeChecker} or
\refclass{framework/source}{AggregateChecker}.
This entry
point, which we call the checker class, serves two
roles:  an interface to the compiler and a factory for constructing
type-system classes.

Because the Checker Framework provides reasonable defaults, oftentimes the
checker class has no work to do.  Here are the complete definitions of the
checker classes for the Interning Checker and the Nullness Checker:

\begin{Verbatim}
  @SupportedLintOptions({"dotequals"})
  public final class InterningChecker extends BaseTypeChecker { }

  @SupportedLintOptions({"flow", "cast", "cast:redundant"})
  public class NullnessChecker extends BaseTypeChecker { }
\end{Verbatim}

(The \refqualclass{framework/source}{SupportedLintOptions} annotation is
optional, and many checker classes do not have one.)

The checker class bridges between the compiler and the rest of the checker.  It
invokes the type-rule check visitor on every Java source file being
compiled, and provides a simple API,
\refmethod{framework/source}{SourceChecker}{report}
{-org.checkerframework.framework.source.Result-java.lang.Object-}, to issue
errors using the compiler error reporting mechanism.

Also, the checker class follows the factory method pattern to
construct the concrete classes (e.g., visitor, factory) and annotation
hierarchy representation.  It is a convention that, for
a type system named Foo, the compiler
interface (checker), the visitor, and the annotated type factory are
named as \<FooChecker>, \<FooVisitor>, and \<FooAnnotatedTypeFactory>.
\refclass{common/basetype}{BaseTypeChecker} uses the convention to
reflectively construct the components.  Otherwise, the checker writer
must specify the component classes for construction.

\begin{sloppypar}
A checker can customize the default error messages through a
\sunjavadoc{java/util/Properties.html}{Properties}-loadable text file named
\<messages.properties> that appears in the same directory as the checker class.
The property file keys are the strings passed to \refmethodterse{framework/source}{SourceChecker}{report}{-org.checkerframework.framework.source.Result-java.lang.Object-}
(like \code{type.incompatible}) and the values are the strings to be
printed (\code{"cannot assign ..."}).
The \<messages.properties> file only need to mention the new messages that
the checker defines.
It is also allowed to override messages defined in superclasses, but this
is rarely needed.
For more details about message keys, see Section~\refwithpageparen{compiler-message-keys}.
\end{sloppypar}

\subsection{Indicating supported annotations\label{creating-indicating-supported-annotations}}

A checker must indicate the annotations that it supports (make up its type
hierarchy), including whether it supports the polymorphic qualifier
\refqualclass{framework/qual}{PolyAll}.

By default, a checker supports \<PolyAll>, and all annotations located in a
subdirectory called \<qual> that's located in the same directory as the checker.
Note that only annotations defined with the \<@Target(ElementType.TYPE\_USE)>
meta-annotation (and optionally with the additional value of
\<ElementType.TYPE\_PARAMETER>, but no other \<ElementType> values)
are automatically considered as supported annotations.

To indicate support for annotations that are located outside of the \<qual>
subdirectory, annotations that have other \<ElementType> values, or to indicate
whether a checker supports the polymorphic qualifier
\refqualclass{framework/qual}{PolyAll}, checker writers can override the
\refmethodterse{framework/type}{AnnotatedTypeFactory}{createSupportedTypeQualifiers}{--}
method (open the link for details).

An aggregate checker (which extends
\refclass{framework/source}{AggregateChecker}) does not need to specify its
type qualifiers, but each of its component checkers should do so.




\subsection{Bundling multiple checkers\label{creating-bundling-multiple-checkers}}

Sometimes, multiple checkers work together and should always be run
together.  There are two different ways to bundle multiple checkers
together, by creating either an ``aggregate checker'' or a ``compound checker''.


\begin{enumerate}
\item
An aggregate checker runs multiple independent, unrelated checkers.  There
is no communication or cooperation among them.

The effect is the same as if a user passes
multiple processors to the \<-processor> command-line option.

For example, instead of a user having to run

\begin{Verbatim}
  javac -processor DistanceUnitChecker,VelocityUnitChecker,MassUnitChecker ... files ...
\end{Verbatim}

\noindent
the user can write

\begin{Verbatim}
  javac -processor MyUnitCheckers ... files ...
\end{Verbatim}

\noindent
if you define an aggregate checker class.  Extend \refclass{framework/source}{AggregateChecker} and override
the \<getSupportedTypeCheckers> method, like the following:

\begin{Verbatim}
  public class MyUnitCheckers extends AggregateChecker {
    protected Collection<Class<? extends SourceChecker>> getSupportedCheckers() {
      return Arrays.asList(DistanceUnitChecker.class,
                           VelocityUnitChecker.class,
                           MassUnitChecker.class);
    }
  }
\end{Verbatim}

% This is the *only* example, as of July 2015.
An example of an aggregate checker is \refclass{checker/i18n}{I18nChecker}
(see \chapterpageref{i18n-checker}), which consists of
\refclass{checker/i18n}{I18nSubchecker} and
\refclass{checker/i18n}{LocalizableKeyChecker}.

\item
Use a compound checker to express dependencies among checkers.  Suppose it
only makes sense to run MyChecker if MyHelperChecker has already been run;
that might be the case if MyHelperChecker computes some information that
MyChecker needs to use.

Override
\<MyChecker.\refmethodterse{common/basetype}{BaseTypeChecker}{getImmediateSubcheckerClasses}{--}>
to return a list of the checkers that MyChecker depends on.  Every one of
them will be run before MyChecker is run.  One of MyChecker's subcheckers
may itself be a compound checker, and multiple checkers may declare a
dependence on the same subchecker.  The Checker Framework will run each
checker once, and in an order consistent with all the dependences.

A checker obtains information from its subcheckers (those that ran before
it) by querying their \refclass{framework/type}{AnnotatedTypeFactory} to
determine the types of variables.  Obtain the \<AnnotatedTypeFactory> by
calling
\refmethodterse{common/basetype}{BaseTypeChecker}{getTypeFactoryOfSubchecker}{-java.lang.Class-}.

\end{enumerate}



\subsection{Providing command-line options\label{creating-providing-command-line-options}}

A checker can provide two kinds of command-line options:
boolean flags and
named string values (the standard annotation processor
options).

\subsubsection{Boolean flags\label{creating-providing-command-line-options-boolean-flags}}

To specify a simple boolean flag, add:

\begin{alltt}
  \refqualclass{framework/source}{SupportedLintOptions}(\{"myflag"\})
\end{alltt}

to your checker subclass.
The value of the flag can be queried using

\begin{Verbatim}
  checker.getLintOption("myflag", false)
\end{Verbatim}

The second argument sets the default value that should be returned.

To pass a flag on the command line, call javac as follows:

\begin{Verbatim}
  javac -processor MyChecker -Alint=myflag
\end{Verbatim}


\subsubsection{Named string values\label{creating-providing-command-line-options-named-string-values}}

For more complicated options, one can use the standard annotation
processing \code{@SupportedOptions} annotation on the checker, as in:

\begin{alltt}
  \refqualclass{framework/source}{SupportedOptions}(\{"myoption"\})
\end{alltt}

The value of the option can be queried using

\begin{Verbatim}
  checker.getOption("myoption")
\end{Verbatim}

To pass an option on the command line, call javac as follows:

\begin{Verbatim}
  javac -processor MyChecker -Amyoption=p1,p2
\end{Verbatim}

The value is returned as a single string and you have to perform the
required parsing of the option.


% TODO: describe -ANullnessChecker_option=value mechanism.


\section{Annotated JDK\label{creating-a-checker-annotated-jdk}}

You will need to supply annotations for relevant parts of the JDK;
otherwise, your type-checker may produce spurious warnings for code that
uses the JDK.  As described in Section~\ref{annotated-libraries-creating},
there are two general ways to supply an annotated library:  in Java files
that will be compiled to \<.class> files, or as stub files (partial Java source files).

It's easier to start out with stub files.  If you need to annotate many
classes (say, more than 20 or so), then you should create an annotated JDK.

To supply an annotated JDK that will be compiled, see Section~\ref{annotating-jdk}.

To supply an annotated JDK as a stub file, create a file \<jdk.astub> in
the checker's main source directory.  It will be automatically used by the
checker.
You can also supply \<.astub> files in that directory for other libraries.
You should list them in a
\refqualclass{framework/qual}{StubFiles} annotation on the checker's main
class, so that they will also be automatically used.


\section{Testing framework\label{creating-testing-framework}}

\begin{sloppypar}
The Checker Framework provides a convenient way to write tests for your
checker.
% Don't repeat the information here, to prevent them from getting out of sync.
It is extensively documented in file \<checker-framework/checker/tests/README>.
\end{sloppypar}

If your checker's source code is within a fork of the Checker Framework
repository, then you can copy the testing infrastructure used by some
existing type system.  Here are some of the tasks you will perform:

\begin{itemize}
\item Add a \code{\emph{mychecker}-tests} build target in file
  \<checker-framework/checker/build.xml> and ensure all tests pass.

\item Make sure \code{all-tests} tests the new checker.
\end{itemize}


\section{Debugging options\label{creating-debugging-options}}

The Checker Framework provides debugging options that can be helpful when
writing a checker. These are provided via the standard \code{javac} ``\code{-A}''
switch, which is used to pass options to an annotation processor.


\subsection{Amount of detail in messages\label{creating-debugging-options-detail}}

\begin{itemize}
\item \code{-AprintAllQualifiers}: print all type qualifiers, including
qualifiers like \code{@Unqualified} which are usually not shown.
(Use the \code{@InvisibleQualifier} meta-annotation on a qualifier to hide it.)

\item \code{-AprintVerboseGenerics}: print more information about type
  parameters and wildcards when they appear in warning messages.  This
  produces a subset of the information than \code{-AprintAllQualifiers} does.

\item \code{-Adetailedmsgtext}: Output error/warning messages in a
  stylized format that is easy for tools to parse.  This is useful for
  tools that run the Checker Framework and parse its output, such as IDE
  plugins.  See the source code of \<SourceChecker.java> for details about
  the format.

\item \code{-AprintErrorStack}: print a stack trace whenever an
internal Checker Framework error occurs.

\item \code{-Anomsgtext}: use message keys (such as ``\code{type.invalid}'')
rather than full message text when reporting errors or warnings.  This is
used by the Checker Framework's own tests, so they do not need to be
changed if the English message is updated.

\end{itemize}

\subsection{Stub and JDK libraries\label{creating-debugging-options-libraries}}

\begin{itemize}

\item \code{-Aignorejdkastub}:
  ignore the \<jdk.astub> file in the checker directory. Files passed
  through the \code{-Astubs} option are still processed. This is useful
  when experimenting with an alternative stub file.

\item \code{-Anocheckjdk}:
  don't issue an error if no annotated JDK can be found.

\item \code{-AstubDebug}:
  Print debugging messages while processing stub files.

\end{itemize}

\subsection{Progress tracing\label{creating-debugging-options-progress}}

\begin{itemize}

\item \code{-Afilenames}: print the name of each file before type-checking it.

\item \code{-Ashowchecks}: print debugging information for each
pseudo-assignment check (as performed by
\refclass{common/basetype}{BaseTypeVisitor}; see
Section~\ref{creating-extending-visitor}).

\item \code{-AshowInferenceSteps}: print debugging information
about intermediate steps in method type argument inference
(as performed by \refclass{framework/util/typeinference}{DefaultTypeArgumentInference}).

\end{itemize}

\subsection{Saving the command-line arguments to a file\label{creating-debugging-options-output-args}}

\begin{itemize}

\item \code{-AoutputArgsToFile}:
  This saves the final command-line parameters as passed to the compiler in a file.
  This file can be used as a script (if the file is marked as executable on Unix, or
  if it includes a \code{.bat} extension on Windows) to re-execute the same compilation command.
  Note that this argument cannot be included in a file containing command-line arguments
  passed to the compiler using the @argfile syntax.

  Example usage: \code{-AoutputArgsToFile=\$HOME/scriptfile}

\end{itemize}

\subsection{Visualizing the dataflow graph\label{creating-debugging-dataflow-graph}}

\begin{itemize}

\item \code{-Aflowdotdir=\emph{somedir}}:
  Specify directory for \<.dot> files visualizing the CFG\@.
  Shorthand for\\
  \<-Acfgviz=org.checkerframework.dataflow.cfg.DOTCFGVisualizer,outdir=\emph{somedir}>.
  % TODO: create the directory if it doesn't exist.
  The directory must already exist.

\item \code{-Averbosecfg}:
  Enable additional output in the CFG visualization.
  Equivalent to passing \<verbose> to \<cfgviz>, e.g. as in
  \<-Acfgviz=MyVisualizer,verbose>

\item \code{-Acfgviz=\emph{VizClassName}[,\emph{opts},...]}:
  Mechanism to visualize the control flow graph (CFG) of
  all the methods and code fragments
  analyzed by the dataflow analysis (Section~\ref{creating-dataflow}).
  The graph also contains information about flow-sensitively refined
  types of various expressions at many program points.

  The argument is a comma-separated sequence of values or key-value pairs.
  The first argument is the fully-qualified name of the
  \<org.checkerframework.dataflow.cfg.CFGVisualizer> implementation
  that should be used. The remaining values or key-value pairs are
  passed to \<CFGVisualizer.init>.

\end{itemize}

You can visualize \<.dot> graph files with the \ahref{http://www.graphviz.org}{Graphviz} program.  For
example, to convert a \<.dot> file to PDF:

\begin{Verbatim}
dot -Tpdf -o myfile.pdf myfile.dot
\end{Verbatim}

\subsection{Miscellaneous debugging options\label{creating-debugging-options-misc}}

\begin{itemize}

\item \code{-AresourceStats}:
  Whether to output resource statistics at JVM shutdown.

\end{itemize}


\subsection{Examples\label{creating-debugging-options-examples}}

The following example demonstrates how these options are used:

%BEGIN LATEX
\begin{smaller}
%END LATEX
\begin{Verbatim}
$ javac -processor org.checkerframework.checker.interning.InterningChecker \
    docs/examples/InternedExampleWithWarnings.java -Ashowchecks -Anomsgtext -Afilenames

[InterningChecker] InterningExampleWithWarnings.java
 success (line  18): STRING_LITERAL "foo"
     actual: DECLARED @org.checkerframework.checker.interning.qual.Interned java.lang.String
   expected: DECLARED @org.checkerframework.checker.interning.qual.Interned java.lang.String
 success (line  19): NEW_CLASS new String("bar")
     actual: DECLARED java.lang.String
   expected: DECLARED java.lang.String
docs/examples/InterningExampleWithWarnings.java:21: (not.interned)
    if (foo == bar)
            ^
 success (line  22): STRING_LITERAL "foo == bar"
     actual: DECLARED @org.checkerframework.checker.interning.qual.Interned java.lang.String
   expected: DECLARED java.lang.String
1 error
\end{Verbatim}
%BEGIN LATEX
\end{smaller}
%END LATEX

\subsection{Using an external debugger\label{creating-debugging-options-external}}

\begin{sloppypar}
You can use any standard debugger to observe the execution of your checker.
Set the execution main class to \code{com.sun.tools.javac.Main}, and insert
the Checker Framework javac.jar (resides in
\code{\$CHECKERFRAMEWORK/checker/dist/javac.jar}).  If using an IDE, it is
recommended that you add \code{.../jsr308-langtools} as a project, so you
can step into its source code if needed.
\end{sloppypar}

You can also set up remote (or local) debugging using the following command as a template:

\begin{Verbatim}
java -jar "$CHECKERFRAMEWORK/checker/dist/checker.jar" \
    -J-Xdebug -J-Xrunjdwp:transport=dt_socket,server=y,suspend=y,address=5005 \
    -processor org.checkerframework.checker.nullness.NullnessChecker \
    src/sandbox/FileToCheck.java

\end{Verbatim}

% TODO: show example -AprintErrorStack usage. Update text above to
% refer to it.

% $ javac -processor org.checkerframework.checker.fenum.FenumChecker IdentityArrayList.java
% error: GraphQualifierHierarchy found an unqualified type.  Please ensure that your implicit rules cover all cases and/or use a @DefaulQualifierInHierarchy annotation.
% 1 error

% $ javac -processor org.checkerframework.checker.fenum.FenumChecker -AprintErrorStack IdentityArrayList.java
%% error: GraphQualifierHierarchy found an unqualified type.  Please ensure that your implicit rules cover all cases and/or use a @DefaulQualifierInHierarchy annotation.
%%   checkers.util.GraphQualifierHierarchy.checkAnnoInGraph(GraphQualifierHierarchy.java:253)
%%   checkers.util.GraphQualifierHierarchy.isSubtype(GraphQualifierHierarchy.java:243)
%%   checkers.fenum.FenumChecker$FenumQualifierHierarchy.isSubtype(FenumChecker.java:129)
%%   checkers.types.QualifierHierarchy.isSubtype(QualifierHierarchy.java:78)
%%   checkers.types.TypeHierarchy.isSubtypeImpl(TypeHierarchy.java:122)
%%   checkers.types.TypeHierarchy.isSubtype(TypeHierarchy.java:67)
%%   checkers.basetype.BaseTypeChecker.isSubtype(BaseTypeChecker.java:323)
%%   checkers.basetype.BaseTypeVisitor.commonAssignmentCheck(BaseTypeVisitor.java:608)
%%   checkers.basetype.BaseTypeVisitor.checkTypeArguments(BaseTypeVisitor.java:680)
%%   checkers.basetype.BaseTypeVisitor.visitMethodInvocation(BaseTypeVisitor.java:299)
%%   checkers.basetype.BaseTypeVisitor.visitMethodInvocation(BaseTypeVisitor.java:1)
%%   com.sun.tools.javac.tree.JCTree$JCMethodInvocation.accept(JCTree.java:1351)
%%   com.sun.source.util.TreePathScanner.scan(TreePathScanner.java:67)
%%   checkers.basetype.BaseTypeVisitor.scan(BaseTypeVisitor.java:122)
%%   checkers.basetype.BaseTypeVisitor.scan(BaseTypeVisitor.java:1)
%%   com.sun.source.util.TreeScanner.visitExpressionStatement(TreeScanner.java:241)
%%   com.sun.tools.javac.tree.JCTree$JCExpressionStatement.accept(JCTree.java:1176)
%%   com.sun.source.util.TreePathScanner.scan(TreePathScanner.java:67)
%%   checkers.basetype.BaseTypeVisitor.scan(BaseTypeVisitor.java:122)
%%   checkers.basetype.BaseTypeVisitor.scan(BaseTypeVisitor.java:1)
%%   com.sun.source.util.TreeScanner.scan(TreeScanner.java:90)
%%   com.sun.source.util.TreeScanner.visitBlock(TreeScanner.java:160)
%%   com.sun.tools.javac.tree.JCTree$JCBlock.accept(JCTree.java:793)
%%   com.sun.source.util.TreePathScanner.scan(TreePathScanner.java:67)
%%   checkers.basetype.BaseTypeVisitor.scan(BaseTypeVisitor.java:122)
%%   checkers.basetype.BaseTypeVisitor.scan(BaseTypeVisitor.java:1)
%%   com.sun.source.util.TreeScanner.scanAndReduce(TreeScanner.java:80)
%%   com.sun.source.util.TreeScanner.visitMethod(TreeScanner.java:143)
%%   checkers.basetype.BaseTypeVisitor.visitMethod(BaseTypeVisitor.java:218)
%%   checkers.basetype.BaseTypeVisitor.visitMethod(BaseTypeVisitor.java:1)
%%   com.sun.tools.javac.tree.JCTree$JCMethodDecl.accept(JCTree.java:693)
%%   com.sun.source.util.TreePathScanner.scan(TreePathScanner.java:67)
%%   checkers.basetype.BaseTypeVisitor.scan(BaseTypeVisitor.java:122)
%%   checkers.basetype.BaseTypeVisitor.scan(BaseTypeVisitor.java:1)
%%   com.sun.source.util.TreeScanner.scanAndReduce(TreeScanner.java:80)
%%   com.sun.source.util.TreeScanner.scan(TreeScanner.java:90)
%%   com.sun.source.util.TreeScanner.scanAndReduce(TreeScanner.java:98)
%%   com.sun.source.util.TreeScanner.visitClass(TreeScanner.java:132)
%%   checkers.basetype.BaseTypeVisitor.visitClass(BaseTypeVisitor.java:158)
%%   checkers.basetype.BaseTypeVisitor.visitClass(BaseTypeVisitor.java:1)
%%   com.sun.tools.javac.tree.JCTree$JCClassDecl.accept(JCTree.java:617)
%%   com.sun.source.util.TreePathScanner.scan(TreePathScanner.java:49)
%%   checkers.source.SourceChecker.typeProcess(SourceChecker.java:337)
%%   com.sun.source.util.AbstractTypeProcessor$AttributionTaskListener.finished(AbstractTypeProcessor.java:211)
%%   com.sun.tools.javac.main.JavaCompiler.flow(JavaCompiler.java:1272)
%%   com.sun.tools.javac.main.JavaCompiler.flow(JavaCompiler.java:1231)
%%   com.sun.tools.javac.main.JavaCompiler.compile2(JavaCompiler.java:885)
%%   com.sun.tools.javac.main.JavaCompiler.compile(JavaCompiler.java:844)
%%   com.sun.tools.javac.main.Main.compile(Main.java:419)
%%   com.sun.tools.javac.main.Main.compile(Main.java:333)
%%   com.sun.tools.javac.main.Main.compile(Main.java:324)
%%   com.sun.tools.javac.Main.compile(Main.java:76)
%%   com.sun.tools.javac.Main.main(Main.java:61)
%% 1 error



\section{Documenting the checker\label{creating-documenting-a-checker}}

This section describes how to write a chapter for this manual that
describes a new type-checker.  This is a prerequisite to having your
type-checker distributed with the Checker Framework, which is the best way
for users to find it and for it to be kept up to date with Checker
Framework changes.  Even if you do not want your checker distributed with
the Checker Framework, these guidelines may help you write better
documentation.

When writing a chapter about a new type-checker, see the existing chapters
for inspiration.  (But recognize that the existing chapters aren't perfect:
maybe they can be improved too.)

A chapter in the Checker Framework manual should generally have the
following sections:

\begin{description}
\item[Chapter: Belly Rub Checker]
  The text before the first section in the chapter should state the
  guarantee that the checker provides and why it is important.  It should
  give an overview of the concepts.  It should state how to run the checker.
\item[Section: Belly Rub Annotations]
  This section includes descriptions of the annotations with links to the
  Javadoc.  Separate type annotations from declaration annotations, and put
  any type annotations that a programmer may not write (they are only used
  internally by the implementation) last within variety of annotation.

  Draw a diagram of the type hierarchy.  A textual description of
  the hierarchy is not sufficient; the diagram really helps readers to
  understand the system.
  The diagram will appear in directory \<docs/manual/figures/>;
  see its \<README> file for tips.

  The Javadoc for the annotations deserves the same care as the manual
  chapter.  Each annotation's Javadoc comment should use the
  \<@checker\_framework.manual> Javadoc taglet to refer to the chapter that
  describes the checker; see \refclass{javacutil/dist}{ManualTaglet}.
\item[Section: What the Belly Rub Checker checks]
  This section gives more details about when an error is issued, with examples.
  This section may be omitted if the checker does not contain special
  type-checking rules --- that is, if the checker only enforces the usual
  Java subtyping rules.
\item[Section: Examples]
  Code examples.
\end{description}

Sometimes you can omit some of the above sections.  Sometimes there are
additional sections, such as tips on suppressing warnings, comparisons to
other tools, and run-time support.

You will create a new \<belly-rub-checker.tex> file,
then \verb|\input| it at a logical place in \<manual.tex> (not
necessarily as the last checker-related chapter).  Also add two references
to the checker's chapter:  one at the beginning of
chapter~\ref{introduction}, and identical text in
Section~\ref{type-refinement-runtime-tests} (both of these lists appear in
the same order as the manual chapters, to help us notice if anything is
missing).

Every chapter and (sub)*section should have a label defined \emph{within} the
\verb|\section| command.  Section labels should start with the checker
name (as in \verb|\label{bellyrub-examples}|) and not with ``\<sec:>''.
These conventions are for the benefit of the Hevea program that produces
the HTML version of the manual.

Don't forget to write Javadoc for any annotations that the checker uses.
That is part of the documentation and is the first thing that many users
may see.  Also ensure that the Javadoc links back to the manual, using the
\<@checker\_framework.manual> custom Javadoc tag.

There are several other miscellaneous tasks:
\begin{itemize}
\item
  Add its \<messages.properties> file to the \<check-compilermsgs> target in
  file \<checker/build.xml>.
\item
  Integrate your new checker with the Eclipse plugin.
  See \<\$CHECKERFRAMEWORK/eclipse/README-developers.html\#add-checker>/ for instructions.
\end{itemize}


\section{javac implementation survival guide\label{creating-javac-tips}}

Since this section of the manual was written, the useful ``The Hitchhiker's
Guide to javac'' has become available at
\url{http://openjdk.java.net/groups/compiler/doc/hhgtjavac/index.html}.
See it first, and then refer to this section.  (This section of the manual
should be revised, or parts eliminated, in light of that document.)


A checker built using the Checker Framework makes use of a few interfaces
from the underlying compiler (Oracle's OpenJDK javac).
This section describes those interfaces.




\subsection{Checker access to compiler information\label{creating-compiler-information}}

The compiler uses and exposes three hierarchies to model the Java
source code and classfiles.


\subsubsection{Types --- Java Language Model API\label{creating-javac-types}}

A \refModelclass{type}{TypeMirror} represents a Java type.
% Java declaration, statement, or expression.

\begin{sloppypar}
There is a \code{TypeMirror} interface to represent each type kind,
e.g., \code{PrimitiveType} for primitive types, \code{ExecutableType}
for method types, and \code{NullType} for the type of the \code{null} literal.
\end{sloppypar}

\code{TypeMirror} does not represent annotated types though.  A checker
should use the Checker Framework types API,
\refclass{framework/type}{AnnotatedTypeMirror}, instead.  \code{AnnotatedTypeMirror}
parallels the \code{TypeMirror} API, but also present the type annotations
associated with the type.

The Checker Framework and the checkers use the types API extensively.


\subsubsection{Elements --- Java Language Model API\label{creating-javac-elements}}

An \refModelclass{element}{Element} represents a potentially-public
declaration that can be accessed from elsewhere:  classes, interfaces, methods, constructors, and
fields.  \<Element> represents elements found in both source
code and bytecode.

There is an \code{Element} interface to represent each construct, e.g.,
\code{TypeElement} for class/interfaces, \code{ExecutableElement} for
methods/constructors, \code{VariableElement} for local variables and
method parameters.

If you need to operate on the declaration level, always use elements rather
than trees
% in same subsection, which is the limit of the numbering.
% (Section~\ref{javac-trees})
(see below).  This allows the code to work on
both source and bytecode elements.

Example: retrieve declaration annotations, check variable
modifiers (e.g., \code{strictfp}, \code{synchronized})


\subsubsection{Trees --- Compiler Tree API\label{creating-javac-trees}}

A \refTreeclass{tree}{Tree} represents a syntactic unit in the source code,
like a method declaration, statement, block, \<for> loop, etc. Trees only
represent source code to be compiled (or found in \code{-sourcepath});
no tree is available for classes read from bytecode.

There is a Tree interface for each Java source structure, e.g.,
\code{ClassTree} for class declaration, \code{MethodInvocationTree}
for a method invocation, and \code{ForEachTree} for an enhanced-for-loop
statement.

You should limit your use of trees. A checker uses Trees mainly to
traverse the source code and retrieve the types/elements corresponding to
them.  Then, the checker performs any needed checks on the types/elements instead.


\subsubsection{Using the APIs\label{creating-using-the-apis}}

The three APIs use some common idioms and conventions; knowing them will
help you to create your checker.

\emph{Type-checking}:
Do not use \code{instanceof} to determine the class of the object,
because you cannot necessarily predict the run-time type of the object that
implements an interface.  Instead, use the \code{getKind()} method.  The
method returns \refModelclass{type}{TypeKind},
\refModelclass{element}{ElementKind}, and \refTreeclass{tree}{Tree.Kind}
for the three interfaces, respectively.

\emph{Visitors and Scanners}:
The compiler and the Checker Framework use the visitor pattern
extensively. For example, visitors are used to traverse the source tree
(\refclass{common/basetype}{BaseTypeVisitor} extends
\refTreeclass{util}{TreePathScanner}) and for type
checking (\refclass{framework/type/treeannotator}{TreeAnnotator} implements
\refTreeclass{tree}{TreeVisitor}).

\emph{Utility classes}:
Some useful methods appear in a utility class.  The Oracle convention is that
the utility class for a \code{Foo} hierarchy is \code{Foos} (e.g.,
\refModelclass{util}{Types}, \refModelclass{util}{Elements}, and
\refTreeclass{util}{Trees}).  The Checker Framework uses a common
\code{Utils} suffix instead (e.g., \refclass{javacutil}{TypesUtils},
\refclass{javacutil}{TreeUtils}, \refclass{javacutil}{ElementUtils}), with one
notable exception: \refclass{framework/util}{AnnotatedTypes}.


\subsection{How a checker fits in the compiler as an annotation processor\label{creating-checker-as-annotation-processor}}

The Checker Framework builds on the Annotation Processing API
introduced in Java 6.  A type annotation processor is one that extends
\refclass{javacutil}{AbstractTypeProcessor}; these get run on each class
source file after the compiler confirms that the class is valid Java code.

The most important methods of \refclass{javacutil}{AbstractTypeProcessor}
are \code{typeProcess} and \code{getSupportedSourceVersion}. The former
class is where you would insert any sort of method call to walk the AST\@,
and the latter just returns a constant indicating that we are targeting
version 8 of the compiler. Implementing these two methods should be enough
for a basic plugin; see the Javadoc for the class for other methods that
you may find useful later on.

The Checker Framework uses Oracle's Tree API to access a program's AST\@.
The Tree API is specific to the Oracle OpenJDK, so the Checker Framework only
works with the OpenJDK javac, not with Eclipse's compiler ecj.
This also limits the tightness of
the integration of the Checker Framework into other IDEs such as \href{http://www.jetbrains.com/idea/}{IntelliJ IDEA}\@.
An implementation-neutral API would be preferable.
In the future, the Checker Framework
can be migrated to use the Java Model AST of JSR 198 (Extension API for
Integrated Development Environments)~\cite{JSR198}, which gives access to
the source code of a method.  But, at present no tools
implement JSR~198.  Also see Section~\ref{creating-ast-traversal}.



\subsubsection{Learning more about javac\label{creating-learning-more-about-javac}}

Sun's javac compiler interfaces can be daunting to a
newcomer, and its documentation is a bit sparse. The Checker Framework
aims to abstract a lot of these complexities.
You do not have to understand the implementation of javac to
build powerful and useful checkers.
Beyond this document,
other useful resources include the Java Infrastructure
Developer's guide at
\url{http://wiki.netbeans.org/Java_DevelopersGuide} and the compiler
mailing list archives at
\url{http://mail.openjdk.java.net/pipermail/compiler-dev/}
(subscribe at
\url{http://mail.openjdk.java.net/mailman/listinfo/compiler-dev}).


\section{Integrating a checker with the Checker Framework\label{creating-integrating-a-checker}}

% First version of how to integrate a new checker into the release.
% TODO: what steps are missing?

To integrate a new checker with the Checker Framework release, perform
the following:

\begin{itemize}

\item Create Ant targets for testing, as described in
  Section~\ref{creating-testing-framework}.

\item Extend the \code{check-compilermsgs} target to include the
compiler messages property file of the new checker in
the \code{checker-args} list.  (Keep the list in alphabetical order.)

\item Make sure \code{check-compilermsgs} and \code{check-purity} run
without warnings or errors.

\end{itemize}


% LocalWords:  plugin javac's SourceChecker AbstractProcessor getMessages quals
% LocalWords:  getSourceVisitor SourceVisitor getFactory AnnotatedTypeFactory
% LocalWords:  SupportedAnnotationTypes SupportedSourceVersion TreePathScanner
% LocalWords:  TreeScanner visitAssignment AssignmentTree AnnotatedClassTypes
% LocalWords:  SubtypeChecker SubtypeVisitor NonNull isSubtype getClass nonnull
% LocalWords:  AnnotatedClassType isAnnotatedWith hasAnnotationAt TODO src jdk
% LocalWords:  processor NullnessChecker InterningChecker Nullness Nullable
% LocalWords:  AnnotatedTypeMirrors BaseTypeChecker BaseTypeVisitor basetype
% LocalWords:  Aqual Anqual java CharSequence getAnnotatedType UseLovely
% LocalWords:  AnnotatedTypeMirror LovelyChecker Anomsgtext Ashowchecks enums
% LocalWords:  Afilenames dereferenced SuppressWarnings declaratively SubtypeOf
% LocalWords:  TypeHierarchy GraphQualifierHierarchy Foo qual UnknownSign
% LocalWords:  QualifierHierarchy QualifierRoot createQualifierHierarchy util
% LocalWords:  createTypeHierarchy ImplicitFor treeClasses TypeMirror Anno
% LocalWords:  LiteralTree ExpressionTree typeClasses addComputedTypeAnnotations nullable
% LocalWords:  createSupportedTypeQualifiers FooChecker nullness PolyAll
% LocalWords:  FooVisitor FooAnnotatedTypeFactory basicstyle InterningVisitor
% LocalWords:  InterningAnnotatedTypeFactory QualifierDefaults TypeKind getKind
% LocalWords:  setAbsoluteDefaults PolymorphicQualifier TreeVisitor subnodes
% LocalWords:  SimpleTreeVisitor TreePath instanceof subinterfaces TypeElement
% LocalWords:  ExecutableElement PackageElement DeclaredType VariableElement
% LocalWords:  TypeParameterElement ElementVisitor javax getElementUtils NoType
% LocalWords:  ProcessingEnvironment ExecutableType MethodTree ArrayType Warski
% LocalWords:  MethodInvocationTree PrimitiveType BlockTree TypeVisitor blog
% LocalWords:  AnnotatedTypeVisitor SimpleAnnotatedTypeVisitor html langtools
% LocalWords:  AnnotatedTypeScanner bootclasspath asType stringPatterns Foos
% LocalWords:  DefaultQualifierInHierarchy invocable wildcards novariant Utils
% LocalWords:  AggregateChecker getSupportedTypeCheckers Uninterned sourcepath
% LocalWords:  DefaultQualifier bytecode NullType strictfp ClassTree TypesUtils
% LocalWords:  ForEachTree ElementKind TreeAnnotator TreeUtils ElementUtils ecj
% LocalWords:  AnnotatedTypes AbstractTypeProcessor gcj hardcoding jsr api
% LocalWords:  typeProcess getSupportedSourceVersion fenum classpath astub
%%  LocalWords:  addAbsoluteDefault BaseAnnotatedTypeFactory superclasses
%%  LocalWords:  SupportedOptions AprintAllQualifiers InvisibleQualifier
%%  LocalWords:  Adetailedmsgtext AprintErrorStack Aignorejdkastub Astubs
%%  LocalWords:  Anocheckjdk AstubDebug Aflowdotdir AresourceStats Regex
%%  LocalWords:  classfiles CHECKERFRAMEWORK RegexUtil asRegex myString
%%  LocalWords:  myInt CFAbstractTransfer RegexTransfer CFAbstractAnalysis
%%  LocalWords:  createTransferFunction RegexAnalysis createFlowAnalysis
%%  LocalWords:  EqualToNode LeftShiftNode VariableDeclarationNode myvar
%%  LocalWords:  MethodInvocationNode visitMethodInvocation TransferResult
%%  LocalWords:  RegexUtils LockTransfer FormatterTransfer CFValue argfile
%%  LocalWords:  RegexTransfer's newResultValue subcheckers taglet tex XXX
%%  LocalWords:  ParameterizedCheckerTest AoutputArgsToFile ManualTaglet
%%  LocalWords:  Hevea Hitchhiker's compilermsgs args Poly MyTypeSystem
%%  LocalWords:  I18nChecker i18n I18nSubchecker LocalizableKeyChecker
%%  LocalWords:  MyChecker MyHelperChecker getImmediateSubcheckerClasses
%%  LocalWords:  MyChecker's subchecker plugins ElementType myClass myflag
%%  LocalWords:  CheckerFrameworkTest GenericAnnotatedTypeFactory MyClass
%%  LocalWords:  addCheckedCodeDefaults RelevantJavaTypes TargetLocations
%%  LocalWords:  TypeUseLocation createExpressionAnnoHelper internalReprOf
%%  LocalWords:  ExpressionAnnotationHelper FlowExpressions CFTransfer
%%  LocalWords:  AnnotationProvider FooTransfer createFlowTransferFunction
%%  LocalWords:  SupportedLintOptions myoption StubFiles scriptfile outdir
%%  LocalWords:  somedir Acfgviz Averbosecfg cfgviz MyVisualizer init
%%  LocalWords:  VizClassName CFGVisualizer MyProp MyPropChecker mypackage
%  LocalWords:  SourceFile NonNegative JavaExpression DependentTypesHelper
%  LocalWords:  createDependentTypesHelper boolean regex subclasses README
%  LocalWords:  formatter nChecker nSubchecker AprintVerboseGenerics
%  LocalWords:  AshowInferenceSteps DefaultTypeArgumentInference Graphviz
%  LocalWords:  javacutil LiteralKind EnsuresQualifier EnsuresQualifierIf
%%  LocalWords:  mychecker

\htmlhr
\chapter{Integration with external tools\label{external-tools}}

This chapter discusses how to run a checker from the command line, from a
build system, or from an IDE\@.  You can skip to the appropriate section:

% Keep this list up to date with the sections of this chapter and with TWO
% copies of the list in file introduction.tex .
\begin{itemize}
\item javac (Section~\ref{javac-installation})
\item Ant (Section~\ref{ant-task})
\item Maven (Section~\ref{maven})
\item Gradle (Section~\ref{gradle})
\item IntelliJ IDEA (Section~\ref{intellij})
\item Eclipse (Section~\ref{eclipse})
\item tIDE (Section~\ref{tide})
\end{itemize}

If your build system or IDE is not listed above, you should customize how
it runs the javac command on your behalf.  See your build system or IDE
documentation to learn how to
customize it, adapting the instructions for javac in Section~\ref{javac-installation}.
If you make another tool support running a checker, please
inform us via the
\href{http://groups.google.com/group/checker-framework-discuss}{mailing
  list} or
\href{http://code.google.com/p/checker-framework/issues/list}{issue tracker} so
we can add it to this manual.

This chapter also discusses type inference tools (see
Section~\ref{type-inference-tools}).

All examples in this chapter are in the public domain, with no copyright nor
licensing restrictions.


\section{Javac compiler\label{javac-installation}}

If you use the \code{javac} compiler from the command line, then you can
instead use the ``Checker Framework compiler'', a variant of the OpenJDK
\code{javac} that recognizes type annotations in comments and that includes
the Checker Framework jar files on its path.
The Checker Framework compiler is backward-compatible, so using it as your
Java compiler, even when you are not doing pluggable type-checking, has no
negative consequences.

You can use the Checker Framework compiler in three ways.  You can use any
one of them.  However, if you are using the Windows command shell, you must
use the last one.
% Is the last one required for Cygwin, as well as for the Windows command shell?


\begin{itemize}
  \item
    Option 1:
    Add directory
    \code{.../checker-framework-1.8.11/checker/bin} to your path, \emph{before} any other
    directory that contains a \<javac> executable.  Now, whenever
    you run \code{javac}, you will use the updated compiler.  If you are
    using the bash shell, a way to do this is to add the following to your
    \verb|~/.profile| (or alternately \verb|~/.bash_profle| or \verb|~/.bashrc|) file:
\begin{Verbatim}
  export CHECKERFRAMEWORK=${HOME}/checker-framework-1.8.11
  export PATH=${CHECKERFRAMEWORK}/checker/bin:${PATH}
\end{Verbatim}
    then log out and back in to ensure that the environment variable
    setting takes effect.
  \item
    Option 2:
    Whenever this document tells you to run \code{javac}, you
    can instead run \code{\$CHECKERFRAMEWORK/checker/bin/javac}.

    You can simplify this by introducing an alias.  Then,
    whenever this document tells you to run \code{javac}, instead use that
    alias.  Here is the syntax for your 
    \verb|~/.bashrc| file:
% No Windows example because this doesn't work under Windows.
\begin{Verbatim}
  export CHECKERFRAMEWORK=${HOME}/checker-framework-1.8.11
  alias javacheck='$CHECKERFRAMEWORK/checker/bin/javac'
\end{Verbatim}

   \item
   Option 3:
   Whenever this document tells you to run \code{javac}, instead
   run checker.jar via \<java> (not \<javac>) as in:

\begin{Verbatim}
  java -jar $CHECKERFRAMEWORK/checker/dist/checker.jar ...
\end{Verbatim}

    You can simplify the above command by introducing an alias.  Then,
    whenever this document tells you to run \code{javac}, instead use that
    alias.  For example:

\begin{Verbatim}
  # Unix
  export CHECKERFRAMEWORK=${HOME}/checker-framework-1.8.11
  alias javacheck='java -jar $CHECKERFRAMEWORK/checker/dist/checker.jar'

  # Windows
  set CHECKERFRAMEWORK = C:\Program Files\checker-framework-1.8.11\
  doskey javacheck=java -jar %CHECKERFRAMEWORK%\checker\dist\checker.jar $*
\end{Verbatim}

   (Explanation for advanced users:  More generally, anywhere that you would use \<javac.jar>, you can substitute
   \<\$CHECKERFRAMEWORK/checker/dist/checker.jar>; 
   the result is to use the Checker
   Framework compiler instead of the regular \<javac>.)

\end{itemize}


To ensure that you are using the Checker Framework compiler, run
\<javac -version> (possibly using the
full pathname to \<javac> or the alias, if you did not add the Checker
Framework \<javac> to your path).
The output should be:

\begin{Verbatim}
  javac 1.8.0-jsr308-1.8.11
\end{Verbatim}




%% Does this work?  Text elsewhere in the manual imples that it does not.
% \item
% \begin{sloppypar}
%   In order to use the updated compiler when you type \code{javac}, add the
%   directory \<C:\ttbs{}Program Files\ttbs{}checker-framework\ttbs{}checkers\ttbs{}binary> to the
%   beginning of your path variable.  Also set a \code{CHECKERFRAMEWORK} variable.
% \end{sloppypar}
% 
% % Instructions stolen from http://www.webreference.com/js/tips/020429.html
% 
% To set an environment variable, you have two options:  make the change
% temporarily or permanently.
% \begin{itemize}
% \item
% To make the change \textbf{temporarily}, type at the command shell prompt:
% 
% \begin{alltt}
% path = \emph{newdir};%PATH%
% \end{alltt}
% 
% For example:
% 
% \begin{Verbatim}
% set CHECKERFRAMEWORK = C:\Program Files\checker-framework
% path = %CHECKERFRAMEWORK%\checker\bin;%PATH%
% \end{Verbatim}
% 
% This is a temporary change that endures until the window is closed, and you
% must re-do it every time you start a new command shell.
% 
% \item
% To make the change \textbf{permanently},
% Right-click the \<My Computer> icon and
% select \<Properties>. Select the \<Advanced> tab and click the
% \<Environment Variables> button. You can set the variable as a ``System
% Variable'' (visible to all users) or as a ``User Variable'' (visible to
% just this user).  Both work; the instructions below show how to set as a
% ``System Variable''.
% In the \<System Variables> pane, select
% \<Path> from the list and click \<Edit>. In the \<Edit System Variable>
% dialog box, move the cursor to the beginning of the string in the
% \<Variable Value> field and type the full directory name (not using the
% \verb|%CHECKERFRAMEWORK%| environment variable) followed by a
% semicolon (\<;>).
% 
% % This is for the benefit of the Ant task.
% Similarly, set the \code{CHECKERFRAMEWORK} variable.
% 
% This is a permanent change that only needs to be done once ever.
% \end{itemize}



\section{Ant task\label{ant-task}}

If you use the \href{http://ant.apache.org/}{Ant} build tool to compile
your software, then you can add an Ant task that runs a checker.  We assume
that your Ant file already contains a compilation target that uses the
\code{javac} task.

\begin{enumerate}
\item
Set the \code{jsr308javac} property:

%BEGIN LATEX
\begin{smaller}
%END LATEX
\begin{Verbatim}
  <property environment="env"/>

  <property name="checkerframework" value="${env.CHECKERFRAMEWORK}" />
  
  <!-- On Mac/Linux, use the javac shell script; on Windows, use javac.bat -->
  <condition property="cfJavac" value="javac.bat" else="javac">
      <os family="windows" />
  </condition>

  <presetdef name="jsr308.javac">
    <javac fork="yes" executable="${checkerframework}/checker/bin/${cfJavac}" >
      <!-- JSR-308-related compiler arguments -->
      <compilerarg value="-version"/>
      <compilerarg value="-implicit:class"/>
    </javac>
  </presetdef>
\end{Verbatim}
%BEGIN LATEX
\end{smaller}
%END LATEX

\item \textbf{Duplicate} the compilation target, then \textbf{modify} it slightly as
indicated in this example:

%BEGIN LATEX
\begin{smaller}
%END LATEX
\begin{Verbatim}
  <target name="check-nullness"
          description="Check for null pointer dereferences"
          depends="clean,...">
    <!-- use jsr308.javac instead of javac -->
    <jsr308.javac ... >
      <compilerarg line="-processor org.checkerframework.checker.nullness.NullnessChecker"/>
      <!-- optional, to not check uses of library methods: <compilerarg value="-AskipUses=^(java\.awt\.|javax\.swing\.)"/> -->
      <compilerarg line="-Xmaxerrs 10000"/>
      ...
    </jsr308.javac>
  </target>
\end{Verbatim}
%BEGIN LATEX
\end{smaller}
%END LATEX

Fill in each ellipsis (\ldots) from the original compilation target.

In the example, the target is named \code{check-nullness}, but you can
name it whatever you like.
\end{enumerate}

\subsection{Explanation\label{ant-task-explanation}}

This section explains each part of the Ant task.

\begin{enumerate}
\item Definition of \code{jsr308.javac}:

The \code{fork} field of the \code{javac} task
ensures that an external javac program is called.  Otherwise, Ant will run
javac via a Java method call, and there is no guarantee that it will get
the Checker Framework compiler that is distributed with the Checker Framework.

The \code{-version} compiler argument is just for debugging; you may omit
it.

The \code{-implicit:class} compiler argument causes annotation processing
to be performed on implicitly compiled files.  (An implicitly compiled file
is one that was not specified on the command line, but for which the source
code is newer than the \code{.class} file.)  This is the default, but
supplying the argument explicitly suppresses a compiler warning.

%% -Awarns was removed above without removing it here.
% The \code{-Awarns} compiler argument is optional, and causes the checker to
% treat errors as warnings so that compilation does not fail even if
% pluggable type-checking fails; see Section~\ref{checker-options}.

\item The \code{check-nullness} target:

The target assumes the existence of a \code{clean} target that removes all
\code{.class} files.  That is necessary because Ant's \code{javac} target
doesn't re-compile \code{.java} files for which a \code{.class} file
already exists.

The \code{-processor ...} compiler argument indicates which checker to
run.  You can supply additional arguments to the checker as well.

\end{enumerate}


\section{Maven\label{maven}}

\textbf{Note: The Maven plugin
was deprecated in favor of the following approach.  The plugin is no
longer maintained and the final release was on June 2, 2014.}

If you use the \href{http://maven.apache.org/}{Maven} tool,
then you can specify pluggable type-checking as part of your build
process. This is done by pointing Maven to a script that
makes Maven use the Type Annotations compiler.

\begin{enumerate}

\item Declare a dependency on the type qualifier annotations.  Find the
  existing \code{<dependencies>} section and add a new
  \code{<dependencies>} item:

\begin{alltt}
    <dependencies>
        ... existing <dependency> items ...

        <!-- annotations from the Checker Framework: nullness, interning, locking, ... -->
        <dependency>
            <groupId>org.checkerframework</groupId>
            <artifactId>checker-qual</artifactId>
            <version>\ReleaseVersion{}</version>
        </dependency>

    </dependencies>
\end{alltt}

\item Direct the Maven compiler plugin to use the \code{javac\_maven} script. Find
the existing reference to the \code{maven-compiler-plugin} within the \code{<plugins>}
section or add it if necessary.

\begin{alltt}
      <plugin>
        <artifactId>maven-compiler-plugin</artifactId>
        <configuration>
          <source>1.8</source>
          <target>1.8</target>
          <fork>true</fork>
          <executable>(absolute path to checker framework installation)/checker/bin/javac_maven</executable>
        </configuration>
      </plugin>
\end{alltt}

On Windows, the \code{javac\_maven.bat} script is automatically used instead
of the \code{javac\_maven} script -- it is not necessary to include the \code{.bat}
extension in the absolute path above.

This script assumes that the Checker Framework is the only annotation processor being run.
If this is not the case, please modify the \code{javac\_maven} script accordingly.

\item Create a text file named \code{argfile} in the same directory as the \code{pom.xml}
and include in it the command-line parameters to pass to the Java compiler.

Example contents of \code{argfile}:

\begin{alltt}
-processor org.checkerframework.checker.nullness.NullnessChecker
-AsuppressWarnings=purity.invalid.overriding
-Alint
-AprintErrorStack
-Awarns
-Xmaxwarns 10000
\end{alltt}

Due to limitations of the \code{pom.xml} syntax, some command-line options are
difficult or impossible to pass via the \code{pom.xml}, hence the need for this file.

If you would like to call this file something other than \code{argfile},
please modify the \code{javac\_maven} script accordingly.

\subsection{Debugging the Maven compiler command-line arguments\label{debugging-maven-args}}

Maven will sometimes hide important Checker Framework and/or Java compiler debugging output.
If Maven is not producing the expected output when using it with the Type Annotations compiler
and Checker Framework, it is possible to output the compiler command-line arguments produced
by Maven to a file.  This file can then be used as a script to execute the compiler in the
same way Maven would have but without running Maven.  This is done through the
\<-AoutputArgsToFile> command-line parameter.  To use it with Maven, modify (or copy) the
\code{javac\_maven} script such that the last line in the script ends with:

\begin{verbatim}
[...] "-AoutputArgsToFile=<path to filename>" "$@"
\end{verbatim}

For \code{javac\_maven.bat}, modify (or copy) it such that the last line ends with:

\begin{verbatim}
[...] -AoutputArgsToFile=<path to filename> %*
\end{verbatim}

Please see Section~\ref{debugging-options-output-args} for more details on how to use
the resulting file.

\end{enumerate}

\section{Gradle\label{gradle}}

% This information came from:
% http://jira.codehaus.org/browse/GRADLE-342
% http://docs.codehaus.org/display/GRADLE/Gradle+0.8+Breaking+Changes

If you fork the compilation task, \href{http://gradle.org/}{Gradle}
lets you specify the executable to compile java programs.

To specify the appropriate executable, set
\code{options.fork = true} and
\code{compile.options.fork.executable = "\$CHECKERFRAMEWORK/checker/bin/javac"}

To specify command-line arguments, set
\code{compile.options.compilerArgs}.  Here is a possible example:

\begin{Verbatim}
allprojects {
  tasks.withType(JavaCompile).all { JavaCompile compile ->
    compile.options.debug = true
    compile.options.compilerArgs = [
      '-version',
      '-implicit:class',
      '-processor', 'org.checkerframework.checker.nullness.NullnessChecker'
    ]
    options.fork = true
    options.forkOptions.executable = "$CHECKERFRAMEWORK/checker/bin/javac"
  }
}
\end{Verbatim}

% extra $ to unconfuse Emacs's LaTeX mode


\section{IntelliJ IDEA\label{intellij}}

IntelliJ IDEA (Maia release)
\href{http://blogs.jetbrains.com/idea/2009/07/type-annotations-jsr-308-support/}{supports}
the Type Annotations (JSR-308) syntax.
See \url{http://blogs.jetbrains.com/idea/2009/07/type-annotations-jsr-308-support/}.

\section{Eclipse\label{eclipse}}

Eclipse supports type annotations.
Eclipse does not directly support running the Checker Framework, 
nor is Eclipse necessary for running the Checker Framework.

There are two ways to run a checker from within the Eclipse IDE:  via Ant
or using an Eclipse plug-in.  These two methods are described below.

No matter what method you choose, we suggest that
all Checker Framework annotations be written in the comments
if you are using a version of Eclipse that
does not support Java 8.  This will avoid many
text highlighting errors with versions of Eclipse that don't support Java 8
and type annotations.

Even in a version of Eclipse that supports Java 8's type annotations, you
still need to run the Checker Framework via Ant or via the plug-in, rather
than by supplying the \<-processor> command-line option to the ejc
compiler.  The reason is that the Checker Framework is built upon javac,
and ejc represents the Java program differently.  (If both javac and ejc
implemented JSR 198~\cite{JSR198}, then it would be possible to build
type-checking plug-ins that work with both compilers.)


\paragraph{Using an Ant task\label{eclipse-ant}}

Add an Ant target as described in Section~\ref{ant-task}.  You can
run the Ant target by executing the following steps
(instructions copied from {\codesize\url{http://help.eclipse.org/luna/index.jsp?topic=%2Forg.eclipse.platform.doc.user%2FgettingStarted%2Fqs-84_run_ant.htm}}):

\begin{enumerate}

\item
  Select \code{build.xml} in one of the navigation views and choose
  {\bf Run As $>$ Ant Build...} from its context menu.

\item
  A launch configuration dialog is opened on a launch configuration
  for this Ant buildfile.

\item
  In the {\bf Targets} tab, select the new ant task (e.g., check-interning).

\item
  Click {\bf Run}.

\item
  The Ant buildfile is run, and the output is sent to the Console view.

\end{enumerate}

\paragraph{Eclipse plug-in for the Checker Framework\label{eclipse-plug-in}}

The Checker Plugin is an Eclipse plugin that enables the use of the Checker
Framework.
Its website (\myurl{http://types.cs.washington.edu/checker-framework/eclipse/}).
The website contains instructions for installing and using the plugin.
% The plugin has been substantially improved through a Google Summer of Code 2010 project
% and supports all checkers that are distributed with the Checker Framework.


\section{tIDE\label{tide}}

\begin{sloppypar}
tIDE, an open-source Java IDE, supports the Checker Framework.  See its
documentation at \myurl{http://tide.olympe.in/}.
\end{sloppypar}


\section{Type inference tools\label{type-inference-tools}}

\subsection{Varieties of type inference\label{type-inference-varieties}}

There are two different tasks that are commonly called ``type inference''.

\begin{enumerate}
\item
  Type inference during type-checking (Section~\ref{type-refinement}):
  During type-checking, if certain variables have no type qualifier, the
  type-checker determines whether there is some type qualifier that would
  permit the program to type-check.  If so, the type-checker uses that type
  qualifier, but never tells the programmer what it was.  Each time the
  type-checker runs, it re-infers the type qualifier for that variable.  If
  no type qualifier exists that permits the program to type-check, the
  type-checker issues a type warning.

  This variety of type inference is built into the Checker Framework.  Every
  checker can take advantage of it at no extra effort.  However, it only
  works within a method, not across method boundaries.

  Advantages of this variety of type inference include:
  \begin{itemize}
  \item
    If the type qualifier is obvious to the programmer, then omitting it
    can reduce annotation clutter in the program.
  \item
    The type inference can take advantage of only the code currently being
    compiled, rather than having to be correct for all possible calls.
    Additionally, if the code changes, then there is no old annotation to
    update.
  \end{itemize}


\item
  Type inference to annotate a program (Section~\ref{type-inference-to-annotate}):
  As a separate step before type-checking, a type inference tool takes the
  program as input, and outputs a set of type qualifiers that would
  type-check.  These qualifiers are inserted into the source code or the
  class file.  They can be viewed and adjusted by the programmer, and can
  be used by tools such as the type-checker.

  This variety of type inference must be provided by a separate tool.  It
  is not built into the Checker Framework.

  Advantages of this variety of type inference include:
  \begin{itemize}
  \item
    The program contains documentation in the form of type qualifiers,
    which can aid programmer understanding.
  \item
    Error messages may be more comprehensible.  With type inference
    during type-checking, error messages can be obscure, because the
    compiler has already inferred (possibly incorrect) types for a number
    of variables.
  \item
    A minor advantage is speed:  type-checking can be modular, which can be
    faster than re-doing type inference every time the
    program is type-checked.
  \end{itemize}

\end{enumerate}

Advantages of both varieties of inference include:
\begin{itemize}
\item
  Less work for the programmer.
\item
  The tool chooses the most general type, whereas a programmer might
  accidentally write a more specific, less generally-useful annotation.
\end{itemize}


Each variety of type inference has its place.  When using the Checker
Framework, type inference during type-checking is performed only
\emph{within} a method (Section~\ref{type-refinement}).  Every method
signature (arguments and return values) and field must have already been explicitly annotated,
either by the programmer or by a separate type-checking tool
(Section~\ref{type-inference-to-annotate}).
This approach enables modular checking (one class or method at a time) and
gives documentation benefits.
The programmer still has to
put in some effort, but much less than without inference:  typically, a
programmer does not have to write any qualifiers
inside the body of a method.


\subsection{Type inference to annotate a program\label{type-inference-to-annotate}}

This section lists tools that take a program and output a set of
annotations for it.

Section~\ref{nullness-inference} lists several tools that infer
annotations for the Nullness Checker.

Section~\ref{javari-inference} lists a tool that infers
annotations for the Javari Checker, which detects mutation errors.

\href{https://github.com/reprogrammer/cascade/}{Cascade}~\cite{VakilianPEJ2014}
is an Eclipse plug-in that implements interactive type qualifier inference.
Cascade is interactive rather than fully-automated:  it makes it easier for
a developer to insert annotations.
Cascade starts with an unannotated program and runs a type-checker.  For each
warning it suggests multiple fixes, the developer chooses a fix, and
Cascade applies it.  Cascade works with any checker built on the Checker
Framework.
You can find installation instructions and a video tutorial at \url{https://github.com/reprogrammer/cascade}.


% LocalWords:  jsr plugin Warski xml buildfile tIDE java Awarns pom lifecycle
% LocalWords:  IntelliJ Maia newdir classpath Unconfuse nullness Gradle
% LocalWords:  compilerArgs Xbootclasspath JSR308 jsr308 jsr308javac mvn
%  LocalWords:  plugins proc procOnly DirectoryScanner setIncludes groupId
%  LocalWords:  setExcludes checkerFrameworkVersion javacParams javaParams
%%  LocalWords:  artifactId quals failOnError ejc

\htmlhr
\chapter{Frequently Asked Questions (FAQs)\label{faq}}

These are some common questions about the Checker Framework and about
pluggable type-checking in general.  Feel free to suggest improvements to
the answers, or other questions to include here.

There is a separate FAQ for the type annotations syntax
(\url{http://types.cs.washington.edu/jsr308/jsr308-faq.html}).

% Not supported by Hevea, so don't bother; instead do by hand:
% \minitoc

%BEGIN LATEX
~
%END LATEX

\noindent
\textbf{Contents:} \\
\ref{faq-ease-of-use}: Are type annotations easy to read and write? \\
\ref{faq-code-clutter}: Will my code become cluttered with type annotations? \\
\ref{faq-no-absolute-guarantee}: Can a pluggable type-checker give an absolute guarantee of correctness? \\
\ref{never-make-type-errors}: I don't make type errors, so would pluggable type checking help me? \\
\ref{faq-qualifiers-vs-subclasses}: When should I use type qualifiers, and when should I use subclasses? \\
\ref{faq-no-annotation-on-types-and-declarations}: Why shouldn't a qualifier apply to both types and declarations? \\
\ref{faq-run-on-all-files}: How do I run a checker on all my source files? \\
\ref{faq-create-a-checker}: How do I create a new checker? \\
\ref{faq-declarative-syntax-for-type-rules}: Why is there no declarative syntax for writing type rules? \\
\ref{faq-type-checking-vs-bug-detectors}: Why not just use a bug detector (like FindBugs)? \\
\ref{faq-jml}: How does pluggable type-checking compare with JML? \\
\ref{faq-list-map-nonnull-typeargs}: Why are the type parameters to \<List> and \<Map> annotated as \<@NonNull>? \\
\ref{faq-run-time-checking}: How can I do run-time monitoring of properties that were not statically checked?



% This FAQ also appears in the JSR 308 FAQ.
\section{Are type annotations easy to read and write?\label{faq-ease-of-use}}

The paper
\ahref{http://www.cs.washington.edu/homes/mernst/pubs/pluggable-checkers-issta2008-abstract.html}{``Practical
  pluggable types for Java''}~\cite{PapiACPE2008} discusses case studies in
which programmers
found type annotations to be natural to read and write.  The code
continued to feel like Java, and the type-checking errors were easy to
comprehend and often led to real bugs.

You don't have to take our word for it, though.  You can try the
Checker Framework for yourself.

The difficulty of adding and verifying annotations depends on your program.
If your program is well-designed and -documented, then skimming the
existing documentation and writing type annotations is extremely easy.
Otherwise, you may find yourself spending a lot of time trying to
understand, reverse-engineer, or fix bugs in your program, and then just a
moment writing a type annotation that describes what you discovered.  This
process inevitably improves your code.  You must decide whether it is a
good use of your time.  For code that is not causing trouble now and is
unlikely to do so in the future (the code is bug-free, and you do not
anticipate changing it or using it in new contexts), then the
effort of writing type annotations for it may not be justified.


% This FAQ also appears in the JSR 308 FAQ.
\section{Will my code become cluttered with type annotations?\label{faq-code-clutter}}

As with any language feature, it is possible to write ugly code that
over-uses annotations.  However, in normal use, very few annotations need
to be written.  Figure 1 of the paper
\ahref{http://www.cs.washington.edu/homes/mernst/pubs/pluggable-checkers-issta2008-abstract.html}{Practical
  pluggable types for Java}~\cite{PapiACPE2008} reports data for over
350,000 lines of type-annotated code:

\begin{itemize}
\item
    1 annotation per 62 lines for nullness annotations (\<@NonNull>, \<@Nullable>, etc.)
    % (/ (+ 4640 3961 10798) (+ 107 35 167))
\item
    1 annotation per 1736 lines for interning annotations (\<@Interned>)
    % (/ 224048 129)
\item
    1 annotation per 27 lines for immutability annotations (IGJ type system)
    % (/ (+ 6236 18159 30507 8691 59221 26828) (+ 315 1125 1386 384 1815 450))
\end{itemize}

Furthermore, these numbers are for annotating existing code.  New code that
is written with the type annotation system in mind is cleaner and more
correct, so it requires even fewer annotations.

In other words, annotations do not clutter code, and they are used much
less frequently than generic types, which Java programmers find acceptable.


\section{Can a pluggable type-checker give an absolute guarantee of correctness?\label{faq-no-absolute-guarantee}}

Each checker looks for certain errors.  You can use multiple checkers, but
even then your program might still contain other kinds of errors.

If you run a pluggable checker on only part of the code of a program, then
you do not get a guarantee that all parts of the program satisfy the type
system (that is, are error-free).  An example is a framework that clients
are intended to extend.  In this case, you should recommend that clients
run the pluggable checker.  There is no way to force users to do so, so you
may want to retain dynamic checks or use other mechanisms to detect errors.

There are other circumstances in which a static type-checker may fail to
detect a possible type error.  In Java, these include arrays, casts, raw
types, reflection, separate compilation (bytecodes from unverified sources),
native code, etc.  (For details, see section~\ref{checker-guarantees}.)
Java uses dynamic checks for most of these, so that the
type error cannot cause a security vulnerability or a crash.  The pluggable
type-checkers inherit many (not all) of these weaknesses of Java
type-checking, but do not currently have built-in dynamic checkers.
Writing dynamic checkers would be an interesting and valuable project.

% This paragraph is weak.

Even if a tool such as a pluggable checker cannot give an ironclad
guarantee of correctness, it is still useful.  It can finding errors, 
excluding certain types of possible problems (e.g., restricting the
possible class of problems), and increasing confidence in a piece of
software.


\section{I don't make type errors, so would pluggable type checking help me?\label{never-make-type-errors}}

Occasionally, a developer says that he makes no errors that typechecking
could catch, or that any such errors are unimportant because they have low
impact and are easy to fix.  When I investigate the claim, I invariably
find that the developer is mistaken.

Very frequently, the developer has underestimated what typechecking can
discover.  Not every type error leads to an exception being thrown; and
even if an exception is thrown, it may not seem related to classical types.
Remember that a type system can discover
null pointer dereferences,
incorrect side effects, 
security errors such as information leakage or SQL injection,
partially-initialized data,
and many other errors.  Even where type-checking does not discover a
problem directly, it can indicate code with bad smells, thus revealing
problems, improving documentation, and making future maintenance easier.

There are other ways to discover errors, including extensive testing and
debugging.  But type-checking is a good complement to these.  It is more
effective for some problems, and less effective for other problems.  It can
reduce (but not eliminate) the time and effort that you spend on other
approaches.  There are many important errors that type checking and other
automated approaches cannot find; pluggable typechecking gives you more
time to focus on those.


\section{Why shouldn't a qualifier apply to both types and declarations?\label{faq-no-annotation-on-types-and-declarations}}

It is bad style for an annotation to apply to both types and declarations.
In other words, every annotation should have a \<@Target> meta-annotation,
and the \<@Target> meta-annotation should list either only declaration
locations or only type annotations.  (It's OK for an annotation to target
both \<ElementType.TYPE\_PARAMETER> and \<ElementType.TYPE\_USE>, but no
other declaration location along with \<ElementType.TYPE\_USE>.)

Sometimes, it may seem tempting for an annotation to apply to both type
uses and (say) method declarations.  Here is a hypothetical example:

\begin{quote}
  ``Each \<Widget> type may have a \<@Version> annotation.
  I wish to prove that versions of widgets don't get assigned to
  incompatible variables, and that older code does not call newer code (to
  avoid problems when backporting).

  A \<@Version> annotation could be written like so:

\begin{Verbatim}
  @Version("2.0") Widget createWidget(String value) { ... }
\end{Verbatim}

\<@Version("2.0")> on the method could mean that the \<createWidget> method
only appears in the 2.0 version.  \<@Version("2.0")> on the return type
could mean that the returned \<Widget> should only be used by code that
uses the 2.0 API of \<Widget>.  It should be possible to specify these
independently, such as a 2.0 method that returns a value that allows the
1.0 API method invocations.''
\end{quote}

Both of these are type properties and should be specified with type
annotations.  No method annotation is necessary or desirable.  The best way
to require that the receiver has a certain property is to use a type
annotation on the receiver of the method.  (Slightly more formally, the
property being checked is compatibility between the annotation on the type
of the formal parameter receiver and the annotation on the type of the
actual receiver.)


Another example of a type-and-declaration annotation that represents poor
design is JCIP's \<@GuardedBy> annotation~\cite{Goetz2006}.  As discussed
in Section~\ref{jcip-annotations}, it means two different things when
applied to a field or a method.  To reduce confusion and increase
expressiveness, the Lock Checker (see Chapter~\ref{lock-checker}) uses the
\<@Holding> annotation for one of these meanings, rather than overloading
\<@GuardedBy> with two distinct meanings.



\section{When should I use type qualifiers, and when should I use subclasses?\label{faq-qualifiers-vs-subclasses}}

In brief, use subtypes when you can, and use type qualifiers when you cannot
use subtypes.
For more details, see section~\ref{when-to-use-type-qualifiers}.


\section{How do I run a checker on all my source files?\label{faq-run-on-all-files}}

The \code{javac} compiler halts compilation as soon as it processes a
source file with an error, including an error issued by a pluggable
type-checker.  Section~\ref{running} describes the \<-Awarns> command-line
option that turns checker errors into warnings, permitting \<javac> to
continue past the first erroneous source file.


\section{How do I create a new checker?\label{faq-create-a-checker}}

In addition to using the checkers that are distributed with the Checker
Framework, you can write your own checker to check specific properties that
you care about.  Thus, you can find and prevent the bugs that are most
important to you.

Chapter~\ref{writing-a-checker} gives
complete details regarding how to write a checker.  It also suggests places
to look for more help, such as the \ahref{doc/}{Checker Framework
API documentation (Javadoc)} and the source code of the distributed
checkers.

To whet your interest and demonstrate how easy it is to get started, here
is an example of a complete, useful type checker.

\begin{Verbatim}
  @TypeQualifier
  @SubtypeOf(Unqualified.class)
  @Target({ElementType.TYPE_PARAMETER, ElementType.TYPE_USE})
  public @interface Encrypted { }
\end{Verbatim}

Section~\ref{basic-example} explains this checker and tells
you how to run it.


\section{Why is there no declarative syntax for writing type rules?\label{faq-declarative-syntax-for-type-rules}}

A type system implementer can declaratively specify the type qualifier
hierarchy (Section~\ref{declarative-hierarchy}) and the type introduction rules
(Section~\ref{declarative-type-introduction}).  However, the Checker
Framework uses a procedural syntax for specifying type-checking
rules (Section~\ref{extending-visitor}).
A declarative syntax might be more concise, more readable, and more
verifiable than a procedural syntax.

We have not found the procedural syntax to be the most important impediment
to writing a checker.

Previous attempts to devise a declarative syntax 
for realistic type systems have failed; see a technical
paper~\cite{PapiACPE2008} for a discussion.  When an
adequate syntax exists, then the Checker Framework can be extended to
support it.


\section{Why not just use a bug detector (like FindBugs)?\label{faq-type-checking-vs-bug-detectors}}

Pluggable type-checking finds more bugs than a bug detector does, for any
given variety of bug.

A bug detector like \ahref{http://findbugs.sourceforge.net/}{FindBugs}~\cite{HovemeyerP2004,HovemeyerSP2006},
\ahref{http://artho.com/jlint/}{JLint}~\cite{Artho2001}, or
\ahref{http://pmd.sourceforge.net/}{PMD}~\cite{Copeland2005} aims to find \emph{some}
of the most obvious bugs in your program.  It uses a lightweight analysis,
then uses heuristics to discard some of its warnings.  Thus, even if the tool
prints no warnings, your code might still have errors --- maybe the
analysis was too weak to find them, or the tool's heuristics classified the
warnings as likely false positives and discarded them.

A type checker aims to find \emph{all} the bugs (of certain varieties).
It requires you to write type qualifiers in your program, or to use a tool
that infers types.  Thus, it requires more work from the programmer, and in
return it gives stronger guarantees.

Each tool is useful in different circumstances, depending on how important
your code is and your desired level of confidence in your code.  For more
details on the comparison, see section~\ref{other-tools}.  For a case study
that compared the nullness analysis of FindBugs, JLint, PMD, and the
Checker Framework, see section 6 of the paper
\ahref{http://www.cs.washington.edu/homes/mernst/pubs/pluggable-checkers-issta2008.pdf}{``Practical pluggable types for Java''}~\cite{PapiACPE2008}.


\section{How does pluggable type-checking compare with JML?\label{faq-jml}}

\ahref{http://www.cs.ucf.edu/~leavens/JML/}{JML}, the Java Modeling
Language~\cite{LeavensBR2006:JML}, is a language for writing formal
specifications.  JML aims to be more expressive than pluggable
type-checking.  JML is not as practical as pluggable type-checking.

A programmer can write a JML specification that
describes arbitrary facts about program behavior.  Then, the programmer can
use formal reasoning or a theorem-proving tool to verify that the code
meets the specification.  Run-time checking is also possible.
By contrast, pluggable type-checking can express a more limited set of
properties about your program.

The JML toolset is less mature.  For instance, if your code uses
generics or other features of Java 5, then you cannot use JML.  
However, JML has a run-time checker, which the Checker Framework currently
lacks.


\section{Why are the type parameters to \<List> and \<Map> annotated as \<@NonNull>?\label{faq-list-map-nonnull-typeargs}}

The annotation on \<java.util.Collection> only allows non-null elements:

\begin{Verbatim}
  public interface Collection<E extends @NonNull Object> {
    ...
  }
\end{Verbatim}

\noindent
Thus, you will get a type error if you write code like
\code{Collection<@Nullable Object>}.
A nullable
type parameter is also forbidden for certain other collections, including
\<AbstractCollection>, \<List>, \<Map>, and \<Queue>.

% AbstractCollection has no documentation of its own regarding nullness,
% but it implements Collection.

% The JML specifications of the add() method says
%       @   signals (NullPointerException)
%       @             (* not allowed to add null *);
%       ...
%       @   signals (NullPointerException)
%       @             (* not allowed to add null *);
% In other words, the method might throw NullPointerException, but the JML
% spec does not say under what circumstances.

The \<extends @NonNull Object> bound is a direct consequence of the design
of the collections classes; it merely formalizes the Javadoc specification.
The Javadoc for \<Collection> states:

\begin{quote}
  Some list implementations have restrictions on the elements that they may
  contain. For example, some implementations prohibit null elements, \ldots
\end{quote}

Here are some consequences of the requirement to detect all nullness errors
at compile time.  If even one subclass of a given collection class may
prohibit null, then the collection class and all its subclasses must
prohibit null.  Conversely, if a collection class is specified to accept
null, then all its subclasses must honor that specification.

The Checker Framework's annotations make apparent a flaw in the JDK
design, and helps you to avoid problems that might be caused by that flaw.


\paragraph{Justification from type theory}
Suppose \<B> is a subtype of \<A>.
Then an overriding method in \<B> must have a stronger (or equal) signature
than the overridden method in~\<A>.  In a stronger signature, the formal
parameter types may be supertypes, and the return type may be a subtype.
Here are examples:

\begin{Verbatim}
  class A           {  @NonNull Object Number m1( @NonNull Object arg) { ... } }
  class B extends A { @Nullable Object Number m1( @NonNull Object arg) { ... } } // error!
  class C extends A {  @NonNull Object Number m1(@Nullable Object arg) { ... } } // OK
  class D           { @Nullable Object Number m2(@Nullable Object arg) { ... } }
  class E extends D {  @NonNull Object Number m2(@Nullable Object arg) { ... } } // OK
  class F extends D { @Nullable Object Number m2( @NonNull Object arg) { ... } } // error!
\end{Verbatim}

According to these rules, since some subclasses of \<Collection> do not
permit nulls, then \<Collection> cannot either:

\begin{Verbatim}
  // does not permit null elements
  class PriorityQueue<E> implements Collection<E> {
    boolean add(E);
    ...
  }
  // must not permit null elements, or PriorityQueue would not be not a subtype of Collection
  interface Collection<E> {
    boolean add(E);    
    ...
  }
\end{Verbatim}


\paragraph{Justification from checker behavior}

Suppose that you changed the bound in the \<Collection> declaration to
\<extends @Nullable Object>.  Then, the checker would issue no warning for
this method:

\begin{Verbatim}
  static void addNull(Collection l) {
    l.add(null);
  }
\end{Verbatim}

\noindent
However, calling this method \emph{can} result in a null pointer exception,
for instance caused by the following code:

\begin{Verbatim}
  addNull(new PriorityQueue());
\end{Verbatim}

\noindent
Therefore, the bound must remain as \<extends @NonNull Object>.

By contrast, this code is OK because \<ArrayList> is documented to support
null elements:

\begin{Verbatim}
  static void addNull(ArrayList l) {
    l.add(null);
  }
\end{Verbatim}

\noindent
Any subclass of \<ArrayList> must also support null elements.




% Every implementation of List seems to permit null.
% Examples of Collection that do not permit null:
% BlockingQueue family:
%   BlockingQueue, BlockingDeque, ArrayBlockingQueue, DelayQueue, LinkedBlockingDeque, LinkedBlockingQueue, PriorityBlockingQueue, SynchronousQueue 
% PriorityQueue
% probably lots of other queues.

% A similar argument applies to \<Map>.
% For example, \<ConcurrentHashMap> and \<Hashtable> implement \<Map> but do
% not permit \<null> to be used as a key or value.  Therefore, \<Map> must
% not permit \<null> to be used as a key or value


% The Checker Framework is designed to warn you whenever your code might
% throw a null pointer exception.  If you want to be safe, you will never put
% \<null> in a \<List> of unknown provenance, because that \<List> might not
% accept null.



\paragraph{Suppressing warnings}

Suppose your program has a list variable, and you know that any list referenced
by that variable will definitely support null.  Then, you can suppress the
warning:

\begin{Verbatim}
  @SuppressWarnings("nullness:generic.argument")
  static void addNull(List l) {
    l.add(null);
  }
\end{Verbatim}

\noindent
You need to use \<@SuppressWarnings("nullness:generic.argument")>
whenever you use a collection that may contain \<null> elements in
contradiction to its documentation.  Fortunately, such uses are relatively
rare.


For more details on suppressing nullness warnings, see
Section~\ref{suppressing-warnings-nullness}.


\section{How can I do run-time monitoring of properties that were not statically checked?\label{faq-run-time-checking}}

Currently, the Checker Framework has no support for adding code to check,
at run time, code that was not checked (see
Chapter~\ref{warnings-and-legacy} for reasons that code might not be
checked).  An exception is the Nullness Checker, which has ways to
dynamically check nullness via assertions and casts (the
\refmethod{nullness}{NullnessUtils}{castNonNull}{(T)} method); see
Section~\ref{suppressing-warnings-with-assertions}.

More general support would be an interesting and valuable project.  If you
are able to add run-time verification functionality, we would gladly
welcome it as a contribution to the Checker Framework.


% LocalWords:  IGJ toolset AbstractCollection ConcurrentHashMap NullnessUtils
% LocalWords:  castNonNull createWidget backporting JCIP's GuardedBy Awarns PMD
% LocalWords:  ElementType nullness bytecodes JLint Hashtable SuppressWarnings

\htmlhr
\chapter{Troubleshooting and getting help\label{troubleshooting}}

Please read the entire manual, including this chapter and the FAQ
(Chapter~\ref{faq}), because the manual might already answer your question.
If not, you can use the mailing list,
\code{checker-framework-discuss@googlegroups.com}, to ask other users for
help.  For archives and to subscribe, see \url{http://groups.google.com/group/checker-framework-discuss}.
To report bugs, use the issue tracker at
\url{http://code.google.com/p/checker-framework/issues/list}.
If you want to help out, you can choose a bug and fix it, or select a
project from the ideas list at
\url{http://code.google.com/p/checker-framework/wiki/Ideas}.


\section{Common problems and solutions\label{common-problems}}

\begin{itemize}
\item
To verify that you are using the compiler you think you are, you can add
\code{-version} to the command line.  For instance, instead of running
\code{javac -g MyFile.java}, you can run \code{javac \underline{-version} -g
  MyFile.java}.  Then, javac will print out its version number in addition
to doing its normal processing.

\end{itemize}



\subsection{Unable to run the checker\label{common-problems-running}}

If you are unable to run the checker, then the problem is likely to be a
problem with your environment.  This section describes some possible
problems and solutions.

\begin{itemize}
\item
If you get the error

%BEGIN LATEX
\begin{smaller}
%END LATEX
\begin{Verbatim}
com.sun.tools.javac.code.Symbol$CompletionFailure: class file for com.sun.source.tree.Tree not found
\end{Verbatim}
% Unconfuse Emacs by matching the "$" in the above Verbatim
%BEGIN LATEX
\end{smaller}
%END LATEX

\noindent
then you are using the source installation and file \code{tools.jar} is not
on your classpath.  See the installation instructions
(Section~\ref{installation}).

\item
If you get an error such as

\begin{Verbatim}
package checkers.nullness.quals does not exist
\end{Verbatim}

  \noindent
  despite no apparent use of \code{import checkers.nullness.quals.*;} in
  the source code, then perhaps
  \code{jsr308\_imports} is set as a Java system property, a shell
  environment variable, or a command-line option (see
  Section~\ref{jsr308_imports}).  You can solve this by unsetting the
  variable/option, or by ensuring that the \code{checkers.jar} file is on
  your classpath.

If the error is 

\begin{Verbatim}
package 'checkers.nullness.quals does not exist
\end{Verbatim}

\noindent
(note the extra apostrophe!), then you have probably mis-used quoting when
supplying the \code{jsr308\_imports} environment variable.

\item
If you get an error like the following when using the Ant task
(Section~\ref{ant-task}),

%BEGIN LATEX
\begin{smaller}
%END LATEX
\begin{Verbatim}
...\build.xml:59: Error running ${env.CHECKERS}\binary\javac.bat compiler
\end{Verbatim}
% Unconfuse Emacs by matching the "$" in the above Verbatim
%BEGIN LATEX
\end{smaller}
%END LATEX

\noindent
then the problem may be that you have not set the CHECKERS environment
variable, as described in Section~\ref{windows-installation}.  Or, maybe
you made it a user variable instead of a system variable.

\item
If you get one of these errors:

\begin{alltt}
The hierarchy of the type \emph{ClassName} is inconsistent

The type com.sun.source.util.AbstractTypeProcessor cannot be resolved.
  It is indirectly referenced from required .class files", 
\end{alltt}

\noindent
then you are missing \code{jsr308-all.jar} from your classpath.

\item
If you get the error

\begin{Verbatim}
  java.lang.ArrayStoreException: sun.reflect.annotation.TypeNotPresentExceptionProxy
\end{Verbatim}

\noindent
% I'm not 100% sure of the following explanation and solution.
then an annotation is not present at run time that was present at compile
time.  For example, maybe when you compiled the code, the \<@Nullable>
annotation was available, but it was not available at run time.
You can use JDK 7 at run time, or compile
with a Java 6 compiler that will ignore the annotations in comments.

\item
A ``class file not found'' error may be due to a JDK version mismatch.
For instance, you might be using JDK 7, but you get an error that refers to a class that was in a
previous version of the JDK but has subsequently been removed, such as:

\begin{Verbatim}
  class file for java.io.File$LazyInitialization not found
\end{Verbatim}
% To unconfuse Emacs's LaTeX mode: $

Or, you might be using JDK 6, but you get an error that refers to a class
that has been introduced in a newer version of the JDK, such as:

\begin{Verbatim}
  class file for java.util.Vector$Itr not found
\end{Verbatim}
% To unconfuse Emacs's LaTeX mode: $

This problem occurs when your classpath contains code that was compiled
with one version of the JDK and refers to its implementation details, but
your classpath does not contain that version of the JDK itself.

You can solve the problem by re-generating \code{jdk/jdk.jar} and
\code{binary/jdk.jar}.  You can do this by running

\begin{Verbatim}
  cd checkers
  ant jdk.jar bindist
\end{Verbatim}

\end{itemize}


\subsection{Unexpected type-checking results\label{common-problems-typechecking}}

This section describes possible problems that can lead the type-checker to
give unexpected results.


\begin{itemize}
\item
  If the Checker Framework is unable to verify a property that you know is
  true, then it is helpful to formulate an argument about why the property
  is true.  Recall that the Checker Framework does modular verification,
  one procedure at a time; it observes the specifications, but not the
  implementations, of other methods.

  If any aspects of your argument are not expressed as annotations, then
  you may need to write more annotations.  If any aspects of your argument
  are not expressible as annotations, then you may need to extend the
  type-checker.

\item
If a checker seems to be ignoring the annotation on a method, then it is
possible that the checker is reading the method's signature from its
\code{.class} file, but the \code{.class} file was not created by the JSR
308 compiler.  You can check whether the annotations actually appear in the
\code{.class} file by using the \code{javap} tool.

If the annotations do not appear in the \code{.class} file, here are two
ways to solve the problem:
\begin{itemize}
\item
  Re-compile the method's class with the Type Annotations compiler.  This will
  ensure that the type annotations are written to the class file, even if
  no type-checking happens during that execution.
\item
  Pass the method's file explicitly on the command line when type-checking,
  so that the compiler reads its source code instead of its \code{.class}
  file.
\end{itemize}

\item
If the compiler reports that it cannot find a method from the
JDK or another external library, then maybe the stub/skeleton file for that
class is incomplete.  You can edit it to add the missing method.  The
libraries appear, for example, at \code{checkers/jdk/nullness/src/} for the
Nullness checker.

The error might take one of these forms:

\begin{Verbatim}
method sleep in class Thread cannot be applied to given types
cannot find symbol: constructor StringBuffer(StringBuffer)
\end{Verbatim}

\end{itemize}


\subsection{Known problems in the framework\label{known-problems}}

\begin{itemize}

\item
  The framework may not parse annotations from skeleton files if the
  skeleton files are older than the classfiles.  Running \code{ant
    touch-jdk} solves this problem, by applying the 
  \code{touch} program to each distributed skeleton file.

% Mahmood will address.  -MDE 3/19/2009
\item The framework is missing a check for type argument subtyping in
  method invocations if the type arguments are inferred.

% Mahmood will address.  -MDE 3/19/2009
\item The checks for enclosed types are not yet fully tested.

\end{itemize}

\subsection{Known problems in the Nullness checker}

\begin{itemize}
\item
  The Nullness checker is often able to determine that a call to
  \code{Map.get()} will not return null.  This enables the checker to avoid
  issuing false positive warnings, in circumstances like the following.

\begin{Verbatim}
    @NonNull String value;
    if (myMap.containsKey(key)) {
      value = myMap.get(key);
    }
    for (String keyInMap : myMap.keySet()) {
        value = myMap.get(keyInMap);
    }
\end{Verbatim}

  The Nullness checker can sometimes fail to issue a warning if the map is
  modified or re-assigned between the check of \code{containsKey} and the
  call to \code{get}.

% This description really needs an example of a case where the checker
% fails.  Right now, it is impossible for a reader to tell what the problem
% is or whether a particular piece of code triggers it.
% Do we have a test case?


% The solution is to merge flow with the Map.get heuristics.
% And to do forward instead of backward analysis.


\end{itemize}



\section{How to report problems\label{reporting-bugs}}

If you have a problem with any checker, or with the Checker Framework,
please file a bug at 
\url{http://code.google.com/p/checker-framework/issues/list}.
(First, check whether there is an existing bug report for that issue.)

Alternately (especially if your communication is not a bug report), you can
send mail to checker-framework-dev@googlegroups.com.
We welcome suggestions, annotated libraries, bug fixes, new
features, new checker plugins, and other improvements.

Please ensure that your bug report is clear and that it is complete.
Otherwise, we may be unable to understand it or to reproduce it, either of
which would prevent us from fixing the bug.  Your bug report will be most
helpful if you:

\begin{itemize}
\item
  Add \code{-version -verbose} to the javac options.  This causes the compiler to output
  debugging information, including its version number.
\item
  Indicate exactly what you did.  Don't skip any steps, and don't merely
  describe your actions in words.  Show the exact commands by attaching a
  file or using cut-and-paste from your command shell;
\item
  Include all files that are necessary to reproduce the problem.  This
  includes every file that is used by any of the commands you reported, and
  possibly other files as well.
\item
  Indicate exactly what the result was by attaching a file or using
  cut-and-paste from your command shell (don't merely describe it in
  words).  Also indicate what you expected the result to be --- remember, a
  bug is a difference between desired and actual outcomes.
\end{itemize}

A particularly useful format for a test case is as a diff, or a new file,
for the existing checker test cases.  For instance, for the Nullness
Checker, see directory \<checker-framework/checkers/tests/nullness/>.


\section{Building from source\label{build-source}}

The Checker Framework release (Section~\ref{installation}) contains
everything that most users need, both to use the distributed checkers and
to write your own checkers.  This section describes how to re-build its
binaries from source.  You will be using the latest development version of
the Checker Framework, rather than an official release.

% Doing
% so permits you to examine and modify the implementation of the distributed
% checkers and of the checker framework.  It may also help you to debug
% problems more effectively.


\subsection{Obtain the source}

Obtain the latest source code from the version control repository:

\begin{Verbatim}
export JSR308=$HOME/jsr308
mkdir -p $JSR308
cd $JSR308
hg clone https://jsr308-langtools.googlecode.com/hg/ jsr308-langtools
hg clone https://checker-framework.googlecode.com/hg/ checker-framework
hg clone https://annotation-tools.googlecode.com/hg/ annotation-tools
\end{Verbatim}
% $ to unconfuse Emacs LaTeX mode

\noindent
(Alternately, you could use the version of the source code that is packaged
in the Checker Framework release.)


\subsection{Build the Type Annotations compiler}

\begin{enumerate}
\item
% Why is this necessary?  What goes wrong if it is not set?  Can I avoid
% the need to set it?  It's used for:
%  * the location of tools.jar, below.
%  * the default location of RTJAR, in checkers/jdk/Makefile.
Set the \<JAVA\_HOME> environment variable to the location of your JDK 6 or
7 installation (not the JRE installation).  (It may already be set for Ant to work.)

%% This does not work, because "java" might be the version in the JDK or in
%% the JRE.
% Here is a command that works in the bash shell, for example:
% % Can someone give a simpler command?
% \begin{Verbatim}
%   export JAVA_HOME=${JAVA_HOME:-$(dirname $(dirname $(dirname $(readlink -f $(/usr/bin/which java)))))}
% \end{Verbatim}

\item
Compile the Type Annotations javac compiler and the javap tool:

\begin{Verbatim}
  cd $JSR308/jsr308-langtools/make
  ant clean build-javac build-javap
\end{Verbatim}

\item
 Add the \<jsr308-langtools/dist/bin> directory to the front of your PATH environment variable.
  Example command:

\begin{Verbatim}
  export PATH=$JSR308/jsr308-langtools/dist/bin:${PATH}
\end{Verbatim}

\end{enumerate}

% JSR 308 extends the Java language to permit annotations to appear on types,
% as in \code{List<@NonNull String>} (see Section~\ref{writing-annotations}).
% This change will be part of the Java 8 language.  We recommend that you
% write annotations in comments, as in \code{List</*@NonNull*/ String>} (see
% Section~\ref{annotations-in-comments}).  The JSR 308 compiler still reads
% such annotations, but this syntax permits you to use a compiler other than
% the JSR 308 compiler.  For example, you can compile your code with a Java 5
% compiler, and you can use a checker as an external tool in an IDE.


\subsection{Build the Checker Framework\label{building}}

% Building (compiling) the checkers and framework from source creates the
% \code{checkers.jar} file.  A pre-compiled \code{checkers.jar} is included
% in the distribution, so building it is optional.  It is mostly useful for
% people who are developing compiler plug-ins (type-checkers).  If you only
% want to \emph{use} the compiler and existing plug-ins, it is sufficient to
% use the pre-compiled version.

\begin{enumerate}
% \item
% Edit \code{checkers/build.properties} file so that the
% \code{compiler.lib} property specifies the location of the JSR 308
% \code{javac.jar} library.  (If you also installed the JSR 308 compiler from
% source, and you made the \code{checkers} and \code{jsr308-langtools} directories
% siblings, then you don't need to edit \code{checkers/build.properties}.)

\item
Run \code{ant} to create \<checkers.jar>:

\begin{Verbatim}
  cd $JSR308/checker-framework/checkers
  ant
\end{Verbatim}
% $ to unconfuse Emacs LaTeX mode

\item Add \code{tools.jar} and \code{checkers.jar} to your classpath
  (If you do not do this, you will have to supply the \code{-cp} option
  whenever you run \code{javac} and use a checker plugin.)
  Example command:

%BEGIN LATEX
\begin{smaller}
%END LATEX
\begin{Verbatim}
  export CLASSPATH=${CLASSPATH}:$JAVA_HOME/lib/tools.jar:$JSR308/checker-framework/checkers/checkers.jar
\end{Verbatim}
% $ to unconfuse Emacs LaTeX mode
%BEGIN LATEX
\end{smaller}
%END LATEX
  %% In Cygwin, are reversed slashes required?

\item Test that everything works:

  \begin{itemize}

  \item Run \code{ant all-tests} in the \code{checkers} directory:
\begin{Verbatim}
  cd $JSR308/checker-framework/checkers
  ant all-tests
\end{Verbatim}
% $ to unconfuse Emacs LaTeX mode

  \item Run the Nullness checker examples (see
    Section~\refwithpage{nullness-example}).

  \end{itemize}

\end{enumerate}

\subsection{Build the Checker Framework manual (this document)}

\begin{enumerate}
\item
To build the manual you will need plume-bib (\myurl{http://code.google.com/p/plume-bib/}) and {\hevea} (\myurl{http://hevea.inria.fr/}) installed.

\item
Run \code{make} in the \code{checkers/manual} directory to build both the PDF and HTML versions of the manual.
\end{enumerate}

% \subsection{Adjust classpath}

% Building the Checker Framework requires use of a Java 8 compiler.  You may
% use either the OpenJDK compiler or the JSR 308 compiler.  The latter has a
% few extra features and tends to get bug fixes more quickly.

% The following instructions give detailed steps for installing the source
% release of the Checker Framework.


% \item Download and install the JSR 308 implementation; follow the instructions at
% % alternative: \urldef{\JsrInstallingUrl}\url{http://types.cs.washington.edu/checker-framework/current/README-jsr308.html#installing}
% {\codesize\url{http://types.cs.washington.edu/checker-framework/current/README-jsr308.html#installing}}.
% This creates a \code{jsr308-langtools} directory.
% 
% \item Download the Checker Framework distribution zipfile from
% \myurl{http://types.cs.washington.edu/checker-framework/current/checkers.zip},
% and unzip it to create a \code{checkers} directory.  We recommend that the
% \code{checkers} directory and the \code{jsr308-langtools} directory be siblings.
% Example commands:
% 
% \begin{Verbatim}
%   cd $JSR308
%   wget http://types.cs.washington.edu/checker-framework/current/checkers.zip
%   unzip checkers.zip
% \end{Verbatim}
% 
% You will also need to adjust the path to \<javac> in any Ant buildfiles,
% etc.

% \item Optionally edit property \code{compiler.lib} in file
%   \code{checkers/build.properties}.  You don't have to do this if the
%   \code{checkers} directory and the \code{jsr308-langtools} directory are
%   siblings.


% (A checkers implementation builds on
% standard mechanisms such as JSR 269 annotation processing, but also
% accesses the compiler's AST. In the long run, a checker built using the
% Checker Framework should not be dependent on any compiler specifics.)
% If you do not place the annotations in 
% then you should also disable Eclipse's on-the-fly syntax checking.




% \subsection{TO DO:  The short instructions (for Linux only)}
% 
% %%% This comment does not seem to be correct any longer.
% %% This text is identically reproduced at ../../jsr308-langtools/README-jsr308.html
% %% so if you change either one, change the other also!
% 
% The following commands install
% the JSR 308 \code{javac} compiler and the Checker
% Framework, or update an existing installation.
% It currently works only on \textbf{Linux}.
% For more details, or if anything goes wrong, see the comments in the 
% \code{Makefile-jsr308-install} file.
% 
% \begin{enumerate}
% 
% \item
%   Execute the following commands:
% 
% \begin{Verbatim}
%   cd
%   wget -nv -N http://types.cs.washington.edu/jsr308/Makefile-jsr308-install
%   make -f Makefile-jsr308-install
% \end{Verbatim}
% 
% \item
% Set some environment variables according to the instructions at the top of file
% \code{Makefile-jsr308-install}.
% 
% \end{enumerate}



\section{Learning more\label{learning-more}}

The technical paper ``Practical pluggable types for Java''~\cite{PapiACPE2008}
(\myurl{http://www.cs.washington.edu/homes/mernst/pubs/pluggable-checkers-issta2008.pdf})
gives more technical detail about many
aspects of the Checker Framework and its implementation.
%
The technical
paper also describes case
studies in which each of the checkers found
previously-unknown errors in real software.


\section{Comparison to other tools\label{other-tools}}

A pluggable type-checker, such as those created by the Checker Framework,
aims to help you prevent or detect all errors of a given variety.  An
alternate approach is to use a bug detector such as
\ahref{http://findbugs.sourceforge.net/}{FindBugs},
\ahref{http://artho.com/jlint/}{JLint}, or
\ahref{http://pmd.sourceforge.net/}{PMD}.

A pluggable type-checker
differs from a bug detector in several ways:
\begin{itemize}
\item
  A type-checker aims to find \emph{all} errors.  Thus, it can verify the
  \emph{absence} of errors:  if the type checker says there are no null
  pointer errors in your code, then there are none.  (This guarantee only
  holds for the code it checks, of course; see
  Section~\ref{checker-guarantees}.)

  A bug detector aims to find \emph{some} of the most obvious errors.  Even
  if it reports no errors, then there may still be errors in your code.

  Both types of tools may issue false positive warnings; see
  Section~\ref{suppressing-warnings}.

\item
  A type-checker requires you to annotate your code with type qualifiers,
  or to run an inference tool that does so for you.  A bug detector may not
  require annotations.  This means that it may be easier to get started
  running a bug detector.

\item
  A type-checker may use a more sophisticated and complete analysis.
  A bug detector typically does a more lightweight analysis, coupled with
  heuristics to suppress false positives.

  As one example, a type-checker can take advantage of annotations on
  generic type parameters, such as \code{List<@NonNull String>}, permitting
  it to be much more precise for code that uses generics.

\end{itemize}

A case study~\cite[\S6]{PapiACPE2008} compared the Checker Framework's nullness
checker with those of FindBugs, JLint, and PMD\@.  The case study was on a
well-tested program in daily use.  The Checker Framework tool found 8
nullness errors.  None of the other tools found any errors.

Also see the
\ahref{http://types.cs.washington.edu/jsr308/}{JSR 308}~\cite{jsr308}
documentation for a detailed discussion of related work.



\section{Credits and changelog\label{credits}}

The key developers of the Checker Framework are Mahmood Ali, Telmo Correa,
Werner M. Dietl, Michael D. Ernst, and Matthew M. Papi.
Many other developers have also contributed, for example by writing
the checkers that are distributed with the Checker Framework.
Many, many users to list have provided valuable feedback, for which we are
grateful.

%% Not so accurate, since Mahmood is really an author of the nullness and
%% interned checkers too.
% The Checker Framework was implemented by 
% The nullness checker was implemented by Matthew M. Papi.
% The interning checker was implemented by Matthew M. Papi.
% The Javari checker was implemented by Telmo Correa.
% The IGJ checker was implemented by Mahmood Ali.
% The basic checker was implemented by Matthew M. Papi.
% The Fake enum checker was written by Werner M. Dietl.
% ... many others ...

Differences from previous versions of the checkers and framework can be found
in the \code{changelog-checkers.txt} file.  This file is included in the
Checker Framework distribution and is also available on the web at
\myurl{http://types.cs.washington.edu/checker-framework/current/changelog-checkers.txt}.






% LocalWords:  jsr unsetting plugins langtools zipfile cp plugin Nullness txt
% LocalWords:  nullness classpath NonNull MyObject javac uref changelog MyEnum
% LocalWords:  subtyping containsKey proc classfiles SourceChecker javap jdk
% LocalWords:  MyFile buildfiles


\htmlhr
\bibliographystyle{alpha}
% The .bib files are available from
% http://code.google.com/p/plume-bib/
% clone the project somewhere and set your BIBINPUTS accordingly.
\bibliography{bibstring-unabbrev,types,ernst,invariants,generals,alias,concurrency}

\end{document}

% LocalWords:  pt TODO JavaDocs Arg api HEVEA html ernst IGJ igj javari
