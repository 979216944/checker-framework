\htmlhr
\chapter{Third-party checkers\label{third-party-checkers}\label{external-checkers}}

The Checker Framework has been used to build other checkers that are not
distributed together with the framework.  This chapter mentions just a few
of them.

They are externally-maintained, so if you have problems or questions, you
should contact their maintainers rather than the Checker Framework
maintainers.

If you want a reference to your checker included in this chapter,
send us a link and a short description.


% Note to maintainers:
% Sections are added to this chapter in chronological order.


\section{Typestate checkers\label{typestate-checker}}

In a regular type system, a variable has the same type throughout its
scope.
In a typestate system, a variable's type can change as operations
are performed on it.

The most common example of typestate is for a \<File> object.  Assume a file
can be in two states, \<@Open> and \<@Closed>.  Calling the \<close()> method
changes the file's state.  Any subsequent attempt to read, write, or close
the file will lead to a run-time error.  It would be better for the type
system to warn about such problems, or guarantee their absence, at compile
time.

Just as you can extend the Subtyping Checker to create a type-checker, you can
extend a typestate checker to create a type-checker that supports typestate
analysis.  An extensible typestate analysis by Adam Warski that builds on
the Checker Framework is available at
\myurl{http://www.warski.org/typestate.html}.


\subsection{Comparison to flow-sensitive type refinement\label{typestate-vs-type-refinement}}

The Checker Framework's flow-sensitive type refinement
(Section~\ref{type-refinement}) implements a form of typestate analysis.
For example, after code that tests a variable against null, the Nullness
Checker (Chapter~\ref{nullness-checker}) treats the variable's type as
\<@NonNull \emph{T}>, for some \<\emph{T}>\@.

For many type systems, flow-sensitive type refinement is sufficient.  But
sometimes, you need full typestate analysis.  This section compares the
two.
% (Dependent types and unused variables
(Unused variables
% (Section~\ref{unused-fields-and-dependent-types})
(Section~\ref{unused-fields})
also have similarities
with typestate analysis and can occasionally substitute for it.  For
brevity, this discussion omits them.)

A typestate analysis is easier for a user to create or extend.
Flow-sensitive type refinement is built into the Checker Framework and is
optionally extended by each checker.  Modifying the rules requires writing
Java code in your checker.  By contrast, it is possible to write a simple
typestate checker declaratively, by writing annotations on the methods
(such as \<close()>) that change a reference's typestate.

A typestate analysis can change a reference's type to something that is not
consistent with its original definition.  For example, suppose that a
programmer decides that the \<@Open> and \<@Closed> qualifiers are
incomparable --- neither is a subtype of the other.  A typestate analysis
can specify that the \<close()> operation converts an \<@Open File> into a
\<@Closed File>.  By contrast, flow-sensitive type refinement can only give
a new type that is a subtype of the declared type --- for flow-sensitive
type refinement to be effective, \<@Closed> would need to be a child of
\<@Open> in the qualifier hierarchy (and \<close()> would need to be
treated specially by the checker).



\section{Units and dimensions checker\label{units-and-dimensions-checker}}

A checker for units and dimensions is available at
\url{http://www.lexspoon.org/expannots/}.

Unlike the Units Checker that is distributed with the Checker Framework
(see Section~\ref{units-checker}), this checker includes dynamic checks and
permits annotation arguments that are Java expressions.  This added
flexibility, however, requires that you use a special version both of the
Checker Framework and of the Type Annotations compiler.


\section{Thread locality checker\label{loci-thread-locality-checker}}

Loci, a checker for thread locality, is available at
\url{http://www.it.uu.se/research/upmarc/loci/}.
Developer resources are available at the project page
\url{http://java.net/projects/loci/}.

% A paper was publishd in ECOOP 2009, release 0.1 was made in March 2011,
% but as of October 2013 the manual is still listed as "forthcoming".


% In a mail from Amanj Mahmud <amanjpro@gmail.com> on 28.03.2011:

% The plugin name: 
% ``Loci: A Pluggable Type Checker for Expressing Thread Locality in
% Java''

% Project homepage: http://www.it.uu.se/research/upmarc/loci

% Project's developer's page: http://java.net/projects/loci


\section{Safety-Critical Java checker\label{safety-critical-java-checker}}

A checker for Safety-Critical Java (SCJ, JSR 302) is available at
\url{http://sss.cs.purdue.edu/projects/oscj/checker/checker.html}.
Developer resources are available at the project page
\url{http://code.google.com/p/scj-jsr302/}.


% In a mail from Aleš Plšek <aplsek@gmail.com> on 29.03.2011:

% Name: SCJ Checker
% WWW: http://sss.cs.purdue.edu/projects/oscj/checker/checker.html
% Source-Code Repository: http://code.google.com/p/scj-jsr302/

% Description: The SCJ Checker implements verification of a set of
% annotations defined by the Safety-Critical Java standard (JSR-302).
% The checker mainly focuses on proving memory safety of Java programs
% that use a region-based memory management.

% Publications: Static checking of safety critical Java annotations:
% http://portal.acm.org/citation.cfm?doid=1850771.1850792


\section{Generic Universe Types checker\label{gut-checker}}

A checker for Generic Universe Types, a lightweight ownership type
system, is available from
\url{https://ece.uwaterloo.ca/~wdietl/ownership/}.


\section{EnerJ checker\label{enerj-checker}}

A checker for EnerJ, an extension to Java that exposes hardware faults
in a safe, principled manner to save energy with only
slight sacrifices to the quality of service, is available from
\url{http://sampa.cs.washington.edu/research/approximation/enerj.html}.


\section{CheckLT taint checker\label{checklt-checker}}

CheckLT uses taint tracking to detect illegal information flows, such as
unsanitized data that could result in a SQL injection attack.
CheckLT is available from \url{http://checklt.github.io/}.


\section{JavaUI GUI threading checker\label{javaui-checker}}

Eclipse and other GUI toolkits require all access to the UI to occur from
the UI event loop thread.  If a different thread accesses a UI element, the
program crashes.  JavaUI is an effect system that ensures that you program
does not suffer such access errors.

The implementation is available at
\url{https://github.com/csgordon/javaui}.  
A fork that fixes some compilation errors appears at
\url{https://github.com/Overruler/javaui}.
You can also read a technical paper about the effect system:  ``JavaUI:
Effects for Controlling UI Object Access'' is available at
\url{http://homes.cs.washington.edu/~mernst/pubs/gui-thread-ecoop2013-abstract.html}.


% LocalWords:  SCJ EnerJ CheckLT unsanitized JavaUI
