\htmlhr
\chapter{Basic checker\label{basic-checker}}

The Basic checker enforces only subtyping rules.  It operates over
annotations specified by a user on the command line.  Thus, users can
create a simple type checker without writing any code beyond definitions of
the type qualifier annotations.

The Basic checker can accommodate all of the type system enhancements that
can be declaratively specified (see Chapter~\ref{writing-a-checker}).
This includes type introduction rules (implicit
annotations, e.g., literals are implicitly considered \code{@\refclass{nullness/quals}{NonNull}}) via
the \code{@\refclass{quals}{ImplicitFor}} meta-annotation, and other features such as
flow-sensitive type qualifier inference (Section~\ref{type-refinement}) and
qualifier polymorphism (Section~\ref{qualifier-polymorphism}).

The Basic checker is also useful to type system designers who wish to
experiment with a checker before writing code; the Basic checker
demonstrates the functionality that a checker inherits from the Checker
Framework.

If you need typestate analysis, then you can extend a typestate checker,
much as you would extend the Basic Checker if you do not need typestate
analysis.  For more details (including a definition of ``typestate''), see
Chapter~\ref{typestate-checker}.

For type systems that require special checks (e.g., warning about
dereferences of possibly-null values), you will need to write code and
extend the framework as discussed in Chapter~\ref{writing-a-checker}.


\section{Using the Basic checker\label{basic-using}}

The Basic checker is used in the same way as other checkers (using the
\code{-processor checkers.basic.BasicChecker} option; see Chapter~\ref{using-a-checker}), except that it
requires an additional annotation processor argument via the standard
``\code{-A}'' switch:

\begin{itemize}

\item
\code{-Aquals}: this option specifies a comma-no-space-separated list of
the fully-qualified class
names of the annotations used as qualifiers in the custom type system.  For
example,

\begin{alltt}
  javac -processor checkers.fenum.BasicChecker
        \textit{-Aquals=myproject.quals.MyQual,myproject.quals.OtherQual} MyFile.java ...
\end{alltt}

It serves the same purpose as the \code{@\refclass{quals}{TypeQualifiers}}
annotation used by other checkers (see section
\ref{writing-compiler-interface}).

The annotations listed in \code{-Aquals} must be accessible to
the compiler during compilation in the classpath.  In other words, they must
already be compiled before you run the Basic checker with \code{javac}; it
is not sufficient to supply their source files on the command line.

\end{itemize}

To suppress a warning issued by the basic checker, use a 
\code{@\sunjavadoc{java/lang/SuppressWarnings.html}{SuppressWarnings}}
annotation, with the argument being the unqualified, uncapitalized name of
any of the annotations passed to \code{-Aquals}.  This will suppress all
warnings, regardless of which of the annotations is involved in the
warning.  (As a matter of style, you should choose one of the annotations
as your \code{@SuppressWarnings} key and stick with it for that entire type
hierarchy.)


\section{Basic checker example\label{basic-example}\label{encrypted-example}}

Consider a hypothetical \code{Encrypted} type qualifier, which denotes that the
representation of an object (such as a \code{String}, \code{CharSequence}, or
\code{byte[]}) is encrypted. To use the Basic checker for the \code{Encrypted}
type system, follow three steps.

\begin{enumerate}
\item
 Define an annotation for the \code{Encrypted} qualifier:

\begin{Verbatim}
package myquals;

import checkers.quals.*;

/**
 * Denotes that the representation of an object is encrypted.
 * ...
 */
@TypeQualifier
@SubtypeOf(Unqualified.class)
@Target({ElementType.TYPE_PARAMETER, ElementType.TYPE_USE})
public @interface Encrypted {}
\end{Verbatim}

Don't forget to compile this class:

\begin{Verbatim}
$ javac myquals/Encrypted.java
\end{Verbatim}

The resulting \<.class> file should either be on your classpath, or on the
processor path (set via the \<-processorpath> command-line option to javac).

\item 
  Write \code{@Encrypted} annotations in your program (\code{YourProgram.java}):

\begin{Verbatim}
import myquals.Encrypted;

...

public @Encrypted String encrypt(String text) {
    // ...
}

// Only send encrypted data!
public void sendOverInternet(@Encrypted String msg) {
    // ...
}

void sendText() {
    // ...
    @Encrypted String ciphertext = encrypt(plaintext);
    sendOverInternet(ciphertext);
    // ...
}

void sendPassword() {
    String password = getUserPassword();
    sendOverInternet(password);
}
\end{Verbatim}

You may also need to add \code{@SuppressWarnings} annotations to the
\code{encrypt} and \code{decrypt} methods.  Analyzing them is beyond the
capability of any realistic type system.

\item
  Invoke the compiler with the Basic checker, specifying the
  \code{@Encrypted} annotation using the \code{-Aquals} option.
  You should add the \code{Encrypted} classfile to the processor classpath:

\begin{Verbatim}
$ javac -processorpath myqualspath -processor checkers.basic.BasicChecker \
        -Aquals=myquals.Encrypted YourProgram.java

YourProgram.java:42: incompatible types.
found   : java.lang.String
required: @myquals.Encrypted java.lang.String
    sendOverInternet(password);
                     ^
\end{Verbatim}

\end{enumerate}


\htmlhr
\chapter{Typestate checker\label{typestate-checker}}

In a regular type system, a variable has the same type throughout its
scope.
In a typestate system, a variable's type can change as operations
are performed on it.

The most common example of typestate is for a \<File> object.  Assume a file
can be in two states, \<@Open> and \<@Closed>.  Calling the \<close()> method
changes the file's state.  Any subsequent attempt to read, write, or close
the file will lead to a run-time error.  It would be better for the type
system to warn about such problems, or guarantee their absence, at compile
time.

Just as you can extend the Basic Checker to create a type checker, you can
extend a typestate checker to create a type checker that supports typestate
analysis.  Two extensible typestate analyses that build on the Checker
Framework are available.  One is by Adam Warski:
\myurl{http://www.warski.org/typestate.html}.
The other is by Daniel Wand:
\myurl{http://typestate.ewand.de/}.


\section{Comparison to flow-sensitive type refinement\label{typestate-vs-type-refinement}}

The Checker Framework's flow-sensitive type refinement
(Section~\ref{type-refinement}) implements a form of typestate analysis.
For example, after code that tests a variable against null, the Nullness
Checker (Chapter~\ref{nullness-checker}) treats the variable's type as
\<@NonNull \emph{T}>, for some \<\emph{T}>\@.

For many type systems, flow-sensitive type refinement is sufficient.  But
sometimes, you need full typestate analysis.  This section compares the
two.  (Dependent types and unused variables
(Section~\ref{unused-fields-and-dependent-types}) also have similarities
with typestate analysis and can occasionally substitute for it.  For
brevity, this discussion omits them.)

A typestate analysis is easier for a user to create or extend.
Flow-sensitive type refinement is built into the Checker Framework and is
optionally extended by each checker.  Modifying the rules requires writing
Java code in your checker.  By contrast, it is possible to write a simple
typestate checker declaratively, by writing annotations on the methods
(such as \<close()>) that change a reference's typestate.

A typestate analysis can change a reference's type to something that is not
consistent with its original definition.  For example, suppose that a
programmer decides that the \<@Open> and \<@Closed> qualifiers are
incomparable --- neither is a subtype of the other.  A typestate analysis
can specify that the \<close()> operation converts an \<@Open File> into a
\<@Closed File>.  By contrast, flow-sensitive type refinement can only give
a new type that is a subtype of the declared type --- for flow-sensitive
type refinement to be effective, \<@Closed> would need to be a child of
\<@Open> in the qualifier hierarchy (and \<close()> would need to be
treated specially by the checker).


% LocalWords:  TODO ImplicitFor Aquals TypeQualifiers sourcepath java NonNull
% LocalWords:  CharSequence classpath nullness quals SuppressWarnings classfile
% LocalWords:  uncapitalized processorpath Warski MyFile YourProgram
