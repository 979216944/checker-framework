\htmlhr
\chapter{Basic checker\label{basic-checker}}

The Basic checker enforces only subtyping rules.  It operates over
annotations specified by a user on the command line.  Thus, users can
create a simple type checker without writing any code beyond definitions of
the type qualifier annotations.

The Basic checker can accommodate all of the type system enhancements that
can be declaratively specified (see Chapter~\ref{writing-a-checker}).
This includes type introduction rules (implicit
annotations, e.g., literals are implicitly considered \code{@\refclass{nullness/quals}{NonNull}}) via
the \code{@\refclass{quals}{ImplicitFor}} meta-annotation, and other features such as
flow-sensitive type qualifier inference (Section~\ref{type-refinement}) and
qualifier polymorphism (Section~\ref{qualifier-polymorphism}).

The Basic checker is also useful to type system designers who wish to
experiment with a checker before writing code; the Basic checker
demonstrates the functionality that a checker inherits from the Checker
Framework.

For type systems that require special checks (e.g., warning about
dereferences of possibly-null values), you will need to write code and
extend the framework as discussed in Chapter~\ref{writing-a-checker}.


\section{Using the Basic checker\label{basic-using}}

The Basic checker is used in the same way as other checkers (using the
\code{-processor} option; see Chapter~\ref{using-a-checker}), except that it
requires an additional annotation processor argument via the standard
``\code{-A}'' switch:

\begin{itemize}

\item
\code{-Aquals}: this option specifies a comma-no-space-separated list of
the fully-qualified class
names of the annotations used as qualifiers in the custom type system.
%
It serves the same purpose as the \code{@\refclass{quals}{TypeQualifiers}}
annotation used by other checkers (see section
\ref{writing-compiler-interface}).

The annotations listed in \code{-Aquals} must be accessible to
the compiler during compilation in the classpath.  In other words, they must
already be compiled before you run the Basic checker with \code{javac}; it
is not sufficient to supply their source files on the command line.

\end{itemize}

To suppress a warning issued by the basic checker, use a 
\code{@\sunjavadoc{java/lang/SuppressWarnings.html}{SuppressWarnings}}
annotation, with the argument being the unqualified, uncapitalized name of
any of the annotations passed to \code{-Aquals}.


\section{Basic checker example\label{basic-example}\label{encrypted-example}}

Consider a hypothetical \code{Encrypted} type qualifier, which denotes that the
representation of an object (such as a \code{String}, \code{CharSequence}, or
\code{byte[]}) is encrypted. To use the Basic checker for the \code{Encrypted}
type system, follow three steps.

\begin{enumerate}
\item
 Define an annotation for the \code{Encrypted} qualifier:

\begin{Verbatim}
package myquals;

import checkers.quals.*;

/**
 * Denotes that the representation of an object is encrypted.
 * ...
 */
@TypeQualifier
@SubtypeOf(Unqualified.class)
public @interface Encrypted {}
\end{Verbatim}

Don't forget to compile this class:

\begin{Verbatim}
$ javac myquals/Encrypted.java
\end{Verbatim}

The resulting \<.class> file should either be on your classpath, or on the
processor path (set via the \<-processorpath> command-line option to javac).

\item 
  Write \code{@Encrypted} annotations in your program:

\begin{Verbatim}
import myquals.Encrypted;

...

public @Encrypted String encrypt(String text) {
    // ...
}

// Only send encrypted data!
public void sendOverInternet(@Encrypted String msg) {
    // ...
}

void sendText() {
    // ...
    @Encrypted String ciphertext = encrypt(plaintext);
    sendOverInternet(ciphertext);
    // ...
}

void sendPassword() {
    String password = getUserPassword();
    sendOverInternet(password);
}
\end{Verbatim}

You may also need to add \code{@SuppressWarnings} annotations to the
\code{encrypt} and \code{decrypt} methods.  Analyzing them is beyond the
capability of any realistic type system.

\item
  Invoke the compiler with the Basic checker, specifying the
  \code{@Encrypted} annotation using the \code{-Aquals} option.
  You should add the \code{Encrypted} classfile to the processor classpath:

\begin{Verbatim}
$ javac -processorpath myqualspath -processor checkers.basic.BasicChecker -Aquals=myquals.Encrypted YourProgram.java

YourProgram.java:42: incompatible types.
found   : java.lang.String
required: @myquals.Encrypted java.lang.String
    sendOverInternet(password);
                     ^
\end{Verbatim}

\end{enumerate}


% LocalWords:  TODO ImplicitFor Aquals TypeQualifiers sourcepath java NonNull
% LocalWords:  CharSequence classpath nullness quals SuppressWarnings classfile
% LocalWords:  uncapitalized processorpath
