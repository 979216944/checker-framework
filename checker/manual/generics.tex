\htmlhr
\chapter{Generics and polymorphism\label{polymorphism}}

This chapter describes support for Java generics (also known as
``parametric polymorphism'').
The chapter also describes polymorphism over type qualifiers.


\section{Generics (parametric polymorphism or type polymorphism)\label{generics}}

The Checker Framework fully supports
type-qualified Java generic types (also known in the research literature as ``parametric
polymorphism'').
When instantiating a generic type,
clients supply the qualifier along with the type argument, as in
\code{List<@NonNull String>}.


\subsection{Raw types\label{generics-raw-types}}

Before running any pluggable type-checker, we recommend that you eliminate
raw types from your code (e.g., your code should use \code{List<...>} as
opposed to \code{List}).
Your code should compile without warnings when using the standard Java
compiler and the \<-Xlint:unchecked -Xlint:rawtypes> command-line options.
Using generics helps prevent type errors just as using a pluggable
type-checker does, and makes the Checker Framework's warnings easier to
understand.

If your code uses raw types, then the Checker Framework will do its best to
infer the Java type parameters and the type qualifiers.  If it infers
imprecise types that lead to type-checking warnings elsewhere, then you have
two options.  You can convert the raw types such as \code{List} to
parameterized types such as \code{List<String>}, or you can supply the 
\<-AignoreRawTypeArguments> command-line option.  That option causes the
Checker Framework to ignore all subtype tests for type arguments that
were inferred for a raw type.


\subsection{Restricting instantiation of a generic class}

When you define a generic class in Java, the \<extends> or \<super> clause
of the generic type parameter restricts
how the class may be instantiated.  For example, given the definition
\verb|class G<T extends Number> {...}|,
a client can instantiate it as \code{G<Integer>} but not \code{G<Date>}.
Similarly, type qualifiers on the generic type parameters can restrict on
how the class may be instantiated.  For example, a generic list class might
indicate that it can hold only non-null values.  Similarly, a generic map
class might indicate it requires an immutable key type, but that it
supports both nullable and non-null value types.


There are two ways to restrict the type qualifiers that may be used on
the actual type argument when instantiating a generic class.

The first technique is the standard Java approach of using the
\code{extends} or \code{super} clause to supply an upper or lower bound.
For example:

\begin{Verbatim}
  MyClass<T extends @NonNull Object> { ... }

  MyClass<@NonNull String> m1;       // OK
  MyClass<@Nullable String> m2;      // error
\end{Verbatim}

The second technique is to write a type annotation on the declaration of a
generic type parameter, which specifies the exact annotation that is
required on the actual type argument, rather than just a bound.  For example:

\begin{Verbatim}
  class MyClassNN<@NonNull T> { ... }
  class MyClassNble<@Nullable T> { ... }

  MyClassNN<@NonNull Number> v1;     // OK
  MyClassNN<@Nullable Number> v2;    // error
  MyClassNble<@NonNull Number> v4;   // error
  MyClassNble<@Nullable Number> v3;  // OK
\end{Verbatim}

\subsection{A qualifier on a type parameter declaration is like two bounds\label{qualifier-is-like-two-bounds}}

A type annotation on a generic type parameter declaration can be thought of
as syntactic sugar for writing the annotation on both the \<extends> and the
\<super> clauses of the declaration.

Suppose that a type parameter declaration or a wildcard is annotated.  Then
the annotation applies to all bounds that have no explicit annotation.  In
other words, the
annotation applies to both the upper and lower bounds, except that any
explicitly-written bound annotation takes precedence.

\subsubsection{Type parameter declaration annotation examples\label{type-parameter-examples}}

For example, the following pairs of declarations have the same meaning
(except that the second declaration in each pair is not legal syntax).

\begin{Verbatim}
  class MyClass<@A T> { ... }
  class MyClass<T extends @A Object super @A void> { ... }

  class MyClass<@A T extends Number> { ... }
  class MyClass<T extends @A Number super @A void> { ... }

  class MyClass<@A T extends @B Number> { ... }
  class MyClass<T extends @B Number super @A void> { ... }

  class MyClass<@A T super Number> { ... }
  class MyClass<T extends @A Object super @A Number> { ... }

  class MyClass<@A T super @B Number> { ... }
  class MyClass<T extends @A Object super @B Number> { ... }
\end{Verbatim}


\subsubsection{Type parameter declaration annotation rationale\label{type-parameter-rationale}}

It is desirable to be able to constrain both the upper and the lower bound
of a type, as in

\begin{Verbatim}
  class MyClass<T extends @C MyUpperBound super @D void> { ... }
\end{Verbatim}

However, doing so is not possible due to two limitations of Java's syntax.
First, it is illegal to specify both the upper and the lower bound of a
type parameter or wildcard.
Second, it is impossible to specify a type annotation for a lower
bound without also specifying a type (use of \<void> is illegal).

Thus, when you wish to specify both bounds, you write one of them
explicitly, and you write the other one in front of the type variable name
or \<?>.  When you wish to specify two identical bounds, you write a
single annotation in front of the type variable name or \<?>.


\subsection{Examples of qualifiers on a type parameter}

Recall that \<@Nullable \emph{X}> is a supertype of \<@NonNull \emph{X}>,
for any \emph{X}\@.
We can see from Section~\ref{qualifier-is-like-two-bounds} that almost all
of the following types mean
different things:

\begin{Verbatim}
  class MyList1<@Nullable T> { ... }
  class MyList2<@NonNull T> { ... }
  class MyList3<T extends @Nullable Object> { ... }
  class MyList4<T extends @NonNull Object> { ... } // same as MyList2
\end{Verbatim}

\<MyList1> must be instantiated with a nullable type. 
The implementation must be able to consume (store) a null
value and produce (retrieve) a null value.

\<MyList2> and \<MyList4> must be instantiated with non-null type.
The implementation has to account for only non-null values --- it
does not have to account for consuming or producing null.

\<MyList3> may be instantiated either way:
with a nullable type or a non-null type.  The implementation must consider
that it may be instantiated either way --- flexible enough to support either
instantiation, yet rigorous enough to impose the correct constraints of the
specific instantiation.  It must also itself comply with the constraints of
the potential instantiations.

One way to express the difference
among \<MyList1>, \<MyList2>, \<MyList3>, and \<MyList4>
is by comparing what expressions are
legal in the implementation of the list --- that is, what expressions may
appear in the ellipsis, such as inside a method's body.  Suppose each class
has, in the ellipsis, these declarations:

\begin{Verbatim}
  T t;
  @Nullable T nble;      // Section "Type annotations on a use of a generic type variable", below,
  @NonNull T nn;         // further explains the meaning of "@Nullable T" and "@NonNull T".
  void add(T arg) { ... }
  T get(int i) { ... }
\end{Verbatim}

\noindent
Then the following expressions would be legal, inside a given implementation.
(Compilable source code appears as file
\<checker-framework/checker/tests/nullness/generics/GenericsExample.java>.)

\begin{tabular}{|l|c|c|c|c|} \hline
                        & MyList1 & MyList2 & MyList3 & MyList4 \\ \hline
  t = null;             & OK      & error   & error   & error   \\ \hline
  t = nble;             & OK      & error   & error   & error   \\ \hline
  nble = null;          & OK      & OK      & OK      & OK      \\ \hline
  nn = null;            & error   & error   & error   & error   \\ \hline
  t = this.get(0);      & OK      & OK      & OK      & OK      \\ \hline
  nble = this.get(0);   & OK      & OK      & OK      & OK      \\ \hline
  nn = this.get(0);     & error   & OK      & error   & OK      \\ \hline
  this.add(t);          & OK      & OK      & OK      & OK      \\ \hline
  this.add(nble);       & OK      & error   & error   & error   \\ \hline
  this.add(nn);         & OK      & OK      & OK      & OK      \\ \hline
\end{tabular}


\medskip

%% This text is not very helpful.
% The
% implementation of \code{MyList2} may only place non-null objects in the
% list and may assume that retrieved elements are non-null.  The
% implementation of \code{MyList3} is similar in that it may only place
% non-null objects in the list, because it might be instantiated as, say,
% \code{MyList3<@NonNull Date>}.  When retrieving elements from the list,
% the implementation of \code{MyList3} must account for the fact that
% elements of \code{MyList3} may be null, because it might be instantiated
% as, say, \code{MyList3<@Nullable Date>}.
The differences are more
significant when the qualifier hierarchy is more complicated than just
\<@Nullable> and \<@NonNull>.


\subsection{Type annotations on a use of a generic type variable}

A type annotation on a use of a generic type variable overrides/ignores any type
qualifier (in the same type hierarchy) on the corresponding actual type
argument.  For example, suppose that \code{T} is a formal type parameter.
Then using \code{@Nullable T} within the scope of \code{T} applies the type
qualifier \code{@Nullable} to the (unqualified) Java type of \code{T}\@.
This feature is only rarely used.

Here is an example of applying a type annotation to a generic type
variable:

\begin{Verbatim}
  class MyClass2<T> {
    ...
    @Nullable T myField = null;
    ...
  }
\end{Verbatim}

\noindent
The type annotation does not restrict how \code{MyClass2} may be
instantiated.  In other words, both
\code{MyClass2<@NonNull String>} and \code{MyClass2<@Nullable String>} are
legal, and in both cases \code{@Nullable T} means \code{@Nullable String}.
In \code{MyClass2<@Interned String>},
\code{@Nullable T} means \code{@Nullable @Interned String}.

% Note that a type annotation on a generic type variable does not act like
% other type qualifiers.  In both cases the type annotation acts as a type
% constructor, but as noted above they act slightly differently.


% %% This isn't quite right because a type qualifier is itself a type
% %% constructor.
% More formally, a type annotation on a generic type variable acts as a type
% constructor rather than a type qualifier.  Another example of a type
% constructor is \code{[]}.  Just as \code{T[]} is not the same type as
% \code{T}, \code{@Nullable T} is not (necessarily) the same type as
% \code{T}.


\subsection{Covariant type parameters\label{covariant-type-parameters}}

Java types are \emph{invariant} in their type parameter.  This means that
\code{A<X>} is a subtype of \code{B<Y>} only if \<X> is identical to \<Y>.  For
example, \code{ArrayList<Number>} is a subtype of \code{List<Number>}, but
neither \code{ArrayList<Integer>} nor \code{List<Integer>} is a subtype of
\code{List<Number>}.  (If they were, there would be a type hole in the Java
type system.)  For the same reason, type parameter annotations are treated
invariantly.  For example, \code{List<@Nullable String>} is not a subtype
of \code{List<String>}.

When a type parameter is used in a read-only way --- that is, when values
of that type are read but are never assigned --- then it is safe for the
type to be \emph{covariant} in the type parameter.  Use the \refqualclass{checker/nullness/qual}{Covariant} annotation to indicate
this.
When a type parameter is covariant, two instantiations of the class with
different type arguments have the same subtyping relationship as the type
arguments do.

For example, consider \<Iterator>.  Its elements can be read but not
written, so \code{Iterator<@Nullable String>} can be a subtype of
\code{Iterator<String>} without introducing a hole in the type system.
Therefore, its type parameter is annotated with \<@Covariant>.
The first type parameter of \<Map.Entry> is also covariant.
Another example would be the type parameter of a hypothetical class
\<ImmutableList>. 

The \<@Covariant> annotation is trusted but not checked.
If you incorrectly specify as covariant a type parameter that that can be
written (say, the class performs a
\<set> operation or some other mutation on an object of that type), then
you have created an unsoundness in the type system.
For example, it would be incorrect to annotate the type parameter of
\<ListIterator> as covariant, because \<ListIterator> supports a \<set>
operation.


\subsection{Method type argument inference and type qualifiers\label{infer-method-type-qualifiers}}

Sometimes method type argument inference does not interact well with
type qualifiers. In such situations, you might need to provide
explicit method type arguments, for which the syntax is as follows:

\begin{alltt}
	Collections.</*@MyTypeAnnotation*/ Object>sort(l, c);
\end{alltt}

\noindent
This uses Java's existing syntax for specifying a method call's type arguments.



\section{Qualifier polymorphism\label{qualifier-polymorphism}}

The Checker Framework also supports type \emph{qualifier} polymorphism for
methods, which permits a single method to have multiple different qualified
type signatures.  This is similar to Java's generics, but is used in
situations where you cannot use Java generics.  If you can use generics,
you typically do not need to use a polymorphic qualifier such as \<@PolyNull>.

To \emph{use} a polymorphic qualifier, just write it on a type.
For example, you can write \<@PolyNull> anywhere in a method that you would write
\<@NonNull> or \<@Nullable>.
A polymorphic qualifier can be used on a method signature or body.
It may not be used on a class or field.

A method written using a polymorphic qualifier conceptually has multiple
versions, somewhat like a template in C++ or the generics feature of Java.
In each version, each instance of the polymorphic qualifier has been
replaced by the same other qualifier from the hierarchy.  See the examples
below in Section~\ref{qualifier-polymorphism-examples}.

The method body must type-check with all signatures.  A method call is
type-correct if it type-checks under any one of the signatures.  If a call
matches multiple signatures, then the compiler uses the most specific
matching signature for the purpose of type-checking.  This is the same as
Java's rule for resolving overloaded methods.

To \emph{define} a polymorphic qualifier, mark the definition with
\refqualclass{framework/qual}{PolymorphicQualifier}.  For example,
\refqualclass{checker/nullness/qual}{PolyNull} is a polymorphic type
qualifier for the Nullness type system:

\begin{Verbatim}
  @PolymorphicQualifier
  @Target({ElementType.TYPE_USE, ElementType.TYPE_PARAMETER})
  public @interface PolyNull { }
\end{Verbatim}

\noindent
See Section~\ref{polyall} for a way you can sometimes avoid defining a new
polymorphic qualifier.


\subsection{Examples of using polymorphic qualifiers\label{qualifier-polymorphism-examples}}

As an example of the use of \<@PolyNull>, method
\sunjavadoc{java/lang/Class.html#cast(java.lang.Object)}{Class.cast}
returns null if and only if its argument is \<null>:

\begin{Verbatim}
  @PolyNull T cast(@PolyNull Object obj) { ... }
\end{Verbatim}

\noindent
This is like writing:

\begin{Verbatim}
   @NonNull T cast( @NonNull Object obj) { ... }
  @Nullable T cast(@Nullable Object obj) { ... }
\end{Verbatim}

\noindent
except that the latter is not legal Java, since it defines two
methods with the same Java signature.


As another example, consider

\begin{Verbatim}
  // Returns null if either argument is null.
  @PolyNull T max(@PolyNull T x, @PolyNull T y);
\end{Verbatim}

\noindent
which is like writing

\begin{Verbatim}
   @NonNull T max( @NonNull T x,  @NonNull T y);
  @Nullable T max(@Nullable T x, @Nullable T y);
\end{Verbatim}

\noindent
At a call site, the most specific applicable signature is selected.

Another way of thinking about which one of the two \code{max} variants is
selected is that the nullness annotations of (the declared types of) both
arguments are \emph{unified} to a type that is a supertype of both, also
known as the \emph{least upper bound} or lub.  If both
arguments are \code{@NonNull}, their unification (lub) is \<@NonNull>, and the
method return type is \<@NonNull>.  But if even one of the arguments is \<@Nullable>,
then the unification (lub) is \<@Nullable>, and so is the return type.



\subsection{Relationship to subtyping and generics\label{qualifier-polymorhism-vs-subtyping}}

Qualifier polymorphism has the same purpose and plays the same role as
Java's generics.  If a method is written using generics, it usually does
not need qualifier polymorphism.  If you have legacy code that is not
written generically, and you cannot change it to use generics, then you can
use qualifier polymorphism to achieve a similar effect, with respect to
type qualifiers only.  The base Java types are still treated non-generically.

Why not use ordinary subtyping to handle qualifier polymorphism?
Ordinarily, when you want a method to work on multiple types, you can just
use Java's subtyping.  For example, the \<equals> method is declared to
take an \<Object> as its first formal parameter, but it can be called on a
\<String> or a \<Date> because those are subtypes of \<Object>.

In most cases, the same subtyping mechanism works with type qualifiers.
\<String> is a supertype of \<@Interned String>, so a method \<toUpperCase>
that is declared to take a \<String> parameter can also be called on a
\<@Interned String> argument.

You use qualifier polymorphism in the same cases when you would use Java's
generics.  (If you can use Java's generics, then that is often better and
you don't also need to use qualifier polymorphism.)  One example is when
you want a method to operate on collections with different types of
elements.  Another example is when you want two different formal parameters
to be of the same type, without constraining them to be one specific type.


\subsection{Using multiple polymorphic qualifiers in a method signature\label{qualifier-polymorphism-multiple-qualifiers}}

%% I can't think of a non-clumsy way to say this.
% Each method containing a polymorphic qualifier is (conceptually) expanded
% into multiple versions completely independently.

Usually, it does not make sense to write only a single instance of a polymorphic
qualifier in a method definition:  if you write one instance of (say)
\<@PolyNull>, then you should use at least two.  (An exception is a
polymorphic qualifier on an array element type; this section ignores that
case, but see below for further details.)

For example, there is no point to writing

\begin{Verbatim}
  void m(@PolyNull Object obj)
\end{Verbatim}

\noindent
which expands to

\begin{Verbatim}
  void m(@NonNull Object obj)
  void m(@Nullable Object obj)
\end{Verbatim}

This is no different (in terms of which calls to the method will
type-check) than writing just

\begin{Verbatim}
  void m(@Nullable Object obj)
\end{Verbatim}

The benefit of polymorphic qualifiers comes when one is used multiple times
in a method, since then each instance turns into the same type qualifier.
Most frequently, the polymorphic qualifier appears on at least one formal
parameter and also on the return type.  It can also be useful to have
polymorphic qualifiers on (only) multiple formal parameters, especially if
the method side-effects one of its arguments.
For example, consider

\begin{Verbatim}
void moveBetweenStacks(Stack<@PolyNull Object> s1, Stack<@PolyNull Object> s2) {
  s1.push(s2.pop());
}
\end{Verbatim}

\noindent
In this example, if it is acceptable to rewrite your code to use Java
generics, the code can be even cleaner:

\begin{Verbatim}
<T> void moveBetweenStacks(Stack<T> s1, Stack<T> s2) {
  s1.push(s2.pop());
}
\end{Verbatim}


%% It would be nice to give an example that isn't too contrived.


\subsection{Using a single polymorphic qualifier on an element type\label{qualifier-polymorphism-element-types}}

There is an exception to the general rule that a polymorphic qualifier
should be used multiple times in a signature.  It can make sense to use a
polymorphic qualifier just once, if it is on an array or generic element
type.

For example, consider a routine that returns the index, in an array, of a
given element:

\begin{Verbatim}
  public static int indexOf(@PolyNull Object[] a, @Nullable Object elt) { ... }
\end{Verbatim}

If \<@PolyNull> were replaced with either \<@Nullable> or \<@NonNull>, then
one of these safe client calls would be rejected:

\begin{Verbatim}
  @Nullable Object[] a1;
  @NonNull Object[] a2;

  indexOf(a1, someObject);
  indexOf(a2, someObject);
\end{Verbatim}

Of course, it would be better style to use a generic method, as in either
of these signatures:

\begin{Verbatim}
 public static <T extends @Nullable Object> int indexOf(T[] a, @Nullable Object elt) { ... }
 public static <T extends @Nullable Object> int indexOf(T[] a, T elt) { ... }
\end{Verbatim}

The examples in this section use arrays, but analogous collection examples exist.

These examples show that use of a single polymorphic qualifier may be
necessary in legacy code, but can often be avoided by use of better code
style.


\subsection{The \code{@PolyAll} qualifier applies to every type system\label{polyall}}

Ordinarily, you have to create a new polymorphic type qualifier for each
type system you write.  This can be tedious.  More seriously, it can lead
to an explosion in the number of type annotations, if some method is
qualifier-polymorphic over multiple type qualifier hierarchies.

For example, a method that only performs \<==> on array elements will work
no matter what the array's element types are:

\begin{Verbatim}
  /** Searches for the first occurrence of the given element in the array,
   *  testing for equality using == (not the equals method). */
  public static int indexOfEq(@PolyAll Object[] a, @Nullable Object elt) {
    for (int i=0; i<a.length; i++)
      if (elt == a[i])
        return i;
    return -1;
  }
\end{Verbatim}

The \<@PolyAll> qualifier takes an optional argument so that you can
specify multiple, independent polymorphic type qualifiers.  For example,
the method also works no matter what the type argument on the second
argument is.  This signature is overly restrictive:

\begin{Verbatim}
  /** Returns true if the arrays are elementwise equal,
   *  testing for equality using == (not the equals method). */
  public static int eltwiseEqualUsingEq(@PolyAll Object[] a, @PolyAll Object elt) {
    for (int i=0; i<a.length; i++)
      if (elt != a[i])
        return false;
    return true;
  }
\end{Verbatim}

\noindent
That signature requires the element type annotation to be identical for the
two arguments.  For example, it forbids this invocation:

\begin{Verbatim}
    @Mutable Object[] x;
  @Immutable Object   y;
  ... indexOf(x, y) ...
\end{Verbatim}

\noindent
A better signature lets the two arrays' element types vary independently:

\begin{Verbatim}
  public static int eltwiseEqualUsingEq(@PolyAll(1) Object[] a, @PolyAll(2) Object elt)
\end{Verbatim}

\noindent
Note that in this case, the \<@Nullable> annotation on \<elt>'s type is no
longer necessary, since it is subsumed by \<@PolyAll>.

The \<@PolyAll> annotation applies to every type qualifier hierarchy for
which no explicit qualifier is written.  For example, a declaration like
\<@PolyAll @NonNull Object elt> is polymorphic over every type system
\emph{except} the nullness type system, for which the type is fixed at
\<@NonNull>.  That would be the proper declaration for \<elt> if the body
had used \<elt.equals(a[i])> instead of \<elt == a[i]>.


% Suppose that some type system has two qualifiers, such as 
% \<@Nullable> and \<@NonNull>.  When a polymorphic type qualifier such
% as \<@PolyNull> is used in a method, then the method conceptually
% has two different versions:  one in which every instance of
% \<@PolyNull> has been replaced by \<@NonNull> and one in
% which every instance of \<@PolyNull> has been replaced by
% \<@Nullable>.


% LocalWords:  nullable MyList nble nn Nullness DefaultQualifier MyClass quals
% LocalWords:  DefaultLocation subtype's ImmutableList ListIterator nullness
% LocalWords:  PolymorphicQualifier PolyNull java lub invariantly supertype's
%  LocalWords:  MyList1 MyList2 MyList4 MyList3 MyClass2 toUpperCase elt
%  LocalWords:  PolyAll arrays' Xlint rawtypes AignoreRawTypeArguments
%  LocalWords:  call's
