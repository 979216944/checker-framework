\htmlhr
\chapter{Constant Value Checker\label{value-checker}}

The Constant Value Checker is a constant propagation analysis: for
each variable, it determines whether that variable's value can be
known at compile time.


The Constant Vlaue Checker is automatically run by every other type
checker, but it is rarely used directly by programmers.

To run just the Value Checker, supply the
\code{-processor org.checkerframework.checker.value.ValueChecker}
command-line option to javac.

\section{Annotations}
\subsection{Value Annotations}

The Value Checker uses annotations to track values for all Java
primitives, their wrappers (e.g. int, and Integer), Strings, and the
length field of arrays. The types covered correspond to the names of
the annotations as shown in Figure~\ref{fig-value-hierarchy}

\begin{figure}
\includeimage{value-hierarchy3}{9cm}
\caption{The overall annotation hierarchy of the Value
annotations. Qualifiers in gray are used
internally by the type system but should never be written by a
programmer.}
\label{fig-value-hierarchy}
\end{figure}

Each annotation has an array of values representing all the values the
annotated type could be. This array is limited to 10 entries for
performance reasons. Therefore, if a variable, due to program logic,
could be more than 10 different values at any given point the Value
Checker gives up and it is annotated as @UnknownVal instead.

In addition to the hierarchy shown above, an annotation is also
considered a subtype of another annotation if its value array is a
subset of the value array of the other annotation (see
Figure~\ref{fig-value-subtype}. This allows the possible values of
variables to continue to be tracked even when code branches cannot be
predicted at compile-time as in Figure~\ref{fig-value-multivalue}.

\begin{figure}
\includeimage{value-subtype}{2.5cm}
\caption{Subtyping relationship between value annotations of the same type.}
\label{fig-value-subtype}
\end{figure}

\begin{figure}
\begin{Verbatim}
public void foo(boolean b) {
   int i = 1;     // i is @IntVal({1})
   if (b) {  
      i = 2;      // i is now @IntVal({2})
   }        
   // i is now @IntVal({1,2})
   
   i = i + 1;     // i is now @IntVal({2,3})
}
\end{Verbatim}
\caption{An example of the Value Checker accumulating possible values
    of a variable.}
\label{fig-value-multivalue}
\end{figure}

The third subtyping relationship among the value annotations is the
hierarchy of numeric value types. If two annotations satisfy the
subset relationship described above and one of them is numerically a
less precise type than the other it will be automatically ``casted''
up to the type of the other. The full extent of this subtyping is
shown in Figure~\ref{fig-value-subtyping}.

\begin{figure}
\includeimage{value-subtyping}{9cm}
\caption{Subtyping of different numeric value annotations.}
\label{fig-value-subtyping}
\end{figure}

\subsection{@Analyzable}
In addition to the value annotations on variables discussed above, the
Value Checker also uses a method annotation, @Analyzable. The Constant
Value Checker resolves operators (e.g. !, +, ==) and method calls when
performed on expressions with constant value types to track the values
returned. 

Annotating a method with @Analyzable causes the Constant Value Checker
to be able to track the return value of that method; if all arguments
to a call of an @Analyzable method are constant-value types then the
return value of that call is also assigned a constant-value type
corresponding to execution of the call.

@Analyzable methods must
be \refqualclass{dataflow/qual}{Pure} (side-effect free and
deterministic).

Additionally, @Analyzable methods and any methods they call must be on
the Classpath for the compiler because they are reflectively called at
compile-time to perform the constant value analysis.

\section{Warnings}
The Value Checker may issue warnings, mostly based on failing to
resolve methods marked as @Analyzable but which were not able to be
resolved when the checker was run. These will be displayed as warnings when
the checker is run but the only negative effect is that the return
value of the function will be @UnknownVal instead of being able to be
resolved to a specific value annotation.

