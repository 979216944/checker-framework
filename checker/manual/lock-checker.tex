\htmlhr
\chapter{Lock Checker\label{lock-checker}}

The Lock Checker prevents certain kinds of concurrency errors.  If the Lock
checker issues no warnings for a given program, then the program holds the
appropriate lock every time that it accesses a variable annotated with
\refqualclass{checker/lock/qual}{GuardedBy}.

Note:  This does \emph{not} mean that your program has \emph{no} concurrency
errors.  (You might have forgotten to annotate that a particular variable
should only be accessed when a lock is held.  You might release and
re-acquire the lock, when correctness requires you to hold it throughout a
computation.  And, there are other concurrency errors that cannot, or
should not, be solved with locks.)  However, ensuring that your
program obeys its locking discipline is an easy and effective way to
eliminate a common and important class of errors.


To run the Lock Checker, supply the
\code{-processor org.checkerframework.checker.lock.LockChecker}
command-line option to javac.


\section{Lock annotations\label{lock-annotations}}

\textbf{Summary of declaration annotations used by the Lock Checker.}\\

\begin{tabular}{l|l|l}
\textbf{Type annotation} & \textbf{Indicates} \\
@GuardedBy(String lock) &
Fields/variables only accessible after acquiring the given lock.
\\
\textbf{Declaration annotation} & \textbf{Indicates} \\
@Holding(String[] locks) &
Locks that must be held before the method is called.
\\
@EnsuresLockHeld(String[] expressions) &
Locks guaranteed to be held when the method returns.
\\
@EnsuresLockHeldIf(String[] expr, boolean result) &
Locks guaranteed to be held when the method returns the given result.
\\
@LockingFree &
The method does not make any use of locks or synchronization.
\\
\end{tabular}

\subsection{Type annotations for objects protected by locks\label{lock-annotations-objects}}

\begin{description}
\item[\refqualclass{checker/lock/qual}{GuardedBy}\small{(String lock)}]
  indicates a type whose value may be accessed only when the given lock is
  held.
  See the \href{api/org/checkerframework/checker/lock/qual/GuardedBy.html}{GuardedBy
    Javadoc} for an explanation of the argument and other details.  The lock
  acquisition and the value access may be arbitrarily far in the future;
  or, if the value is never accessed, the lock never need be held.
\end{description}

\subsection{Lock method annotations\label{lock-annotations-methods}}

The Lock Checker supports several annotations that specify method behavior.
These are declaration annotations, not type annotations: they apply to the
method itself rather than to some particular type.

\begin{description}
\item[\refqualclass{checker/lock/qual}{EnsuresLockHeld}\small{(String[] expressions)}]
\item[\refqualclass{checker/lock/qual}{EnsuresLockHeldIf}\small{(String[] expressions, boolean result)}]
  indicate a method postcondition. With \code{@EnsuresLockHeld}, the given
  expressions are known to be objects used as locks and are known to be in
  a locked state after the method returns; this is useful for annotating a
  method that takes a lock.
  With \code{@EnsuresLockHeldIf}, if the annotated method returns the given
  boolean value (true or false), the given
  expressions are known to be objects used as locks and are known to be in
  a locked state after the method returns; this is useful for annotating a
  method that conditionally takes a lock.
  See Section~\ref{ensureslockheld-examples} for examples.

\item[\refqualclass{dataflow/qual}{LockingFree}]
  indicates that the method does not use synchronization/locking,
  directly or indirectly.  This is used to facilitate dataflow
  analysis and is less restrictive than \refqualclass{dataflow/qual}{SideEffectFree}.
  It is especially useful for annotating library methods, including JDK methods.
  Since \code{@SideEffectFree} implies \code{@LockingFree}, if both
  are applicable then you should only write \code{@SideEffectFree}.
  
  \textbf{It is critical not to use this annotation
  for any method that uses synchronization/locking,
  directly or indirectly.}  This is because even methods that are guaranteed
  to release all locks they acquire could cause deadlocks.
  Although the Lock Checker currently does not aid with deadlock
  detection, this annotation must be used in anticipation that
  the Lock Checker eventually could.
\end{description}


\subsection{Discussion of \<@Holding>\label{lock-checker-holding}}

A programmer might choose to use the \code{@Holding} method annotation in
two different ways:  to specify a higher-level protocol, or to summarize
intended usage.  Both of these approaches are useful, and the Lock Checker
supports both.

\paragraph{Higher-level synchronization protocol\label{lock-checker-holding-highlevel}}

  \code{@Holding} can specify a higher-level synchronization protocol that
  is not expressible as locks over Java objects.  By requiring locks to be
  held, you can create higher-level protocol primitives without giving up
  the benefits of the annotations and checking of them.

\paragraph{Method summary that simplifies reasoning\label{lock-checker-holding-method-summary}}

  \code{@Holding} can be a method summary that simplifies reasoning.  In
  this case, the \code{@Holding} doesn't necessarily introduce a new
  correctness constraint; the program might be correct even if the lock
  were acquired later in the body of the method or in a method it calls, so
  long as the lock is acquired before accessing the data it protects.

  Rather, here \code{@Holding} expresses a fact about execution:  when
  execution reaches this point, the following locks are already held.  This
  fact enables people and tools to reason intra- rather than
  inter-procedurally.

  In Java, it is always legal to re-acquire a lock that is already held,
  and the re-acquisition always works.  Thus, whenever you write 

\begin{Verbatim}
  @Holding("myLock")
  void myMethod() {
    ...
  }
\end{Verbatim}

\noindent
it would be equivalent, from the point of view of which locks are held
during the body, to write

\begin{Verbatim}
  void myMethod() {
    synchronized (myLock) {   // no-op:  re-aquire a lock that is already held
      ...
    }
  }
\end{Verbatim}

The advantages of the \<@Holding> annotation include:
\begin{itemize}
\item
  The annotation documents the fact that the lock is intended to already be
  held.
\item
  The Lock Checker enforces that the lock is held when the method is
  called, rather than masking a programmer error by silently re-acquiring
  the lock.
\item
  The \<synchronized> statement can deadlock if, due to a programmer error,
  the lock is not already held.  The Lock Checker prevents this type of
  error.
\item
  The annotation has no run-time overhead.  Even if the lock re-acquisition
  succeeds, it still consumes time.
\end{itemize}

\section{Examples\label{lock-examples}}

\subsection{Examples of @GuardedBy and @Holding\label{lock-examples-guardedby-and-holding}}

The most common use of \code{@GuardedBy} is to annotate a field declaration
type.  However, other uses of \code{@GuardedBy} are possible.

\paragraph{Return types\label{lock-examples-guardedby-return}}

A return type may be annotated with \code{@GuardedBy}:

\begin{Verbatim}
  @GuardedBy("MyClass.myLock") Object myMethod() { ... }

  // reassignments without holding the lock are OK.
  @GuardedBy("MyClass.myLock") Object x = myMethod();
  @GuardedBy("MyClass.myLock") Object y = x;
  x.toString(); // ILLEGAL because the lock is not held
  synchronized(MyClass.myLock) {
    y.toString();  // OK: the lock is held
  }
\end{Verbatim}

\paragraph{Formal parameters\label{lock-examples-guardedby-formal-parameters}}

A parameter type may be annotated with \code{@GuardedBy}, which indicates that
the method body must acquire the lock before accessing the parameter.  A
client may pass a non-\code{@GuardedBy} reference as an argument, since it
is legal to access such a reference after the lock is acquired.

\begin{Verbatim}
  void helper1(@GuardedBy("MyClass.myLock") Object a) {
    a.toString(); // ILLEGAL: the lock is not held
    synchronized(MyClass.myLock) {
      a.toString();  // OK: the lock is held
    }
  }
  @Holding("MyClass.myLock")
  void helper2(@GuardedBy("MyClass.myLock") Object b) {
    b.toString(); // OK: the lock is held
  }
  void helper3(Object c) {
    helper1(c); // OK: passing a subtype in place of a the @GuardedBy supertype
    c.toString(); // OK: no lock constraints
  }
  void helper4(@GuardedBy("MyClass.myLock") Object d) {
    d.toString(); // ILLEGAL: the lock is not held
  }
  void myMethod2(@GuardedBy("MyClass.myLock") Object e) {
    helper1(e);  // OK to pass to another routine without holding the lock
    e.toString(); // ILLEGAL: the lock is not held
    synchronized (MyClass.myLock) {
      helper2(e);
      helper3(e);
      helper4(e); // OK, but helper4's body still does not type-check
    }
  }
\end{Verbatim}

\subsection{Examples of @EnsuresLockHeld and @EnsuresLockHeldIf\label{ensureslockheld-examples}}

\code{@EnsuresLockHeld} and \code{@EnsuresLockHeldIf} are primarily intended
for annotating JDK locking methods, as in:

\begin{Verbatim}
package java.util.concurrent.locks;
 
class ReentrantLock {
 
    @EnsuresLockHeld("this")
    public void lock();
 
    @EnsuresLockHeldIf (expression="this", result=true)
    public boolean tryLock();
 
[...]
}
\end{Verbatim}

They can also be used to annotate user methods, particularly for
higher-level lock constructs such as a Monitor, as in this simplified example:

\begin{Verbatim}
public class Monitor {
 
    private ReentrantLock lock; // Initialized in the constructor
 
[...]
 
    @EnsuresLockHeld("lock")
    public void enter() {
           lock.lock();
    }
 
[...]
}
\end{Verbatim}

\subsection{Example of @LockingFree\label{lockingfree-example}}

\code{@LockingFree} is useful when a method does not make any use of synchronization
or locks but causes other side effects (hence \code{@SideEffectFree} is not appropriate).
\code{@SideEffectFree} implies \code{@LockingFree}, therefore if both are applicable,
you should only write \code{@SideEffectFree}.


\begin{verbatim}
private Object myField;
private ReentrantLock lock; // Initialized in the constructor
private @GuardedBy("lock") Object x; // Initialized in the constructor
 
[...]
 
@LockingFree
// This method does not use locks or synchronization but cannot
// be annotated as @SideEffectFree since it alters myField.
void myMethod() {
    myField = new Object();
}

@SideEffectFree
int mySideEffectFreeMethod() {
    return 0;
}

void myUnlockingMethod() {
    lock.unlock();
}

void myUnannotatedEmptyMethod() {
}

void myOtherMethod() {
    if (lock.tryLock()) {
        x.toString(); // OK: the lock is held
        myMethod();
        x.toString(); // OK: the lock is still known to be held since myMethod is locking-free
        mySideEffectFreeMethod();
        x.toString(); // OK: the lock is still known to be held since mySideEffectFreeMethod
                      // is side-effect-free
        myUnlockingMethod();
        x.toString(); // ILLEGAL: myLockingMethod is not locking-free
    }
    if (lock.tryLock()) {
        x.toString(); // OK: the lock is held
        myUnannotatedEmptyMethod();
        x.toString(); // ILLEGAL: even though myUnannotatedEmptyMethod is empty, since it is
                      // not annotated with @LockingFree, the Lock Checker no longer knows
                      // the state of the lock.
        if (lock.isHeldByCurrentThread()) {
            x.toString(); // OK: the lock is known to be held
        }
    }
}
\end{verbatim}

\section{Other lock annotations\label{other-lock-annotations}}

The Checker Framework's lock annotations are similar to annotations used
elsewhere.

If your code is already annotated with a different lock
annotation, you can reuse that effort.  The Checker Framework comes with
cleanroom re-implementations of annotations from other tools.  It treats
them exactly as if you had written the corresponding annotation from the
Lock Checker, as described in Figure~\ref{fig-lock-refactoring}.


% These lists should be kept in sync with LockAnnotatedTypeFactory.java .
\begin{figure}
\begin{center}
% The ~ around the text makes things look better in Hevea (and not terrible
% in LaTeX).

\begin{tabular}{ll}
\begin{tabular}{|l|}
\hline
 ~net.jcip.annotations.GuardedBy~ \\ \hline
 ~javax.annotation.concurrent.GuardedBy~ \\ \hline
\end{tabular}
&
$\Rightarrow$
~org.checkerframework.checker.lock.qual.GuardedBy~
\end{tabular}
\end{center}
%BEGIN LATEX
\vspace{-1.5\baselineskip}
%END LATEX
\caption{Correspondence between other lock annotations and the
  Checker Framework's annotations.}
\label{fig-lock-refactoring}
\end{figure}

Alternately, the Checker Framework can process those other annotations (as
well as its own, if they also appear in your program).  The Checker
Framework has its own definition of the annotations on the left side of
Figure~\ref{fig-lock-refactoring}, so that they can be used as type
annotations.  The Checker Framework interprets them according to the right
side of Figure~\ref{fig-lock-refactoring}.


\subsection{Relationship to annotations in \emph{Java Concurrency in Practice}\label{jcip-annotations}}

The book \href{http://jcip.net/}{\emph{Java Concurrency in Practice}}~\cite{Goetz2006} defines a
\href{http://jcip.net.s3-website-us-east-1.amazonaws.com/annotations/doc/net/jcip/annotations/GuardedBy.html}{\code{@GuardedBy}} annotation that is the inspiration for ours.  The book's
\code{@GuardedBy} serves two related but distinct purposes:

\begin{itemize}
\item
  When applied to a field, it means that the given lock must be held when
  accessing the field.  The lock acquisition and the field access may be
  arbitrarily far in the future.
\item
  When applied to a method, it means that the given lock must be held by
  the caller at the time that the method is called --- in other words, at
  the time that execution passes the \code{@GuardedBy} annotation.
\end{itemize}

The Lock Checker renames the method annotation to
\refqualclass{checker/lock/qual}{Holding}, and it generalizes the 
\refqualclass{checker/lock/qual}{GuardedBy} annotation into a type annotation
that can apply not just to a field but to an arbitrary type (including the
type of a parameter, return value, local variable, generic type parameter,
etc.).  This makes the annotations more expressive and also more amenable
to automated checking.  It also accommodates the distinct
meanings of the two annotations, and resolves ambiguity when \<@GuardedBy>
is written in a location that might apply to either the method or the
return type.

(The JCIP book gives some rationales for reusing the annotation name for
two purposes.  One rationale is
that there are fewer annotations to learn.  Another rationale is
that both variables and methods are ``members'' that can be ``accessed'';
variables can be accessed by reading or writing them (putfield, getfield),
and methods can be accessed by calling them (invokevirtual,
invokeinterface):  in both cases, \code{@GuardedBy} creates preconditions
for accessing so-annotated members.  This informal intuition is
inappropriate for a tool that requires precise semantics.)

% It would not work to retain the name \code{@GuardedBy} but put it on the
% receiver; an annotation on the receiver indicates what lock must be held
% when it is accessed in the future, not what must have already been held
% when the method was called.


\section{Possible extensions\label{lock-extensions}}

The Lock Checker validates some uses of locks, but not all.  It would be
possible to enrich it with additional annotations.  This would increase the
programmer annotation burden, but would provide additional guarantees.

Lock ordering:  Specify that one lock must be acquired before or after
another, or specify a global ordering for all locks.  This would prevent
deadlock.

Not-holding:  Specify that a method must not be called if any of the listed
locks are held.

These features are supported by 
\href{http://clang.llvm.org/docs/ThreadSafetyAnalysis.html}{Clang's
  thread-safety analysis}.


\section{A note on Lock Checker internals\label{lock-checker-internals}}

The following type qualifiers are inferred and used internally by the Lock Checker
and should never need to be written by the programmer.  They are presented
here for reference on how the Lock Checker works and to help understand
warnings produced by the Lock Checker.  You may skip this section if you
are not seeing a warning mentioning \code{@LockHeld} or \code{@LockPossiblyHeld}.

These type qualifiers are used on the types of the objects that will be used
as locks to protect other objects.  The Lock Checker uses them to track the
current state of locks at a given point in the code.

\begin{description}

\item[\refqualclass{checker/lock/qual}{LockPossiblyHeld}]
  indicates a type that may be used as a lock to protect a field/variable
  (i.e. an object of this type may be used as the expression in
  a \code{@GuardedBy} annotation)
  and the lock may or may not be currently held.  Since any
  object can potentially be used as a lock, it in fact applies to all
  non-primitive types.  This is the default type qualifier in the
  hierarchy and it is the top type.

\item[\refqualclass{checker/lock/qual}{LockHeld}]
  indicates a type that may be used as a lock to protect a field/variable,
  and is currently in a locked state on the current thread.  It is a
  subtype of \code{@LockPossiblyHeld} and is the bottom type.

\end{description}

% LocalWords:  quals GuardedBy JCIP putfield getfield invokevirtual 5cm
% LocalWords:  invokeinterface threadsafety Clang's GuardedByTop cleanroom
%%  LocalWords:  api 5cm
