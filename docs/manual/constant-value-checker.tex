\htmlhr
\chapter{Constant Value Checker\label{constant-value-checker}}

The Constant Value Checker is a constant propagation analysis: for
each variable, it determines whether that variable's value can be
known at compile time.

There are two ways to run the Constant Value Checker.
\begin{itemize}
\item
Typically, it is automatically run by another type checker.
%% This is an implementation detail that does not belong in the user
%% manual, because users don't have to do anything.
% When using the Constant Value Checker as part of another checker, the
% \code{statically-executable.astub} file in the Constant Value Checker directory must
% be passed as a stub file for the checker.
\item
Alternately, you can run just the Constant Value Checker, by
supplying the following command-line options to javac:
\code{-processor org.checkerframework.common.value.ValueChecker -Astubs=statically-executable.astub}
\end{itemize}

\section{Annotations\label{constant-value-checker-annotations}}

The Constant Value Checker uses type annotations to indicate the value of
an expression (Section~\ref{constant-value-checker-type-annotations}), and
it uses method annotations to indicate methods that the Constant Value
Checker can execute at compile time
(Section~\ref{constant-value-staticallyexecutable-annotation}).


\subsection{Type Annotations\label{constant-value-checker-type-annotations}}

Typically, the programmer does not write any type annotations.  Rather, the
type annotations are inferred by the Constant Value Checker.
The programmer is also permitted to write type annotations.  This is only necessary in
locations where the Constant Value Checker does not infer annotations:  on fields
and method signatures.

The main type annotations are
\refqualclass{common/value/qual}{BoolVal},
\refqualclass{common/value/qual}{IntVal},
\refqualclass{common/value/qual}{IntRange},
\refqualclass{common/value/qual}{DoubleVal}, and
\refqualclass{common/value/qual}{StringVal}.
Additional type annotations for arrays and strings are
\refqualclass{common/value/qual}{ArrayLen},
\refqualclass{common/value/qual}{ArrayLenRange},
and \refqualclass{common/value/qual}{MinLen}.
A polymorphic qualifier (\refqualclass{common/value/qual}{PolyValue})
is also supported (see Section~\ref{qualifier-polymorphism}).

Each \<*Val> type annotation takes as an argument a set of values, and its
meaning is that at run time, the expression evaluates to one of the values.  For
example, an expression of type
\<\refqualclass{common/value/qual}{StringVal}("a", "b")> evaluates to
one of the values \<"a">, \<"b">, or \<null>.
The set is limited to 10 entries; if a variable
could be more than 10 different values, the Constant Value
Checker gives up and its type becomes
\refqualclass{common/value/qual}{IntRange} for integral types,
\refqualclass{common/value/qual}{ArrayLenRange} for array types,
\refqualclass{common/value/qual}{ArrayLen} or \refqualclass{common/value/qual}{ArrayLenRange} for \<String>, and
\refqualclass{common/value/qual}{UnknownVal} for all other types.
The \<@ArrayLen> annotation means that at run time, the expression
evaluates to an array or a string whose length is one of the annotation's arguments.

In the case of too many strings in \<@StringVal>, the values are forgotten
and just the lengths are used in \<@ArrayLen>.
If this would result in too many lengths,
only the minimum and maximum lengths are used in \<@ArrayLenRange>,
giving a range of possible lengths of the string.

The \<@StringVal> annotation may be applied to a char array.  Although byte
arrays are often converted to/from strings, the \<@StringVal> annotation may
not be applied to them.  This is because the conversion depends on the
platform's character set.

% \refqualclass{checker/value/qual}{BottomVal}, meaning that the expression
% is dead or always has the value \<null>.

\refqualclass{common/value/qual}{IntRange} takes two arguments --- a lower
bound and an upper bound.  Its meaning is that at run time, the expression
evaluates to a value between the bounds (inclusive).  For example, an
expression of type \<@IntRange(from=0, to=255)> evaluates to
0, 1, 2, \ldots, 254, or 255.
An \refqualclass{common/value/qual}{IntVal} and
\refqualclass{common/value/qual}{IntRange} annotation that represent the
same set of values are semantically identical and interchangeable:  they
have exactly the same meaning, and using either one has the same effect.
\refqualclass{common/value/qual}{ArrayLenRange} has the same relationship
to \refqualclass{common/value/qual}{ArrayLen} that
\refqualclass{common/value/qual}{IntRange} has to
\refqualclass{common/value/qual}{IntVal}.
The \<@MinLen> annotation is an alias for \<@ArrayLenRange> (meaning that every \<@MinLen> annotation
 is automatically converted to an \<@ArrayLenRange> annotation) that only takes
one argument, which is the lower bound of the range. The upper bound of the
range is the maximum integer value.

Figure~\ref{fig-value-hierarchy} shows the
subtyping relationship among the type annotations.
For two annotations of the same type, subtypes have a smaller set of
possible values, as also shown in the figure.
Because \<int> can be casted to \<double>, an \<@IntVal> annotation is a
subtype of a \<@DoubleVal> annotation with the same values.

\begin{figure}
\includeimage{value-subtyping}{7.9cm}
\caption{At the top, the type qualifier hierarchy of the Constant Value Checker
annotations. Qualifiers in gray are used
internally by the type system but should never be written by a
programmer.  At the bottom are examples of additional subtyping
relationships that depend on the annotations' arguments.}
\label{fig-value-hierarchy}
\end{figure}

Figure~\ref{fig-value-multivalue} illustrates how the Constant Value Checker
infers type annotations (using flow-sensitive type qualifier refinement, Section~\ref{type-refinement}).

\begin{figure}
\begin{Verbatim}
public void foo(boolean b) {
    int i = 1;     // i has type:  @IntVal({1}) int
    if (b) {
        i = 2;     // i now has type:  @IntVal({2}) int
    }
                   // i now has type:  @IntVal({1,2}) int
    i = i + 1;     // i now has type:  @IntVal({2,3}) int
}
\end{Verbatim}
\caption{The Constant Value Checker infers different types
  for a variable on different lines of the program.}
\label{fig-value-multivalue}
\end{figure}

If your code is already annotated with a different constant value or range
annotation, the Checker Framework can type-check your code.
It treats annotations from other tools
as if you had written the corresponding annotation from the
Constant Value Checker, as described in Figure~\ref{fig-constant-value-refactoring}.
If the other annotation is a declaration annotation, it may be moved; see
Section~\ref{declaration-annotations-moved}.


% These lists should be kept in sync with ValueAnnotatedTypeFactory.java .
\begin{figure}
\begin{center}
% The ~ around the text makes things look better in Hevea (and not terrible
% in LaTeX).
\begin{tabular}{ll}
\begin{tabular}{|l|}
\hline
 ~android.support.annotation.IntRange~ \\ \hline
\end{tabular}
&
$\Rightarrow$
~org.checkerframework.checker.common.value.qual.IntRange~
\end{tabular}
\end{center}
%BEGIN LATEX
\vspace{-1.5\baselineskip}
%END LATEX
\caption{Correspondence between other constant value and range annotations
  and the Checker Framework's annotations.}
\label{fig-constant-value-refactoring}
\end{figure}

\ifonbuffalo{\relax}\else{
  The Constant Value Checker trusts the
  \refqualclass{checker/index/qual}{Positive},
  \refqualclass{checker/index/qual}{NonNegative},
  and \refqualclass{checker/index/qual}{GTENegativeOne} annotations.  If your code
  contains any of these annotations, then
  in order to guarantee soundness, you must run the Index Checker whenever
  you run the Constant Value Checker.
}\fi

\subsection{Compile-time execution of expressions\label{constant-value-compile-time-execution}}

Whenever all the operands of an expression are compile-time constants (that
is, their types have constant-value type annotations), the Constant Value
Checker attempts to execute the expression.  This is independent of any
optimizations performed by the compiler and does not affect the code that
is generated.

The Constant Value Checker statically executes operators that do
not throw exceptions (e.g., \<+>, \<->, \code{<\relax<}, \<!=>).


\subsection{\<@StaticallyExecutable> methods and the classpath\label{constant-value-staticallyexecutable-annotation}}

The Constant Value Checker statically executes methods annotated with
\refqualclass{common/value/qual}{StaticallyExecutable}, \emph{if the
method has already been compiled and is on the classpath}.

\begin{figure}
\begin{Verbatim}
@StaticallyExecutable @Pure
public int foo(int a, int b) {
    return a + b;
}

public void bar() {
    int a = 5;          // a has type:  @IntVal({5}) int
    int b = 4;          // b has type:  @IntVal({4}) int
    int c = foo(a, b);  // c has type:  @IntVal({9}) int
}
\end{Verbatim}
\caption{The
  \refqualclass{common/value/qual}{StaticallyExecutable} annotation enables
  constant propagation through method calls.}
\label{fig-staticallyexecutable}
\end{figure}

A \<@StaticallyExecutable> method must
be \refqualclass{dataflow/qual}{Pure} (side-effect-free and
deterministic).

Additionally, a \<@StaticallyExecutable> method and any method it calls must be on
the classpath for the compiler, because they are reflectively called at
compile-time to perform the constant value analysis.
% Standard library methods (such as those annotated as \<@StaticallyExecutable>
% in file \<statically-executable.astub>) will already be on the classpath.
To use \<@StaticallyExecutable> on methods in your own code, you should
first compile the code without the Constant Value Checker and then add
the location of the resulting \code{.class} files to the
classpath. For example, the command-line arguments to the Checker Framework
might include:
\begin{Verbatim}
  -processor org.checkerframework.common.value.ValueChecker
  -Astubs=statically-executable.astub
  -classpath $CLASSPATH:MY_PROJECT/build/
\end{Verbatim}


\section{Warnings\label{value-checker-warnings}}

If the option \code{-AreportEvalWarns} options is used, the Constant Value Checker issues a warning if it cannot load and run, at
compile time, a method marked as \<@StaticallyExecutable>.  If it issues
such a warning, then the return value of the method will be \<@UnknownVal>
instead of being able to be resolved to a specific value annotation.
Some examples of these:
% This section describes potentially-confusing messages, not every message.

\begin{sloppypar}
\begin{itemize}
\item \code{[class.find.failed] Failed to find class named Test.}

  The checker could not find the class
  specified for resolving a \<@StaticallyExecutable> method. Typically
  this is caused by not providing the path of a class-file needed to
  the classpath.

\item \code{[method.find.failed] Failed to find a method named foo with argument types [@IntVal(3) int].}

  The checker could not find the method \code{foo(int)} specified for
  resolving a \<@StaticallyExecutable> method, but could find the
  class. This is usually due to providing an outdated version of the
  class-file that does not contain the
  method that was annotated as \<@StaticallyExecutable>.

\item \code{[method.evaluation.exception] Failed to evaluate method public static int Test.foo(int) because it threw an exception: java.lang.ArithmeticException: / by zero.}

  An exception was thrown when trying to statically execute the
  method. In this case it was a divide-by-zero exception. If the
  arguments to the method each only had one value in their annotations
  then this exception will always occur when the program is actually
  run as well. If there are multiple possible values then the exception
  might not be thrown on every execution, depending on the run-time values.

\end{itemize}
\end{sloppypar}

There are some other situations in which the Constant Value Checker produces a
warning message:

\begin{sloppypar}
\begin{itemize}
\item \code{[too.many.values.given] The maximum number of arguments permitted is 10.}

  The Constant Value Checker only tracks up to 10 possible values for an
  expression.  If you write an annotation with more values than will be
  tracked, the annotation is replaced with \<@IntRange>, \<@ArrayLen>, \<@ArrayLenRange>, or \<@UnknownVal>.

\end{itemize}
\end{sloppypar}


\section{Unsoundly ignoring overflow\label{value-checker-overflow}}

The Constant Value Checker takes Java's overflow rules into account when
computing the possible values of expressions.
%
The \code{-AignoreRangeOverflow} command-line option makes it ignore the
possibility of overflow for range annotations
\refqualclass{common/value/qual}{IntRange} and
\refqualclass{common/value/qual}{ArrayLenRange}.
%
Figure~\ref{fig-value-ignore-overflow} gives an example of behavior with
and without the \code{-AignoreRangeOverflow} command-line option.

\begin{figure}
\begin{Verbatim}
  ...
  if (i > 5) {
    // i now has type:  @IntRange(from=5, to=Integer.MAX_VALUE)
    i = i + 1;
    // If i started out as Integer.MAX_VALUE, then i is now Integer.MIN_VALUE.
    // i's type is now @IntRange(from=Integer.MIN_VALUE, to=Integer.MAX_VALUE).
    // When ignoring overflow, i's type is now @IntRange(from=6, to=Integer.MAX_VALUE).
  }
\end{Verbatim}
\caption{With the \code{-AignoreRangeOverflow} command-line option,
the Constant Value Checker ignores overflow
for range types, which gives smaller ranges to range types.}
\label{fig-value-ignore-overflow}
\end{figure}

As with any unsound behavior in the Checker Framework, this option reduces
the number of warnings and errors produced, and may reduce the number of
\<@IntRange> qualifiers that you need to write in the source code.
However, it is possible that at run time, an expression might evaluate to a
value that is not in its \<@IntRange> qualifier.  You should either accept
that possibility, or verify the lack of overflow using some other tool or
manual analysis.


%%  LocalWords:  UnknownVal StringValue BottomVal astub Astubs IntRange
%  LocalWords:  StaticallyExecutable BoolVal IntVal DoubleVal StringVal
%%  LocalWords:  classpath AreportEvalWarns ArrayLen ArrayLenRange casted
%  LocalWords:  qual AignoreRangeOverflow MinLen PolyValue
