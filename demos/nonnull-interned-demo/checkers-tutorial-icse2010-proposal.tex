%% This tutorial proposal was accepted, but not presented due to
%% difficulties with travel.

% -*- Mode: LaTeX -*-
\documentclass{sig-alternate} % 10pt if not specified
% \usepackage{pslatex}
% \usepackage{fullpage}
\usepackage{relsize}
\usepackage{url}

\begin{document}

%
% --- Author Metadata here ---
% \conferenceinfo{WOODSTOCK}{'97 El Paso, Texas USA}
%\CopyrightYear{2007} % Allows default copyright year (200X) to be over-ridden - IF NEED BE.
%\crdata{0-12345-67-8/90/01}  % Allows default copyright data (0-89791-88-6/97/05) to be over-ridden - IF NEED BE.
% --- End of Author Metadata ---

\title{Building and using pluggable type systems}
\numberofauthors{2}
\author{
\alignauthor
Michael D. Ernst\\
       \affaddr{University of Washington}\\
       \affaddr{CS\&E, Box 352350}\\
       \affaddr{Seattle, WA, USA}\\
       \affaddr{+1-206-221-0965; fax: +1-206-616-3804}\\
       \affaddr{\url{http://homes.cs.washington.edu/~mernst/}}\\
       \email{mernst@cs.washington.edu}
% 2nd. author
\alignauthor
Mahmood Ali\\
       \affaddr{MIT CSAIL}\\
       \affaddr{32 Vassar St.}\\
       \affaddr{Cambridge, MA, USA}\\
       \affaddr{+1-617-253-5895; fax: +1-617-258-8682}\\
       \email{mali@csail.mit.edu}
}

\maketitle
\begin{abstract}
Are you a practitioner who is tired of null pointer exceptions,
unintended side effects, SQL injections, concurrency errors, mistaken
equality tests, and other run-time errors that appear during testing or
in the field?  A pluggable type system can guarantee the absence of these
errors, and many more.

Are you a researcher who wants to be able to quickly and easily implement
a type system, giving you the ability to evaluate it in practice and to
field it?  You need a framework that supports these essential activities.

This tutorial is aimed at both audiences.  It describes both the theory
of pluggable types and also the practice of implementing them.  A simple
pluggable type-checker can be implemented in 2 minutes, and can be
enhanced thereafter.  The type checkers are easy to create, easy for
programmers to use, and effective in finding real, important bugs.

The specific examples and exercises use the Checker Framework, which
enables you to create pluggable type systems for Java.  However, the
ideas translate to other languages and toolsets.  Java 7 will contain
syntactic support for type annotations, but in the meanwhile your code
remains backward-compatible with all versions of Java.

\end{abstract}

% % A category with the (minimum) three required fields
% \category{H.4}{Information Systems Applications}{Miscellaneous}
% %A category including the fourth, optional field follows...
% \category{D.2.8}{Software Engineering}{Metrics}[complexity measures, performance measures]
% 
% \terms{Delphi theory}
% 
% \keywords{ACM proceedings, \LaTeX, text tagging}

\section{Presenters}

Mahmood Ali is a graduate student in the MIT Computer Science and
Artificial Intelligence Lab.  He designed and implemented the Type
Annotations (JSR 308) compiler that has been incorporated into Sun's
OpenJDK, and the Checker Framework that enables the creation of
user-defined pluggable type systems.

Michael D. Ernst is an Associate Professor in the Computer Science and
Engineering department at the University of Washington.  He is the
specification lead for the Type Annotations language extension (``JSR
308'') that is part of Java 7 --- the first non-Sun-employee to be the
specification lead for a Java language change.  Ernst's research aims to
make software more reliable, more secure, and easier (and more fun!) to
produce. His primary technical interests are in software engineering and
related areas, including programming languages, type theory, security,
program analysis, bug prediction, testing, and verification. Ernst's
research combines strong theoretical foundations with realistic
experimentation, with an eye to changing the way that software developers
work.  Dr. Ernst was previously a tenured professor at MIT, and before that
a researcher at Microsoft Research.  More information is available at his
homepage:  \url{http://homes.cs.washington.edu/~mernst/}.

Ernst received a ``Java Rock Star'' award for his presentation on the
Checker Framework at the JavaOne 2009 conference.  He and/or Ali have also
given short (1-hour) presentations on the topic at Dagstuhl (4/2008),
J-Spring (4/2008), ECOOP (7/2008), OOPSLA (10/2008), Devoxx (12/2008),
Google (12/2008), and SeaJUG (4/2009).  Additionally, they have done longer
(2.5--4 hour) tutorials at PPPJ (8/2009) and OOPSLA (10/2009).  This list
does not include older presentations (over 18 months ago).


\section{Duration}

Either half-day or full-day.  The latter permits more depth, and more time
for hands-on exercises.


\section{Goal and objectives}

Our aim is to bring the research and practitioner communities closer
together, in the concrete context of type systems, by offering each
community something that it has lacked so far.

For researchers, it is too hard to evaluate a proposed type system in a
realistic way, which requires integration with a full, real programming
language and evaluation on real, sizable programs.  It is also too hard to
package type systems for use by practitioners, even after they have been
demonstrated to be useful.

For practitioners, many important properties of a program must be expressed
informally in English, and no tools exist to detect and prevent these
problems.

We will demonstrate that all of these problems are soluble and have even
been solved.  We will show how to achieve each of these aims, and will lead
participants through solving whichever ones are of most interest to them.


\section{Scope}

The audience is twofold:  researchers who wish to evaluate their type
systems in a realistic manner or transition them to practical use; and
practitioners who wish to create reliable code.

Participants should be comfortable with the type system of an
object-oriented language such as Java.



\section{Teaching method}

The tutorial will use a mix of teaching methods.  There will be a short
preliminary lecture to introduce concepts, and we will return to this
format periodically to convey more background theory.  There will be
numerous demos to illustrate how both type systems, and tools for
implementing them, work in practice.  There will be hands-on exercises to
get participants started in using pluggable type-checkers and writing their
own.


\section{Summary of contents}

The Java type system helps programmers to detect and prevent
errors. However, Java's built-in type system does not detect and prevent
\emph{enough} errors. It cannot express important properties such as: a reference
is non-null, a data structure should not be modified, or a string has been
properly sanitized.

Java's type system can be extended to enforce such properties statically,
preventing whole classes of errors. 
A user-defined, or pluggable, type system enriches a
language's built-in type system via type qualifiers.  Pluggable types
permit more expressive compile-time checking, and they can guarantee
the absence of additional errors.

Despite considerable interest in user-defined type qualifiers,
previous implementations have been too inexpressive, unscalable, or
incompatible with existing languages or tools.  This has hindered the
evaluation, understanding, and uptake of pluggable types.

More recently, pluggable types (that operate as an optional plug-in to the
compiler) have become practical for Java. This benefits two constituencies:

\begin{itemize}
\item Programmers can detect and prevent errors in their code.
\item Type system designers can easily evaluate new type systems in the
  context of an industrial programming language.
\end{itemize}

The line between these constituencies is porous: a developer may create a
simple type system to enforce an application-specific property.

This tutorial will explain how to both \emph{use} and \emph{create}
pluggable type systems. Afterward, you will be ready to use pluggable types
with confidence and to create simple type-checkers of your own, and you
will know where to look for more information.

The Checker Framework is expressive and flexible.  It builds in many
features needed by type system designers and programmers, including:
\begin{itemize}
\item backward-compatible Java and classfile syntax for type qualifiers
\item integration with javac and Eclipse
\item three type inference tools to ease programmer annotation burden
\item declarative and procedural syntax for writing type-checking rules
\item flow-sensitive type qualifier inference
\item polymorphism over types (Java generics)
\item polymorphism over type qualifiers
\item implicit and default qualifiers
\end{itemize}

The hands-on activities will use the Checker Framework, which provides
expressive, concise declarative and procedural mechanisms for creating
pluggable type checkers for Java. However, the ideas extend to other
frameworks and languages. The Checker Framework is the basis of many
practical checkers that are in daily use and have found hundreds of
important errors in millions of lines of code. The Checker Framework is
being used by researchers worldwide, and it has yielded new insight into
research type systems. It is usable with any version of the Java language,
but pluggable types are so important that Java 7 will contain special
syntax (designed by the presenter of this tutorial) to support type
qualifiers.

The tutorial will cover both theory and practice, and is applicable both to
researchers and also to practitioners who wish to verify domain-specific
properties in their software. For example, you will learn how type system
designers can build scalable, robust, and easily deployable
type-checkers. During the tutorial, you will build your own
type-checker. After the tutorial, you will be prepared to use, create, or
customize pluggable type-checkers.

Attendees are suggested to bring laptop computers, though it is also
possible to look on with another attendee.

The Checker Framework is freely available at
\url{http://types.cs.washington.edu/jsr308/}.
The distribution includes source code, binaries, extensive
documentation, and example type-checkers.



\section{Areas of special interest}

Among the areas of special interest to the Southern African software
engineering community, ours does not relate to ``cost effectiveness
analysis''.  It does relate, in greater or lesser degree, to three:
\begin{itemize}
\item security:  we will demonstrate how pluggable type-checking can reveal
  and prevent a variety of security flaws, including information flows,
  injection/XSS, and representation exposure of mutable state.
\item advanced software testing:
  testing is one aspect of verification and validation; static analyses are
  another side of the same coin.  Our approach complements dynamic
  approaches, and we have effectively combined them.
\item model-driven language engineering:
  the types of a program are an abstraction --- a model --- of the program
  and/or of its specification; and when these are machine-checkable, they
  are of more immediate use.
\end{itemize}

The tutorial is relevant because it is a rare example of academic research
having a direct effect on industrial practice and becoming available to
every Java developer via support in the Java language standard.


\section{Structure of contents and sample slides}

I have attached a slide deck that has the same content that I would provide
in a ``structure of contents'' section --- I would take slide titles and
contents as the section titles and bullet points.


\section{References}

The two most relevant documents are the Type Annotations (JSR 308)
Specification~\cite{JSR308} that is part of Java 7, and a research paper on
building and using pluggable type-systems~\cite{PapiACPE2008}.  Also see
the 82-page Checker Framework Manual, available from
\url{http://types.cs.washington.edu/jsr308/} in PDF and HTML form.

\bibliographystyle{alpha}
\bibliography{bibstring-unabbrev,ernst}

\end{document}





Tutorial Proposal Submission Template

Goal and Objectives
The overall goal of the tutorial and some concrete objectives to be achieved. Please ensure that it is clear why this tutorial would be of interest and benefit to ICSE 2010 attendees
Scope
Summary of Contents (not exceeding two pages)
Here you summarize your tutorial in the style of an extended abstract. The summary must not exceed 1500 words or two pages in ICSE conference proceedings format.

Please note that this section should be the preliminary version of the two-page summary in the proceedings that you will be required to submit if your proposal is accepted.
Structure of Contents
Here you should provide a structured overview of your planned tutorial, organized into numbered sections and sub-sections. For each sub-section, you should sketch its contents in a few sentences or bullets.
Sample Slides

Please also prepare three sample slides from the presentation that you would give if your tutorial were accepted. Select slides that are typical for your presentation style. These slides must be submitted in a separate PDF file with all fonts included.



%%% Local Variables: 
%%% mode: latex
%%% TeX-master: t
%%% End: 

% LocalWords:  Devoxx SeaJUG PPPJ deployable nonnull readonly classfile XSS
% LocalWords:  checkable toolsets Dagstuhl ernst
