% -*- Mode: LaTeX -*-
\documentclass{article}

\usepackage{url}

% \|name| or \mathid{name} denotes identifiers and slots in formulas
\def\|#1|{\mathid{#1}}
\newcommand{\mathid}[1]{\ensuremath{\mathit{#1}}}
% \<name> or \codeid{name} denotes computer code identifiers
\def\<#1>{\codeid{#1}}
\newcommand{\codeid}[1]{\ifmmode{\mbox{\ttfamily{#1}}}\else{\ttfamily #1}\fi}
\newcommand{\code}[1]{\ifmmode{\mbox{\ttfamily{#1}}}\else{\ttfamily #1}\fi}


\begin{document}

\title{Preventing runtime errors at compile time
% with the Checker Framework
}

\author{
Michael D. Ernst
\qquad
Werner Dietl\\
University of Washington\\
\url{{mernst,wmdietl}@cs.washington.edu}
}

\maketitle


This document contains the data that we will submit in the OSCON Java
web form.


\section{Proposal title}

== document title.


\section{Description}

Brief overview for marketing purposes, max. length 400
characters---about 65 words:

\medskip

Are you tired of null pointer exceptions, unintended side effects, SQL
injections, concurrency errors, mistaken equality tests, and other run-time
errors that appear during testing or in the field?  A compile-time tool
named the Checker Framework has found hundreds of such errors.  Oracle
plans to include it in the Java 8 javac, but you can use it today to
improve your code and avoid errors.



\section{Topics}

Select the most relevant to your proposal:

\noindent
Java: Tools \& techniques

\medskip

\noindent
Other options:\\
Java: Cloud\\
Java: Community\\
Java: Groovy\\
Java: JVM languages\\
Java: Mobile\\
Java: Rich client\\
Java: Scala\\
Java: Server side\\
Java: Other


\section{Session type}

40 min presentation\\
90 min workshop

We can do either, preferably the 90 min workshop.


\section{Abstract}

The form doesn't say how long this can be...

\medskip

Are you tired of null pointer exceptions, unintended side effects, SQL
injections, concurrency errors, mistaken equality tests, and other run-time
errors that appear during testing or in the field?  A pluggable type system
can guarantee the absence of these errors, and many more real, important
bugs.

Are you a software architect who wants to be able to quickly and easily
implement custom checks that prevent more errors at compile time?  You need
a framework that supports you in creating a formally correct code checker.

This presentation is aimed at both audiences.  A pluggable type system can
give a compile-time guarantee of important properties.  We will explain
what it is, how to use it, and even how to create your own.  You can create
a simple pluggable type-checker in 2 minutes, and you can enhance it
thereafter.

The demo uses the Checker Framework, which enables you to create pluggable
type systems for Java.
It takes advantage of features planned by Oracle for Java 8, but your code
remains backward-compatible.
The pluggable type-checker can be run as part of javac or via an Eclipse
plug-in, and integration with build tools such as Ant and Maven is
provided.
The tools are freely available at
\url{http://types.cs.washington.edu/checker-framework/}.

The Checker Framework provides 12 pluggable type systems that are
ready to use, including nullness, immutability, and
locking checkers.
The presentation will first develop a simple declarative type checker
that checks the consistency of method signature strings.
The presentation will then discuss the design and usage of more
advanced checkers.

The Checker Framework has found hundreds of bugs in over 3 million lines of
open source code, including from Oracle, Google, Apache, etc.  Overall, we
found that the type checkers were easy to write, easy for novices to
productively use, and effective in finding real bugs and verifying program
properties, even for widely tested and used open source projects.  It is
easy to improve the quality of your Java code, and you can start using the
Checker Framework today!



\section{Skill Level of this session}

Beginning\\
X Intermediate\\
Advanced


\section{What knowledge/skills should attendees have in order to get
  the most from this session?}

A basic understanding of the Java type system is assumed.

Attendants are encouraged to come with a laptop computer.  We will walk
through the installation and use of the Checker Framework, including both
exercises and running it on participants' code.



\section{Additional Tags}

Please provide any related tags to your presentation 
(comma separated list):

\medskip

Design, Documentation, Experimentation, Languages, Reliability,
Security, Verification,
User-defined type system, Static type system


\section{Audience level}

Novice\\
X Intermediate\\
Expert

\medskip

What is the difference to the Skill level??


\section{O'Reilly author}

Are you an O'Reilly author? If so, please list your main contact with us.

\medskip

No.


\section{Additional Notes}

Any other information you wish the organizer to know? If you are
proposing a workshop, please tell us what type of learning experience
you have in mind (hands-on, demo, etc.). If you selected ``Other'' in
the Session type field, please provide more details here.

\medskip

We can do either a 90 minute workshop or a 40 minute presentation.  We
prefer a 90 minute workshop, which would be more interactive and provide
hands-on sessions to install and use the Checker Framework and walks
through building a custom type checker.  In a 40-minute presentation, we
would do a demo but the audience would not get practice in using the tools
themselves.


\section{Speakers}

Michael D. Ernst


\end{document}














Categories:
{D.2.4}{Software Engineering}{Software/Program Verification}
{D.3.3}{Programming Languages}{Language Constructs and Features}
{D.3.4}{Programming Languages}{Processors}

Terms:
{Design, Documentation, Experimentation, Languages, Reliability,
  Security, Verification}

Keywords:
{Pluggable type-system, user-defined type system, Checker
  Framework, Java, type checking}


\section{Motivation}

Buggy software is costly to society:  in 2002, insufficient software
testing and verification were estimated to cost the US economy \$22--60
billion annually~\cite{NIST02-3}.
One approach to reducing this cost is program verification via type systems.
%
A static type system helps programmers to detect and prevent errors.
However, a language's built-in type system does not help to detect and
prevent \emph{enough} errors, because it cannot express
certain important invariants.  A user-defined, or pluggable, type
system enriches the built-in type system by expressing extra information
about types via type qualifiers.  Example qualifiers include
\code{nonnull}, \code{readonly}, \code{interned}, \code{locked}, and
\code{tainted}, as well as much more complex type systems.  Pluggable types
permit more expressive compile-time checking and guarantee the absence of
additional errors.


We present the Checker Framework for defining pluggable types in a
backward-compatible way, in the context of a real-world programming
language.  Our implementation is for Java, but the ideas and techniques
translate to other languages.
The Checker Framework is available in source and binary forms from
\url{http://types.cs.washington.edu/checker-framework/}.

The framework makes it easy to define type-checkers either
declaratively or procedurally.
The framework makes use of a syntax for type qualifiers that is usable
today and is planned for inclusion in a future version of the Java
language.
The framework is in daily academic use by type system designers who are
creating new type systems  and evaluating them in a realistic context.
The framework is in daily industrial use by programmers who want a
guarantee that certain types of errors cannot occur.
The framework has been applied to millions of lines of code and has found
errors in every codebase to which it has been applied.

The framework ships with the following checkers:
\begin{itemize}    \itemsep 0pt \parskip 0pt
\item Nullness checker for null pointer errors
\item Interning checker for errors in equality testing and interning
\item IGJ checker for mutation errors (incorrect side effects), based on the IGJ type system
\item Javari checker for mutation errors (incorrect side effects), based on the Javari type system
\item Lock checker for concurrency and lock errors, inspired by the Java Concurrency in Practice (JCIP)~\cite{Goetz2006} annotations
\item Fake enum checker for integers used as enumerations
\item Tainting checker for trust and security errors
\item Linear checker to control aliasing and prevent re-use
\item Regex checker to prevent use of syntactically invalid regular expressions
\item Internationalization checker to ensure that code is properly internationalized: user-visible text is obtained from a localization resource, and proper keys are used for a localization resource
\item Property file checker to check keys used for property files and
  resource bundles, and to check internationalization and compiler messages
\item Basic checker for customized checking without writing any code
\end{itemize}
Additional checkers written by third parties are also available.
The Checker Framework comes with a 100-page manual, mostly describing each
of the pluggable type systems.


\section{Benefits to the profession}

Program types are the shining success of formal methods.  Types are widely
adopted by rank-and-file programmers to detect errors and --- more
importantly --- to verify that no more errors (of particular varieties)
exist.  However, the uptake of types into practice has been
limited by their migration into mainstream programming languages, a
process that takes decades.  Our work shortcuts this process,
offering benefits both to practitioners and to researchers.

For practitioners, we offer the ability to adopt new type systems without
breaking compatibility with existing programs, systems, and languages.
This enables faster adoption of new ideas.  Our work also permits
programmers to define application-specific type systems that verify
important properties of their systems.  We believe that these changes have
the potential to transform the way that software is written, and the way
that programmers think about their programs.  Rather than testing to try to
eliminate as many bugs as possible, types provide a pathway to verification
and a mindset of writing correct code.

For researchers, we offer the opportunity to experiment with type systems
in the context of a widely-used industrial language, Java.  To date,
evaluating a new type system has generally required either defining a new
language, or extensive and incompatible changes to an existing language.
Incompatibility with existing tools and programs limits the scope of
experimentation, and it limits adoption in practice.  This has too often led to
flawed theory --- say, the approach is
unscalable, or it does not handle features like iterators, generics, or the
visitor pattern, or a proof of soundness makes unrealistic simplifications.
Empirical evaluation is not optional --- it is a necessary part of
programming language research.  By lowering the bar to experimentation,
while retaining compatibility with existing tools and programs, we hope to
change the way that researchers think about implementing and evaluating
their ideas.  The result should be more relevant and worthwhile theory and
systems, achieved more quickly.



% \textbf{Intellectual merit.}
% The proposed research will advance the state of the art, both in terms of
% theory and in terms of practice.  The creation of a uniform and
% backward-compatible framework for type systems is a long-standing research
% goal, but previous attempts have fallen short.  Our work takes a novel
% and more practical approach, both in terms of algorithms and in terms of
% implementation strategy.  Our framework for type checking and inference can be
% instantiated for use in many 
% domains, as indicated by our collaborations with other researchers
% to support their work.  The type systems raise interesting theoretical
% questions about program specification, verification, and correctness.
% Experiments enabled by our work will yield insight into the applicability
% of theory to practice.  Already, users of our prototype implementation have
% revised theory based on experimental evaluation.  Future users will be able
% to develop even deeper theoretical insights that have the promise to aid in
% understanding of program analysis and the behavior operation of computer
% systems.


% \section{Goal and objectives}
% 
% Our aim is to bring the research and practitioner communities closer
% together, in the concrete context of type systems, by offering each
% community something that it has lacked so far.
% 
% For researchers, it is too hard to evaluate a proposed type system in a
% realistic way, which requires integration with a full, real programming
% language and evaluation on real, sizable programs.  It is also too hard to
% package type systems for use by practitioners, even after they have been
% demonstrated to be useful.
% 
% For practitioners, many important properties of a program must be expressed
% informally in English, and no tools exist to detect and prevent these
% problems.
% 
% We will demonstrate a solution to many of these problems.





\section{Related work}

This document only skims the surface of related work.
For a more extensive discussion, please see~\cite{PapiACPE2008}.

The most closely related frameworks are JQual~\cite{GreenfieldboyceF2007},
and JavaCOP~\cite{AndreaeNMM2006}.  These two frameworks, like the Checker
Framework, have been used to implement the Javari~\cite{TschantzE2005} type
system for enforcing reference immutability.  The version implemented in
our framework supports the entire Javari language (5 keywords).  The JQual
and JavaCOP versions have only partial support for 1 keyword (\<readonly>),
and neither one properly implements method overriding, a key feature of an
object-oriented language.  Neither JQual nor JavaCOP scales to real
programs --- either in terms of program size or of language features.

The JastAdd extensible Java compiler~\cite{EkmanH2007:JastAdd} includes a
module for checking and inferencing of non-null
types~\cite{EkmanH2007:NonNull}, though it is less featureful and correct
than our nullness checker.  JastAdd could theoretically be used as a
framework to build other type systems.

The goal of a bug-finder such as FindBugs~\cite{HovemeyerSP2005} is to find
just a few errors, not all errors.  FindBugs discards most reports to avoid
false positives.  Its benefit is that it requires few user annotations.

The Type Annotations (JSR 308) Specification~\cite{JSR308} explains how
Java will support expressing pluggable type systems.

\bibliographystyle{plain}
\bibliography{bibstring-unabbrev,ernst,testing,types,concurrency}

% \newpage
% 
% \section{Demo setup}
% 
% The demo will mix slides (that introduce concepts) with live use of the
% pluggable type-checkers, under multiple environments such as Emacs and
% Eclipse.  We will show multiple checkers, including ones for nullness,
% immutability, and tainting.  In each case, we will show real code that
% contains real errors committed by programmers, and will show how pluggable
% type-checking can detect these errors and verify the absence of further
% errors.
% 
% 
% \section{Presenters}
% 
% Mahmood Ali is a graduate student in the MIT Computer Science and
% Artificial Intelligence Lab.  He designed and implemented the Type
% Annotations (JSR 308) compiler that has been incorporated into Sun's
% OpenJDK, and the Checker Framework that enables the creation of
% user-defined pluggable type systems.
% 
% Michael D. Ernst is an Associate Professor in the Computer Science and
% Engineering department at the University of Washington.  He is the
% specification lead for the Type Annotations language extension (``JSR
% 308'') that is part of Java 7 --- the first non-Sun-employee to be the
% specification lead for a Java language change.  Ernst's research aims to
% make software more reliable, more secure, and easier (and more fun!) to
% produce. His primary technical interests are in software engineering and
% related areas, including programming languages, type theory, security,
% program analysis, bug prediction, testing, and verification. Ernst's
% research combines strong theoretical foundations with realistic
% experimentation, with an eye to changing the way that software developers
% work.  Dr. Ernst was previously a tenured professor at MIT, and before that
% a researcher at Microsoft Research.  More information is available at his
% homepage:  \url{http://www.cs.washington.edu/homes/mernst/}.
% 
% Ernst received a ``Java Rock Star'' award for his presentation on the
% Checker Framework at the JavaOne 2009 conference.  He and/or Ali have
% recently given presentations on the topic at Dagstuhl (4/2008), J-Spring
% (4/2008), ECOOP (7/2008), OOPSLA (10/2008), Devoxx (12/2008), Google
% (12/2008), and SeaJUG (4/2009), PPPJ (8/2009), OOPSLA (10/2009), and Devoxx
% (11/2009).



\end{document}



% \section{Summary of contents}
% 
% The Java type system helps programmers to detect and prevent
% errors. However, Java's built-in type system does not detect and prevent
% \emph{enough} errors. It cannot express important properties such as: a reference
% is non-null, a data structure should not be modified, or a string has been
% properly sanitized.
% 
% Java's type system can be extended to enforce such properties statically,
% preventing whole classes of errors. 
% A user-defined, or pluggable, type system enriches a
% language's built-in type system via type qualifiers.  Pluggable types
% permit more expressive compile-time checking, and they can guarantee
% the absence of additional errors.
% 
% Despite considerable interest in user-defined type qualifiers,
% previous implementations have been too inexpressive, unscalable, or
% incompatible with existing languages or tools.  This has hindered the
% evaluation, understanding, and uptake of pluggable types.
% 
% More recently, pluggable types (that operate as an optional plug-in to the
% compiler) have become practical for Java. This benefits two constituencies:
% 
% \begin{itemize}
% \item Programmers can detect and prevent errors in their code.
% \item Type system designers can easily evaluate new type systems in the
%   context of an industrial programming language.
% \end{itemize}
% 
% The line between these constituencies is porous: a developer may create a
% simple type system to enforce an application-specific property.
% 
% This demo will explain how to both \emph{use} and \emph{create}
% pluggable type systems. Afterward, you will be ready to use pluggable types
% with confidence and to create simple type-checkers of your own, and you
% will know where to look for more information.
% 
% The Checker Framework is expressive and flexible.  It builds in many
% features needed by type system designers and programmers, including:
% \begin{itemize}
% \item backward-compatible Java and classfile syntax for type qualifiers
% \item integration with javac and Eclipse
% \item three type inference tools to ease programmer annotation burden
% \item declarative and procedural syntax for writing type-checking rules
% \item flow-sensitive type qualifier inference
% \item polymorphism over types (Java generics)
% \item polymorphism over type qualifiers
% \item implicit and default qualifiers
% \end{itemize}
% 
% The hands-on activities will use the Checker Framework, which provides
% expressive, concise declarative and procedural mechanisms for creating
% pluggable type checkers for Java. However, the ideas extend to other
% frameworks and languages. The Checker Framework is the basis of many
% practical checkers that are in daily use and have found hundreds of
% important errors in millions of lines of code. The Checker Framework is
% being used by researchers worldwide, and it has yielded new insight into
% research type systems. It is usable with any version of the Java language,
% but pluggable types are so important that Java 7 will contain special
% syntax (designed by the presenter of this demo) to support type
% qualifiers.
% 
% The demo will cover both theory and practice, and is applicable both to
% researchers and also to practitioners who wish to verify domain-specific
% properties in their software. For example, you will learn how type system
% designers can build scalable, robust, and easily deployable
% type-checkers. During the demo, you will build your own
% type-checker. After the demo, you will be prepared to use, create, or
% customize pluggable type-checkers.
% 
% Attendees are suggested to bring laptop computers, though it is also
% possible to look on with another attendee.
% 
% The Checker Framework is freely available at
% \url{http://types.cs.washington.edu/jsr308/}.
% The distribution includes source code, binaries, extensive
% documentation, and example type-checkers.
% 
% 
% 
% 
% The demo is relevant because it is a rare example of academic research
% having a direct effect on industrial practice and becoming available to
% every Java developer via support in the Java language standard.



%%% Local Variables: 
%%% mode: latex
%%% End: 

% LocalWords:  Devoxx SeaJUG PPPJ deployable nonnull readonly classfile XSS
% LocalWords:  checkable toolsets Dagstuhl ernst
